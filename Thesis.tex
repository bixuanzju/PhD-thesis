\documentclass{mimosis}

\usepackage{metalogo}

%%%%%%%%%%%%%%%%%%%%%%%%%%%%%%%%%%%%%%%%%%%%%%%%%%%%%%%%%%%%%%%%%%%%%%%%
% Some of my favourite personal adjustments
%%%%%%%%%%%%%%%%%%%%%%%%%%%%%%%%%%%%%%%%%%%%%%%%%%%%%%%%%%%%%%%%%%%%%%%%
%
% These are the adjustments that I consider necessary for typesetting
% a nice thesis. However, they are *not* included in the template, as
% I do not want to force you to use them.

% This ensures that I am able to typeset bold font in table while still aligning the numbers
% correctly.
\usepackage{etoolbox}

\usepackage[binary-units=true]{siunitx}
\DeclareSIUnit\px{px}

\sisetup{%
  detect-all           = true,
  detect-family        = true,
  detect-mode          = true,
  detect-shape         = true,
  detect-weight        = true,
  detect-inline-weight = math,
}

%%%%%%%%%%%%%%%%%%%%%%%%%%%%%%%%%%%%%%%%%%%%%%%%%%%%%%%%%%%%%%%%%%%%%%%%
% Hyperlinks & bookmarks
%%%%%%%%%%%%%%%%%%%%%%%%%%%%%%%%%%%%%%%%%%%%%%%%%%%%%%%%%%%%%%%%%%%%%%%%

\usepackage[%
  colorlinks = true,
  citecolor  = RoyalBlue,
  linkcolor  = RoyalBlue,
  urlcolor   = RoyalBlue,
  ]{hyperref}

\usepackage{url}
\usepackage{nameref}
\usepackage[capitalise]{cleveref}



\usepackage{bookmark}

%%%%%%%%%%%%%%%%%%%%%%%%%%%%%%%%%%%%%%%%%%%%%%%%%%%%%%%%%%%%%%%%%%%%%%%%
% Bibliography
%%%%%%%%%%%%%%%%%%%%%%%%%%%%%%%%%%%%%%%%%%%%%%%%%%%%%%%%%%%%%%%%%%%%%%%%
%
% I like the bibliography to be extremely plain, showing only a numeric
% identifier and citing everything in simple brackets. The first names,
% if present, will be initialized. DOIs and URLs will be preserved.

\usepackage[%
  hyperref     = true,
  natbib       = true,
  sortcites    = true,
  maxbibnames  = 9,
  maxcitenames = 2,
  style        = numeric,
  ]{biblatex}

% \input{bibliography-mimosis}
% \bibliographystyle{ACM-Reference-Format}
\bibliography{Thesis}

%%%%%%%%%%%%%%%%%%%%%%%%%%%%%%%%%%%%%%%%%%%%%%%%%%%%%%%%%%%%%%%%%%%%%%%%
% Fonts
%%%%%%%%%%%%%%%%%%%%%%%%%%%%%%%%%%%%%%%%%%%%%%%%%%%%%%%%%%%%%%%%%%%%%%%%

\ifxetexorluatex
  \setmainfont{Minion Pro}
\else
  \usepackage[lf]{ebgaramond}
  \usepackage[oldstyle,scale=0.7]{sourcecodepro}
  \singlespacing
\fi

\renewcommand{\th}{\textsuperscript{\textup{th}}\xspace}

\newacronym[description={Principal component analysis}]{PCA}{PCA}{principal component analysis}
\newacronym                                            {SNF}{SNF}{Smith normal form}
\newacronym[description={Topological data analysis}]   {TDA}{TDA}{topological data analysis}

\newglossaryentry{LaTeX}{%
  name        = {\LaTeX},
  description = {A document preparation system},
  sort        = {LaTeX},
}

\newglossaryentry{Real numbers}{%
  name        = {$\real$},
  description = {The set of real numbers},
  sort        = {Real numbers},
}

\makeindex
\makeglossaries

% 12pt, please
% \KOMAoptions{fontsize=12pt}


% Basics
\usepackage{fixltx2e}
\usepackage{url}
\usepackage{fancyvrb}
\usepackage{mdwlist}  % Miscellaneous list-related commands

% https://www.nesono.com/?q=book/export/html/347
% Package for inserting TODO statements in nice colorful boxes - so that you
% won’t forget to fix/remove them. To add a todo statement, use something like
% \todo{Find better wording here}.
\usepackage{todonotes}

%% Math
\usepackage{bm}       % Bold symbols in maths mode
\usepackage{amssymb}

% http://tex.stackexchange.com/questions/114151/how-do-i-reference-in-appendix-a-theorem-given-in-the-body
\usepackage{thmtools, thm-restate}

%% Theoretical computer science
\usepackage{stmaryrd}
\usepackage{mathtools}  % For "::=" ( \Coloneqq )

%% Font
% \usepackage[euler-digits,euler-hat-accent]{eulervm}

\usepackage{pifont}

\usepackage{ottalt}

\usepackage{comment}

% Code highlighting
\usepackage{listings}

\lstset{%
  % backgroundcolor=\color{white},
  basicstyle=\small\ttfamily,
  keywordstyle=\sffamily\bfseries,
  captionpos=none,
  columns=flexible,
  keepspaces=true,
  showspaces=false,               % show spaces adding particular underscores
  showstringspaces=false,         % underline spaces within strings
  showtabs=false,                 % show tabs within strings adding particular underscores
  breaklines=true,                % sets automatic line breaking
  breakatwhitespace=true,         % sets if automatic breaks should only happen at whitespace
  escapeinside={(*}{*)},
  literate={->}{{$\rightarrow$}}1 {Top}{{$\top$}}1 {=>}{{$\Rightarrow$}}1 {/\\}{{$\Lambda$}}1,
  tabsize=2,
  commentstyle=\color{purple}\ttfamily,
  stringstyle=\color{red}\ttfamily,
  sensitive=false
}

\lstdefinelanguage{sedel}{
  keywords={Int, Bool, String, this, inherits, super, type, Trait, override, new, if, then, else, let, in, letrec},
  identifierstyle=\color{black},
  morecomment=[l]{--},
  morecomment=[l]{//},
  morestring=[b]",
  xleftmargin  = 3mm,
  morestring=[b]'
}

\lstdefinelanguage{gbeta}{%
  language     = java,
  morekeywords = {virtual,refine},
  xleftmargin  = 3mm
}


\lstdefinelanguage{JavaScript}{
  keywords={const, extends, super, class, export, boolean, throw, implements, import, this, typeof, new, true, false, catch, function, return, null, catch, switch, var, if, in, while, do, else, case, break},
  identifierstyle=\color{black},
  comment=[l]{//},
  morecomment=[s]{/*}{*/},
  morestring=[b]',
  xleftmargin  = 3mm,
  morestring=[b]"
}


\lstset{language=sedel}

\theoremstyle{plain}
\newtheorem{theorem}{Theorem}
\newtheorem{lemma}{Lemma}
\newtheorem{corollary}{Corollary}
\theoremstyle{definition}
\newtheorem{definition}{Definition}
\newtheorem{example}{Example}
\theoremstyle{remark}
\newtheorem*{remark}{Remark}
\newtheorem{observation}{Observation}

\input{pl-theory.tex}

% Ott includes
\inputott{ott-rules}
\renewcommand\ottaltinferrule[4]{
  \inferrule*[narrower=0.5,lab=#1,#2]
    {#3}
    {#4}
}

%%----------------------------------------------------------------------------%%
%%    Environment for Declaration                                             %%
%%----------------------------------------------------------------------------%%


\makeatletter
\newcommand{\makedeclaration}
{
\chapter*{Declaration}
% \addcontentsline{toc}{chapter}{Declaration}
\noindent I declare that this thesis represents my own work, except
where due acknowledgement is made, and that it has not been
previously included in a thesis, dissertation or report submitted
to this University or to any other institution for a degree,
diploma or other qualifications.
\vspace*{1.5in}

\noindent%
\begin{tabular}{@{}l@{}}
\dotfill \\
\@author\hspace*{3cm}\\
\@date\\
\end{tabular}
}

\makeatother

%%----------------------------------------------------------------------------%%
%%    Environment for Acknowledgments                                         %%
%%----------------------------------------------------------------------------%%

\makeatletter
\newcommand{\makeAck}
{
\chapter*{Acknowledgments}
% \addcontentsline{toc}{chapter}{Acknowledgements}
\def\CTeXPreproc{Created by ctex v0.2.14, don't edit!}
\vspace{-5mm}

\noindent First and foremost, I would like to thank my PhD advisor Dr. Chi Kwong
Hui for his continued support and mentorship throughout my studies. Under his
guidance, I had the opportunity to explore many interesting research directions.
His feedbacks and continued guidances greatly helped shape my research and
dissertation.\\
\\
I would also like to thank Dr. Siu Ming Yiu, for his valuable suggestions and
guidelines. He helped revise my papers and illustrated to me the way to make
papers clearer. His ideas and feedbacks were nothing short of inspiring, which I
greatly enjoyed. Apart from my study, he also provided valuable advices about my
career path.\\
\\
During the years I spent at HKU, I had the opportunity to collaborate with a
number of excellent researchers: Dr. Wenting Tu, Dr. Min Yang, Dr. Xingmin Cui,
for their valuable discussions in general. I also thank my friends, to name a
few, Zhuolan Bao, Pengyu Chen, Zhendong Feng, Yuzhi Liang, Xuyan Qiu, Shiru Tao,
Dr. Yanbin Tang, Xin Wang, Yanlin Wang, Dr. Jun Xing, Leo Yeung, Dr. Xiaoqi Yu,
Su Yuan, Jingyu Zhan, Ruoqing Zhang for their encouragements and supports.\\
\\ 
Last but not least, I would love to thank my family for being extremely
supportive throughout my studies. I consider myself lucky to have such amazing
parents. Without the help and vision of them, I shall never be able make it this
far.

}
\makeatother


%%----------------------------------------------------------------------------%%
%%    Environment for Title Page                                              %%
%%----------------------------------------------------------------------------%%

\makeatletter
\renewcommand{\maketitle}
{%
  \begin{titlepage}
    \renewcommand{\baselinestretch}{1}
    \begin{center}
      \vspace*{\stretch{3}}
      {\LARGE\@title\par}
      \vspace*{1cm}
      {\large\textit{by}\par}
      \vspace*{1cm}
      {\Large\@author\par}
      \vspace*{\stretch{5}}
      {{\includegraphics[width=30mm]{figures/hku}} \par}
      {\hbox{}\par}
      \vspace*{\stretch{5}}
      {
      	\textsl{(Temporary Binding for Examination Purposes)} \\
      	\vspace*{3cm}
      	{\normalsize
      	A thesis submitted in partial fulfillment of the requirements for \\
        the degree of Doctor of Philosophy \\
        at The University of Hong Kong \\
        \par
        }
      }
      \vspace*{\stretch{1}}
      {\large\@date\par}
      \vspace*{\stretch{1}}
    \end{center}
  \end{titlepage}
} \makeatother



%%%%%%%%%%%%%%%%%%%%%%%%%%%%%%%%%%%%%%%%%%%%%%%%%%%%%%%%%%%%%%%%%%%%%%%%
% Incipit
%%%%%%%%%%%%%%%%%%%%%%%%%%%%%%%%%%%%%%%%%%%%%%%%%%%%%%%%%%%%%%%%%%%%%%%%

\title{Disjoint Intersection Types: Theory and Practice}
\author{Xuan Bi}

\begin{document}

\frontmatter
  \maketitle
  % \begin{titlepage}
  \vspace*{5cm}
  \makeatletter
  \begin{center}
    \begin{Huge}
      \@title
    \end{Huge}\\[0.1cm]
    %
    \begin{Large}
      \@subtitle
    \end{Large}\\
    %
    \emph{by}\\
    \@author
    %
    \vfill
    A document submitted in partial fulfillment
    of the requirements for the degree of\\
    \emph{Doctor of Philosophy}\\
    at\\
    \textsc{The University of Hong Kong}
  \end{center}
  \makeatother
\end{titlepage}

\newpage
\null
\thispagestyle{empty}
\newpage

  \begin{center}
  \textsc{Abstract}
\end{center}

Programs are hard to write. It was so 50 years ago at the time of the so-called
\textit{software crisis}; it still remains so nowadays, as the software we use
daily is getting more and more complex and hard to maintain. Over the years, we
have learned---the hard way---that software should be constructed in a
\textit{modular} way, i.e., as a network of smaller and loosely connected
modules. To help programmers write modular code, researchers and software
practitioners have developed new methodologies; new programming paradigms; more
expressive type systems; as well as better tooling support. Still, this is not
enough to cope with today's needs. Several reasons have been raised for the lack
of satisfactory solutions, but one that is constantly pointed out is the
inadequacy of existing programming language features for the construction of
modular software.

This thesis investigates \textit{disjoint intersection types}, a variant of
intersection types. Disjoint intersections types have great potential to serve
as a foundation for powerful, flexible and yet type-safe and easy to reason
object-oriented languages, suitable for writing modular software. On the
theoretical side, this thesis shows how to significantly increase the
expressiveness of disjoint intersection types by adding support for
\textit{nested composition}, along with parametric polymorphism. Nested
composition extends inheritance to work on a whole family of classes, enabling
high degrees of modularity and code reuse. The combination with parametric
polymorphism further improves the state-of-art encodings of extensible designs.
However, the extension with nested composition and parametric polymorphism is
challenging, for two different reasons. Firstly, the subtyping relation that
supports these features is non-trivial. Secondly, the syntactic method used to
prove coherence for previous calculi with disjoint intersection types is too
inflexible. This thesis addresses the first problem by adapting and extending
the well-known BCD subtyping relation with records, universal quantification and
coercions. To address the second problem, this thesis proposes a powerful proof
method to establish coherence. Thus this thesis puts disjoint intersection types
on a solid footing by thoroughly exploring their meta-theoretical properties.

On the pragmatic side, this thesis propose a new language design with support
for \textit{first-class traits}, \textit{dynamic inheritance} and nested
composition. First-class traits allows two objects of statically unknown types
to be composed without conflicts. Dynamic inheritance allows a class to inherit
from other classes at \textit{run time}. To address the challenges of typing
first-class traits and detecting conflicts statically, this thesis shows how to
model source language constructs for first-class traits and dynamic inheritance
by leveraging the fine-grained expressiveness of disjoint intersection types. To
illustrate the applicability of the new design, this thesis conducts a case
study that modularizes programming language features using a highly modular form
of visitors.

All the results and metatheory presented (unless otherwise indicated) in this
thesis are mechanized in the Coq proof assistant in order to show the
rigorousness of the approach. This thesis unifies ideas that are seemingly
unrelated but powerful on their own---dynamic inheritance, first-class traits,
family polymorphism---by a single lightweight mechanism, thus providing new
insights into software modularity and extensibility.


  \makedeclaration
  \makeAck
  \tableofcontents
  \listoffigures
  \listoftables

\mainmatter

  
%%%%%%%%%%%%%%%%%%%%%%%%%%%%%%%%%%%%%%%%%%%%%%%%%%%%%%%%%%%%%%%%%%%%%%%%
\chapter{Introduction}
%%%%%%%%%%%%%%%%%%%%%%%%%%%%%%%%%%%%%%%%%%%%%%%%%%%%%%%%%%%%%%%%%%%%%%%%


% \begin{center}
%   \begin{minipage}{0.5\textwidth}
%     \begin{small}
%       In which the reasons for creating this package are laid bare for the
%       whole world to see and we encounter some usage guidelines.
%     \end{small}
%   \end{minipage}
%   \vspace{0.5cm}
% \end{center}


This thesis investigates disjoint intersection types---a variant of intersection
types---focusing on its theoretical foundation and applications in the context
of object-oriented programming. The results are three new typed calculi, the
first two being core calculi and the last one a source calculus, combining the
power of parametric polymorphism, a rich subtyping relation with the
fine-grained expressiveness of disjoint intersection types. The key contribution
of the thesis is that it unifies ideas that are seemingly unrelated but
powerful on their own in object-oriented programming---dynamic inheritance,
first-class traits, family polymorphism, extensible design patterns---by a
single lightweight mechanism, thus providing new insights into software
modularity and extensibility.

\section{Motivation}

Programs are hard to write. It was so 50 years ago at the time of the so-called
\emph{software crisis}~\citep{Naur:1969:SER:1102020}; it still remains so
nowadays, as the software we use daily is getting more and more complex and
harder to maintain. Over the years, we have learned---the hard way---that
software should be constructed in a \emph{modular} way, i.e., as a network of
smaller and loosely connected modules. To facilitate writing modular code,
researchers and software practitioners have developed new methodologies; new
programming paradigms; more expressive type systems; as well as better tooling
support. Still, this is not enough to cope with today's needs. We will mention
some limitations of existing mainstream languages on supporting modular
programming shortly. But before that, let us identify the following
well-established requirements for construction of modular software:
\begin{enumerate}
\item \textbf{Extensibility in both dimensions:} Extensions may require new
  variants to the datatype and new operations on the datatype.
\item \textbf{Strong static type safety:} Extensions cannot cause run-time type errors.
\item \textbf{No modification or duplication:} Existing code must not be
  modified nor duplicated.
\item \textbf{Separate compilation and type-checking:} Safety checks or
  compilation steps must not be deferred until linking or at run time.
\item \textbf{Independent extensibility:} Independently developed extensions
  should be composable so that they can be used jointly.
\item \textbf{Scalability:} Extension should be scalable. The amount of code
  needed should be proportional to the functionality added.
\item \textbf{Non-destructive extension:} The base system should still be
  available for use within the extended system.
\end{enumerate}
The first four of these requirements correspond to
\citeauthor{wadler1998expression}'s expression
problem~\citep{wadler1998expression}. \citet{Zenger-Odersky2005} added the 5\th
requirement. The last two requirements were proposed by \citet{Nystrom:2006}.
Scalability (6\th) is often but not necessarily satisfied by separate compilation;
it is important for extending large software. Non-destructive extension (7\th) is
an important requirement for legacy and performance reasons: it enables clients
of the extended system to reuse code and data of the base system, allowing some
interoperability between new functionality and legacy code. To address the
requirements, many solutions have been proposed over the years (for example, see
\citet{oliveira2012extensibility, wang2016expression, oliveira09modular,
  swierstra_2008, Zenger-Odersky2005}, to cite a few). They differ considerably
in the language context with varying degrees of extensibility they offer, as
well as the limitations they impose. Building on the previous solutions, this
thesis proposes a lightweight language design that addresses all of these
requirements.

Various programming language features support modular programming, with varying
degrees of limitations. Functional languages, notably ML and OCaml, use module
systems~\citep{MacQueen_1984} for flexible program construction. In particular,
ML ``functors''---which are functions over modules---allow
one to develop and compile a module independently from the modules on which it
depends. One functor can then be instantiated with multiple different modules
during the execution of the program, enabling a powerful form of code reuse. One
prominent weakness of ML modules (at least in current module implementations) is
that they cannot be defined recursively, that is, mutually recursive functions
and datatypes must be written in the same module, even though they may belong
conceptually to different modules. Another limitation is that modules form a
separate, higher-order functional language on top of the core and therefore ML
is actually two languages in one. Moreover, module systems usually put more
emphasis on supporting data abstraction, which adds considerable complexity to
languages adopting module systems as the primary way to construct modular
programs.
% Datatype-genetic programming is another
% approach to modularity, where programs can be parameterized over
% datatypes. By abstracting from the differences in what would otherwise be
% separate but similar programs, one can write a single unified piece of program
% and instantiate it in various ways to retrieve specific programs. However, this
% approach usually requires quite advanced type system features, e.g., type classes, type
% families, or even dependent types!

Object-oriented languages, on the other hand, use classes and inheritance as primary
mechanisms to support code extensibility and reuse. Single inheritance found in
mainstream object-oriented languages (such as Java or C++) is perhaps the most
well-known and well-studied mechanism. However, programmers have long realized
that single inheritance is not flexible enough when it comes to structuring a
class hierarchy: it works for small and simple extensions, but does not work
well for larger software systems such as compilers and operating systems. There
has been great interest in the past several years in mechanisms for providing
greater extensibility in object-oriented languages. Of particular relevance to
the subject of this thesis are three powerful linguistic mechanisms for software
extensibility, providing increasing order of flexibility, as well as complexity:
first-class classes~\citep{DBLP:conf/oopsla/TakikawaSDTF12}, (first-class)
mixins/traits~\citep{bracha1990mixin, scharli2003traits}, and family
polymorphism~\citep{Ernst_2001}.


\subsection{First-Class Classes}

Many dynamically typed languages (including JavaScript, Ruby, Python or Racket)
support \emph{first-class classes}. In those languages classes are first-class
values and, like any other values, they can be passed as an argument, or
returned from a function. Furthermore, first-class classes support \emph{dynamic inheritance}:
i.e., they can inherit from other classes at \emph{run time},
enabling programmers to abstract over the inheritance hierarchy. % For example,
% mixins~\citep{bracha1990mixin} become programmer-defined constructs---a mixin is
% simply a function that takes a class as an argument and returns a subclass.
In contrast, most statically typed languages do not have first-class classes and
dynamic inheritance. While all statically typed object-oriented languages allow first-class
\emph{objects} (i.e., objects can be passed as arguments and returned as
results), the same is not true for classes. Classes in languages such as Scala,
Java or C++ are typically a second-class construct, and the inheritance
hierarchy is \emph{statically determined}.

Despite the popularity and expressive power of first-class classes in
dynamically typed languages, there is surprisingly little work on typing of
first-class classes. First-class classes and dynamic inheritance pose well-known
difficulties in terms of typing. For example, in his thesis,
\citet{bracha1992programming} comments several times on the difficulties of
typing dynamic inheritance and first-class mixins, and proposes the restriction
to static inheritance that is also common in statically typed languages. One of
the motivations for this thesis is to propose a type discipline that can encode
first-class classes. Moreover, we push this one step further: for the first
time, this thesis shows how to encode \emph{first-class traits} in a
statically typed setting. But first things first, let us briefly explain what are
traits, and the related concept ``mixins''.

% Racket~\citep{DBLP:conf/aplas/FlattFF06} supports typed first-class classes, as
% well as a \emph{dynamically typed model} of first-class traits. However, unlike
% Racket's first-class classes and mixins, there is no type system supporting the
% use of first-class traits. A key difficulty is \emph{statically} detecting
% conflicts, even in the presence of polymorphism. Type system limitations prevent
% most statically typed languages from having typed first-class traits.

\subsection{(First-Class) Mixins and Traits}

As remarked earlier, single inheritance is inadequate and inflexible to write
large software. To overcome this limitation, multiple
inheritance was proposed as a generalization of
single inheritance. However, multiple inheritance is renowned for being tricky
to get right, largely because of the possible ambiguity issues that arise when
conflicting features are inherited along different paths.
Mixins~\citep{bracha1990mixin} provide a simple mechanism for multiple
inheritance without the ambiguity issue. A mixin is a class declaration
parameterized over a superclass, able to extend a variety of parent classes with
the same set of features. Mixins are composed \emph{linearly}, and that methods defined
in mixins appearing later override all the identically named methods of earlier
mixins. Because of the linear order of composition, a class may not be able
to access a member of a given super-mixin because the member is overridden by
another mixin.

Traits~\citep{scharli2003traits, Ducasse_2006} are an alternative to mixins, and
other models of multiple inheritance. The key difference between traits and
mixins lies on the treatment of conflicts when composing multiple traits/mixins.
Mixins adopt an \emph{implicit} resolution strategy for conflicts, where the
compiler automatically picks one implementation in case of conflicts. Traits, on
the other hand, employ an \emph{explicit} resolution strategy, where the
compositions with conflicts are rejected, and the conflicts are explicitly
resolved by programmers. \citet{scharli2003traits} make a good case for the
advantages of the trait model. In particular, traits avoid bugs that could arise
from accidental conflicts that were not noticed by programmers. With the mixin
model, such conflicts would be silently resolved, possibly resulting in
unexpected run-time behavior due to a wrong method implementation choice. From a
modularity point of view, the trait model also ensures that composition is
\emph{commutative}, thus the order of composition is irrelevant and does not
affect the semantics. \citet{bracha1992programming} claims that ``\emph{The only
  modular solution is to treat the name collisions as errors...}'',
strengthening the case for the use of a trait model of composition. Otherwise,
if the semantics is affected by the order of composition (like in the mixin model), global knowledge about
the full inheritance graph is required to determine which implementations are
chosen.

Mixins and traits as found in most statically typed languages/calculi are
typically a second-class construct. Promoting mixins/traits to first-class
citizens adds considerable expressiveness and flexibility in terms of software
extensibility, as will be illustrated throughout this thesis. Only recently some
progress has been made in statically typing first-class classes and dynamic
inheritance~\citep{DBLP:conf/oopsla/TakikawaSDTF12,DBLP:conf/ecoop/LeeASP15}.
However, prior to this thesis, no previous work supports \emph{typed first-class traits}.
A key challenge, compared to models with first-class
classes or mixins, is how to detect conflicts at compile time even when \emph{not}
knowing all components being composed statically. This is important because in
the setting with dynamic inheritance and polymorphism, the possibility of
accidental conflicts caused by programmers is extremely high.



\subsection{Family Polymorphism and Nested Composition}

The last mechanism---also the most powerful and complex one---is \emph{family
  polymorphism}. In family polymorphism~\citep{Ernst_2001}, inheritance is
extended to work on a \emph{whole family of classes}, rather than just a single
class. This enables high degrees of modularity and code reuse, enabling simple
solutions to hard programming language problems, like the expression
problem~\citep{wadler1998expression}. An essential feature of family
polymorphism is \emph{nested composition}~\citep{Corradi_2012,
  ErnstVirtual,Nystrom_2004}, which allows the automatic inheritance/composition
of nested (or inner) classes when the enclosing classes are composed.
\citet{Nystrom_2004} call this \emph{scalable extensibility}: ``the ability to
extend a body of code while writing new code proportional to the differences in
functionality''.

Not many mechanisms that support family polymorphism are available in existing
mainstream languages. The \textsc{Cake} pattern~\citep{odersky2005scalable,
  Zenger-Odersky2005} in Scala provides some form of family polymorphism. In
order to model this modest form of family polymorphism, this pattern uses
\emph{virtual types}, \emph{self types}, \emph{path-dependent types} and
\emph{static mixin composition}. Even with so many sophisticated features,
composition of families is still quite heavyweight and manual. The problem is
due to the lack of \emph{deep} mixin composition. Though solutions do
exist~\citep{oliveira2013feature}, they usually require low-level type-unsafe
programming features such as dynamic proxies, reflection or other
meta-programming techniques. It is known that designing a sound type system that
fully supports family polymorphism and nested composition is notoriously hard;
there are only a few, quite sophisticated, research languages that manage
this~\citep{ErnstVirtual, Nystrom_2004, pubsdoc:tribe-virtual-calculus,SAITO_2007}.
But those mechanisms usually focus
on getting a relatively complex Java-like language with support for family polymorphism. Instead, one
of the motivations for the work presented in this thesis is to come up with a \textit{minimal} calculus that
supports nested composition.



\section{Our Proposed Solution}

This thesis sets out to explore an alternative object-oriented language design
that makes it easy and safe to extend and compose existing code on the
language level. More specifically, we seek to rein in ideas that are seemingly
unrelated but powerful in object-oriented programming---dynamic inheritance,
first-class traits, family polymorphism---under a simple unifying mechanism:
they are but different manifestations of a single underlying type discipline:
\emph{disjoint intersection types}. Through a serious of examples and rigorous
analysis in this thesis, we hope to convince readers that disjoint intersection
types are a feasible semantic tool to facilitate code reuse and modularity. In
particular, for family polymorphism, we show that the combination of the
\emph{merge operator} and a rich subtyping relation captures the essence of
nested composition; for traits, we show that the merge operator and
disjoint intersection types enable encodings of \emph{typed first-class traits}.
Combined with parametric polymorphism, we can further
express a very dynamic form of mixin-style compositions, enabling programmers to write highly
modular and reusable software components.

So what are disjoint intersecting types? Here only highlights are given---more
details are to be delivered in later chapters.

\subsection{Disjoint Intersection Types}

One recurring theme of this thesis are \emph{intersection types} (usually written
$\inter{A}{B}$). Intersection types~\citep{pottinger1980type, coppoInter} have a
long history in programming languages. They were originally introduced to
characterize exactly all strongly normalizing lambda terms. Since then, starting
with \citeauthor{reynolds1988preliminary}'s work on
Forsythe~\citep{reynolds1988preliminary}, they have also been employed to
express useful programming language constructs, such as key aspects of multiple
inheritance~\citep{compagnoni1996higher} in object-oriented programming. One
notable example is the Scala language~\citep{odersky2004overview} and its DOT
calculus~\citep{amin2012dependent}, which make fundamental use of intersection
types to express a class/trait that extends multiple other traits. Other modern
programming languages, such as TypeScript~\citep{typescript}, Flow~\citep{flow}
and Ceylon~\citep{ceylon}, also adopt some form of intersection types.

Intersection types come in different varieties in the literature. A far more
common form of intersection types are the so-called \emph{refinement types}~\citep{Freeman_1991, Davies_2000, dunfield2003type}. Refinement types
restrict the formation of intersection types so that the two types in an
intersection are refinements of the same simple (unrefined) type.
% For example,
% we can refine a type $\mathsf{Int}$ of integers with a subtype $\mathsf{Odd}$ of
% odd numbers, then an integer $1$ can be typed as follows:
% \[
%   1 : \inter{\mathsf{Int}}{\mathsf{Odd}}
% \]
% which satisfies the restriction: both $\mathsf{Int}$ and $\mathsf{Odd}$ refine a
% single simple type $\mathsf{Int}$.
Refinement intersections increase only the expressiveness of types (more precise
properties can be checked) and not of terms. For this reason,
\citet{dunfield2014elaborating} argues that refinement intersections are unsuited
for encoding various useful language features that require the \emph{merge operator}
(or an equivalent term-level operator).

Unrestricted intersection types with a term-level ``merge'' operator as an \emph{explicit}
introduction form increase the expressiveness of the term
language. This operator was introduced by \citeauthor{reynolds1988preliminary}
in Forsythe~\citep{reynolds1988preliminary} and adopted by a few other
calculi~\citep{Castagna_1992, dunfield2014elaborating, oliveira2016disjoint,
  alpuimdisjoint}. Unfortunately, while the merge operator is powerful, it also
makes it hard to get a \emph{coherent}~\citep{Reynolds_1991} (or unambiguous)
semantics. As a first approximation, a semantics is said to be coherent if a
valid program has exactly \emph{one} meaning (i.e., one value when run).
Unrestricted uses of the merge operator can be ambiguous, leading to an
incoherent semantics where the same program can evaluate to different values.
%Perhaps because of this
%issue the merge operator has not been adopted by many language designs.
We shall come back to this form of intersection types in more details in
\cref{bg:sec:intersection}.

Recently, \citet{oliveira2016disjoint} proposed \oname: a calculus with a
variant of intersection types called \emph{disjoint intersection types}. Calculi
with disjoint intersection types also feature the merge operator, with restrictions
that all expressions in a merge operator must have disjoint types and all
well-formed intersections are also disjoint. With the
disjointness restrictions, \oname is proved to be coherent. As shown by \citet{alpuimdisjoint}, calculi with disjoint
intersection types are very expressive and can be used to statically type-check
JavaScript-style programs using mixins. Yet they retain both type safety and
coherence. While coherence may seem at first of mostly theoretical relevance, it
turns out to be very relevant for object-oriented programming. As remarked
earlier, a key issue for multiple inheritance is \emph{ambiguity} caused by
conflicting features inherited from different parents. Disjoint intersection
types enforce that the types of parents are disjoint and thus that no conflicts
exist. Any violations are statically detected and can be manually resolved by
the programmer (for example by dropping one of the conflicting field/methods
from one of the parents). This is very similar to existing trait
models~\citep{scharli2003traits, Ducasse_2006}. Therefore in an object-oriented language
modeled on top of disjoint intersection types, coherence implies that no
ambiguity arises from multiple inheritance. This makes reasoning a lot simpler.

The main goal of this thesis is to significantly increase the expressiveness of
disjoint intersection types by extending the simple forms of multiple
inheritance/composition supported by previous work~\citep{alpuimdisjoint, oliveira2016disjoint}
into a more powerful form supporting nested composition
and parametric polymorphism. On the pragmatic side, the outcome is a programming
language with support for first-class traits, dynamic inheritance and nested
composition. On the theoretical side, we put disjoint intersection types on a
solid footing by thoroughly exploring their meta-theoretical properties.


\section{Contributions}

In this thesis, we present three new typed calculi, starting from a simple
calculus with disjoint intersection types, then adding parametric polymorphism
and finally ending up with a relatively sophisticated object-oriented language
with support for first-class traits, dynamic inheritance and nested composition.

\paragraph{The \namee calculus.}

The first one, named \namee, is a simple calculus with records and disjoint
intersection types that supports \emph{nested composition}. The essential
novelty of \namee is the adoption of the
Barendregt, Coppo and Dezani (BCD) subtyping~\citep{Barendregt_1983}, which includes distributivity
rules between function/record types and intersection types. These rules are the
delta that enables extending the simple forms of multiple inheritance/composition
supported by previous work~\citep{oliveira2016disjoint} into a more powerful form
supporting nested composition. The incorporation of BCD subtyping is highly
challenging for two different reasons. The first difficulty is how to preserve
coherence. Although previous work on disjoint intersection types proposes a
solution to coherence, the solution imposes several ad-hoc restrictions to
guarantee the uniqueness of the elaboration and thus allows for a simple
syntactic proof. However such restrictions make it hard or
impossible to adapt the proof to extensions of the calculus with distributivity
rules. To deal with coherence, a more semantic proof method, called the \emph{canonicity} relation, is employed.
The second difficulty is that BCD subtyping is non-algorithmic: the presence of a transitivity axiom in the
rules makes it hard to get an algorithmic version. To address it, we adapt and extend
Pierce's decision procedure~\citep{pierce1989decision} (closely related to BCD)
with subtyping of records and coercions, and propose an equivalent algorithmic
subtyping relation.

\paragraph{The \fnamee calculus.}

The second one, named \fnamee, is a polymorphic calculus with disjoint
intersection types. \fnamee is essentially \namee enriched with a variant of
parametric polymorphism called disjoint polymorphism~\citep{alpuimdisjoint}. The
addition of parametric polymorphism greatly increases the expressiveness of
\namee: \fnamee improves upon the finally tagless~\citep{CARETTE_2009} and
object algebra~\citep{oliveira2012extensibility} approaches and support advanced
compositional designs, and enables the development of highly modular and
reusable programs. \fnamee is a generalization and extension of the
\fname calculus~\citep{alpuimdisjoint}, which proposed the idea of
\emph{disjoint polymorphism}. The main novelty of \fnamee is a novel subtyping
algorithm with distributivity laws. Distributivity plays a fundamental
role in improving compositional designs, by enabling the automatic composition
of multiple operations/interpretations. The main technical challenge is the
proof of coherence as impredicativity makes it hard to develop a
well-founded logical relation for coherence. However, by restricting the system
to predicative instantiations only we are able to develop a suitable logical
relation and show coherence. Besides coherence, we show several other important
meta-theoretical results, such as type-safety, sound and complete algorithmic
subtyping, and decidability of the type system. Remarkably, unlike \fsub's
\emph{bounded polymorphism}~\citep{cardelli1985understanding}, disjoint polymorphism in \fnamee supports decidable
type-checking.

% The
% addition of parametric polymorphism greatly increases the expressiveness of
% \namee: \fnamee is able to express \emph{deep} conflict-free mixin composition
% in the presence of parametric polymorphism, which is extremely useful in the
% encodings of extensible designs. The key contribution is thus the extension of BCD
% subtyping with disjointness polymorphism. The extension is non-trivial in that
% we need to carefully retain coherence. The technical difficulty is
% \emph{well-foundedness}, stemming from the interaction between impredicativity
% and disjointness. To address this, we extend the canonicity relation with the
% restriction of predicativity.


\paragraph{Typed first-class traits.}

Lastly we present the design of \sedel: a polymorphic language with
\emph{first-class traits}, supporting \emph{parametric polymorphism}, \emph{dynamic inheritance} as well as
conventional object-oriented features such as \emph{dynamic dispatching} and \emph{abstract
  methods}. Traits pose additional challenges when compared to models with
first-class classes or mixins, because method conflicts should be detected
\emph{statically}, even in the presence of features such as dynamic inheritance and
parametric polymorphism. To address the challenges of
typing first-class traits and detecting conflicts statically, \sedel adopts the
well-established approach of elaborating high-level language constructs to a
low-level core calculus. The main contribution of \sedel is to show how to model
source language constructs for first-class traits and dynamic inheritance. The
work on \namee and \fnamee aimed at core record calculi, and omits important
features for practical object-oriented languages, including (dynamic) inheritance, dynamic
dispatching and abstract methods. Based on \citeauthor{cookthesis}'s
work on the denotational semantics for inheritance~\citep{cook1989denotational, cookthesis},
we show how to design a source language that is elaborated into \fnamee.
\sedel's elaboration into \fnamee is proved to be both type-safe and coherent.
Coherence ensures that the semantics of \sedel is unambiguous. In particular
this property is useful to ensure that programs using traits are free of
conflicts/ambiguities (even when the types of the object parts being composed
are not fully statically know). We illustrate the applicability of \sedel with
several example uses for first-class traits. Furthermore, we conduct a case study
that modularizes programming language interpreters using a highly modular form
of \visitor~\citep{oliveira09modular, togersen:2004}.

In summary the contributions of this thesis are:

\begin{itemize}

\item We present \namee, a calculus with disjoint intersection types that
  features both \emph{BCD-style subtyping} and \emph{the merge operator}. This
  calculus is both type-safe and coherent, and supports \emph{nested composition}.

\item We present \fnamee, a polymorphic calculus with disjoint intersection
  types. \fnamee is incorporated with a BCD-like subtyping relation extended
  with disjoint polymorphism. \fnamee is both type-safe and coherent, and
  supports nested composition.

\item We present \sedel, an object-oriented language design that supports
  \emph{typed first-class traits}, dynamic inheritance, as well as standard
  object-oriented features such as dynamic dispatching and abstract methods. We
  show how the semantics of \sedel can be defined by elaboration into \fnamee.

\item A more flexible notion of disjoint intersection types where only merges
  need to be checked for disjointness. This removes the need for enforcing
  disjointness for all well-formed types, making calculi with disjoint
  intersections more easily extensible.

\item The canonicity relation: a powerful proof method for establishing
  coherence of calculi with disjoint intersection types, BCD-like subtyping and
  polymorphism.

\item A comprehensive Coq mechanization of all metatheory, including type
  safety, coherence, algorithmic soundness and completeness, etc.\footnote{For
    convenience, whenever possible, definitions, lemmas and theorems have hyperlinks (click
    \href{https://github.com/bixuanzju/phd-thesis-artifact}{\leftpointright}) to their Coq counterparts. Also since \fnamee completely
    subsumes \namee, to save work, for \namee metatheory we provide cross
    references to the corresponding \fnamee Coq definitions, instead.} This has
  notably revealed several missing lemmas and oversights in Pierce's manual
  proof of BCD's algorithmic subtyping~\citep{pierce1989decision}. As a
  by-product, we obtain the first mechanically verified BCD-style subtyping
  algorithm with coercions.

\item A full-blown implementation of \sedel; it runs and type-checks all the
  examples in this thesis. We also conduct a case study, which shows that
  support for composition of object algebras~\citep{oliveira2012extensibility}
  is greatly improved in \sedel. Using such improved design patterns we re-code
  the interpreters from an undergraduate textbook on programming
  languages~\citep{poplcook} in a modular way. The implementation, Coq
  formalization and all code presented in this thesis are available at
  \url{https://github.com/bixuanzju/phd-thesis-artifact}.

\end{itemize}


% The author also contributed to the following publications that do not directly
% relate to the topics of this thesis:
% \begin{itemize}
% \item Ningning Xie, Xuan Bi, Bruno C. d. S. Oliveira. 2018. ``Consistent Subtyping for All''.
%   In \emph{European Symposium on Programming (ESOP)}.
% \item Yanpeng Yang, Xuan Bi, Bruno C. d. S. Oliveira. 2016. ``Unified Syntax with
%   Iso-Types''. In \emph{Asian Symposium on Programming Languages and
%     Systems (APLAS)}.
% \item Tomas Tauber, Xuan Bi, Zhiyuan Shi, Weixin Zhang, Huang Li, Zhenrui Zhang,
%   Bruno C. d. S. Oliveira. 2015. ``Memory-efficient Tail Calls in the JVM with
%   Imperative Functional Objects''. In \emph{Asian Symposium on
%     Programming Languages and Systems (APLAS)}.
% \end{itemize}


\section{Organization}

We begin with some background in the main topics of this thesis in
\cref{chap:background} in order to keep this thesis as self-contained as
possible and also to put our methods and contributions into context. The
structure of the technical content in the thesis is divided into three parts:
\begin{description}
\item[\cref{part:typesystem}:] \Cref{chap:nested,chap:fi} formally define the type systems of
  \namee and \fnamee, respectively. We first give the syntax and semantics of
  the two calculi. The semantics is defined in two parts. The ``target''
  languages are two standard type systems (simply-typed lambda calculus and
  System F, respectively) that do not have intersection types, the merge
  operator or subtyping. The ``source'' languages, defined by translation into
  the target languages, contain intersection types, the merge operator and
  subtyping. We then prove some basic properties such as type safety of
  the elaboration, soundness and completeness of the algorithmic subtyping, etc.
\item[\cref{part:coherence}:] \Cref{chap:coherence:simple,chap:coherence:poly} explore the
  issue of coherence. In \cref{chap:coherence:simple} we first propose a
  semantically founded definition of coherence. We then propose a proof method
  called the canonicity relation to establish coherence of \namee. In
  \cref{chap:coherence:poly} we follow the same technique in
  \cref{chap:coherence:simple} but encounter a severe issue of impredicativity.
  We impose a predicativity restriction and adapt the canonicity relation to
  establish coherence of \fnamee.
\item[\cref{part:applications}:] In \cref{chap:traits} we present the syntax and semantics of
  \sedel. In particular we show how to elaborate source-level constructs for
  first-class traits into expressions of \fnamee. In \cref{chap:case_study} we
  conduct a case study of modularizing programming language features using a
  highly modular form of \visitor.
\end{description}
\Cref{sec:related} reviews related work, \cref{chap:future} discusses future
work and \cref{chap:conclusion} concludes.

This thesis is largely based on two
publications~\citep{bi_et_al:LIPIcs:2018:9214, bi_et_al:LIPIcs:2018:9227} by the
author and one draft~\citep{xuanbiesop}, currently under review as of this writing.
In comparison to the original publications, this thesis contains a
more in-depth and consistent treatment of disjoint intersection types.

\begin{description}
\item[\cref{chap:nested,chap:coherence:simple}:] Xuan Bi, Bruno C. d. S.
  Oliveira, and Tom Schrijvers. 2018. ``The Essence of Nested Composition''. In
  \emph{European Conference on Object-Oriented Programming (ECOOP)}.
\item[\cref{chap:fi,chap:coherence:poly}:] Xuan Bi, Ningning Xie, Bruno C. d. S.
  Oliveira, and Tom Schrijvers. 2019. ``Distributive Disjoint Polymorphism for Compositional Programming''. Submitted to
  \emph{European Symposium on Programming (ESOP)}.
\item[\cref{chap:traits,chap:case_study}:] Xuan Bi and Bruno C. d. S. Oliveira.
  2018. ``Typed First-Class Traits''. In \emph{European Conference on Object-Oriented Programming (ECOOP)}.
\end{description}

\noindent\makebox[\linewidth]{\rule{0.7\textwidth}{0.4pt}}

\vspace{1.5\baselineskip}

This thesis assumes familiarity with basic knowledge of programming language
theory and object-oriented programming. We recommend
\citeauthor{DBLP:books/daglib/0005958}'s excellent textbook on programming
languages~\citep{DBLP:books/daglib/0005958} for a general introduction.


%%% Local Variables:
%%% mode: latex
%%% TeX-master: "../Thesis"
%%% org-ref-default-bibliography: "../Thesis.bib"
%%% End:

  
%%%%%%%%%%%%%%%%%%%%%%%%%%%%%%%%%%%%%%%%%%%%%%%%%%%%%%%%%%%%%%%%%%%%%%%%
\chapter{Background}
\label{chap:background}
%%%%%%%%%%%%%%%%%%%%%%%%%%%%%%%%%%%%%%%%%%%%%%%%%%%%%%%%%%%%%%%%%%%%%%%%

The chapter sets the stage for the three typed calculi that we are going to
present in later chapters by first reviewing intersection types in more details.
We start with


\section{Intersection Types}
\label{bg:sec:intersection}


\subsection{BCD Type System}

talk about intersection types as capturing  strongly normalizing terms, BCD subtyping



\subsection{Coherence}

Talk about previous approach


\subsection{Disjoint Intersection Types}


talk about Dunfield's system and lambda i

talk about Fi


\begin{comment}
The merge operator was introduced by Reynolds
and Forsythe and adopted by a few other calculi as well~\citep{}.
Unfortunately, while the merge operator is powerful, it makes
it hard to get a \emph{coherent} semantics. \bruno{what is coherence}
Perhaps because
of this issue the merge operator has not been adopted by
many language designs. Disjoint intersection types provide
a remedy for the coherence problem, by imposing restrictions
on the uses of merges and on the formation of intersection types.
\bruno{merge operator ==> models inheritance; intersection types ==>
model subtyping}

In essence disjoint intersection types retain most of the
expressive power of the merge operator.
For example, they can
be used to model powerful forms of extensible records~\citep{}.
\end{comment}


\section{Family Polymorphism and Nested Composition}
\label{sec:ernst}
% %-------------------------------------------------------------------------------
% \subsection{Motivation: Family Polymorphism}

\emph{Family polymorphism} is the ability to simultaneously refine a family of
related classes through inheritance. This is motivated by a need to not only
refine individual classes, but also to preserve and refine their mutual
relationships. \citet{Nystrom_2004} call this \emph{scalable extensibility}:
``the ability to extend a body of code while writing new code proportional to
the differences in functionality''.
%
A well-studied mechanism to achieve family inheritance is \emph{nested
inheritance}~\citep{Nystrom_2004}. Nested inheritance combines two aspects.
Firstly, a class can have nested class members; the outer class is then a
family of (inner) classes. Secondly, when one family extends another, it
inherits (and can override) all the class members, as well as the relationships
within the family (including subtyping) between the class members. However,
the members of the new family do not become subtypes of those in the parent family.

\paragraph{The Expression Problem, Scandinavian Style.}

\citet{Ernst_2001} illustrates the benefits of nested inheritance for modularity
and extensibility with one of the most elegant and concise solutions to the
\emph{Expression Problem}~\citep{wadler1998expression}.


The objective of the
Expression Problem is to extend a datatype, consisting of several cases,
together with several associated operations in two dimensions: by adding more
cases to the datatype and by adding new operations for the datatype. Ernst
solves the Expression Problem in the gbeta language, which he adorns with a
Java-like syntax for presentation purposes, for a small abstract syntax tree
(AST) example. His starting point is the code shown in \cref{fig:lang}. The
outer class \lstinline{Lang} contains a family of related AST classes: the
common superclass \lstinline{Exp} and two cases, \lstinline{Lit} for literals
and \lstinline{Add} for addition. The AST comes equipped with one operation,
\lstinline{toString}, which is implemented by both cases. Notice that all the
inner classes are \textit{virtual}, in the same sense of virtual methods, which
means that they may be redefined in subclasses of the enclosing class.


\begin{figure}[t]
    \centering
    \begin{subfigure}[b]{0.45\textwidth}
\begin{lstlisting}[language=gbeta]
class Lang {
  virtual class Exp {
    String toString() {}
  }
  virtual class Lit extends Exp {
    int value;
    Lit(int value) {
      this.value = value;
    }
    String toString() {
      return value;
    }
  }
  virtual class Add extends Exp {
    Exp left,right;
    Add(Exp left, Exp right) {
      this.left = left;
      this.right = right;
    }
    String toString() {
      return left + "+" + right;
    }
  }
}
\end{lstlisting}
\subcaption{Base family: the language \lstinline{Lang}} \label{fig:lang}
    \end{subfigure} ~
    \begin{subfigure}[b]{0.5\textwidth}
\begin{lstlisting}[language=gbeta,  xleftmargin=1mm]
// Adding a new operation
class LangEval extends Lang {
  refine class Exp {
    int eval() {}
  }
  refine class Lit {
    int eval { return value; }
  }
  refine class Add {
    int eval { return
      left.eval() + right.eval();
    }
  }
}
// Adding a new case
class LangNeg extends Lang {
  virtual class Neg extends Exp {
    Neg(Exp exp) { this.exp = exp; }
    String toString() {
      return "-(" + exp + ")";
    }
    Exp exp;
  }
}
\end{lstlisting}
\subcaption{Extending in two dimensions} \label{fig:extend}
    \end{subfigure}
    \caption{The Expression Problem, Scandinavian Style}
\end{figure}

\paragraph{Adding a New Operation.}

One way to extend the family is to add an additional evaluation operation, as
shown in the top half of \cref{fig:extend}. This is done by subclassing the
\lstinline{Lang} class and refining all the contained classes by implementing
the additional \lstinline{eval} method. The semantics of the keyword
\lstinline[language=gbeta]{refine} is that the virtual class is constrained to
be a subclass of the new declaration. In other words, \lstinline{Exp},
\lstinline{Lit} and \lstinline{Add} are all extended with the \lstinline{eval}
method. Note that the inheritance between, e.g., \lstinline{Lang.Exp} and
\lstinline{Lang.Lit} is transferred to \lstinline{LangEval.Exp} and
\lstinline{LangEval.Lit}. Similarly, the \lstinline{Lang.Exp} type of the
\lstinline{left} and \lstinline{right} fields in \lstinline{Lang.Add} is
automatically refined to \lstinline{LangEval.Exp} in \lstinline{LangEval.Add}.

\paragraph{Adding a New Case.}

A second dimension to extend the family is to add a case for negation, shown in
the bottom half of \cref{fig:extend}. This is similarly achieved by subclassing
\lstinline{Lang}, and now adding a new contained virtual class \lstinline{Neg}
that represents the unary negation operator. Note that \lstinline{Neg} is
declared to be a subclass of \lstinline{Exp}, which means that the extension to
\lstinline{Exp} will also be added to \lstinline{Neg}.


\paragraph{Combining Both Extensions.}

Finally, the two extensions are naturally combined by means of
multiple inheritance, closing the diamond.
\begin{lstlisting}[language=gbeta]
class LangNegEval extends LangEval & LangNeg {
  refine class Neg {
    int eval() { return -exp.eval(); }
  }
}
\end{lstlisting}
The only effort required is to implement the one missing operation
case, evaluation of negated expressions.

% \paragraph{A solution in \namee using nested composition:} Show a
% solution in \namee with records implemented in SEDEL. Justify the connection to the
% class-based solution. Mention that type system support
% for family polymorphism is known to be hard.


\section{Mixins and Traits}

talk about the connection with mixins/traits in OO.



%%% Local Variables:
%%% mode: latex
%%% TeX-master: "../Thesis"
%%% org-ref-default-bibliography: ../Thesis.bib
%%% End:



  \part{Type Systems}

  
%%%%%%%%%%%%%%%%%%%%%%%%%%%%%%%%%%%%%%%%%%%%%%%%%%%%%%%%%%%%%%%%%%%%%%%%
\chapter{Semantics of the \namee Calculus}
\label{chap:nested}
%%%%%%%%%%%%%%%%%%%%%%%%%%%%%%%%%%%%%%%%%%%%%%%%%%%%%%%%%%%%%%%%%%%%%%%%

This chapter presents \namee,\footnote{It was also called \name in the original
  publication~\citep{bi_et_al:LIPIcs:2018:9227}. Also the ``+'' symbol stands
  for two extra features compared to \oname: \emph{BCD subtyping} and
  \emph{unrestricted intersections}.} a calculus based on
\oname~\citep{oliveira2016disjoint} that
features unrestricted intersections, BCD-style subtyping and a merge operator, which we believe
captures the essence of nested composition. We illustrate this by presenting a
solution to the expression problem based on family polymorphism. We then discuss
the algorithmic aspects of \namee. The coherence property of \namee is discussed
in \cref{chap:coherence:simple}.

\section{Introduction}

\namee is a simple calculus with records and disjoint intersection types that
supports \emph{nested composition}. Nested composition enables encoding simple
forms of family polymorphism. More complex forms of family polymorphism,
involving binary methods~\citep{bruce1995binary} and mutable state are not yet
supported, but are interesting avenues for future work. Nevertheless, in \namee,
it is possible, for example, to encode \citeauthor{ernst2004expression}'s elegant family-polymorphism
solution to the expression problem. Compared to \oname the essential novelty of
\namee are distributivity rules between function/record types and intersection
types. These rules are the delta that enables extending the simple forms of
multiple inheritance/composition supported by \oname into a more powerful form
supporting nested composition. The distributivity rule between function types
and intersections is common in calculi with intersection types aimed at
capturing the set of all strongly normalizable terms, and was first proposed by
\citet{Barendregt_1983} (BCD). However the distributivity rule is not common in
calculi or languages with intersection types aimed at programming. For example
the rules employed in languages that support intersection types (such as Scala,
TypeScript, Flow or Ceylon) lack distributivity rules. Moreover distributivity
is also missing from several calculi with a merge operator. This includes all
calculi with disjoint intersection types~\citep{oliveira2016disjoint, alpuimdisjoint}
and \citeauthor{dunfield2014elaborating}'s work on elaborating
intersection types~\citep{dunfield2014elaborating}, which was the original
foundation for \oname. A possible reason for this omission in the past is that
distributivity adds substantial complexity (both algorithmically and
meta-theoretically), without having any obvious practical applications. This
chapter shows how to deal with the complications of BCD subtyping, while
identifying a major reason to include it in a programming language: BCD enables
nested composition and subtyping, which is of significant practical interest.

%The distributivity rules for records are
%new. Moreover, as far as we know, no previous work
%establishes the relation between BCD-style subtyping and nested composition.

\namee differs significantly from previous BCD-based calculi in that it has to
deal with the possibility of incoherence, introduced by the merge operator. Incoherence
is a non-issue in the previous BCD-based calculi because they do not feature
this merge operator or any other source of incoherence.
Although previous work on disjoint intersection types
proposes a solution to coherence, the solution imposes several ad-hoc restrictions (cf. \cref{sec:comparision})
to guarantee the uniqueness of the elaboration and thus allows for a simple
syntactic proof of coherence. Most
importantly, it makes it hard or impossible to adapt the proof to extensions of
the calculus, such as the new subtyping rules required by the BCD system.
We shall return to this point in
\cref{chap:coherence:simple}.



\section{\namee by Examples}
\label{nested:sec:overview}

This section illustrates \namee with an encoding of a family polymorphism
solution to the expression problem, and informally presents its salient
features.


%-------------------------------------------------------------------------------
\subsection{The expression problem, \namee Style}

The \namee calculus allows us to solve the expression problem in a way that is
very similar to \citeauthor{Ernst_2001}'s \textsf{gbeta} solution in \cref{sec:ernst}.
However, the underlying mechanisms of \namee are quite different from those of
\textsf{gbeta}. In particular, \namee features a structural type system in which we can
model objects with records, and object types with record types. For instance, we
model the interface of \lstinline{Lang.Exp} with the singleton record type
\lstinline${ print : String }$. For the sake of conciseness, we use \lstinline{type} aliases
to abbreviate types.
\lstinputlisting[linerange=4-4]{./examples/overview.sl}% APPLY:linerange=PRINT_INTERFACE
Similarly, we capture the interface of the \lstinline{Lang} family in a record,
with one field for each case's constructor.
\lstinputlisting[linerange=8-8]{./examples/overview.sl}% APPLY:linerange=LANG_FAMILY
Here is the implementation of \lstinline{Lang}.
\lstinputlisting[linerange=17-24]{./examples/overview.sl}% APPLY:linerange=LANG_IMPL
We assume several primitive types: fixed width integers \lstinline{Int},
\lstinline{Double} for numeric operations and \lstinline{String} for text
manipulation. A \namee program consists of a collection of definitions and
declarations, separated by semicolon \lstinline{;}.

% - - - - - - - - - - - - - - - - - - - - - - - - - - - - - - - - - - - - - - - -
\paragraph{Adding Evaluation.}
We obtain \lstinline{IPrint & IEval}, which is the corresponding type for \lstinline{LangEval.Exp}, by
intersecting \lstinline{IPrint} with \lstinline{IEval} where
\lstinputlisting[linerange=29-29]{./examples/overview.sl}% APPLY:linerange=EVAL_INTERFACE
The type for \lstinline{LangEval} is then
\lstinputlisting[linerange=34-37]{./examples/overview.sl}% APPLY:linerange=EVAL_PRINT_INTERFACE
We obtain an implementation for \lstinline{LangEval} by merging the existing
\lstinline{Lang} implementation \lstinline{implLang} with the new evaluation
functionality \lstinline{implEval} using the merge operator \lstinline{,,}.
\lstinputlisting[linerange=45-53]{./examples/overview.sl}% APPLY:linerange=EVAL_PRINT_IMPL

% - - - - - - - - - - - - - - - - - - - - - - - - - - - - - - - - - - - - - - - -
\paragraph{Adding Negation.}
Adding negation to \lstinline{Lang} works similarly.
\lstinputlisting[linerange=57-65]{./examples/overview.sl}% APPLY:linerange=LANG_NEG
% \begin{Verbatim}[xleftmargin=10mm,fontsize=\relscale{.80}]
% type LangNeg = Lang & { neg : IPrint -> IPrint }

% implLangNeg : LangNeg
% implLangNeg = implLang ,, implNeg

% implNeg = { neg = \a.{print = "-" ++ a.print } }
% \end{Verbatim}

% - - - - - - - - - - - - - - - - - - - - - - - - - - - - - - - - - - - - - - - -
\paragraph{Putting Everything Together.}
Finally, we can combine the two extensions and provide the missing
implementation of evaluation for the negation case.
\lstinputlisting[linerange=70-80]{./examples/overview.sl}% APPLY:linerange=LANG_FINAL
We can test \lstinline{implLangNegEval} by creating an object \lstinline{e} of expression $-2 + 3$ that is able to print and evaluate at the same time.
\lstinputlisting[linerange=98-100]{./examples/overview.sl}% APPLY:linerange=TEST



%- - - - - - - - - - - - - - - - - - - - - - - - - - - - - - - - - - - - - - - -
\paragraph{Multi-Field Records.} One relevant remark is that
\namee does not have multi-field record types built in. They are merely syntactic
sugar for intersections of single-field record types. Hence, the following is an
equivalent definition of \lstinline{Lang}:
\lstinputlisting[linerange=13-13]{./examples/overview.sl}% APPLY:linerange=LANG_FAMILY2
Similarly, the multi-field record expression in the definition of
\lstinline{implLang} is syntactic sugar for the explicit merge of two
single-field records.
\begin{lstlisting}
implLang : Lang = { lit = ... } ,, { add = ... };
\end{lstlisting}

%- - - - - - - - - - - - - - - - - - - - - - - - - - - - - - - - - - - - - - - -
\paragraph{Subtyping.}
A big difference compared to \textsf{gbeta} is that many more \namee types are related through
subtyping. Indeed, \textsf{gbeta} is unnecessarily conservative~\citep{ernst_hoh}: none of the families is related
through subtyping, nor is any of the class members of one family related to any
of the class members in another family. For instance, \lstinline{LangEval} is
not a subtype of \lstinline{Lang}, nor is \lstinline{LangNeg.Lit} a subtype
of \lstinline{Lang.Lit}.

In contrast, subtyping in \namee is much more nuanced and depends entirely on the
structure of types. The primary source of subtyping are intersection types:
any intersection type is a subtype of its components. For instance, 
\lstinline{IPrint & IEval} is a subtype of both \lstinline{IPrint} and
\lstinline{IEval}. Similarly \lstinline{LangNeg = Lang & NegPrint} is a subtype
of \lstinline{Lang}. Compare this to \textsf{gbeta} where \lstinline{LangEval.Expr} is
not a subtype of \lstinline{Lang.Expr}, nor is the family \lstinline{LangNeg} a
subtype of the family \lstinline{Lang}.

However, \textsf{gbeta} and \namee agree that \lstinline{LangEval} is not a subtype of
\lstinline{Lang}. The \namee-side of this may seem contradictory at first, as we
have seen that intersection types arise from the use of the merge operator, and
we have created an implementation for \lstinline{LangEval} with
\lstinline{implLang ,, implEval} where \lstinline{implLang : Lang}. That
suggests that \lstinline{LangEval} is a subtype of \lstinline{Lang}.
Yet, there is a flaw in our reasoning:
strictly speaking, \lstinline{implLang ,, implEval} is not of
type \lstinline{LangEval} but instead of type \lstinline{Lang & EvalExt}, where
\lstinline{EvalExt} is the type of \lstinline{implEval}:
\lstinputlisting[linerange=41-41]{./examples/overview.sl}% APPLY:linerange=EVAL_INTERFACE2
Nevertheless, the definition of \lstinline{implLangEval} is valid because
\lstinline{Lang & EvalExt} is a subtype of \lstinline{LangEval}.
Indeed, if we consider for the sake of simplicity only the \lstinline{lit}
field, we have that \lstinline{(Int -> IPrint) & (Int -> IEval)} is a
subtype of \lstinline{Int -> IPrint & IEval}. This follows from a standard
subtyping axiom for distributivity of functions and intersections in the BCD system inherited by \namee.
In conclusion, \lstinline{Lang & EvalExt} is a subtype of both \lstinline{Lang}
and of \lstinline{LangEval}. However, neither of the latter two types is a subtype of the other.
Indeed, \lstinline{LangEval} is not a subtype of \lstinline{Lang} as the type
of \lstinline{add} is not covariantly refined and thus admitting the subtyping
is unsound. For the same reason \lstinline{Lang} is not a subtype of \lstinline{LangEval}.


A summary of the various relationships between the language components is shown
in \cref{fig:diagram}. Admittedly, the figure looks quite complex because our
calculus has a structural type system (as often more foundational calculi
do) where more types are related through subtyping, whereas mainstream OO
languages have nominal type systems.



\begin{figure}[t]
  \centering
\includegraphics[scale=0.75]{figures/diagram.eps}
\caption{Summary of the relationships between language components}
\label{fig:diagram}
\end{figure}


\paragraph{Stand-Alone Extensions.}
Unlike in \textsf{gbeta} and other class-based inheritance systems, in \namee
the extension \lstinline{implEval} is not tied to \lstinline{LangEval}. In that
sense, it resembles trait and mixin systems that can apply the same extension
to different classes. However, unlike those systems, \lstinline{implEval} can also
exist as a value on its own, i.e., it is not an extension per se.

%-------------------------------------------------------------------------------
% \subsection{Disjoint Intersection Types and Ambiguity}

% The above example shows that intersection types and the merge operator
% are closely related to multiple
% inheritance. Indeed, they share a major concern with multiple inheritance,
% namely ambiguity. When a subclass inherits an implementation of the same
% method from two different parent classes, it is unclear which of the two
% methods is to be adopted by the subclass. In the case where the two parent classes
% have a common superclass, this is known as the \emph{diamond problem}.
% The ambiguity problem also appears in \namee,
% e.g., if we merge two numbers to obtain $\mer{1}{2}$ of type
% $\inter{\mathsf{Int}}{\mathsf{Int}}$. Is the result of $\mer{1}{2} + 3$
% either $4$ or $5$?

% Disjoint intersection types offer to statically detect potential ambiguity and
% to ask the programmer to explicitly resolve the ambiguity by rejecting the
% program in its ambiguous form. In the previous work on \oname, ambiguity is
% avoided by dictating that all intersection types have to be disjoint, i.e.,
% $\inter{\mathsf{Int}}{\mathsf{Int}}$ is ill-formed because the first component
% has the same type as the second.


% Disjoint intersection types ensure unambiguity and conflicts are
% statically detected and manually resolved by programmers. This
% is similar to the trait model.


% Local Variables:
% TeX-master: "../../Thesis"
% org-ref-default-bibliography: ../../Thesis.bib
% End:


\section{Syntax and Semantics}

\begin{figure}[t]
  \centering
\begin{tabular}{llll} \toprule
  Types & $[[A]], [[B]], [[C]]$ & $\Coloneqq$ & $[[nat]] \mid [[Top]] \mid [[A -> B]]  \mid [[A & B]] \mid [[{l : A}]] \mid [[X]] \mid [[\ X ** A . B]] $\\
  Monotypes & $[[t]]$ & $\Coloneqq$ & $[[nat]] \mid [[Top]] \mid [[t1 -> t2]]  \mid [[t1 & t2]] \mid [[X]] \mid [[{l : t}]]$\\
  Expressions & $[[ee]]$ & $\Coloneqq$ & $[[x]] \mid [[i]] \mid [[Top]] \mid [[\x . ee]] \mid [[ee1 ee2]] \mid [[ ee1 ,, ee2 ]]   \mid [[ ee : A ]] $ \\
        & & $\mid$ & $ [[{l = ee}]] \mid [[ ee.l  ]] \mid [[\X ** A . ee]] \mid [[ ee A ]]  $ \\
  Value Contexts & $[[GG]]$ & $\Coloneqq$ &  $[[empty]] \mid [[GG , x : A]] $ \\
  Type Contexts & $[[DD]]$ & $\Coloneqq$ &  $[[empty]] \mid [[DD , X ** A]] $  \\ \bottomrule
  % Expression Contexts & $[[CC]]$ & $\Coloneqq$ &  $[[__]] \mid [[\ x . CC]] \mid [[\ X ** A. CC]] \mid [[ CC A  ]] \mid [[CC ee]] \mid [[ee CC]] \mid [[ CC ,, ee  ]]  $ \\
  % & & $\mid$ & $[[ ee ,, CC  ]] \mid  [[ { l = CC}  ]]  \mid [[ CC . l]] $
\end{tabular}
  \caption{Syntax of \fnamee}
  \label{fig:syntax:fi}
\end{figure}

\Cref{fig:syntax:fi} shows the syntax of \fnamee. Metavariables $[[A]], [[B]],
[[C]]$ range over types. Apart from \namee types, \fnamee also includes type
variables $[[X]]$ and disjoint quantification $[[ \X ** A . B ]]$. Monotypes
$[[t]]$ are the same, less the universal quantification. Metavariable $[[ee]]$
ranges over expressions. We extend \namee expressions with two standard
constructs in System F: type abstractions $[[ \X ** A . ee ]]$ and type
applications $[[ee A]]$. The former also includes an extra disjointness
constraint $[[A]]$ associated with the type variable $[[X]]$.

\paragraph{Contexts.}

In the traditional formulation of System F, there is a single context that is
used to keep track of both type variables and term variables. Here we use
another style of presentation~\citep[chap. 16]{Harper_2016} where contexts are
split into \textit{value contexts} $[[GG]]$ and \textit{type contexts} $[[DD]]$.
The former track bound term variables $[[x]]$ with their types $[[A]]$; and the
latter track bound type variables $[[X]]$ with their disjointness constraints
$[[A]]$. This formulation is also convenient for the presentation of logical
relations in \cref{chap:coherence:poly}.

\begin{figure}
  \centering
  \drules[swft]{$[[DD |- A]]$}{Well-formedness of types}{top, int, var, arrow, all, and, rcd}
  \drules[swfe]{$[[DD ||- GG]]$}{Well-formedness of value contexts}{empty, var}
  \drules[swfte]{$[[||- DD]]$}{Well-formedness of type contexts}{empty, var}
  \caption{Well-formedness of contexts and types}
  \label{fig:well-formedness:fi}
\end{figure}

\paragraph{Well-formedness of contexts.}

The well-formedness judgments for contexts and types, as shown in
\cref{fig:well-formedness:fi} are quite standard. They together ensure that each
type appearing in the contexts is well-formed in the sense that there are no
unbound free variables.


\paragraph{Declarative Subtyping.}


\begin{figure}[h]
  \centering
  \drules[FS]{$[[ A <|: B ~~> c]]$}{Declarative subtyping}{refl,trans,top,rcd, arr,andr,andl,and,distArr,topArr,distRcd,topRcd,forall}
  \caption{Declarative subtyping of \fnamee}
  \label{fig:subtyping:fi}
\end{figure}

\Cref{fig:subtyping:fi} presents the subtyping relation of \fnamee. For now, we
ignore the coercion parts ($[[~~>]] [[c]]$) and explain them in
\cref{sec:elaboration:fi}. We naturally extend the subtyping rules of \namee
with only one rule \rref*{FS-forall}, which specifies the subtyping relation
between two universal quantifiers. In \rref{FS-forall}, a universal quantifier
is covariant in its body, and contravariant in its disjointness constraint. A
minor comment is that since \fnamee features explicit polymorphism, type
variables are neutral to subtyping, i.e., $[[X <: X]]$, which is contained in
\rref{FS-refl}. As with \namee subtyping, the subtyping relation of \fnamee is
trivially \textit{reflexive} and \textit{transitive}.

\begin{remark}
  In our Coq formalization, we require that the two types $[[A]]$ and $[[B]]$ are
  well-formed with respect to some type context, resulting in the subtyping
  judgment $[[DD |- A <: B]]$. But this is not very important
  for the purpose of presentation, thus we omit contexts.
\end{remark}


\paragraph{Typing.}

\begin{figure}
  \centering
  \drules[FT]{$[[DD; GG |- ee => A ~~> e]]$}{Inference}{top, int, var, app, merge, anno, tabs, tapp, rcd, proj}
  \drules[FT]{$[[DD ; GG |- ee <= A ~~> e]]$}{Checking}{abs, sub}
  \caption{Bidirectional type system of \fnamee}
  \label{fig:typing:fi}
\end{figure}


The bidirectional type system of \fnamee follows that of \namee, as shown
in \cref{fig:typing:fi}. Again we ignore the translation parts ($[[~~>]] [[e]]$) and explain them in
\cref{sec:elaboration:fi}. The inference judgment $[[ DD; GG |- ee => A  ]]$
says that we can synthesize the type $[[A]]$ in the contexts $[[DD]]$ and $[[GG]]$. The checking judgment
$[[ DD ; GG |- ee <= A  ]]$ asserts that $[[ee]]$ checks against the type $[[A]]$
in the contexts $[[DD]]$ and $[[GG]]$. The rules directly ported from \namee are inferring rules \rref*{FT-top} for top values,
\rref*{FT-int} for integers, \rref*{FT-var} for variables, \rref*{FT-app} for applications, \rref*{FT-merge} for merges,
\rref*{FT-anno} for annotated terms, \rref*{FT-rcd,FT-proj} for records; checking rules \rref*{FT-abs} for term abstractions, and
the subsumption rule \rref*{FT-sub}. Note that in \rref{FT-merge}, the disjointness judgment has an extra type context, which will be
explained in \cref{sec:disjoint:fi}.

\paragraph{Disjoint quantification.}

The new rules are the inferring rules for type abstractions \rref*{FT-tabs} and
type applications \rref*{FT-tapp}. In \rref{FT-tabs}, the disjointness
constraint is added to the type context. During a type application in
\rref{FT-tapp}, the type system checks that the type argument agrees with the
disjointness constraint. This, together with \rref{FT-merge} are the only two
rules that use the disjointness checking. Moreover, since \fnamee is
predicative, we require that the type being instantiated is a monotype.



% Local Variables:
% TeX-master: "../../Thesis"
% org-ref-default-bibliography: ../../Thesis.bib
% End:


\section{Algorithmic Subtyping}
\label{sec:alg}

This section considers the algorithmic aspects of \namee. The bidirectional type
system is obviously syntax directed, so the only source of non-determinism comes
from the subtyping relation. In this section, we present an algorithm that
implements the subtyping relation. While BCD subtyping is well-known, the
presence of a transitivity axiom in the rules means that the system is not
algorithmic. This raises an obvious question: how to obtain an algorithm for
this subtyping relation? \citet{Laurent12note} has shown that simply dropping
the transitivity rule from the BCD system is not possible without losing
expressivity. Hence, this avenue for obtaining an algorithm is not available.
%Moreover, even if transitivity elimination
%would be possible, the remaining rules are still highly overlapping, and pose
%difficulties for an implementation.  
Instead, we adapt \citeauthor{pierce1989decision}'s decision
procedure~\citep{pierce1989decision} for a subtyping system (closely
related to BCD) to obtain a sound and complete algorithm for our
BCD extension. Our algorithm extends \citeauthor{pierce1989decision}'s decision
procedure with subtyping of singleton records and
coercion generation. We prove in Coq that the algorithm is sound and complete with
respect to the declarative version. At the same time we
find some errors and missing lemmas in \citeauthor{pierce1989decision}'s original manual proofs.

%The algorithm is implemented in our
%prototype implementation. \jeremy{should i say more about implementation?}

%See \cref{sec:alg} for the details. 
%\bruno{The meaning of the paragraph is somewhat obscure to me. After
%  discussing with Tom, it seems that what may be meant here is that we
%cannot do cut elimination, which is a common process that you can try
%for certain systems with subtyping. However Pierce managed to find
%another way to get a sound/complete algorithmic system. Maybe 
%the text can be improved.}


\begin{figure}[t]
  \centering
  \drules[A]{$[[fs |- A <: B ~~> c]]$}{Algorithmic subtyping}{prim, and, arr, rcd, top, arrR, rcdR, andROne, andRTwo}
  \caption{Algorithmic subtyping of \namee}
  \label{fig:algorithm}
\end{figure}


\subsection{The Subtyping Algorithm}

\Cref{fig:algorithm} shows the algorithmic subtyping judgment $[[fs |- A <: B ~~> c]]$.
This judgment is the algorithmic counterpart of the declarative
judgment $[[A <: fs -> B ~~> c]]$, where the symbol $[[fs]]$ stands for a
queue tracking types and labels. Definition~\ref{def:fs} converts a queue to a type.
For instance, if $[[fs]] = [[A]] , [[B]] , \{[[l]]\} $, then $[[fs -> C]]$ abbreviates $ [[A -> B -> {l : C}]]$.

\begin{definition} $[[fs -> A]]$ is inductively defined as follows: \label{def:fs}
  \begin{mathpar}
    [[ [] -> A]] = [[A]] \and
    [[ (fs , B) -> A]] = [[fs -> (B -> A)]] \and
    [[ (fs , {l}) -> A]] = [[fs -> {l : A}]]
  \end{mathpar}
\end{definition}

The basic idea of $[[fs |- A <: B ~~> c]]$ is to first perform a structural
analysis of $[[B]]$, which descends into both sides of $[[&]]$'s (\rref{A-and}),
into the right side of $[[->]]$'s (\rref{A-arr}), and into the fields of records
(\rref{A-rcd}) until it reaches one of the two base cases, $[[Top]]$ or $[[  B rigid ]]$.
If the base case is $[[Top]]$, then the subtyping holds trivially
(\rref{A-top}). For the other base case, we introduce the notion of \textit{rigid types}---those that do not have distributivity rules---captured
by the predicate ``$ [[A rigid]] $'' over types, and defined as follows:

\begin{definition}[Rigid types] \label{def:rigid}
  \[
    [[  pri rigid  ]]
  \]
\end{definition}

For now, the only rigid type is the primitive type $[[pri]]$, but will be
extended when we have more rich types. If the base case is a rigid type, the
algorithm performs a structural analysis of $[[A]]$, in which $[[fs]]$ plays an
important role. The left sides of $[[->]]$'s are pushed onto $[[fs]]$ as they
are encountered in $[[B]]$ and popped off again later, left to right, as
$[[->]]$'s are encountered in $[[A]]$ (\rref{A-arrR}). Similarly, the labels are
pushed onto $[[fs]]$ as they are encountered in $[[B]]$ and popped off again
later, left to right, as records are encountered in $[[A]]$ (\rref{A-rcdR}). The
remaining rules are similar to their declarative counterparts. Let us illustrate
the algorithm in \cref{fig:example_deri} with an example derivation (for
formatting reasons we use $[[N]]$ and $[[S]]$ to denote $[[nat]]$ and
$[[string]]$ respectively, which are both rigid types), which is essentially the
one used by the \lstinline{add} field in \cref{nested:sec:overview}. The readers
can try to give a corresponding derivation using the declarative subtyping and
see how \rref{S-trans} plays an essential role there.


\begin{figure}[t]
  \centering
\begin{footnotesize}
  \begin{tabular}{l}
\begin{mathpar}
  \inferrule*[right=\rref*{A-rcd}]
  { \inferrule*[right=\rref*{A-arr}]
    {  \inferrule*[right=\rref*{A-arr}]
      { \inferrule*[right=\rref*{A-and}]
        {D \\ D'}
        { \{ [[l]]  \} , [[N & S]] , [[N & S]] \vdash [[{l : N -> N -> N} & {l : S -> S -> S} ]] \prec : [[N & S]] }
      }
      {  \{ [[l]]  \}, [[N & S]] \vdash [[{l : N -> N -> N} & {l : S -> S -> S} ]] \prec : [[N & S -> N & S]] }
    }
    { \{ [[l]]  \} \vdash [[{l : N -> N -> N} & {l : S -> S -> S} ]] \prec : [[ N & S -> N & S -> N & S ]]}      }
  {  [[ [] |- {l : N -> N -> N} & {l : S -> S -> S} <: {l : N & S -> N & S -> N & S} ]] }
\end{mathpar}

    \\ \\

    \begin{mathpar}
      D =
\inferrule*[right=\rref*{A-andR1}]
        { \inferrule*[right=\rref*{A-rcdR}]
          { \inferrule*[right=\rref*{A-arrR}]
            { \inferrule*{ [[  [] |- N <: N  ]] } { [[ [] |- N & S <: N]]  }
              \\
              \inferrule*
              { \inferrule*{ [[  [] |- N <: N   ]]   }{[[  []   |- N & S <: N   ]]} \\ [[  []  |- N <: N  ]]}
              { [[N & S]] \vdash [[N -> N]] \prec : [[N]] }
            }
            {[[N & S]] ,[[N & S]] \vdash [[N -> N -> N]] \prec : [[N]]} }
          { \{ [[l]]  \}, [[N & S]] ,[[N & S]] \vdash [[{l : N -> N -> N}]] \prec : [[N]] } }
        { \{ [[l]]  \}, [[N & S]] ,[[N & S]] \vdash [[{l : N -> N -> N} & {l : S -> S -> S} ]] \prec : [[N]] }
    \end{mathpar}

    \\ \\

    \begin{mathpar}
          D' =
\inferrule*[right=\rref*{A-andR2}]
        { \inferrule*[right=\rref*{A-rcdR}]
          { \inferrule*[right=\rref*{A-arrR}]
            { \inferrule*{ [[ [] |- S <: S    ]] } { [[ [] |- N & S <: S]]  }
              \\
              \inferrule* {  \inferrule*{  [[  [] |- S <: S  ]]    }{[[  []  |- N & S <: S    ]]}  \\ [[  [] |- S <: S   ]]     } { [[N & S]] \vdash [[S -> S]] \prec : [[S]] }     }
            {[[N & S]] ,[[N & S]] \vdash [[S -> S -> S]] \prec : [[S]]} }
          { \{ [[l]]  \}, [[N & S]] ,[[N & S]] \vdash [[{l : S -> S -> S}]] \prec : [[S]] } }
        { \{ [[l]]  \}, [[N & S]] ,[[N & S]] \vdash [[{l : N -> N -> N} & {l : S -> S -> S} ]] \prec : [[S]] }

      \end{mathpar}

  \end{tabular}
\end{footnotesize}
  \caption{Example derivation}
  \label{fig:example_deri}
\end{figure}


Now consider the coercions. Algorithmic subtyping uses the same set of
coercions as declarative subtyping. However, because algorithmic
subtyping has a different structure, the rules generate slightly more
complicated coercions. Two meta-functions $\llbracket \cdot \rrbracket_{\top}$
and $\llbracket \cdot \rrbracket_{\&}$ used in \rref{A-top,A-and} respectively,
are meant to generate correct forms of coercions. They are defined recursively
on $[[fs]]$ and are shown in \cref{fig:coercion}.



\begin{figure}[t]
    \centering
    \begin{subfigure}[b]{0.5\textwidth}
      \begin{align*}
        [[ < [] >1 ]] &=  [[top]] \\
        [[ < { l } , fs >1 ]] &= [[ < fs >1 o id  ]] \\
        [[ < A , fs >1 ]] &= [[(top -> < fs >1) o (topArr o top)]]
      \end{align*}
    \end{subfigure} ~
    \begin{subfigure}[b]{0.45\textwidth}
      \begin{align*}
        [[ < [] >2 ]] &=  [[id]] \\
        [[ < { l } , fs >2 ]] &= [[ < fs >2 o id  ]] \\
        [[ < A , fs >2 ]] &= [[(id -> < fs >2) o distArr]]
      \end{align*}
    \end{subfigure}
    \caption{Meta-functions of coercions}\label{fig:coercion}
\end{figure}

\subsection{Correctness of the Algorithm}

To establish the correctness of the algorithm, we must show that the algorithm
is both sound and complete with respect to the declarative specification. While
soundness follows quite easily, completeness is much harder. The proof of
completeness essentially follows that of \citet{pierce1989decision}
%%\footnote{
%%While transferring \cite{pierce1989decision}'s manual proofs to Coq,
%%we discovered several errors, which will be reported along the way.}
in that we
need to show the algorithmic subtyping is reflexive and
transitive. 


\paragraph{Soundness of the Algorithm.}

The following two lemmas connect the declarative subtyping with the meta-functions.

\begin{lemma} \label{lemma:top}
  $[[ Top <: fs -> Top ~~> < fs >1]]$
\end{lemma}
\begin{proof}
  By induction on the length of $[[fs]]$.
\end{proof}

\begin{lemma} \label{lemma:and}
  $[[(fs -> A) & (fs -> B) <: fs -> (A & B) ~~> < fs >2]]$
\end{lemma}
\begin{proof}
  By induction on the length of $[[fs]]$.
\end{proof}

The proof of soundness is straightforward.
\begin{theorem}[Soundness] \label{thm:soundness}
  If $[[ fs |- A <: B ~~> c]]$ then $ [[   A <: fs -> B ~~> c  ]]   $.
\end{theorem}
\begin{proof}
  By induction on the derivation of the algorithmic subtyping and applying \cref{lemma:top,lemma:and} where appropriate.
\end{proof}


\paragraph{Completeness of the Algorithm.}


\newcommand{\UU}[1]{\mathcal{U}(#1)}

Completeness, however, is much harder. The reason is that, due to the use of
$[[fs]]$, reflexivity and transitivity are not entirely obvious. We need to
strengthen the induction hypothesis by introducing the notion of a set,
$\UU{[[A]]}$, of ``reflexive supertypes'' of $[[A]]$, as defined below:
\begin{mathpar}
  \UU{[[Top]]} \defeq \{ [[Top]]  \} \and
  \UU{[[nat]]} \defeq \{ [[nat]]  \} \and
  \UU{[[{l : A}]]} \defeq \{ [[{l : B}]] \mid [[B]] \in \UU{[[A]]}  \} \and
  \UU{[[A & B]]} \defeq \UU{[[A]]} \cup \UU{[[B]]} \cup \{ [[A & B]] \} \and
  \UU{[[A -> B]]} \defeq \{ [[A -> C]] \mid [[C]] \in \UU{[[B]]} \}
\end{mathpar}
We show two lemmas about $\UU{[[A]]}$ that are crucial in the subsequent proofs.
\begin{lemma} \label{lemma:set_refl}
  $[[A]] \in \UU{[[A]]}$
\end{lemma}
\begin{proof}
  By induction on the structure of $[[A]]$.
\end{proof}

\begin{lemma} \label{lemma:set_trans}
  If $[[A]] \in \UU{[[B]]}$ and $[[B]] \in \UU{[[C]]}$, then $[[A]] \in \UU{[[C]]}$.
\end{lemma}
\begin{proof}
  By induction on the structure of $[[B]]$.
\end{proof}

\begin{remark}
  \cref{lemma:set_trans} is not found in \citeauthor{pierce1989decision}'s proofs~\citep{pierce1989decision}, which is
  crucial in \cref{lemma:refl0}, from which reflexivity (\cref{lemma:refl})
  follows immediately.
\end{remark}

% Next we show the following lemma from which reflexivity (\cref{lemma:refl})

\begin{lemma} \label{lemma:refl0}
  If $[[fs -> B]] \in \UU{[[A]]}$ then there exists $[[c]]$ such that $[[fs |- A <: B ~~> c]]$.
\end{lemma}
\begin{proof}
  By induction on $\mathsf{size}([[A]]) + \mathsf{size}([[B]]) + \mathsf{size}([[fs]])$.
\end{proof}
% \begin{remark}
%   \cite{pierce1989decision}'s proof is wrong in one case~\cite[pp.~10, Case~ii]{pierce1989decision} because we need \cref{lemma:set_trans} to be able
%   to apply the inductive hypothesis.
% \end{remark}

Now it immediately follows that the algorithmic subtyping is reflexive.

\begin{lemma}[Reflexivity] \label{lemma:refl}
  For every $[[A]]$ there exists $[[c]]$ such that $[[ [] |- A <: A ~~> c]]$.
\end{lemma}
\begin{proof}
  Immediate from \cref{lemma:set_refl} and \cref{lemma:refl0}.
\end{proof}

% We omit the details of the proof of transitivity.
The proof of transitivity is, to quote \citet{pierce1989decision}, typically
``the hardest single piece'' of metatheory. We omit the details here and
refer the interested reader to our Coq development.

\begin{lemma}[Transitivity] \label{lemma:trans}
  If $[[ [] |- A1 <: A2 ~~> c1]]$ and $[[ [] |- A2 <: A3 ~~> c2]]$, then there
  exists $[[c]]$ such that $[[ [] |- A1 <: A3 ~~> c]]$.
\end{lemma}

With reflexivity and transitivity in position, we show the main theorem.

\begin{theorem}[Completeness] \label{thm:complete}
  If $[[A <: B ~~> c]]$ then there exists $[[c']]$ such that $[[ [] |- A <: B ~~> c']]$.
\end{theorem}
\begin{proof}
  By induction on the derivation of the declarative subtyping and applying \cref{lemma:refl,lemma:trans} where appropriate.
\end{proof}
\begin{remark}
  \citeauthor{pierce1989decision}'s proof is wrong~\cite[pp.~20, Case~F]{pierce1989decision} in the case
  \begin{mathpar}
  \drule{S-arr}
  \end{mathpar}
  where he concludes from the inductive
  hypotheses $[[ [] |- B1 <: A1]]$ and $[[ [] |- A2 <: B2]]$ that $[[ [] |- A1 -> A2 <: B1 -> B2]]$ (his rules 6a and 3).
  However his rule 6a (our \rref{A-arrR}) only works for \textit{primitive types}, and is thus not applicable in this case. Instead we
  need a few technical lemmas to support the argument.
\end{remark}

\begin{remark}
  It is worth pointing out that the two coercions $[[c]]$ and $[[c']]$ in
  \cref{thm:complete} are \textit{contextually equivalent} (the precise
  definition is found in \cref{chap:coherence:simple}), which follows from
  \cref{thm:soundness} and \cref{lemma:coercion_same}.
\end{remark}

% Local Variables:
% TeX-master: "../../Thesis"
% org-ref-default-bibliography: ../../Thesis.bib
% End:

% 
\section{Conclusions and Future Work}
\label{sec:conclusion}

We have proposed \name, a type-safe and coherent calculus with disjoint
intersection types, and support for nested composition/subtyping. \name
improves upon earlier work with a more
flexible notion of disjoint intersection types, which leads to
a clean and elegant formulation of the type system. Due to the added
flexibility we have had to employ a more powerful proof method based on logical
relations to rigorously prove coherence.
We also show how \name supports essential features of family
polymorphism, such as nested composition. We believe \name provides insights into family polymorphism, and
has potential for practical applications for extensible software designs.

A natural direction for future work is to enrich \name with parametric
polymorphism. There is abundant literature on logical relations for parametric
polymorphism~\cite{reynolds1983types} and we foresee no fundamental difficulties
in extending our proof method.\footnote{ Our prototype implementation already
  supports polymorphism, but we are still in the process of extending our Coq
  development with polymorphism.} The main challenge in the definition of the
logical relation is the clause for type variables with arbitrary types. Careful
measures are to be taken to avoid the potential circularity due to
impredicativity. With the combination of parametric polymorphism and nested
composition, an interesting application that we intend to investigate is native
support for a highly modular form of \textit{Object Algebras}~\cite{oliveira2012extensibility, xuan_traits} and \textsc{Visitor}s
(or the finally tagless approach~\cite{CARETTE_2009}).

Another direction for future work is to add mutable references, which would
touch two places in our metatheory: type safety and coherence. For type safety,
we expect that lessons learned from previous work on family polymorphism and
mutability on OO to apply to our work. For example, it is well-known that
subtyping in the presence of mutable state often needs restrictions. Given such
suitable restrictions we expect that type-safety in the presence of mutability
still works. For coherence, it would be a major technical challenge to adjust
our coherence proof and its Coq mechanisation. Logical relations that account
for mutable state (e.g., see Ahmed's thesis~\cite{ahmed2004semantics}) introduce significant complexity.





% For example, we can
% define the object algebra interfaces for the Expression Problem example in
% \cref{sec:overview} as follows:
% \lstinputlisting[linerange=77-78]{../../impl/examples/overview.sl}% APPLY:linerange=LANG_EXT_INTER
% By instantiating \lstinline{E} with \lstinline{IPrint}, i.e.,
% \lstinline{ExpAlg[IPrint]}, we get the interface of the \lstinline{Lang} family.
% In that sense, object algebra interfaces can be viewed as family interfaces.
% Moreover, combining algebras implementing \lstinline{ExpAlg[IPrint]} and
% \lstinline{ExpAlg[IEval]} to form \lstinline{ExpAlg[IPrint & IEval]} is trivial
% with nested composition. Polymorphism also improves code reuse across expressions in the
% base and extended languages. For example, the following creates two expressions,
% one in the base language, the other in the extended language:
% \lstinputlisting[linerange=83-84]{../../impl/examples/overview.sl}% APPLY:linerange=LANG_EXT
% Notice how we can  reuse \lstinline{e1} of the base language in the definition
% of \lstinline{e2}.



% \jeremy{creating expressions using base and extended expressions, and show more reuse}

% \jeremy{future work} \jeremy{mention in passing this rule is unsound with
%   effects, see ``Intersection types and computational effects''}

% Local Variables:
% mode: latex
% TeX-master: "../paper"
% org-ref-default-bibliography: ../paper.bib
% End:




%%% Local Variables:
%%% mode: latex
%%% TeX-master: "../Thesis"
%%% org-ref-default-bibliography: ../Thesis.bib
%%% End:

  
%%%%%%%%%%%%%%%%%%%%%%%%%%%%%%%%%%%%%%%%%%%%%%%%%%%%%%%%%%%%%%%%%%%%%%%%
\chapter{Semantics of the \fnamee Calculus}
%%%%%%%%%%%%%%%%%%%%%%%%%%%%%%%%%%%%%%%%%%%%%%%%%%%%%%%%%%%%%%%%%%%%%%%%




\section{Motivation}

Parametric polymorphism~\citep{reynolds1983types} is a well-beloved (and
well-studied) programming feature. It enables a single piece of code to be
reused on data of different types. So it is quite natural and theoretically
interesting to study combining parametric polymorphism with disjoint
intersection types, especially how it affects disjointness and coherence. On a more
pragmatic note, the combination of parametric polymorphism and disjoint intersection types also reveals new
insights into practical applications. Dynamic languages (such as JavaScript)
usually embrace quite flexible programming patterns, e.g., mixin composition
where objects can be composed at run time, and their types are not necessarily
statically known. The use of intersection types in TypeScript is inspired by
such flexible programming patterns. For example, an important use of
intersection types in TypeScript is the following function for mixin
composition:
\begin{lstlisting}[language=JavaScript]
function extend<T, U>(first: T, second : U) : T & U {...}
\end{lstlisting}
which is analogous to our merge operator in that it takes two objects and
produces an object with the intersection of the types of the argument objects.
However, the types of the two objects are not known, i.e., they are generic. An
important point is that, while it is possible to define such function in
TypeScript (albeit using some low-level (and type-unsafe) features of
JavaScript), it can also cause, as pointed out by \citet{alpuimdisjoint},
run-time type errors! Clearly a well-defined meaning for intersection types with
type variables is needed.


\paragraph{Disjoint Polymorphism.}

Motivated by the above two points, \citet{alpuimdisjoint} proposed disjoint
polymorphism, a variant of parametric polymorphism. The
main novelty is \textit{disjoint (universal) quantification} of the form $[[ \ X ** A . B ]]$.
Inspired by bounded quantification~\citep{cardelli1994extension}
where a type variable is constrained by a type bound, disjoint quantification
allows type variables to be associated with \textit{disjointness constraints}.
Correspondingly, a term construct of the form $[[ \ X ** A. ee ]]$ is used to
create values of disjoint quantification.
To understand the purpose of disjointness constraints, consider the following program (adapted from \citet{alpuimdisjoint}):
\begin{lstlisting}
mergeBad X (x : X) : X & Int = x ,, 2;
\end{lstlisting}
\lstinline{mergeBad} takes an argument \lstinline{x} of type \lstinline{X} (which is itself a type variable), and merges it with \lstinline{2}.
However, if we were to allow such definition, we could easily create an example where incoherence occurs again:
\begin{lstlisting}
(mergeBad Int 1) : Int -- 1 or 2
\end{lstlisting}
This is essentially the same problem of allowing $\mer{1}{2}$, which as we
discussed in \cref{bg:sec:intersection} will cause ambiguity. For \namee, we
know the concrete type for each variable and thus disjointness checking can help
avoid this problematic expression. However, with parametric polymorphism, a
variable could have any types,
including those that are already in the intersection. So a question to ask is to
decide under what conditions a type variable is disjoint with, say,
\lstinline{Int}. This is where disjointness constraints come into stage. The key
idea is that since we do not know \textit{a priori} what is the type with which
a type variable can be instantiated, we can restrict the set of types it can be
instantiated to. Let us rewrite the above program as follows:
\begin{lstlisting}
mergeGood [X * Int] (x : X) : X & Int = x ,, 2;
\end{lstlisting}
The only change is the notation \lstinline{[X * Int]}, where the left-side of
\lstinline{*} denotes the type variable being declared, and the right-side
denotes the disjointness constraint(s). Here the disjointness constraint
(\lstinline{Int}) effectively states that the type variable \lstinline{X} can be
instantiated to any types disjoint with \lstinline{Int}. For instance, the expression \lstinline{mergeGood Bool True}
type checks but the expression \lstinline{mergeGood Int 1}
is rejected because \lstinline{Int} (the type argument) is not disjoint with
\lstinline{Int} (the disjointness constraint). What is more, we can express multiple
constraints using intersection types, for example,
\begin{lstlisting}
mergeThree [X * Int & Bool] (x : X) : X & Int & Bool = x ,, 2 ,, True;
\end{lstlisting}
Here the type variable \lstinline{X} can only be instantiated to types disjoint with both
\lstinline{Int} and \lstinline{Bool}.


With disjointness constraints and a built-in merge operator, a type-safe and
conflict-free \lstinline{extend} function can be naturally defined as follows:
\begin{lstlisting}
extend T [U * T] (first : T) (second : U) : T & U = first ,, second;
\end{lstlisting}
The disjointness constraint on the type variable \lstinline{U} ensures that no
conflicts can occur when composing two objects, which is quite similar to
trait-based approach~\citep{scharli2003traits} in object-orientated programming.
We shall devote a whole chapter (\cref{chap:traits}) to further this point.


\paragraph{Adding BCD Subtyping.}

While \citet{alpuimdisjoint} studied the combination of disjoint intersection
types and parametric polymorphism, they follow the then-standard approach
of \citet{oliveira2016disjoint} to ensure coherence, thus excluding the
possibility of adding BCD subtyping. The combination of BCD subtyping and
disjoint polymorphism opens doors for more expressiveness. For example, we can
define the following function
\begin{lstlisting}
combine A [B * A] (f : {foo : Int -> A})
                  (g : {foo : Int -> B}) : {foo : Int -> A & B} = f ,, g;
\end{lstlisting}
which ``combines'' two singleton records with parts of types unknown and returns
another singleton record containing an intersection type. A variant of this
function plays a fundamental role in defining Object Algebra combinators (cf.
\cref{chap:case_study}).




In what follows, we first present the syntax and semantics (subtyping and
typing) of \fnamee. We then discuss the disjointness judgment in detail, in
particular, the disjointness relation between type variables and arbitrary
types. Finally we talk about the elaboration semantics of \fnamee and its target
calculus \tnamee, a variant of System F with explicit coercions.



% Local Variables:
% TeX-master: "../../Thesis"
% org-ref-default-bibliography: ../../Thesis.bib
% End:



\section{Syntax and Semantics}

\begin{figure}[t]
  \centering
\begin{tabular}{llll} \toprule
  Types & $[[A]], [[B]], [[C]]$ & $\Coloneqq$ & $[[nat]] \mid [[Top]] \mid [[A -> B]]  \mid [[A & B]] \mid [[{l : A}]] \mid [[X]] \mid [[\ X ** A . B]] $\\
  Monotypes & $[[t]]$ & $\Coloneqq$ & $[[nat]] \mid [[Top]] \mid [[t1 -> t2]]  \mid [[t1 & t2]] \mid [[X]] \mid [[{l : t}]]$\\
  Expressions & $[[ee]]$ & $\Coloneqq$ & $[[x]] \mid [[i]] \mid [[Top]] \mid [[\x . ee]] \mid [[ee1 ee2]] \mid [[ ee1 ,, ee2 ]]   \mid [[ ee : A ]] $ \\
        & & $\mid$ & $ [[{l = ee}]] \mid [[ ee.l  ]] \mid [[\X ** A . ee]] \mid [[ ee A ]]  $ \\
  Value Contexts & $[[GG]]$ & $\Coloneqq$ &  $[[empty]] \mid [[GG , x : A]] $ \\
  Type Contexts & $[[DD]]$ & $\Coloneqq$ &  $[[empty]] \mid [[DD , X ** A]] $  \\ \bottomrule
  % Expression Contexts & $[[CC]]$ & $\Coloneqq$ &  $[[__]] \mid [[\ x . CC]] \mid [[\ X ** A. CC]] \mid [[ CC A  ]] \mid [[CC ee]] \mid [[ee CC]] \mid [[ CC ,, ee  ]]  $ \\
  % & & $\mid$ & $[[ ee ,, CC  ]] \mid  [[ { l = CC}  ]]  \mid [[ CC . l]] $
\end{tabular}
  \caption{Syntax of \fnamee}
  \label{fig:syntax:fi}
\end{figure}

\Cref{fig:syntax:fi} shows the syntax of \fnamee. Metavariables $[[A]], [[B]],
[[C]]$ range over types. Apart from \namee types, \fnamee also includes type
variables $[[X]]$ and disjoint quantification $[[ \X ** A . B ]]$. Monotypes
$[[t]]$ are the same, less the universal quantification. Metavariable $[[ee]]$
ranges over expressions. We extend \namee expressions with two standard
constructs in System F: type abstractions $[[ \X ** A . ee ]]$ and type
applications $[[ee A]]$. The former also includes an extra disjointness
constraint $[[A]]$ associated with the type variable $[[X]]$.

\paragraph{Contexts.}

In the traditional formulation of System F, there is a single context that is
used to keep track of both type variables and term variables. Here we use
another style of presentation~\citep[chap. 16]{Harper_2016} where contexts are
split into \textit{value contexts} $[[GG]]$ and \textit{type contexts} $[[DD]]$.
The former track bound term variables $[[x]]$ with their types $[[A]]$; and the
latter track bound type variables $[[X]]$ with their disjointness constraints
$[[A]]$. This formulation is also convenient for the presentation of logical
relations in \cref{chap:coherence:poly}.

\begin{figure}
  \centering
  \drules[swft]{$[[DD |- A]]$}{Well-formedness of types}{top, int, var, arrow, all, and, rcd}
  \drules[swfe]{$[[DD ||- GG]]$}{Well-formedness of value contexts}{empty, var}
  \drules[swfte]{$[[||- DD]]$}{Well-formedness of type contexts}{empty, var}
  \caption{Well-formedness of contexts and types}
  \label{fig:well-formedness:fi}
\end{figure}

\paragraph{Well-formedness of contexts.}

The well-formedness judgments for contexts and types, as shown in
\cref{fig:well-formedness:fi} are quite standard. They together ensure that each
type appearing in the contexts is well-formed in the sense that there are no
unbound free variables.


\paragraph{Declarative Subtyping.}


\begin{figure}[h]
  \centering
  \drules[FS]{$[[ A <|: B ~~> c]]$}{Declarative subtyping}{refl,trans,top,rcd, arr,andr,andl,and,distArr,topArr,distRcd,topRcd,forall}
  \caption{Declarative subtyping of \fnamee}
  \label{fig:subtyping:fi}
\end{figure}

\Cref{fig:subtyping:fi} presents the subtyping relation of \fnamee. For now, we
ignore the coercion parts ($[[~~>]] [[c]]$) and explain them in
\cref{sec:elaboration:fi}. We naturally extend the subtyping rules of \namee
with only one rule \rref*{FS-forall}, which specifies the subtyping relation
between two universal quantifiers. In \rref{FS-forall}, a universal quantifier
is covariant in its body, and contravariant in its disjointness constraint. A
minor comment is that since \fnamee features explicit polymorphism, type
variables are neutral to subtyping, i.e., $[[X <: X]]$, which is contained in
\rref{FS-refl}. As with \namee subtyping, the subtyping relation of \fnamee is
trivially \textit{reflexive} and \textit{transitive}.

\begin{remark}
  In our Coq formalization, we require that the two types $[[A]]$ and $[[B]]$ are
  well-formed with respect to some type context, resulting in the subtyping
  judgment $[[DD |- A <: B]]$. But this is not very important
  for the purpose of presentation, thus we omit contexts.
\end{remark}


\paragraph{Typing.}

\begin{figure}
  \centering
  \drules[FT]{$[[DD; GG |- ee => A ~~> e]]$}{Inference}{top, int, var, app, merge, anno, tabs, tapp, rcd, proj}
  \drules[FT]{$[[DD ; GG |- ee <= A ~~> e]]$}{Checking}{abs, sub}
  \caption{Bidirectional type system of \fnamee}
  \label{fig:typing:fi}
\end{figure}


The bidirectional type system of \fnamee follows that of \namee, as shown
in \cref{fig:typing:fi}. Again we ignore the translation parts ($[[~~>]] [[e]]$) and explain them in
\cref{sec:elaboration:fi}. The inference judgment $[[ DD; GG |- ee => A  ]]$
says that we can synthesize the type $[[A]]$ in the contexts $[[DD]]$ and $[[GG]]$. The checking judgment
$[[ DD ; GG |- ee <= A  ]]$ asserts that $[[ee]]$ checks against the type $[[A]]$
in the contexts $[[DD]]$ and $[[GG]]$. The rules directly ported from \namee are inferring rules \rref*{FT-top} for top values,
\rref*{FT-int} for integers, \rref*{FT-var} for variables, \rref*{FT-app} for applications, \rref*{FT-merge} for merges,
\rref*{FT-anno} for annotated terms, \rref*{FT-rcd,FT-proj} for records; checking rules \rref*{FT-abs} for term abstractions, and
the subsumption rule \rref*{FT-sub}. Note that in \rref{FT-merge}, the disjointness judgment has an extra type context, which will be
explained in \cref{sec:disjoint:fi}.

\paragraph{Disjoint quantification.}

The new rules are the inferring rules for type abstractions \rref*{FT-tabs} and
type applications \rref*{FT-tapp}. In \rref{FT-tabs}, the disjointness
constraint is added to the type context. During a type application in
\rref{FT-tapp}, the type system checks that the type argument agrees with the
disjointness constraint. This, together with \rref{FT-merge} are the only two
rules that use the disjointness checking. Moreover, since \fnamee is
predicative, we require that the type being instantiated is a monotype.



% Local Variables:
% TeX-master: "../../Thesis"
% org-ref-default-bibliography: ../../Thesis.bib
% End:



\section{Disjointness}
\label{sec:disjoint:fi}


\begin{figure}[t]
  \centering
  \drules[FD]{$[[DD |- A ** B]]$}{Disjointness}{topL, topR, arr, andL, andR, rcdEq, rcdNeq, tvarL, tvarR, forall,ax}
  \drules[Dax]{$[[A **a B]]$}{Disjointness axioms}{sym, intArr, intRcd,intAll,arrAll,arrRcd,allRcd}
  \caption{Disjointness of \fnamee}
  \label{fig:disjoint:fi}
\end{figure}


In this section we present the formal rules for disjointness. As show in
\cref{fig:disjoint:fi}, the disjointness rules of \fnamee are directly inherited
from \fname, which consists of two judgments.


\paragraph{Main judgment.}

The main judgment $[[DD |- A ** B]]$ says that the two types $[[A]]$ and $[[B]]$
are disjoint in the context $[[DD]]$. As a precondition, we require the two types $[[A]]$
and $[[B]]$ are both well-formed in the context $[[DD]]$.
Most of the rules are similar to those of
\namee. The major additions are the two rules (\rref*{FD-tvarL,FD-tvarR}) for
type variables, and \rref{FD-forall} for disjoint quantification.
\Rref{FD-tvarL} and its symmetric one \rref*{FD-tvarR} both state that a type
variable $[[X]]$ is disjoint with some type $[[B]]$ if and only if the
disjointness constraint of $[[X]]$ in the context $[[DD]]$ is a subtype of
$[[B]]$. These two rules are a specialization of a more general lemma, which
says that disjointness is covariant with respect to subtyping. In a more precise
sense, we have the following:

\begin{lemma}[Covariance of disjointness] \label{lemma:covariance:disjoint}
  If $[[DD |- A ** B]]$ and $[[B <: C]]$, then $[[DD |- A ** C]]$.
\end{lemma}
\begin{proof}
  By double induction, first on the subtyping derivation, and then on the
  type $[[A]]$. In the case for \rref{FS-forall}, we need \cref{lemma:narrow:disjoint}.
\end{proof}

\begin{lemma}[Narrowing of disjointness] \label{lemma:narrow:disjoint}
  If $[[DD, X ** C1 |- A ** B]]$ and $[[C2 <: C1]]$, then $[[DD, X ** C2 |- A ** B]]$.
\end{lemma}
\begin{proof}
  We need to slightly generalize the lemma in the sense that the type variable is inserted
  in the middle, then by induction on the disjointness derivation.
\end{proof}

An intuition of the following may help better understanding
\cref{lemma:covariance:disjoint}. As we will see in \cref{sec:category}, another
way to interpret disjointness of two types is that their least upper bound is
(isomorphic to) $[[Top]]$. Following this interpretation, it is obvious that if
the least upper bound of two given types is already $[[Top]]$, a supertype of
one of them will not change this fact.

Let us now turn to \rref{FD-forall}. To illustrate this rule, consider the following two types:
\begin{mathpar}
  [[ \X ** nat . X & nat ]] \and  [[ \X ** char . X & char ]]
\end{mathpar}
Under what conditions are the two types disjoint? In the first type, $[[X]]$
cannot be instantiated to $[[nat]]$ (among others) and in the second type
$[[X]]$ cannot be instantiated to $[[char]]$. Therefore for both bodies to be disjoint,
$[[X]]$ can only be instantiated to types that are disjoint with both $[[nat]]$
and $[[char]]$. More formally, in \rref{FD-forall}, we add a new
constraint by intersecting the two constraints to the context and check for disjointness
in the extended context.

\paragraph{Disjointness axioms.}

Disjointness axioms $[[ A **a B ]]$  take care of two types with different type constructs,
except for when one of them is $[[Top]]$, an intersection type or a type
variable, which are all dealt with by the main judgment.

To conclude this section, we show that disjointness is symmetric:

\begin{lemma}[Symmetry of disjointness]
  If $[[ DD |- A ** B  ]]$, then $[[  DD |- B ** A   ]]$.
\end{lemma}
\begin{proof}
  By induction on the disjointness derivation. In the case for \rref{FD-forall},
  apply \cref{lemma:narrow:disjoint}.
\end{proof}


\section{Elaboration and Type Safety}
\label{sec:elaboration:fi}



\begin{figure}[t]
  \centering
\begin{tabular}{llll} \toprule
  Types & $[[T]]$ & $\Coloneqq$ & $[[nat]] \mid [[Unit]] \mid [[T1 -> T2]]  \mid [[T1 * T2]] \mid [[X]] \mid [[\ X . T]]$\\
  Expressions & $[[e]]$ & $\Coloneqq$ & $[[x]] \mid [[i]] \mid [[unit]] \mid [[\x . e]] \mid [[e1 e2]] \mid [[< e1 , e2>]] \mid [[\X . e]] \mid [[ e T ]] \mid [[c e]]$ \\
  Coercions & $[[c]]$ & $\Coloneqq$ & $[[id]] \mid [[c1 o c2]] \mid [[top]] \mid [[c1 -> c2]] \mid [[< c1 , c2 >]] \mid [[pp1]] \mid [[pp2]] \mid [[\ c]]$ \\
  & & $\mid$ & $ [[distArr]] \mid [[topArr]] $ \\
  Values & $[[v]]$ & $\Coloneqq$ & $[[i]] \mid [[unit]] \mid [[\x . e]] \mid [[< v1 , v2>]] \mid [[\X . e]] \mid [[ (c1 -> c2) v ]] \mid [[\c v]]  $ \\
  & & $\mid$ & $ [[distArr v]] \mid [[topArr v]] $ \\
  Value Contexts & $[[gg]]$ & $\Coloneqq$ &  $[[empty]] \mid [[gg , x : T]] $ \\
  Type Contexts & $[[dd]]$ & $\Coloneqq$ &  $[[empty]] \mid [[dd , X ]] $ \\ \bottomrule
  % Expression Contexts & $[[cc]]$ & $\Coloneqq$ &  $[[__]] \mid [[\ x . cc]] \mid [[\ X . cc]] \mid [[ cc T  ]] \mid [[cc e]] \mid [[e cc]] \mid [[< cc , e>]] \mid [[<e , cc>]] \mid [[c cc]] $
\end{tabular}
\caption{Syntax of \tnamee}
\label{fig:syntax:fco}
\end{figure}


The syntax is shown in \cref{fig:syntax:fco}.


% \drules[ct]{$[[ c |- T1 tri T2  ]]$}{Coercion typing}{refl,trans,top,arr,pair,projl,projr,forall,topArr, distArr}

% \drules[wfe]{$[[ dd |- gg   ]]$}{Well-formedness of value contexts}{empty, var}

% \drules[wft]{$[[ dd |- T   ]]$}{Well-formedness of types}{int, var, arrow,prod, all}

% \drules[s]{$[[ e --> e' ]]$}{Coercion reduction}{id, trans, top, arr, pair, topArr, distArr, projl, projr, forall}

% \drules[Ft]{$[[ dd ; gg |- e : T ]]$}{Typing}{unit, int, var, abs, app, tabs, tapp, pair, capp}


%%% Local Variables:
%%% mode: latex
%%% TeX-master: "../Thesis"
%%% End:


  \part{Coherence}

  

%%%%%%%%%%%%%%%%%%%%%%%%%%%%%%%%%%%%%%%%%%%%%%%%%%%%%%%%%%%%%%%%%%%%%%%%
\chapter{Coherence for \namee}
\label{chap:coherence:simple}
%%%%%%%%%%%%%%%%%%%%%%%%%%%%%%%%%%%%%%%%%%%%%%%%%%%%%%%%%%%%%%%%%%%%%%%%

This chapter constructs a logical relation to
establish coherence of \namee. Finding a
suitable definition of coherence for \namee is already challenging in its own
right. In what follows we reproduce the steps of finding a definition for coherence
that is both intuitive and applicable. Then we present the
construction of the logical (equivalence) relation tailored to this
definition, and the connection between logical equivalence and coherence.
\Cref{chap:coherence:poly} builds on the idea in this chapter to prove coherence for
\fnamee.

%-------------------------------------------------------------------------------
\section{The Intuition}

\paragraph{Duplication is Harmless.}

While requiring that all intersections are disjoint as in \oname is sufficient
to guarantee coherence, it is not necessary. In fact, such requirement
unnecessarily encumbers the subtyping definition with disjointness constraints
and an ad-hoc treatment of ``top-like'' types. Indeed, the value $[[1 ,, 1]]$ of
the non-disjoint type $[[ nat & nat ]]$ is entirely unambiguous, and
$(\mer{1}{1}) + 3$ can obviously only result in $4$. More generally, when the
overlapping components of an intersection type have the same value, there is no
ambiguity problem. \namee uses this idea to relax \oname's enforcement of
disjointness. In the case of a merge, it is hard to statically decide whether
the two arguments have the same value, and thus \namee still requires
disjointness. Yet, disjointness is no longer required for the well-formedness of
types and overlapping intersections can be created implicitly through subtyping,
which results in duplicating values at run time. For instance, while $[[ 1,, 1]]$
is not expressible $[[ 1 : nat & nat]]$ creates the equivalent value implicitly.
% The consequence of this relaxation of disjointness is a much simplified
% type system for \namee.
In short, duplication is harmless and subtyping only generates duplicated values
for non-disjoint types.


% Coherence is easy to establish for \oname as its rigid rules mean that there is
% at most one possible subtyping derivation between any two types.  As a
% consequence there is only one possible elaboration and thus one
% possible behavior for any program.

Two factors make establishing coherence for \namee much more difficult: the
relaxation of disjointness and the adoption of the more expressive subtyping
rules from the BCD system (for which \oname lacks). These two factors mean that
subtyping proofs are no longer unique and hence that there are multiple
elaborations of the same source program. For instance, $[[ nat & nat ]]$ is a
subtype of $[[nat]]$ in two ways: by projection on either the first or second
component. Hence the fact that all elaborations yield the same result when
evaluated has become a much more subtle property that requires sophisticated
reasoning. For instance, we can see that coherence holds because at run time any
value of type $[[nat & nat]]$ has identical components, and
thus both projections yield the same result.




For \namee in general, we show coherence by capturing the non-ambiguity
invariant in a \emph{logical relation}~\citep{tait, plotkin1973lambda,
  statman1985logical} and showing that it is preserved by the operational
semantics. In doing so, we remove the brittleness of the previous syntactic
method to prove coherence. This new proof method has several advantages.
Firstly, with the new proof method, several restrictions that were enforced by
\oname to enable the syntactic proof are removed. For example, the
aforementioned \emph{top-like types} are not necessary; top-like types are
handled like all other types. Secondly, the new proof method is more powerful
because it is based on observational equivalence rather than syntactic equality;
it is more robust as the type system is extended. Finally, the removal of the
ad-hoc side-conditions makes adding new extensions, such as support for
BCD-style subtyping, easier. A complicating factor is that not one, but two
languages are involved: the source language and the target language. In order to
deal with the complexity of the elaboration semantics of \namee, we employ
binary logical relations that are heterogeneous, parameterized by two types; the
fundamental property is also reformulated to account for bidirectional
type-checking. A caveat is that our logical relation does not hold for target programs and
program contexts in general, but only for those that are the image of a
corresponding source program or program context. Thus we must view everything
through the lens of elaboration.



\section{In Search of Coherence}

In \oname the definition of coherence is based on
$\alpha$-equivalence. More specifically, the coherence property in \oname states that
any two target terms that a source expression elaborates into must be exactly the same (up to
$\alpha$-equivalence). Unfortunately this syntactic notion of coherence is
very fragile with respect to extensions.
For example, it is not obvious how to retain this notion of coherence when adding more subtyping
rules such as those in \cref{fig:subtype_decl}.

If we permit ourselves to consider only the syntactic aspects of expressions,
then very few expressions can be considered equal. The syntactic view also conflicts
with the intuition that ``the significance of an expression lies in its
contribution to the \textit{outcome} of a computation''~\citep{Harper_2016}.
Drawing inspiration from a wide range of literature on contextual
equivalence~\citep{morris1969lambda}, we want a context-based notion of
coherence. It is helpful to consider several examples before presenting the
formal definition of our new semantically-founded notion of coherence.

\begin{example} \label{eg:1}
The same \namee expression $1$ can be typed $[[nat]]$ in many ways: for instance, by \rref{T-lit}; by
\rref{T-sub,S-refl}; or by \rref{T-sub,S-trans,S-refl}, resulting in translations
$[[1]]$, $[[id 1]]$ and $[[ (id o id) 1 ]]$, respectively. It is apparent
that these three \tname terms are ``equal'' in the sense that all reduce to the same numeral $[[1]]$.
\end{example}

\subsection{Expression Contexts and Contextual Equivalence.}

To formalize the intuition, we turn to \textit{expression contexts}, as
introduced in \cref{sec:bg:lr}. The syntax of \tname contexts $[[cc]]$ can be found in
\cref{fig:contexts}. The static semantics of \tname is extended to expression
contexts by defining the typing judgment
\[
  [[cc : (gg |- T) ~> (gg' |- T')]]
\]
where $([[gg |- T]])$ indicates the type of the hole. This judgment is
inductively defined so that if $[[gg |- e : T]]$, then $[[gg' |- cc{e} : T']]$.

\begin{figure}[t]
  \centering
\begin{tabular}{llll}\toprule
  \tname contexts & $[[cc]]$ & $\Coloneqq$ & $[[__]] \mid [[\x . cc]] \mid [[cc e]] \mid [[e cc]] \mid [[< cc, e >]] \mid [[< e, cc >]] \mid [[c cc]] $ \\
  \namee contexts & $[[CC]]$ & $\Coloneqq$ & $[[__]] \mid [[\x . CC]] \mid [[CC ee]] \mid [[ee CC]] \mid [[ee ,, CC]] \mid [[CC ,, ee]] \mid [[CC : A]] $ \\
  & & $\mid$ & $ [[ { l = CC } ]] \mid [[CC.l]]$ \\ \bottomrule
\end{tabular}
  \caption{Expression contexts of \tname and \namee}
  \label{fig:contexts}
\end{figure}

We define a \textit{complete program} to mean any closed term of type $[[nat]]$.
Recall the definitions of Kleene equality and contextual equivalence in \cref{sec:bg:lr}.
For ease of reference, we restate them below:

\kleene*

\kleenee*

Regarding \cref{eg:1}, it seems adequate to say that $3$ and $\app{[[id]]}{3}$
are contextually equivalent. Does this imply that coherence can be based on
\cref{def:cxtx}? Unfortunately it cannot, as demonstrated by the following
example.


\begin{example} \label{eg:2} It may be counter-intuitive that two \tname terms
  $[[\x . pp1 x]]$ and $[[\x . pp2 x]]$ should also be considered equal. To see
  why, first note that they are both the translations of the same \namee
  expression: $[[(\x . x) : nat & nat -> nat]]$. What can we do with this lambda
  abstraction? We can apply it to $[[1]]$ for example, which leads
  to two translations $[[  (\x . pp1 x) <1 , 1>  ]]$ and $[[ (\x . pp2 x) <1, 1>  ]]$. It is obvious that both reduce to the same numeral
  $[[1]]$. However, $[[\x . pp1 x]]$ and $[[\x . pp2 x]]$ are definitely \textit{not} equal
  according to \cref{def:cxtx}, as one can find a context
  $[[ __ <1, 2> ]]$ in which the two terms reduce to two different
  numerals. The problem is that
  % not every well-typed \tname term
  % can be obtained from a well-typed \namee expression through the
  % elaboration semantics. For example,
  $[[ __ <1, 2>  ]]$ should not be considered because the
  (non-disjoint) source expression $[[ 1 ,, 2 ]]$ is rejected by the type system
  of the source calculus \namee and thus never gets elaborated into $[[ < 1, 2>  ]]$.
\end{example}




\subsection{\namee Contexts and Refined Contextual Equivalence.}

\cref{eg:2} hints at a shift from \tname contexts to \namee contexts $[[C]]$,
whose syntax is shown in \cref{fig:contexts}. Due to the bidirectional
nature of the type system, the typing judgment of $[[C]]$ features 4
different forms:
\begin{mathpar}
  [[CC : (GG => A) ~> (GG' => A') ~~> cc]] \and
  [[CC : (GG <= A) ~> (GG' => A') ~~> cc]] \and
  [[CC : (GG => A) ~> (GG' <= A') ~~> cc]] \and
  [[CC : (GG <= A) ~> (GG' <= A') ~~> cc]]
\end{mathpar}
We write $[[CC : (GG dir A) ~> (GG' dir' A') ~~> cc]]$ to abbreviate the above 4
different forms. Take $[[CC : (GG => A) ~> (GG' => A') ~~> cc]]$ for example
(whose typing rules are shown in \cref{fig:ctyp}), it reads that if
$[[GG |- ee => A]]$, then $[[GG' |- CC{ee} => A']]$. The judgment also generates
a \tname context $[[cc]]$ so that $[[cc : (|GG| |- |A|) ~> (|GG'| |- |A'|)]]$
holds by construction. The full typing rules appear in \cref{appendix:lambdai}. Now we are
ready to refine \cref{def:cxtx}'s contextual equivalence to take into
consideration both \namee and \tname contexts.


\begin{figure}[t]
  \centering
  \begin{small}
\drules[CTyp]{$[[CC : ( GG => A ) ~> ( GG' => B ) ~~> cc]]$}{Context typing I}{emptyOne, appLOne, appROne, mergeLOne, mergeROne, rcdOne, projOne, annoOne}
  \end{small}
\caption{\namee context typing (excerpt)}
\label{fig:ctyp}
\end{figure}



\begin{definition}[\namee Contextual Equivalence] \label{def:cxtx2}
  \begin{align*}
    [[GG |- ee1 ~= ee2 : A]]  & \defeq \forall [[e1]], [[e2]].\  [[GG |- ee1 => A ~~> e1]] \land [[GG |- ee2 => A ~~> e2]] \ \land \\
                                & \qquad (\forall [[C]], [[cc]]. \ [[CC : (GG => A) ~> (empty => nat) ~~> cc]]  \Longrightarrow \kleq{[[cc{e1}]]}{[[cc{e2}]]})
  \end{align*}
\end{definition}

In other words, two source expressions are contextually equivalent if their
translations are equivalent in all possible source contexts. For brevity we only
consider expressions in the inference mode. Our Coq formalization is complete
with two modes.
Now regarding \cref{eg:2}, a possible \namee context is
\[
[[ __ 1 : (empty => nat & nat -> nat) ~> (empty => nat) ~~> __ <1 , 1>]]
\]
We can verify that both $[[\x . pp1 x]]$ and $[[\x . pp2 x]]$ produce $1$ in the context $[[__ <1 , 1>]]$.
Of course we should consider all possible contexts to be certain that they are truly equal. From now on, we
use the symbol $\backsimeq_{ctx}$ to refer to contextual equivalence in
\cref{def:cxtx2}. With \cref{def:cxtx2} we can formally state that \namee is coherent
in the following sense:

\begin{restatable}[Coherence]{theorem}{coherence} \label{thm:coherence}
  We have that
  \begin{itemize}
  \item If $[[GG |- ee => A ]]$ then $[[GG |- ee ~= ee : A]]$.
  \item If $[[GG |- ee <= A ]]$ then $[[GG |- ee ~= ee : A]]$.
  \end{itemize}
\end{restatable}

That is, coherence is just a special case of \cref{def:cxtx2} where we set
$[[ee1]]$ and $[[ee2]]$ to be the same source expression. At first glance, this
appears underwhelming: of course $[[ee]]$ behaves the same as itself! The tricky part is
that, if we expand it according to \cref{def:cxtx2}, it is not $[[ee]]$
itself but all its translations $[[e]]$ that should behave the same!
The rest of the chapter is devoted to proving the validity of \cref{thm:coherence}.

\section{The Canonicity Logical Relation, Formally Defined}

As intuitive as \cref{def:cxtx2} may seem, it is generally very hard to prove
contextual equivalence directly, since it involves quantification over
\textit{all} possible contexts. Worse still, two kinds of contexts are involved
in \cref{thm:coherence}, which makes reasoning even more tedious. The key to
simplifying the reasoning is to exploit types using logical
relations~\citep{tait, statman1985logical, plotkin1973lambda}.


\paragraph{In Search of a Logical Relation.}

\begin{figure}
  \centering
  \begin{tabular}{lll}
  $[[(v1 , v2) in V ( nat ; nat ) ]]$  & $\defeq$ & $\exists [[i]].\, [[v1]] = [[v2]] = [[ii]]$ \\
  $[[(v1, v2) in V ( {l : A}  ; {l : B} ) ]]$ & $\defeq$ & $[[ (v1, v2) in V ( A ; B ) ]]$\\
  $[[(v1 , v2) in V ( A1 -> B1 ; A2 -> B2 ) ]]$  & $\defeq$ & $\forall [[(v2' , v1') in V ( A2 ; A1 ) ]].\, [[ (v1 v1' , v2 v2') in E ( B1 ; B2 ) ]]$ \\
  $[[( < v1 , v2 > , v3  )  in V ( A & B ;  C  ) ]]$  & $\defeq$ & $[[ (v1, v3)  in V (A ; C) ]] \land [[ (v2, v3)  in V (B ; C) ]]$  \\
  $[[( v3 , < v1 , v2 >  )  in V ( C; A & B  ) ]]$  & $\defeq$ & $[[ (v3, v1)  in V (C ; A) ]] \land [[ (v3, v2)  in V (C ; B) ]]$  \\
  $[[(v1 , v2) in V (A; B) ]]$  & $\defeq$ & $\mathsf{true} \quad \text{otherwise}$ \\ \\
    $[[(e1, e2) in E (A; B)]]$ & $\defeq$ & $\exists [[v1]], [[v2]].\, [[e1 -->> v1]] \land [[e2 -->> v2]] \ \land $ \\
                                       & & $[[(v1, v2) in V (A; B)]]$
  \end{tabular}
  \caption{The canonicity logical relation for \namee}
  \label{fig:logical}
\end{figure}



It is worth pausing to ponder what kind of relation we are looking for. % From
% \cref{eg:2}, it is clear that pairs have a special status in \tname. Indeed they
% ought to be, since pairs originate from merges or subtyping. Also disjointness
% on intersection types should correspond to some sort of constraints over pairs.
The high-level intuition behind the relation is to capture the notion of
``coherent'' values. These values are unambiguous in all possible (source)
contexts. A moment of thought leads us to the following observations:

\begin{observation}[Disjoint values are unambiguous] \label{ob:1}

  The relation should relate values originating from disjoint intersection
  types. Those values are essentially translated from merges, and since
  \rref{T-merge} ensures disjointness, they are unambiguous. For example, two
  values of types $[[nat]]$ and $[[ { l : nat}]]$ can always be distinguished by
  any source context.
\end{observation}

\begin{observation}[Duplication is unambiguous] \label{ob:2}

  The relation should also relate values originating from non-disjoint intersection
  types, only if the values are duplicates. This may sound baffling, since the
  whole point of disjointness is to rule out (ambiguous) expressions such as
  $[[ 1 ,, 2   ]]$. However, $[[ 1,, 2 ]]$ never gets elaborated, and the only values
  corresponding to $[[nat & nat]]$ are those pairs such as $[[ <1 , 1>   ]]$,
  $[[ <2 , 2>   ]]$, etc. Those values are essentially generated from \rref{T-sub} by subtyping
  and are also unambiguous.
\end{observation}

\paragraph{Canonicity.}

In order to deal with the complexity of the elaboration semantics, we introduce
in \cref{fig:logical} what we call the \textit{canonicity} logical relation to capture ``canonical''
values based on the above observations. Canonicity is a family of binary
relations over \tname values that are \textit{heterogeneous}, i.e., indexed by
two \namee types. Heterogeneity allows us to relate values of different types,
and in particular values of disjoint types. Canonicity seeks to combine equality
checking from traditional (homogeneous) logical relations (\cref{ob:2}) with
disjointness checking (\cref{ob:1}). It consists of two relations. The
value relation $\valR{[[A]]}{[[B]]}$ relates \textit{closed} values, i.e.,
well-typed values with no free variables. Similarly, the expression relation
$\eeR{[[A]]}{[[B]]}$ relates closed expressions. For brevity, we write
$\valRR{[[A]]}$ to mean $\valR{[[A]]}{[[A]]}$, and $\eeRR{[[A]]}$ for
$\eeR{[[A]]}{[[A]]}$.

% \begin{remark}
%   The logical relations resemble those given by Biernacki and
%   Polesiuk~\citep{biernacki2015logical}, as both are heterogeneous. However, two
%   important differences are worth pointing out. Firstly, our value relation for
%   product types ($\valR{[[T1 * T2]]}{[[T3]]}$ and $\valR{[[T3]]}{[[T1 * T2]]}$)
%   is unusual. Secondly, their value relation disallows relating functions with
%   natural numbers, while ours does not. As we explain shortly, both points are
%   related to disjointness.
% \end{remark}



First let us consider the relation $\valR{[[A]]}{[[B]]}$, which specifies when
two closed values $[[v1]]$ and $[[v2]]$ are related at the types $[[A]]$ and
$[[B]]$. The definition for integers and records are straightforward. Two
integers are related if they are equal. For records, recall that in
\cref{sec:elaboration}, record labels are erased during translation. Therefore
two values are related at two record types of the same label if they are related
at the two field types.

Functions $[[v1]]$ and $[[v2]]$ are related at the types $[[A1 -> B1]]$ and
$[[A2 -> B2]]$ if given two arguments $[[v1']]$ and $[[v2']]$ related at the
argument types $[[A1]]$ and $[[A2]]$, the functions applied to the arguments are
related expressions at the result types $[[B1]]$ and $[[B2]]$. Note that in
\tname, the values $[[v1]]$ and $[[v2]]$ may each be a lambda abstraction, or a
coercion application of a function type.

% Two functions are related if they map
% related arguments to related results.  These
% cases reflect \cref{ob:2}: values of the same type are duplicates.


The definition of $\valR{[[A]]}{[[B]]}$ is made more interesting when one of the
indexed types is an intersection type. In that case, the relation distributes
over the type constructor $[[&]]$. It is instructive to compare the type constructor $[[&]]$ with product
types $[[*]]$. The traditional way of relating pairs is by relating their components
pairwise. That is, $[[<v1,v2>]]$ and $[[<v1', v2'>]]$ are related at $[[ A * B  ]]$ if (1)
$[[v1]]$ and $[[v1']]$ are related at $[[A]]$ and (2) $[[v2]]$ and $[[v2']]$ are related at $[[B]]$.
According to our definition, we also require that (3) $[[v1]]$ and $[[v2']]$ are
related and (4) $[[v2]]$ and $[[v1']]$ are related. To see why this is the case, consider
whether $(\pair{1}{2}, \pair{1}{2}) \in \valRR{[[nat & nat]]}$. If we regard
$[[nat & nat]]$ as a normal product type, then these two pairs are related.
However, as we mentioned, $\pair{1}{2}$ should not be considered as the image of
some source expression at the type $[[nat & nat]]$, and our definition correctly
rejects it because $1$ is not equal to $2$, while accepting pairs such as
$\pair{1}{1}, \pair{2}{2}$, etc.

The acute reader may have noticed the structural similarity between the two
clauses for intersection types and the disjointness rules for intersection types:
\begin{mathpar}
 \drule{D-andL} \and \drule{D-andR}
\end{mathpar}
This is not a coincidence---we can show that disjointness and the value relation
are connected by the following lemma:

\begin{lemma}[Disjoint values are related] \label{lemma:disjoint}
  If $[[A ** B]]$ and $[[  v1 : |A|  ]]$ and
  $[[  v2 : |B|  ]]$,
  then $[[   (v1, v2) in V ( A ; B  )    ]]$.
\end{lemma}
\begin{proof}
  By induction on the derivation of disjointness.
\end{proof}

Next we consider $\eeR{[[A]]}{[[B]]}$, which is standard. Informally it
expresses that two closed terms $[[e1]]$ and $[[e2]]$ are related if
they evaluate to two values $[[v1]]$ and $[[v2]]$ that are related.



\paragraph{Logical Equivalence.}

The logical relation can be lifted to open terms in the usual way. First we give
the semantic interpretation of typing contexts. A \textit{closing substitution}
$[[ g ]]$ for the typing context $ [[GG]] = [[x1]] : [[A1]], \dots , [[xn]] :
[[An]] $ is a finite function assigning closed values
$[[ v1 ]] : [[ |A1|]], \dots , [[vn]] : [[ |An| ]]$ to $[[ x1 ]] , \dots , [[xn]]$, respectively.
We write $[[ g(e) ]]$ for the substitution $[  [[v1]], \dots, [[vn]]  / [[x1]], \dots, [[xn]]   ] [[e]]  $.
The interruption of typing contexts, written $[[  (g1, g2) in GG ]]$ is inductively defined as follows:

\begin{definition}[Interpretation of value contexts]
  \begin{mathpar}
    \ottaltinferrule{}{}{  }{ [[(emp, emp) in empty ]]  } \and
    \ottaltinferrule{}{}{ [[(g1, g2) in GG  ]] \\ [[(v1, v2) in V (A) ]] }{ [[(g1 [ x -> v1 ] , g2 [ x -> v2 ]  )  in GG , x : A  ]] }
  \end{mathpar}
\end{definition}

Two open terms are related if every pair of related closing substitutions
makes them related:
\begin{definition}[Logical equivalence]
  \begin{align*}
    [[GG |- e1 == e2 : A ; B]] & \defeq [[|GG| |- e1 : |A|]] \land [[|GG| |- e2 : | B | ]] \ \land \\
                                 & \qquad (\forall [[g1]], [[g2]] .\, [[(g1, g2) in GG ]] \Longrightarrow [[(g1 (e1), g2 (e2))  in E (A ; B) ]])
  \end{align*}
\end{definition}
For succinctness, we write $[[GG |- e1 == e2 : A]]$ to mean $[[GG |- e1 == e2 : A ; A]]$.


\section{Establishing Coherence}

With all the machinery in place, we are now ready to prove \cref{thm:coherence}.
But we need several lemmas to set the stage.

Firstly we need the compatibility lemmas, which state that logical equivalence is
preserved by language constructs. Most of them are standard and are thus omitted.
We show only two compatibility lemmas that are specific to our logical relation:

\begin{lemma}[Coercion Compatibility]   \label{lemma:co-compa}
  Suppose that $[[A1 <: A2 ~~> c]]$,
  \begin{itemize}
  \item If $[[GG |- e1 == e2 : A1 ; A0]]$ then $[[GG |- c e1 == e2 : A2 ; A0]]$.
  \item If $[[GG |- e1 == e2 : A0 ; A1]]$ then $[[GG |- e1 == c e2 : A0 ; A2]]$.
  \end{itemize}
\end{lemma}
\begin{proof}
  By induction on the subtyping derivation.
\end{proof}

\begin{lemma}[Merge compatibility]
  If $[[   GG |- e1 == e1' : A ]]$, $[[  GG |- e2 == e2' : B ]]$ and $[[ A ** B ]]$,
  then $[[   GG |- < e1, e2 > == <e1', e2'> : A & B ]]$.
\end{lemma}
\begin{proof}
  By the definition of logical relation and \cref{lemma:disjoint}.
\end{proof}



The ``Fundamental Property'' states that any well-typed expression is related to
itself by the logical relation. In our elaboration setting, we rephrase it so
that any two \tname terms elaborated from the \textit{same} \namee expression are related
by the logical relation. To prove it, we require \cref{thm:uniq}.

\begin{theorem}[Inference Uniqueness] \label{thm:uniq}
  If $[[GG |- ee => A1]]$ and $[[GG |- ee => A2]]$, then $[[A1]] \equiv [[A2]]$.
\end{theorem}

\begin{theorem}[Fundamental Property]  \label{thm:co-log} We have that:
  \begin{itemize}
  \item If $[[GG |- ee => A ~~> e]]$ and $[[GG |- ee => A ~~> e']]$, then $[[GG |- e == e' : A ]]$.
  \item If $[[GG |- ee <= A ~~> e]]$ and $[[GG |- ee <= A ~~> e']]$, then $[[GG |- e == e' : A ]]$.
  \end{itemize}
\end{theorem}
\begin{proof}
  The proof follows by induction on the first derivation. The most interesting
  case is \rref{T-sub}
  \begin{mathpar}
    \drule{T-sub}
  \end{mathpar}
  where we need \cref{thm:uniq} to be able to apply the induction hypothesis.
  Then we apply \cref{lemma:co-compa} to say that the coercion generated
  preserves the relation between terms. For the other cases we use the
  appropriate compatibility lemmas.
\end{proof}


We show that logical equivalence is preserved by \namee contexts:

\begin{lemma}[Congruence] \label{lemma:cong}
 If $[[CC : (GG dir A) ~> (GG' dir' A') ~~> cc]]$, $[[GG |- ee1 dir A ~~> e1]]$, $[[GG |- ee2 dir A ~~> e2]]$
 and $[[GG |- e1 == e2 : A]]$, then $[[GG' |- cc{e1} == cc{e2} : A']]$.
\end{lemma}
\begin{proof}
  By induction on the typing derivation of the context $[[C]]$, and applying
  the compatibility lemmas where appropriate.
\end{proof}


\begin{lemma}[Adequacy] \label{lemma:ade}
  If $[[  empty |- e1 == e2 : nat ]]$ then $\kleq{[[e1]]}{[[e2]]}$.
\end{lemma}
\begin{proof}
  Adequacy follows easily from the definition of the logical relation.
\end{proof}


Next up is the proof that logical relation is sound with respect to contextual
equivalence---that is, if two programs are logically related then they are
contextually equivalent---which justifies the use of logical relation for
proving contextual equivalence of programs.

\begin{theorem}[Soundness w.r.t. Contextual Equivalence] \label{thm:log-sound}
  Given $[[GG |- e1 == e2 : A]]$, we have
  \begin{itemize}
  \item If $[[ GG |- ee1 => A ~~> e1]]$ and $[[ GG |- ee2 => A ~~> e2]]$ then
    $[[ GG |- ee1 ~= ee2 : A ]]$.
  \item If $[[ GG |- ee1 <= A ~~> e1]]$ and $[[ GG |- ee2 <= A ~~> e2]]$ then
    $[[ GG |- ee1 ~= ee2 : A ]]$.
  \end{itemize}
\end{theorem}
\begin{proof}
  From \cref{def:cxtx2}, we are given a context $[[  CC : (GG => A) ~> (empty => nat) ~~> cc ]]$. By \cref{lemma:cong}
  we have $[[  empty |- cc{e1} == cc{e2} : nat  ]]$, thus $  \kleq{[[ cc{e1} ]]}{ [[cc{e2} ]]}    $ by \cref{lemma:ade}.
\end{proof}


Armed with \cref{thm:co-log} and \cref{thm:log-sound}, coherence follows directly.
\coherence*
\begin{proof}
  Immediate from \cref{thm:co-log} and \cref{thm:log-sound}.
\end{proof}

\section{Some Interesting Corollaries}

To showcase the strength of the new proof method, we can derive some
interesting corollaries. For the most part, they are direct consequences of
logical equivalence which carry over to contextual equivalence.


\cref{lemma:neutral} says that merging an expression $[[ee1]]$ of some type with
an arbitrary expression $[[ee2]]$ does not affect the semantics of $[[ee1]]$ at
the same type. \cref{lemma:commu} and \cref{lemma:assoc} express that merges are
commutative and associative, respectively. \cref{lemma:coercion_same} states
that coercions from the same types are ``coherent'', i.e., they can be used
interchangeably.

\begin{corollary}[Neutrality] \label{lemma:neutral}
  If $[[GG |- ee1 => A ]]$ and $[[GG |- ee1 ,, ee2 => A ]]$, then
  $[[GG |- ee1 ~= ee1 ,, ee2 : A]]$
\end{corollary}

\begin{corollary}[Commutativity] \label{lemma:commu}
  If $[[GG |- ee1 ,, ee2 => A ]]$ and $[[GG |- ee2 ,, ee1 => A ]]$, then
  $[[GG |- ee1 ,, ee2 ~= ee2 ,, ee1 : A]]$.
\end{corollary}


\begin{corollary}[Associativity] \label{lemma:assoc}
  If $[[GG |- (ee1 ,, ee2) ,, ee3 => A  ]]$ and $[[GG |- ee1 ,, (ee2 ,, ee3) => A ]]$, then
  $[[GG |- (ee1 ,, ee2) ,, ee3 ~= ee1 ,, (ee2 ,, ee3) : A]]$.
\end{corollary}

\begin{corollary}[Coercions Preserve Semantics]
  \label{lemma:coercion_same}
  If $[[A <: B ~~> c1]]$ and $[[A <: B ~~> c2]]$, then $[[GG |- \x . c1 x == \x . c2 x :  A  ->  B ]]$.
\end{corollary}


% Local Variables:
% TeX-master: "../../Thesis"
% org-ref-default-bibliography: ../../Thesis.bib
% End:


%%%%%%%%%%%%%%%%%%%%%%%%%%%%%%%%%%%%%%%%%%%%%%%%%%%%%%%%%%%%%%%%%%%%%%%%
\chapter{Coherence for \fnamee}
\label{chap:coherence:poly}
%%%%%%%%%%%%%%%%%%%%%%%%%%%%%%%%%%%%%%%%%%%%%%%%%%%%%%%%%%%%%%%%%%%%%%%%

In this chapter, we extend the proof technique introduced in
\cref{chap:coherence:simple} to prove coherence for \fnamee. Firstly we review
the parametric logical relation for System F~\citep{reynolds1983types} in
\cref{sec:para:lr}. We then talk about a failed attempt on a natural extension
to deal with disjoint intersection types in \cref{sec:failed:lr}. The technical
difficulty is \textit{well-foundedness}, stemming from the interaction between
impredicativity and disjointness. Finally in \cref{sec:succeed:lr}, we present
our (predicative) logical relation that is specially crafted to prove coherence
for \fnamee, and talk about a potential solution to lift the predicativity
restriction.


\section{Review of the Parametric Logical Relation}
\label{sec:para:lr}

\begin{figure}[t]
  \centering
  \begin{tabular}{lll}
  $[[(v1 , v2) in V ( X ) with pq ]]$  &$\defeq$ & $ ([[v1]], [[v2]]) \in [[pq]]([[X]])   $ \\
  $[[(v1 , v2) in V ( nat ) with pq ]]$  &$\defeq$ & $\exists [[i]].\, [[v1]] = [[v2]] = [[ii]]$ \\
  $[[(v1 , v2) in V ( T1 -> T2 ) with pq ]]$  & $\defeq$ & $\forall [[(v'1, v'2) in V (T1) with pq  ]].\, [[  (v1 v1' , v2 v2') in E (T2) with pq   ]]$ \\
  $[[(v1 , v2) in V ( T1 * T2 ) with pq ]]$  & $\defeq$ & $[[( pp1 v1, pp1 v2 ) in E (T1) with pq ]]  \land [[ (pp2 v1, pp2 v2) in E (T2) with pq ]]   $ \\
  $[[(v1 , v2) in V ( \X . T ) with pq ]]$  & $\defeq$ & $ \forall [[T1]], [[T2]], [[R]] \subseteq [[T1]] \times [[T2]].\, [[ (v1 T1, v2 T2) in E (T) with pq [X -> R ]     ]]  $ \\ \\
  $[[(e1, e2) in E (T) with pq ]]$ & $\defeq$ & $\exists [[v1]], [[v2]].\, [[e1 -->> v1]] \land [[e2 -->> v2]] \land [[(v1, v2) in V (T) with pq ]]$
  \end{tabular}
  \caption{A logical relation for System F}
  \label{fig:logical:f}
\end{figure}

System F provides a reasoning principle called \textit{relational
  parametricity}~\citep{reynolds1983types} for establishing when two expressions
of the same type have identical behavior. The principle is expressed in terms of
a logical relation. It is well-known that the logical relation of System F
induces the \textit{abstraction theorem} (also called the \textit{parametricity
  theorem}) of \citet{reynolds1983types}, which roughly says that every
well-typed expression behaves the same as itself according to its type. The
stringency of parametricity ensures that a polymorphic type has very few
inhabitants, so few that we can deduce the behavior of a program by just looking
at its type, which \citet{wadler1989theorems} christened \textit{theorems for free}.

Consider an expression $[[e]]$ of type $[[ \X . X -> nat ]]$. The following
``free'' theorem completely determines the behavior of $[[e]]$ and any other
function of the same type --- i.e., all expressions with this type have to be constant
functions.

\begin{theorem}[A free theorem of $[[ \X . X -> nat ]]$] \label{thm:free}
  Suppose $[[e]]$ is any expression of type $[[\X . X -> nat]]$. Let $[[T1]]$
  and $[[T2]]$ be arbitrary types. For any $[[v1]]$ of type $[[T1]]$ and
  $[[v2]]$ of type $[[T2]]$, if $[[e T1 v1 -->> v3]]$ and $[[e T2 v2 -->> v4]]$ then $[[v3]] = [[v4]]$.
\end{theorem}


\Cref{fig:logical:f} defines the logical relation for call-by-value System F.
Compared with the logical relation for simple types, it is parameterized by a
relational mapping $[[pq]]$, which plays a central role for type abstractions
and type variables. Intuitively, for two values $[[v1]]$ and $[[v2]]$ of type
$[[\X . T]]$ to behave the same, their instantiations must behave the same.
However, because $[[v1]]$ and $[[v2]]$ do not manipulate the values of type
$[[X]]$, it is not required that they should behave the same at the
\textit{same} instantiation. Indeed, as we saw in \cref{thm:free}, we shall be
able to consider \textit{separately} instances of $[[v1]]$ and $[[v2]]$ by types
$[[T1]]$ and $[[T2]]$, and treat the type variable $[[X]]$ as standing for any
relation $[[R]]$ between $[[T1]]$ and $[[T2]]$. Now it is apparent that the
relational mapping $[[pq]]$ maps each type variable $[[X]]$ to a relation
$[[R]]$ that says how values at type $[[X]]$ should be related. This explains
why for type variables, values of type $[[X]]$ are related according to the
relation associated with $[[X]]$ in the mapping $[[pq]]$.



\section{Impredicativity and Disjointness at Odds}
\label{sec:failed:lr}

Based on the parametric logical relation of System F, it is not hard to come up
with a heterogeneous version, similar to \cref{fig:logical}, but with a
relational mapping.
\begin{align*}
 [[(v1, v2)  in V(A1 -> B1; A2 -> B2) with pq ]] &\defeq \forall [[(v'1, v'2) in V (A1; A2) with pq  ]].\, [[  (v1 v1' , v2 v2') in E (B1 ; B2) with pq   ]] \\
  [[( < v1 , v2 > , v3  )  in V ( A & B ;  C  ) with pq ]]  &\defeq [[ (v1, v3)  in V (A ; C) with pq ]] \land [[ (v2, v3)  in V (B ; C) with pq ]] \\
  [[( v3 , < v1 , v2 >  )  in V ( C ; A & B  ) with pq ]]  &\defeq [[ (v3, v1)  in V (C ; A) with pq ]] \land [[ (v3, v2)  in V (C ; B) with pq ]] \\
  [[(v1, v2)  in V(X; X) with pq ]] &\defeq ([[v1]], [[v2]]) \in [[pq]]([[X]])
\end{align*}
The first three cases do not manipulate $[[pq]]$ and are directly ported from
\cref{fig:logical}. The case for type variables is directly copied from
\cref{fig:logical:f}. For type abstractions, we follow the definition for System
F, but also take care of disjointness constraints, as shown below.
\begin{align*}
  [[(v1, v2)  in V(\X ** A1 . B1; \X ** A2 . B2) with pq ]] &\defeq \forall [[empty |- C1 ** pq(A1 & A2) ]], [[empty |- C2 ** pq(A1 & A2)]], \\
                                                            & \quad [[R]] \subseteq [[C1]] \times [[C2]]. \\
                                                            & \quad [[ ( v1 |C1|, v2 |C2|) in E (B1 ; B2) with pq[X -> R]    ]]
\end{align*}
Compared with the one for System F, here we cannot pick arbitrary two types
except those that respect the disjointness constraints, due to the typing rule
for type applications (\rref{FT-tapp}). Let us see an example in action to get a
taste of how this definition works. Suppose $[[e]]$ is any expression
corresponding to type $[[ \X ** bool . \Y ** X . X & Y -> Y & X ]]$, for any
integer $[[v1]]$ and character $[[v2]]$, it can be shown that
\[
  [[e nat char <v1 , v2> -->> <v2 , v1>   ]]
\]
First, we have $[[ (e, e) in E( \X ** bool . \Y ** X . X & Y -> Y & X ) ]]$.
Choose $\mathsf{R}_1 = \{ ([[v1]], [[v1]]) \}$ and $\mathsf{R}_2 = \{ ([[v2]], [[v2]]) \}$.
This is allowed because $[[nat]]$ and $[[char]]$ both respect the
disjointness constraints. Suppose $[[ e nat char -->> v  ]]$ for some $[[v]]$ and we have
\[
  [[(v, v) in V (X & Y -> Y & X) with emp[X -> R1][Y -> R2] ]]
\]
Note that the input $[[  <v1, v2>  ]]$ is related to itself at $[[ X & Y   ]]$
\[
  [[ (<v1, v2>, <v1, v2>) in V (X & Y) with emp[X -> R1][Y -> R2] ]]
\]
So the outputs of $[[v]]$ are related at $[[ Y & X ]]$
\[
  [[ (v <v1, v2>, v <v1, v2>) in E (Y & X) with emp[X -> R1][Y -> R2] ]]
\]
Suppose $[[  v <v1, v2> -->> <v3 , v4>  ]]$, we have
\[
  [[ (<v3, v4>, <v3, v4>) in V (Y & X) with emp[X -> R1][Y -> R2] ]]
\]
We have $([[v3]], [[v3]]) \in \mathsf{R}_2$ and $([[v4]], [[v4]]) \in \mathsf{R}_1$, which means
$[[v3]] = [[v2]]$ and $[[v4]] = [[v1]]$. So we have shown
\[
  [[ e nat char <v1, v2> -->> v <v1, v2> ]] [[-->>]] [[  <v2, v1>  ]]
\]

Heterogeneity forces us to consider the case for
$\valR{[[X]]}{[[A]]}$ where $[[A]] \neq [[X]]$. A seemingly innocuous definition
is as follows:
\begin{align*}
  [[(v1, v2)  in V(X; A) with pq ]] &\defeq [[ (v1, v2) in V(pq(X); pq(A)) with emp  ]]
\end{align*}
That is, given $[[v1]]$ of type $[[X]]$ and $[[v2]]$ of type $[[A]]$, they are
related at $[[X]]$ and $[[A]]$ if and only if they are related at $[[pq(X)]]$
and $[[pq(A)]]$.
% \footnote{Here we are abusing the notation with the intention
%   that $[[pq(X)]]$ means the assigned type of $[[X]]$ in $[[pq]]$; and
%   $[[pq(A)]]$ means substituting all the variables mentioned in $[[pq]]$ with
%   the corresponding types in $[[A]]$.}
Let us see another example to motivate this definition. Suppose
\[
  [[ee]] = [[\ X ** bool . (\x . x) : X & nat -> X & nat]]
\]
we should expect the following to
hold:\footnote{The reader is advised to try it out in our prototype interpreter.}
\[
  [[ee nat 1 -->> <1 , 1> ]]
\]
It boils down to verifying the following:
\[
  [[  (1 , 1) in V (X ; nat) with pq  ]]
\]
where the mapping $[[pq]]$ maps $[[X]]$ to a relation over integers. If we
replace $[[X]]$ with $[[nat]]$ then the above holds obviously.


Unfortunately, the seemingly innocuous definition for type variables has a serious
issue with impredicativity --- in other words, the relation in question is \textit{not well-founded}.
We could pick a bad instantiation to make the relation ``go
into a loop''. For example, suppose $[[pq]]$ only contains a mapping from $[[X]]$ to
$[[\X . \Y . Y]]$, we then have the following infinite chain of equational reasoning:
\[
 \valR{[[X]]}{[[\Y . Y]]}_{[[pq]]} = \valR{[[\X . \Y . Y]]}{[[\Y. Y]]}_{[[emp]]} = \eeR{[[\Y . Y]]}{[[ X ]]}_{[[pq]]} = \valR{[[\Y . Y]]}{[[ X ]]}_{[[pq]]} = \dots
\]
The culprit is the polymorphic instantiation $[[\X . \Y . Y]]$, which is larger
than $[[\Y . Y]]$. This is the exact circumstance where in \cref{fig:logical:f}
substitution is deliberately avoided for type abstractions.

\section{Predicative Logical Relation}
\label{sec:succeed:lr}

In light of the fact that substitution in the logical relation seems unavoidable
in our setting, and that impredicativity conflicts with substitution, we turn
to, for the lack of a better logical relation, \textit{predicativity}. The
restriction to predicativity, though reducing the expressiveness in theory, does
not cost much in practice. Languages based on the Hindley–Milner type
system~\citep{milner1978theory, hindley1969principal}, such as Haskell and ML,
have such restriction. We also plan to study a variant of \fnamee with implicit polymorphism,
as briefly discussed in \cref{sec:implicit}.

\begin{figure}
  \centering
  \begin{small}
  \begin{tabular}{lll}
  $[[(v1 , v2) in V ( nat ; nat ) ]]$  & $\defeq$ & $\exists [[i]].\, [[v1]] = [[v2]] = [[ii]]$ \\
  $[[(v1, v2) in V ( {l : A}  ; {l : B} ) ]]$ & $\defeq$ & $[[ (v1, v2) in V ( A ; B ) ]]$\\
  $[[(v1 , v2) in V ( A1 -> B1 ; A2 -> B2 ) ]]$  & $\defeq$ & $\forall [[(v2' , v1') in V ( A2 ; A1 ) ]].\, [[ (v1 v1' , v2 v2') in E ( B1 ; B2 ) ]]$ \\
    $\hlmath{[[(v1, v2)  in V ( \ X ** A1 . B1; \ X ** A2 . B2 ) ]]}$  &$\defeq$ & $\hlmath{\forall [[empty |- t ** A1 & A2 ]].}$ \\
                                       && $\hlmath{[[  (v1 |t| , v2 |t|) in E ( [t / X] B1 ; [t / X] B2) ]]}$ \\
  $[[( < v1 , v2 > , v3  )  in V ( A & B ;  C  ) ]]$  & $\defeq$ & $[[ (v1, v3)  in V (A ; C) ]] \land [[ (v2, v3)  in V (B ; C) ]]$  \\
  $[[( v3 , < v1 , v2 >  )  in V ( C; A & B  ) ]]$  & $\defeq$ & $[[ (v3, v1)  in V (C ; A) ]] \land [[ (v3, v2)  in V (C ; B) ]]$  \\
  $[[(v1 , v2) in V (A; B) ]]$  & $\defeq$ & $\mathsf{true} \quad \text{otherwise}$ \\ \\
    $[[(e1, e2) in E (A; B)]]$ & $\defeq$ & $\exists [[v1]], [[v2]].\, [[e1 -->> v1]] \land [[e2 -->> v2]] \ \land $ \\
                                       & & $[[(v1, v2) in V (A; B)]]$
  \end{tabular}
  \end{small}
  \caption{Logical relation for \fnamee}
  \label{fig:logical:fi}
\end{figure}

Substitution with predicative instantiations does not prevent well-foundedness.
\Cref{fig:logical:fi} presents the logical relation for \fnamee. We extend the
logical relation in \cref{fig:logical} with a new clause (highlighted) for
disjoint quantification. Notice that it does not quantify over arbitrary
relations, which means that our logical relation \textit{does not} imply
parametricity, and in particular, \cref{thm:free} is impossible to prove using
our logical relation. However, it suffices for our purposes to prove coherence.
Also notice that we directly substitute $[[X]]$ with $[[t]]$ in both $[[B1]]$
and $[[B2]]$. It can be shown that the relation is well-founded.

\begin{lemma}[Well-foundedness]
  The logical relation as defined in \cref{fig:logical:fi} is well-founded.
\end{lemma}
\begin{proof}
  Let $| \cdot |_{\forall}$ and $| \cdot |_s$ denote the number of
  $\forall$-quantifies and the size of types, respectively. The following
  measure suffices to prove well-foundedness:
  \[
\langle | \cdot |_{\forall} ,  | \cdot |_s   \rangle
  \]
  where $\langle \dots \rangle$ denotes lexicographic order. We can verify that
  for the clause of disjoint quantification, the number of $\forall$-quantifies
  decreases, because the monotype $[[t]]$ does not contain $\forall$-quantifies.
  For the other clauses, either the measure of $| \cdot |_{\forall}$ decreases,
  or it remains the same but the measure of $| \cdot |_s$ decreases.
\end{proof}


Another noticeable thing is that in the value relation $ \valR{[[A]]}{[[B]]} $
(and $\eeR{[[A]]}{[[B]]} $), we keep the invariant that $[[A]]$ and $[[B]]$
are closed types. This is why we do not need to consider type variables in the
logical relation, which simplifies the proof a lot. We show that the logical
relation is symmetric.


\begin{lemma}[Symmetry of logical relation]
  If $[[ (v1, v2) in V ( A ; B ) ]]$ then $[[ (v2, v1) in V ( B ; A ) ]]$.
\end{lemma}
\begin{proof}
  Symmetry of \cref{fig:logical} is trivial. For \cref{fig:logical:fi}, the
  proof proceeds by first induction on $ | [[A]] |_{\forall} $ then simultaneous
  induction on the structures of $[[A]]$ and $[[B]]$.
\end{proof}

\section{Establishing Coherence}

We are now ready to prove coherence for \fnamee. The proof of coherence
basically follows that in \cref{chap:coherence:simple}. We first give the
interpretations of type and value contexts ($[[p]]$ is a mapping from type
variables to monotypes).

\begin{definition}[Interpretation of type contexts]
  \begin{mathpar}
    \ottaltinferrule{}{}{  }{ [[emp in empty]] } \and
    \ottaltinferrule{}{}{ [[p in DD]] \\ [[empty |- t ** p(B)]] \\  }{ [[p [ X -> t ] in DD , X ** B]]  }
  \end{mathpar}
\end{definition}


\begin{definition}[Interpretation of value contexts]
  \begin{mathpar}
    \ottaltinferrule{}{}{  }{ [[(emp, emp) in empty with p ]]  } \and
    \ottaltinferrule{}{}{ [[(g1, g2) in GG with p ]] \\ [[(v1, v2) in V (p(A)) ]] }{ [[(g1 [ x -> v1 ] , g2 [ x -> v2 ]  )  in GG , x : A with p ]] }
  \end{mathpar}
\end{definition}


\paragraph{Logical Equivalence.}

Logical equivalence is defined in terms of logical relation by considering all possible interpretations of
free type and term variables.

\begin{definition}[Logical equivalence]
  \begin{align*}
    [[DD ; GG |- e1 == e2 : A ; B]]  & \defeq  [[|DD| ; |GG| |- e1 : |A|]] \land [[ |DD | ; |GG| |- e2 : | B | ]] \ \land \\
                                       & \qquad (\forall [[p]], [[g1]], [[g2]]. \ ([[p in DD]] \land [[(g1, g2) in GG with p ]]) \\
                                       & \qquad \Longrightarrow [[(g1 (p1 (e1)), g2 (p2 (e2)))  in E (p(A) ; p(B)) ]])
  \end{align*}
\end{definition}


\paragraph{Contextual Equivalence.}

\begin{figure}
  \centering
\begin{tabular}{llll}\toprule
  \tnamee contexts & $[[cc]]$ & $\Coloneqq$ &  $[[__]] \mid [[\ x . cc]] \mid \hlmath{[[\ X . cc]]}  \mid \hlmath{[[ cc T  ]]} \mid [[cc e]] \mid [[e cc]] \mid [[< cc , e>]] \mid [[<e , cc>]] \mid [[c cc]] $ \\
  \fnamee contexts & $[[CC]]$ & $\Coloneqq$ &  $[[__]] \mid [[\ x . CC]] \mid \hlmath{[[\ X ** A. CC]]} \mid \hlmath{[[ CC A  ]]} \mid [[CC ee]] \mid [[ee CC]] \mid [[ CC ,, ee  ]] \mid [[ ee ,, CC  ]]  $ \\
  & & $\mid$ & $[[ { l = CC}  ]]  \mid [[ CC . l]]   $  \\ \bottomrule
\end{tabular}
  \caption{Expression contexts of \tnamee and \fnamee}
  \label{fig:contexts:fi}
\end{figure}


To define the contextual equivalence, we must define the expression contexts for
\fnamee and \tnamee, which are shown in \cref{fig:contexts:fi} (highlighted for
the differences from \namee contexts). The typing judgment of
\fnamee contexts has 4 different forms:
\begin{mathpar}
  [[CC : (DD; GG => A) ~> (DD'; GG' => A') ~~> cc]] \and
  [[CC : (DD; GG <= A) ~> (DD'; GG' => A') ~~> cc]] \and
  [[CC : (DD; GG => A) ~> (DD'; GG' <= A') ~~> cc]] \and
  [[CC : (DD; GG <= A) ~> (DD'; GG' <= A') ~~> cc]]
\end{mathpar}
The full typing rules are similar to \cref{fig:ctyp} and appear in \cref{appendix:fi}.
Now we may give the definition of contextual equivalence for \fname as follows:

\begin{definition}[\fnamee Contextual Equivalence]
  \begin{align*}
    [[DD ; GG |- ee1 ~= ee2 : A]]  & \defeq \forall [[e1]], [[e2]].\  [[DD ; GG |- ee1 => A ~~> e1]] \land [[DD ; GG |- ee2 => A ~~> e2]] \ \land   \\
                                     & \qquad (\forall [[C]], [[cc]].\ [[CC : (DD; GG => A) ~> (empty ; empty => nat) ~~> cc]]   \\
                                     & \qquad \Longrightarrow \kleq{[[cc{e1}]]}{[[cc{e2}]]})
  \end{align*}
\end{definition}


The connection between disjointness and the value relation becomes a bit
complicated due to the addition of type variables. We first prove that disjoint
values of monotypes are related.


\begin{lemma}[Disjoint values of monotypes are related] \label{lemma:disjoint:mono}
  If $[[empty |- t1 ** t2]]$,
  $[[  empty ; empty |-  v1 : |t1|  ]]$ and
  $[[  empty ; empty |-  v2 : |t2|  ]]$
  then $[[   (v1, v2) in V ( t1 ; t2  )    ]]$.
\end{lemma}
\begin{proof}
  By simultaneous induction on $[[t1]]$ and $[[t2]]$.
\end{proof}

Then we can prove a more general lemma.

\begin{lemma}[Disjoint values are related]
  If $[[DD |- A ** B]]$, $[[ p in DD  ]]$, $[[  empty ; empty |-  v1 : |p (A)|  ]]$ and $[[  empty ; empty |-  v2 : |p (B)|  ]]$
  then $[[   (v1, v2) in V ( p(A) ; p(B)  )    ]]$.
\end{lemma}
\begin{proof}
  By induction on the derivation of disjointness. The most interesting case is the variable rule:
  \[
    \drule{FD-tvarL}
  \]
  By the definition of $[[p]]$, we know $[[p(X)]]$ is a monotype. If $[[B]]$ is
  a polytype, then it follows easily from the definition of logical relation. If
  $[[B]]$ is also a monotype, we know $[[p(X)]]$ and $[[p(A)]]$ are disjoint by
  definition. Then by \cref{lemma:covariance:disjoint} and $[[A <: B]]$,
  we have $[[p(X)]]$ and $[[p(B)]]$ are also disjoint. Finally we apply
  \cref{lemma:disjoint:mono}.
\end{proof}


Next up are the compatibility lemmas. They are essentially the same as in
\cref{chap:coherence:simple}. So we skip them and state the last two important
theorems without giving proofs. The proofs have mostly the same structures as in the
simply-typed case, but with more cases. The interested reader can refer to our
Coq formalization.

\begin{theorem}[Fundamental Property] We have that:
  \begin{itemize}
  \item If $[[DD; GG |- ee => A ~~> e]]$ and $[[DD; GG |- ee => A ~~> e']]$, then $[[DD; GG |- e == e' : A ]]$.
  \item If $[[DD ; GG |- ee <= A ~~> e]]$ and $[[DD ; GG |- ee <= A ~~> e']]$, then $[[DD; GG |- e == e' : A ]]$.
  \end{itemize}
\end{theorem}

\begin{theorem}[Congruence]
 If $[[CC : (DD ; GG dir A) ~> (DD' ; GG' dir' A') ~~> cc]]$, $[[DD ; GG |- ee1 dir A ~~> e1]]$, $[[DD ; GG |- ee2 dir A ~~> e2]]$
 and $[[DD ; GG |- e1 == e2 : A ]]$, then $[[DD' ; GG' |- cc{e1} == cc{e2} : A']]$.
\end{theorem}

Finally \fnamee is coherent.

\begin{theorem}[Coherence of \fnamee] We have that
  \begin{itemize}
  \item If $[[DD ; GG |- ee => A ]]$ then $[[DD ; GG |- ee ~= ee : A]]$.
  \item If $[[DD ; GG |- ee <= A ]]$ then $[[DD ; GG |- ee ~= ee : A]]$.
  \end{itemize}
\end{theorem}


\paragraph{Final remarks.}

It would be interesting to study parametricity of \fnamee. As we have seen, it
is not obvious how to extend the parametric logical relation as defined in
\cref{fig:logical:f} to account for disjointness, and avoid potential
circularity due to impredicativity. A possible solution is to use step-indexed
logical relations. We have yet investigated further on that direction.

% Local Variables:
% TeX-master: "../../Thesis"
% org-ref-default-bibliography: ../../Thesis.bib
% End:


%%% Local Variables:
%%% mode: latex
%%% TeX-master: "../Thesis"
%%% End:




  \part{Applications}

  %%%%%%%%%%%%%%%%%%%%%%%%%%%%%%%%%%%%%%%%%%%%%%%%%%%%%%%%%%%%%%%%%%%%%%%%
\chapter{First-Class Traits}
\label{chap:traits}
%%%%%%%%%%%%%%%%%%%%%%%%%%%%%%%%%%%%%%%%%%%%%%%%%%%%%%%%%%%%%%%%%%%%%%%%

\renewcommand\ottaltinferrule[4]{
  \inferrule*[narrower=0.9,lab=#1,#2]
    {#3}
    {#4}
}

In this chapter and \cref{chap:case_study}, we present two applications of
\fnamee. This chapter is concerned with building a source-level language named
\sedel that features \textit{typed first-class traits}. In particular we show
how to model source-level constructs for first-class traits and dynamic
inheritance, supporting standard object-oriented features such as dynamic
dispatching and abstract methods. It is remarkable that all of these can be
explained by plain \fnamee expressions, showing its expressive power.
In \cref{chap:case_study} we conduct a case study of modularizing programming
language features by the means of first-class traits.



\section{Motivation: First-Class Classes and Dynamic Inheritance}

Many dynamically typed-languages (including JavaScript, Ruby, Python
or Racket) support \emph{first-class classes}~\citep{DBLP:conf/aplas/FlattFF06}, or related concepts
such as first-class mixins and/or traits. In those languages classes
are first-class values and, like any other values, they can be
passed as an argument, or returned from a function. Furthermore
first-class classes support \emph{dynamic inheritance}: i.e., they
can inherit from other classes at \emph{run time}, enabling
programmers to abstract over the inheritance hierarchy.
Those features make first-class classes very powerful and expressive,
and enable highly modular and reusable pieces of code, such as:
\begin{lstlisting}[language=JavaScript]
const mixin = Base => {
  return class extends Base { ... }
};
\end{lstlisting}
In this piece of JavaScript code, \lstinline{mixin} is
parameterized by a class \lstinline{Base}. Note that the concrete
implementation of \lstinline{Base} can be
even dynamically determined at run time, for example
after reading a configuration file to decide which
class to use as the base class.  When applied to an argument,
\lstinline{mixin} will create a new class on-the-fly and return that
as a result. Later that class can be instantiated and used to create
new objects, as any other classes.

In contrast, most statically-typed
languages do not have first-class classes and dynamic
inheritance. While all statically-typed OO languages allow first-class
\emph{objects} (i.e., objects can be passed as arguments and returned
as results), the same is not true for classes. Classes in languages such as
Scala, Java or C++ are typically a second-class construct, and the
inheritance hierarchy is \emph{statically determined}. The closest thing
to first-class classes in
languages like Java or Scala are classes such as
\lstinline[language=java]{java.lang.Class} that enable representing classes and
interfaces as part of their reflective framework. \lstinline[language=java]{java.lang.Class} can be used to
mimic some of the uses of first-class classes, but in an essentially
dynamically-typed way. Furthermore simulating first-class classes
using such mechanisms is highly cumbersome because classes need to be
manipulated programmatically. For example instantiating a new class
cannot be done using the standard \lstinline{new} construct, but
rather requires going through API methods of
\lstinline[language=java]{java.lang.Class}, such as \lstinline{newInstance}, for
creating a new instance of a class.

Despite the popularity and expressive power of first-class classes in dynamically-typed
languages, there is surprisingly little work on typing of first-class
classes (or related concepts such as first-class mixins or traits).
First-class classes and dynamic inheritance pose well-known
difficulties in terms of typing. For example, in his thesis,
\citet{bracha1992programming} comments several times on the difficulties of typing
dynamic inheritance and first-class mixins, and proposes the
restriction to static inheritance that is also common in
statically-typed languages. He also observes that such restriction
poses severe limitations in terms of expressiveness, but that appeared
(at the time)
to be a necessary compromise when typing was also desired.
Only recently some progress has been made in statically typing
first-class classes and dynamic inheritance. In particular there are
two works in this area: Racket's gradually
typed first-class classes~\citep{DBLP:conf/oopsla/TakikawaSDTF12}; and \citeauthor{DBLP:conf/ecoop/LeeASP15}'s model of
typed first-class classes~\citep{DBLP:conf/ecoop/LeeASP15}. Both works provide typed models of
first-class classes, and they enable encodings of mixins~\citep{bracha1990mixin}
similar to those employed in dynamically-typed languages.

However, as far as we known no previous work supports statically-typed
\emph{first-class traits}. Traits~\citep{scharli2003traits, Ducasse_2006} are an
alternative to mixins, and other models of (multiple) inheritance. The key
difference between traits and mixins lies on the treatment of conflicts when
composing multiple traits/mixins. Mixins adopt an \emph{implicit} resolution
strategy for conflicts, where the compiler automatically picks one
implementation in case of conflicts. For example, Scala uses the order of mixin
composition to determine which implementation to pick in case of conflicts.
Traits, on the other hand, employ an \emph{explicit} resolution strategy, where
the compositions with conflicts are rejected, and the conflicts are explicitly
resolved by programmers. This gives programmers fine-grained control, when
conflicts arise, of selecting desired features from different components. Thus
we believe traits are a better model for multiple inheritance in
statically-typed object-oriented languages. In what follows, we present \sedel:
the first design of typed first-class traits.




\section{Overview}
\label{sec:trait:overview}

This section aims at introducing first-class classes and traits, their possible
uses and applications, as well as the typing challenges that arise
from their use.
We start by describing a hypothetical JavaScript library for text editing
widgets, inspired and adapted from Racket's GUI
toolkit~\citep{DBLP:conf/oopsla/TakikawaSDTF12}. The example is illustrative of
typical uses of dynamic inheritance/composition, and also the typing challenges
in the presence of first-class classes/traits. Without diving into
technical details, we then give the corresponding typed version in
\sedel, and informally presents its salient features.

\subsection{First-Class Classes in JavaScript}

A class construct was officially added to JavaScript in the ECMAScript
2015 Language Specification~\citep{EcmaScript:15}. One purpose of
adding classes to JavaScript was to support a construct that is more
familiar to programmers who come from mainstream class-based languages,
such as Java or C++. However classes in JavaScript are
\emph{first-class} and support functionality not easily mimicked in
statically-typed class-based languages.

\paragraph{Conventional Classes.}

Before diving into the more advanced features of JavaScript classes, we first
review the more conventional class declarations supported in JavaScript as well
as many other languages. Even for conventional classes there are some
interesting points to note about JavaScript that will be important when we move
into a typed setting. An example of a JavaScript class declaration is:
\begin{lstlisting}[language=JavaScript]
class Editor {
  onKey(key) {
    return "Pressing " + key;
  }
  doCut() {
    return this.onKey("C-x") + " for cutting text";
  }
  showHelp() {
    return "Version: " + this.version() + " Basic usage...";
  }
};
\end{lstlisting}
This form of class definition is standard and very similar to declarations in
class-based languages (for example Java). The \lstinline{Editor} class
defines three methods: \lstinline{onKey} for handling key events,
\lstinline{doCut} for cutting text and \lstinline{showHelp} for displaying help
message. For the purpose of demonstration, we elide the actual implementation,
and replace it with plain messages.

We wish to bring the readers' attention to two points in the above class.
Firstly, note that the \lstinline{doCut} method is defined in terms of the
\lstinline{onKey} method via the keyword
\lstinline[language=JavaScript]{this}. In other words the call to
\lstinline{onKey} is enabled by the \emph{self} reference and is
\emph{dynamically dispatched} (i.e., the particular implementation of
\lstinline{onKey} will only be determined when the class or subclass
is instantiated). % Typically an
% OO programmer seeing this definition would expect the \lstinline{doCut} method
% to call the \lstinline{onKey} method of a subclass of \lstinline{Editor}, even though
% the subclass does not exist when the superclass \lstinline{Editor} is being
% defined.
Secondly, notice that there is no definition of
the \lstinline{version} method in the class body, but such method is used in the body of the
\lstinline{showHelp} method. In an untyped language, such as JavaScript, using
undefined methods is error prone---accidentally instantiating \lstinline{Editor}
and then calling \lstinline{showHelp} will cause a run-time error!
Statically-typed languages usually provide some means to protect us from this
situation. For example, in Java, we would need an \textit{abstract} \lstinline{version}
method, which effectively makes \lstinline{Editor} an abstract class and
prevents it from being instantiated. As we will see, \sedel's treatment of
abstract methods is quite different from mainstream languages. In fact, \sedel
has a unified (typing) mechanism for dealing with both dynamic dispatch and abstract
methods. We will describe \sedel's mechanism for dealing with both features and
justify our design in \cref{sec:traits}.

% A couple of things worth pointing out in the above code snippet: (1) the class
% \lstinline{Editor} has no definition of the method
% \lstinline{version}, but such method
% is used in the body of the method \lstinline{showHelp}. In a strongly-typed OO
% language, such as Java, we would need to define an abstract method for
% \lstinline{version}. (2) The \lstinline{Editor} class requires
% \emph{dynamic dispatching}.
%  In the body of the method \lstinline{doCut} we invoke
% the method \lstinline{onKey} defined in the same class through the keyword
% \lstinline[language=JavaScript]{this}. This has the implication that when a
% subclass of \lstinline{Editor} overrides the method \lstinline{onKey}, a call to
% \lstinline{doCut} should invoke \lstinline{onKey} defined in the subclass
% instead of the original one.\bruno{punchline?}
%As we will see later, the type system of \sedel correctly handles it.

\paragraph{First-Class Classes and Class Expressions.}

Another way to define a class in JavaScript is via a \emph{class expression}. This is where the class
model in JavaScript is very different from the traditional class model found in
many mainstream OO languages, such as Java, where classes are second-class
(static) entities. JavaScript embraces a dynamic class model that treats classes
as \emph{first-class} expressions: a function can take classes as arguments,
or return them as a result. First-class classes enable programmers to
abstract over patterns in the class hierarchy and to experiment with new forms of OOP
such as mixins and traits. In particular, mixins become programmer-defined
constructs. We illustrate this by presenting a simple mixin that adds
spell checking to an editor:
\begin{lstlisting}[language=JavaScript]
const spellMixin = Base => {
  return class extends Base {
    check() {
      return super.onKey("C-c") + " for spell checking";
    }
    onKey(key) {
      return "Process " + key + " on spell editor";
    }
  }
};
\end{lstlisting}

\paragraph{Dynamic Inheritance.}

In JavaScript, a mixin is simply a function with a superclass as input and a
subclass extending that superclass as an output. Concretely, \lstinline{spellMixin}
adds a method \lstinline{check} for spell checking. It also provides
a method \lstinline{onKey}.
The function \lstinline{spellMixin} shows the typical use of what we call \emph{dynamic inheritance}.
Note that \lstinline{Base}, which is supposed to be a superclass being inherited, is \emph{parameterized}.
Therefore \lstinline{spellMixin} can be applied to any base class at
\emph{run time}. This is impossible to do, in a type-safe way, in
conventional statically-typed class-based languages like Java or
C++.\footnote{With C++ templates, it is possible to
  implement a so-called mixin pattern~\citep{DBLP:conf/gcse/SmaragdakisB00}, which enables extending
a parameterized class. However C++ templates defer type-checking until
instantiation, and such pattern still does not allow selection of the
base class at run time (only at up to class instantiation time).}

It is noteworthy that not all applications of \lstinline{spellMixin} to base
classes are successful. Notice the use of the \lstinline{super} keyword in the
\lstinline{check} method. If the base class does not implement the
\lstinline{onKey} method, then mixin application fails with a run-time error. In
a typed setting, a type system must express this requirement (i.e., the presence of
the \lstinline{onKey} method) on the (statically unknown) base class being inherited.


% The class expression inside the function body has no
% definition of the method \lstinline{version}, but which is used in the body of
% the method \lstinline{showHelp}. In a statically-typed OO language, such as Java,
% we would need an \emph{abstract method} for
% \lstinline{version}.


We invite the readers to pause for a while and think about what the type of
\lstinline{spellMixin} would look like. Clearly our type system should be
flexible enough to express this kind of dynamic pattern of composition in order
to accommodate mixins (or traits), but also not too lenient to allow any
composition.


\paragraph{Mixin Composition and Conflicts.}
The powerful part of mixins is that \lstinline{spellMixin}'s functionality is not
tied to a particular class hierarchy and is composable with other features. For
example, we can define another mixin that adds simple modal editing---as in Vim---to an arbitrary editor:
\begin{lstlisting}[language=JavaScript]
const modalMixin = Base => {
  return class extends Base {
    constructor() {
      super();
      this.mode = "command";
    }
    toggleMode() {
      return "toggle succeeded";
    }
    onKey(key) {
      return "Process " + key + " on modal editor";
    }
  };
};
\end{lstlisting}
\lstinline{modalMixin} adds a \lstinline{mode} field that controls which
keybindings are active, initially set to the command mode, and a method
\lstinline{toggleMode} that is used to switch between modes. It also provides a method \lstinline{onKey}.

Now we can compose \lstinline{spellMixin} with \lstinline{modalMixin} to produce
a combination of functionality, mimicking some form of multiple inheritance:
\begin{lstlisting}[language=JavaScript]
class IDEEditor extends modalMixin(spellMixin(Editor)) {
  version() {
    return 0.2;
  }
}
\end{lstlisting}
The class \lstinline{IDEEditor} extends the base class \lstinline{Editor} with
modal editing and spell checking capabilities. It also defines the missing
\lstinline{version} method.

At first glance, \lstinline{IDEEditor} looks quite fine, but it has a subtle
issue. Recall that two mixins \lstinline{modalMixin} and \lstinline{spellMixin}
both provide a method \lstinline{onKey}, and the \lstinline{Editor} class also
defines an \lstinline{onKey} method of its own. Now we have a name clash. A
question arises as to which one gets picked inside the \lstinline{IDEEditor}
class. A typical mixin model resolves this issue by looking at the order of mixin applications. Mixins appearing later in the order
overrides \emph{all} the identically named methods of earlier mixins. So in our
case, \lstinline{onKey} in \lstinline{modalMixin} gets picked. If we
change the order of application to \lstinline{spellMixin(modalMixin(Editor))},
then \lstinline{onKey} in \lstinline{spellMixin} is inherited.

\paragraph{The Problem of Mixin Composition.}
From the above discussion, we can see that mixin are composed linearly: all the
mixins used by a class must be applied one at a time. However, when we wish to
resolve conflicts by selecting features from different mixins, we may not be
able to find a suitable order. For example, when we compose the two mixins to
make the class \lstinline{IDEEditor}, we can choose which of them comes first,
but in either order, \lstinline{IDEEditor} cannot access to the \lstinline{onKey}
method from the \lstinline{Editor} class.

\paragraph{Trait Model.}
Because of the total ordering and the limited means for resolving conflicts imposed by the mixin model,
researchers have proposed a simple compositional model called
traits~\citep{scharli2003traits, Ducasse_2006}. Traits are lightweight entities and serve as
the primitive units of code reuse. Among others, the key difference from
mixins is that the order of trait composition is irrelevant, and conflicting
methods must be resolved \emph{explicitly}. This gives programmers
fine-grained control, when conflicts arise, of selecting desired features from
different components. Thus we believe traits are a better model for multiple
inheritance in statically-typed object-oriented languages, and in \sedel we realize this
vision by giving traits a first-class status in the language,
achieving more expressive power compared with traditional (second-class) traits.


\paragraph{Summary of Typing Challenges.}
From our previous discussion, we can identify the following typing challenges
for a type system to accommodate the programming patterns (first-class classes/mixins)
we have just seen in a typed setting:
\begin{itemize}
\item How to account for, in a typed way, abstract methods and dynamic dispatch.
\item What are the types of first-class classes or mixins.
\item How to type dynamic inheritance.
\item How to express constraints on method presence and absence (the use of
  \lstinline{super} clearly demands that).
% \item How to ensure that composition of mixins is going to be valid, i.e., how
%   to reflect linearity in a type system.
\item In the presence of first-class traits, how to detect conflicts statically,
  even when the traits involved are not statically known.
\end{itemize}
\sedel elegantly solves the above challenges in a unified way, as
we will see next.


% From a pragmatic point of view, this implicit conflict resolution
% sometimes give programmers more surprises than convenience. What if the compiler can alarm us when a
% potential conflict may occur. Because of the dynamic nature of JavaScript, we
% would not know before actually running the code that there is a conflict. We
% miss the guarantee that a static type system can provide: such conflict can be
% detected at compile-time.

% Given the flexibility of first-class classes in dynamically-typed languages, we
% -- being advocates of statically-typed languages -- were wondering how to
% incorporate this same expressive power into statically-typed
% languages. As it
% turns out, designing a sound type system that fully supports first-class classes
% is notoriously hard; there are only a few, quite sophisticated, languages that
% manage this~\citep{DBLP:conf/oopsla/TakikawaSDTF12, DBLP:conf/ecoop/LeeASP15}. We
% pushed it further: \sedel has support for typed first-class
% traits.\bruno{Better to say there's no work on typed first-class
%   traits, and little work on first-class classes/mixins, despite
%  many dynamic languages prominently supporting such features.}

\subsection{A Glance at Typed First-Class Traits in \sedel}

We now rewrite the above code in \sedel, but this time with types. The resulting code has the same functionality as the dynamic version, but it is
statically typed. All code snippets in this and later sections are runnable in
our prototype. Before proceeding, we ask the reader to bear in mind that in this section we are not using traits
in the most canonical way, i.e., we use traits as if they are classes (but with
built-in conflict detection). This is because we are trying to stay as close as possible
to the structure of the JavaScript version for ease of comparison. In
\cref{sec:traits} we will remedy this to make better use of traits.

\paragraph{Simple Traits.}
Below is a simple trait \lstinline{editor}, which corresponds to the JavaScript
class \lstinline{Editor}. The \lstinline{editor} trait defines the same set of
methods: \lstinline{on_key}, \lstinline{do_cut} and \lstinline{show_help}:
\lstinputlisting[linerange=23-27]{./examples/overview2.sl}% APPLY:linerange=OVERVIEW_EDITOR
The first thing to notice is that \sedel uses a syntax (similar to Scala's
self type annotations~\citep{odersky2004overview}) where we can give a type annotation to the
\lstinline{self} reference. In the type of \lstinline{self} we use
\lstinline{&} construct to create intersection types. \lstinline{Editor} and \lstinline{Version} are two record types:
\lstinputlisting[linerange=10-17]{./examples/overview2.sl}% APPLY:linerange=OVERVIEW_EDITOR_TYPES
For the sake of conciseness, \sedel uses \lstinline{type} aliases to abbreviate types.

\paragraph{The type of \lstinline{self} Encodes Abstract Methods.}
Recall that in the JavaScript class \lstinline{Editor}, the \lstinline{version}
method is undefined, but is used inside \lstinline{showHelp}. How can we express
this in the typed setting, if not with an abstract method? In \sedel, the type of \lstinline{self}
plays the role of trait requirements. As the first approximation, we
can justify the invocation of \lstinline{version} on \lstinline{self} by noticing that (part of) the
type of \lstinline{self} (i.e., \lstinline{Version}) contains the declaration of
\lstinline{version}. An interesting aspect of \sedel's trait model is that there
is no need for abstract methods. Instead, abstract methods can be simulated as
requirements of a trait. Later, when the trait is composed with other
traits, \emph{all} requirements on the type of \lstinline{self} must be
satisfied and one of the traits in the composition must provide an
implementation of the method \lstinline{version}.
%to this point in \cref{sec:traits}.

As with the JavaScript version, the \lstinline{on_key} method is invoked on
\lstinline{self} in the body of \lstinline{do_cut}. This is allowed as (part of)
the type of \lstinline{self} (i.e., \lstinline{Editor}) contains the signature
of \lstinline{on_key}. Compared to the JavaScript class
\lstinline{Editor}, almost everything stays the same, except that we now have
a typed version. As a side note, since \sedel is currently a pure functional OO
language, there is no difference between fields and methods, so we can omit
empty arguments and parameter parentheses.

\paragraph{First-Class Traits and Trait Expressions.}

\sedel treats traits as first-class expressions, putting them in the same
syntactic category as objects, functions, and other primitive forms. To
illustrate this, we give the \sedel version of \lstinline{spellMixin}:
\lstinputlisting[linerange=31-43]{./examples/overview2.sl}% APPLY:linerange=OVERVIEW_HELP
This looks daunting at first, but \lstinline{spell_mixin} has almost the same structure as
its JavaScript cousin \lstinline{spellMixin}, albeit with
some type annotations. In \sedel, we use capital letters (\lstinline{A}, \lstinline{B}, $\dots$) to denote type variables, and trait
expressions \lstinline$trait [self : ...] inherits ... => {...}$ to create
first-class traits. Trait expressions have trait
types of the form \lstinline{Trait[T1, T2]} where \lstinline{T1} and \lstinline{T2} denote trait requirements and functionality respectively.
We will explain trait types in \cref{sec:traits}. Despite the structural similarities, there are several significant
features that are unique to \sedel (e.g., the disjointness operator \lstinline{*}).
We discuss these in the following.



\paragraph{Disjoint Polymorphism and Conflict Detection.}

\sedel uses a type system based on \emph{disjoint intersection types} (cf. \cref{chap:nested}) and
\emph{disjoint polymorphism} (cf. \cref{chap:fi}). Disjoint intersections
empower \sedel to detect conflicts statically when trying to compose two
traits with identically named features. For example, composing two traits
\lstinline{a} and \lstinline{b} that both provide \lstinline{foo} gives a
type error (the overloaded \lstinline{&} operator denotes trait composition):
\begin{lstlisting}
trait a => { foo = 1 };
trait b => { foo = 2 };
trait c inherits a & b => {}; -- type error!
\end{lstlisting}
Disjoint polymorphism, as a more advanced mechanism, allows detecting conflicts
even in the presence of polymorphism---for example when a trait is parameterized and its
full set of methods is not statically known. As can be seen,
\lstinline{spell_mixin} is actually a polymorphic function. Unlike ordinary
parametric polymorphism, in \sedel, a type variable can also have a disjointness
constraint. For instance, \lstinline{A * Spelling & OnKey}
means that \lstinline{A} can be instantiated to any type as long as it \emph{does not}
contain \lstinline{check} and \lstinline{on_key}. Note that these are the minimal constraints of \lstinline{A}, as
\begin{inparaenum}[(1)]
  \item \lstinline{A} cannot contain the \lstinline{on_key} method because otherwise it will conflict with \lstinline{Editor};
  \item \lstinline{A} cannot contain the \lstinline{check} method because otherwise it will conflict with that in the trait body.
\end{inparaenum}
To mimic mixins, the
argument \lstinline{base}, which is supposed to be some trait, serves as the
``base'' trait being inherited. Notice that the type variable
\lstinline{A} appears in the type of \lstinline{base}, which essentially states
that \lstinline{base} is a trait that contains at least those methods specified
by \lstinline{Editor}, and possibly more (which we do not know statically).
% In summary, \lstinline{Trait[Editor & Version, Editor & A]} (the assigned type
% of \lstinline{base}) specifies that both method \emph{presence} and \emph{absence}.
Also note that leaving out the \lstinline{override} keyword will result in a
type error. The type system is forcing us to be very specific as to what is the
intention of the \lstinline{on_key} method because it sees the same method is
also declared in \lstinline{base}, and blindly inheriting \lstinline{base}
will definitely cause a method conflict. As a final note, the use of \lstinline{super}
inside \lstinline{check} is allowed because the ``super'' trait \lstinline{base}
implements \lstinline{on_key}, as can be seen from its type.


\paragraph{Dynamic Inheritance.}

Disjoint polymorphism enables us to correctly type dynamic inheritance:
\lstinline{spell_mixin} is able to take any trait that conforms with its
assigned type, equips it with the \lstinline{check} method and overrides its
old \lstinline{on_key} method. As a side note, the use of disjoint polymorphism
is essential to correctly model the mixin semantics. From the type we know
\lstinline{base} has some features specified by \lstinline{Editor}, plus
something more denoted by \lstinline{A}. By inheriting \lstinline{base}, we are
guaranteed that the resulting trait will have everything that is already contained
in \lstinline{base}, plus more features. This is in some sense similar to row
polymorphism~\citep{wand1994type} in that the result trait is prohibited from
forgetting methods from the argument trait.
% As we will discuss in \cref{sec:related}, disjoint polymorphism is more expressive than row polymorphism.


\paragraph{Typing Mixin Composition.}
Next we give the typed version of \lstinline{modalMixin} as follows:
\lstinputlisting[linerange=48-56]{./examples/overview2.sl}% APPLY:linerange=OVERVIEW_MODAL
Now the definition of \lstinline{modal_mixin} should be self-explanatory.
Finally we can apply both ``mixins'' to \lstinline{editor} one at a time to create
an IDE editor:
\lstinputlisting[linerange=61-66]{./examples/overview2.sl}% APPLY:linerange=OVERVIEW_LINE
As with the JavaScript class \lstinline{IDEEditor}, we need to fill in the missing
\lstinline{version} method. It is easy to verify that the \lstinline{on_key} method
in \lstinline{modal_mixin} is inherited. Compared with the untyped version,
here this behaviour is reasonable because we specifically tag each
\lstinline{on_key} method to be an overriding method. Let us take a close look
at the mixin applications. Since \sedel is currently explicitly typed, we need to
provide concrete types when applying \lstinline{modal_mixin} and \lstinline{spell_mixin}.
In the inner application (\lstinline{spell_mixin Top editor}), we use the top
type \lstinline{Top} to instantiate \lstinline{A} because the \lstinline{editor} trait
provides exactly those method specified by \lstinline{Editor} and nothing more
(hence \lstinline{Top}). In the outer application, we use \lstinline{Spelling}
to instantiate \lstinline{A} because the resulting trait of the inner application
contains the \lstinline{check} method.
In summary, mixin applications are simply normal function applications,
and conflict resolution code is implicitly embedded via the keyword \lstinline{override}
and the order of mixin applications.
Unsurprisingly, changing the mixin application order to
\begin{lstlisting}
  inherits spell_mixin ModalEdit (modal_mixin Top editor)
\end{lstlisting}
gives the same (expected) behavior.


Admittedly the typed version is unnecessarily complicated as we were
mimicking mixins by functions over traits. The final trait
\lstinline{ide_editor} suffers from the same problem as the class
\lstinline{IDEEditor}, since there is no obvious way to access the
\lstinline{on_key} method in the \lstinline{editor} trait.\footnote{In fact, as
  we will see in \cref{sec:traits}, we can still access \lstinline{on_key} in
  \lstinline{editor} by the forwarding operator.} \cref{sec:traits}
makes better use of traits to simplify the editor code.



% Note that the use of \lstinline{override} is valid because the type system knows the inherit clause contains \lstinline{on_key}.
% As a bonus, since \sedel guarantees that there are no potential conflicts in a program,
% we can reason that the version number in \lstinline{modal_editor} is
% \lstinline{0.1}.

%%% Local Variables:
%%% mode: latex
%%% TeX-master: "../paper"
%%% org-ref-default-bibliography: ../paper.bib
%%% End:

\input{Sources/Traits/traits.tex}
\renewcommand{\rulehl}[2][gray!40]{%
  \colorbox{#1}{$\displaystyle#2$}}

\section{Formalizing Typed First-Class Traits}
\label{sec:typesystem}

This section presents the syntax and semantics of \sedel. In particular,
we show how to elaborate high-level source language constructs (self-references, abstract methods, first-class traits, dynamic inheritance, etc)
in \sedel to \fname, a pure record calculus with disjoint
polymorphism. The treatment of the self-reference and dynamic dispatching is
inspired by Cook and Palsberg's work on the denotational semantics for
inheritance~\cite{cook1989denotational}. We then prove the elaboration is type
safe, i.e., well-typed \sedel expressions are translated to well-typed \fname
terms. Finally we show that \sedel is coherent. Full proofs can be found in the appendix.

\subsection{Syntax}

\begin{figure}[t]
\centering
\begin{tabular}{lrcl}
  Types  & $[[AA]], [[BB]], [[CT]]$ & $\Coloneqq$ & $[[Top]] \mid [[nat]] \mid [[AA -> BB]] \mid [[AA & BB]] \mid  [[{ l : AA }]] \mid [[X]] \mid [[\ X ** AA  . BB]] \mid \hlmath{[[ Trait[AA,BB] ]]}$ \\
  Expressions & $[[E]]$ & $\Coloneqq$ & $[[Top]] \mid [[ii]] \mid [[x]] \mid [[\ x . E]] \mid [[E1 E2]] \mid [[\ X ** AA  . E]] \mid [[E AA]] \mid [[E1 ,, E2]] \mid [[E : AA]] $ \\
         & & $\mid$ & $[[{ l = E }]] \mid [[E . l]] \mid [[letrec x : AA = E1 in E2]] \mid \hlmath{[[new [ AA ] (</ Ei // i />) ]]} \mid \hlmath{[[E1 ^ E2]]} $ \\
  Value contexts & $[[SG]]$ & $\Coloneqq$ & $[[empty]] \mid [[SG , x : AA]] $ \\
  Type contexts & $[[SD]]$ & $\Coloneqq$ & $[[empty]] \mid [[SD , X ** AA]]$ \\ \\
\end{tabular}
\begin{tabular}{llll}
  Record types & $[[ { l1 : AA1 , ... , ln : AAn } ]] $ & := & $[[ { l1 : AA1} & ... & { ln : AAn } ]]$ \\
  Records &  $[[ { l1 = E1 , ... , ln = En } ]] $ & := & $ [[ { l1 = E1 } ,, ... ,, { ln = En } ]]$
\end{tabular}
\caption{\sedel core syntax and syntactic abbreviations}
\label{fig:sedel_syntax}
\end{figure}

The core syntax of \sedel is shown in \cref{fig:sedel_syntax}, with trait related
constructs \hll{highlighted}. For brevity of the meta-theoretic study, we do not
consider definitions, which can be added in standard ways.
%We omit mutable fields and other practical
%constructs in order to focus on the basic mechanisms of traits. The omitted
%constructs can be added in standard ways~\cite{DBLP:books/daglib/0005958}.

\paragraph{Types.}
Metavariables $[[AA]]$, $[[BB]]$, $[[CT]]$ range over types. Types include a top
type $[[Top]]$, type of integers $[[nat]]$, function types $[[AA -> BB]]$, intersection types $[[AA & BB]]$,
singleton record types $[[{l : AA}]]$,  type variables $[[X]]$ and disjoint
(universal) quantification $[[\ X ** A . B]]$. The main
novelty is the type of first-class traits $[[ Trait[AA, BB] ]]$, which expresses
the requirement $[[AA]]$ and the functionality $[[BB]]$. We will use $[[ [ AA / X ] BB ]]$
to denote capture-avoiding substitution of $[[AA]]$ for $[[X]]$ inside $[[BB]]$.


\paragraph{Expressions.}
Metavariable $[[E]]$ ranges over expressions. We start with constructs required
to encode objects based on records: term variables $[[x]]$, lambda abstractions $[[\x. E]]$, function
applications $[[E1 E2]]$, singleton records $[[{l = E}]]$, record projections
$[[E.l]]$, recursive let bindings $[[letrec x : AA = E1 in E2]]$, disjoint type
abstraction $[[\ X ** AA . E]]$ and type application $[[E AA]]$.
The calculus also supports a merge construct $[[E1 ,, E2]]$ for creating values of intersection
types and annotated expressions $[[E : AA]]$. We also include a canonical top
value $[[Top]]$ and integer literals $[[ii]]$.

\paragraph{First-class traits and trait expressions.}
The central construct of \sedel is the trait
expression%\footnote{The abstract syntax of trait expressions is slightly different from the concrete syntax.}
$[[ trait [ self : BB ] inherits </ Ei // i /> { </ lj = Ej' // j /> } : AA]]$,
which specifies a (possibly empty) list
of trait expressions $\overline{[[Ei]]}$ in the \lstinline{inherits} clause, an explicit
$[[self]]$ reference (with type annotation $[[B]]$), and a set of
methods $\{ \overline{l_j = E'_j} \}$. Intuitively this trait expression has
type $[[ Trait[BB, AA] ]]$. Unlike the conventional trait model, a trait
expression denotes a first-class value: it may occur anywhere where an
expression is expected. Trait instantiation expressions $[[new [ AA ] (</ Ei // i />) ]]$
instantiate a composition of trait expressions $\overline{[[Ei]]}$ to create an
object of type $[[AA]]$. Finally $[[E1 ^ E2]]$ is the forwarding expression,
where $[[E1]]$ should be some trait.

\paragraph{Abbreviations.}
For ease of programming, multiple-field record types are merely syntactic sugar
for intersections of single-field record types. Similarly, multi-field record
expressions are syntactic sugar for merges of single-field records.

\subsection{Semantics}

\begin{figure}[t]
  \centering
  \drules[TS]{$[[ AA <: BB ]]$}{Subtyping}{arr, trait}
  \drules[WF]{$[[ SD |- AA ]]$}{Well formedness}{and, trait}
  \caption{Subtyping and well-formedness of \sedel (excerpt)}
  \label{fig:typesystem}
\end{figure}


\paragraph{Subtyping and Well-formedness.}
\Cref{fig:typesystem} shows the most relevant subtyping and well-formedness
rules for \sedel. Omitted rules are standard and can be found in previous
work~\cite{alpuimdisjoint}. The
subtyping rule for trait types (\rref{TS-trait}) resembles the one for function
types (\rref{TS-arr}) in that it is contravariant on the first type $[[AA]]$
and covariant on the second type $[[BB]]$. The well-formedness rule for trait
types is straightforward.

\begin{figure}[t]
  \centering
\begin{small}
  \drules[SD]{$[[ SD |- AA ** BB ]]$}{Disjointness}{top, topSym, var, varSym, forall, rec, recn, arrow, andL, andR, trait, traitArrOne, traitArrTwo, ax}
  \drules[Dax]{$[[ AA **a BB ]]$}{Disjointness axiom}{intTrait, traitForall, traitRec}
\end{small}
\caption{Disjointness rules of \sedel (excerpt)}
  \label{fig:disjoint}
\end{figure}


\paragraph{Disjointness.}
\Cref{fig:disjoint} shows the disjointness judgment $[[SD |- AA ** BB]]$, which is
used for example in \rref{WF-and}. The disjointness checking is the underlying
mechanism of conflict detection. We naturally extend the disjointness rules in
\fname to cover trait types. We refer to
their paper~\cite{alpuimdisjoint} for further explanation. Here we discuss
the rules
related with traits. \Rref{SD-trait} says that as long as the functionalities
that two traits provide are disjoint, the two trait types are disjoint.
\Rref{SD-traitArr1,SD-traitArr2} deal with situations where one of the two types
is a function type. At first glance, these two look strange because a trait type is
\textit{different} from a function type, and they ought to be disjoint as an axiom. The reason
is that \sedel has an elaboration semantics, and as we will see, trait types are translated to function
types. In order to ensure the elaboration is type-safe, we have to have special treatment for trait
and function types. In principle, if \sedel has its own semantics, then trait types are always disjoint
to function types. The axiom rules of the form $[[ AA **a BB ]]$ take care of two types with different language constructs.
% These rules capture the notion that any two types are disjoint unless one of
% them is an intersection types, a type variable or $[[top]]$.

\begin{figure}[t]
  \centering
  \begin{small}
  \drules[ST]{$[[ SD ; SG  |- E => AA ~~> ee]]$}{Infer}{trait,traitSuper,forward,new}
  \end{small}
  \caption{Typing of \sedel (excerpt)}
  \label{fig:type}
\end{figure}

\paragraph{Typing Traits.}
The typing rules of trait related constructs are shown in \cref{fig:type}. The full set of rules can be found in the appendix.
The reader is advised to ignore the \hll{highlighted} parts for now.
% As with conventional bidirectional type systems,
\sedel employs two modes: the inference mode
($[[=>]]$) and the checking mode ($[[<=]]$). The inference judgment $[[ SD ; SG |- E => AA]]$
says that we can synthesize a type $[[AA]]$ for expression $[[E]]$.
The checking judgment $[[SD; SG |- E <= AA]]$ checks $[[E]]$ against $[[AA]]$. One representative of inference rules is
% \begin{mathpar}
%  \drule{st-merge}
% \end{mathpar}
which says that a merge of two expressions is valid only if their types are disjoint. This is the underlying
mechanism for conflict detection. One representative of checking rules is
% \begin{mathpar}
%  \drule{ST-sub}
% \end{mathpar}
% typically known as the subsumption rule,
where subtyping is used to coerce expressions of one type to another.


To type-check a trait (\rref{ST-trait}) we first type-check if its inherited traits $\overline{[[Ei]]}$ are valid
traits. Note that each trait $[[Ei]]$ can possibly refer to $[[self]]$. Methods
must all be well-typed in the usual sense. Apart from these, we have several
side-conditions to make sure traits are well-behaved. The well-formedness
judgment $[[SD |- CT1 & .. & CTn & CT]]$ ensures that we do not have conflicting
methods (in inherited traits and the body). The subtyping judgments $\overline{[[BB <: BBi]]}$ ensure that the
$[[self]]$ parameter satisfies the requirements imposed by each
inherited trait. Finally the subtyping judgment $[[CT1 & .. & CTn & CT <: AA]]$
sanity-checks that the assigned type $[[AA]]$ is compatible.

Trait instantiation (\rref{ST-new}) requires that each instantiated trait is valid.
There are also several side-conditions, which serve the same
purposes as in \rref{ST-trait}.
\Rref{ST-forward} says that the first operand $[[E1]]$ of the forwarding operator must be a trait. Moreover, the type of the second operand
$[[E2]]$ must satisfy the requirement of $[[E1]]$.



\paragraph{Treatments of Exclusion, Super and Override.}
One may have noticed that in \cref{fig:sedel_syntax} we did not include the
exclusion operator in the core \sedel syntax, neither do \lstinline{super} and
\lstinline{override} appear. The reason is that in principle all
uses of the exclusion operator can be replaced by type annotations. For example
to exclude a \lstinline{bar} field from \lstinline${foo = a, bar = b, baz = c}$,
all we need is to annotate the record with type \lstinline${foo : A, baz : C}$
(suppose \lstinline{a} has type \lstinline{A}, etc). By \rref{Chk-sub}, the resulting
record is guaranteed to contain no \lstinline{bar} field. In the same vein,
the use of \lstinline{override} can be explained using the exclusion operator.
The \lstinline{super} keyword is internally a variable pointing to the \lstinline{inherits} clause
(its typing rule is similar to \rref{inf-trait} and can be found in the appendix).
We omit all of these features in the meta-theoretic study in order to focus our attention on
the essence of first-class traits.
However in practice, this is rather inconvenient as we need to write down
all types we wish to retain rather than the one to exclude. So in our
implementation we offer all of them.

\paragraph{Elaboration.}

The operational semantics of \sedel is given by means of a type-directed
translation into \fname extended with (lazy) recursive let bindings.
This extension is standard and type-safe. Let us go back to
\cref{fig:type}, now focusing on the \hll{highlighted} parts, which
denote the elaborated \fname terms. Most of them
are straightforward translations and are thus omitted. We explain the most
involved rules regarding traits. In \rref{Inf-trait}, a trait is translated into
a lambda abstraction with $[[self]]$ as the formal parameter.
In essence a trait corresponds to what Cook and Palsberg~\cite{cook1989denotational} call a \emph{generator}.
 The translations
of the inherited traits (i.e., $\overline{[[eei]]}$) are each applied to
$[[self]]$ and then merged with the translation of the trait body $[[ee]]$. Now
it is clear why we require $[[BB]]$ (the type of $[[self]]$) to be a subtype of each
$[[BBi]]$ (the requirement of each inherited trait). Note that we abuse the bar
notation here with the intention that $[[</ (eei self) // i IN 1..n />]]$ means
$[[ee1 self ,, .. ,, een self]]$.
Here is an example of translating the \lstinline{ide_editor} trait from \cref{sec:overview} into
plain \fname terms equipped with definitions (suppose \lstinline{modal_mixin} and \lstinline{spell_mixin}
have been translated accordingly):
\lstinputlisting[linerange=62-63]{./examples/overview2.sl}% APPLY:linerange=TRANS

The gray parts in \rref{ST-new} show the translation of trait instantiation.
First we apply every translation (i.e., $[[eei]]$) of the instantiated traits to the $[[self]]$ parameter,
and then merge the applications together. The bar notation is
interpreted similarly to the translation in \rref{ST-trait}. Finally we compute the \emph{lazy}
fixed-point of the resulting merge term, i.e., self-reference must be updated to refer to
the whole composition. Taking the fixed-point of the
traits/generators again follows the denotational inheritance model by
Cook and Palsberg.
 This is the key to the correct implementation of dynamic
 dispatching. Finally,
\rref{ST-forward} translates forwarding expressions to function
applications. We show the translation of the
\lstinline{a_editor1} object in \cref{sec:traits} to illustrate the
translation of instantiation:
\lstinputlisting[linerange=71-71]{./examples/overview2.sl}% APPLY:linerange=NEW

One remarkable point is that, while Cook and Palsberg work is done in
an untyped setting, here we apply their ideas in a setting with
disjoint intersection types and disjoint polymorphism. Our work shows that
disjoint intersection types blend in quite nicely with Cook and
Palsberg's denotational model of inheritance.

\paragraph{Flattening Property.}

In the literature of traits~\cite{Ducasse_2006, scharli2003traits, JOT:issue_2006_05/article4},
a distinguished feature of traits is the
so-called \textit{flattening property}. This property says that a (non-overridden) method in a
trait has the same semantics as if it were implemented directly in the class
that uses the trait. It would be interesting to see if our trait model has this
property. One problem in formulating such a property is that flattening is a
property that talks about the equivalence between a flattened class (i.e., a
class where all trait methods have been inlined) and a class that reuses code
from traits. Since \sedel does not have classes, we cannot state exactly the same
property. However, we believe that one way to talk about a similar property for \sedel is to have something
along the lines of the following example:
\begin{example}[Flattening]
  Suppose we have \lstinline$m$ well-typed (i.e, conflict-free) traits \lstinline$trait t1 {l11 = E11, ..}, ..., trait tm {lm1 = Em1, ..}$,
  each with some number of methods, then
  \begin{center}
   \lstinline|new (trait inherits t1 & ... & tm {})|  $=$  \lstinline|new (trait {l11 = E11,..,lm1 = Em1,..})|
  \end{center}
\end{example}
If we elaborate these two expressions, the property boils down to whether two merge terms
$[[(ee1 ,, ee2) ,, ee2]]$ and $[[ee1 ,, (ee2 ,, ee3)]]$
have the same semantics. As is shown by Bi et al.~\cite{xuan_nested}, merges are
associative and commutative, so it is not hard to see that the above two expressions
are semantically equivalent. We leave it as future work to formally state and prove flattening.


% no gray anymore after this point
\renewcommand{\rulehl}[1]{#1}

\subsection{Type Soundness and Coherence}

Since the semantics of \sedel is defined by elaboration into \fname it
is easy to show that key properties of \fname are also guaranteed by \sedel.
In particular, we show that the type-directed elaboration is
type-safe in the sense that well-typed \sedel expressions are elaborated into
well-typed \fname terms. We also show that the source language is
coherent and each valid source program has a unique (unambiguous)
elaboration.

We need a meta-function $| \cdot |$ that translates \sedel types to \fname types, whose definition is
straightforward. Only the translation of trait types deserves attention:
\begin{mathpar}
  | [[Trait[AA,BB] ]] | = [[|AA| -> |BB|]]
\end{mathpar}
That is, trait types are translated to
function types. $| \cdot |$ extends naturally to typing contexts.
Now we show several lemmas that are useful in the type-safety proof.

\begin{lemma}
  If $[[SD |- AA]]$ then $[[|SG| |- |AA|]]$.
\end{lemma}
\begin{proof}
  By structural induction on the well-formedness judgment.
\end{proof}

\begin{lemma}
  If $[[AA <: BB]]$ then $[[|AA| <: |BB|]]$.
\end{lemma}
\begin{proof}
  By structural induction on the subtyping judgment.
\end{proof}

\begin{lemma}
  If $[[SD |- AA ** BB]]$ then $[[ |SD| |- |AA| ** |BB| ]]$.
\end{lemma}
\begin{proof}
  By structural induction on the disjointness judgment.
\end{proof}
% \begin{remark}
%  Due to the elaboration semantics, \rref{D-traitArr1,D-traitArr2} are needed to make this lemma hold.
% \end{remark}


Finally we are in a position to establish the type safety property:
\begin{theorem}[Type-safe translation]
  We have that:
  \begin{itemize}
  \item If $[[SD ; SG  |- E => AA ~~> ee]]$ then $ [[ |SD| ;  |SG|  |- ee => |AA| ]] $.
  \item If $[[SD ; SG  |- E <= AA ~~> ee]]$ then $ [[ |SD| ;  |SG|  |- ee <= |AA| ]] $.
  \end{itemize}
\end{theorem}
\begin{proof}
    By structural induction on the typing judgment.
\end{proof}

\begin{theorem}[Coherence] Each well-typed \sedel expression has a unique elaboration.
\end{theorem}
\begin{proof}
  By examining every elaboration rule, it is easy to see that the elaborated
  \fname term in the conclusion is uniquely determined by the elaborated \fname
  terms in the premises. Then by the coherence property of \fname, we conclude
  that each well-typed \sedel expression has a unique unambiguous elaboration,
  thus \sedel is coherent.
\end{proof}

% \input{Sources/Traits/applications.tex}
% 
\section{Conclusions and Future Work}
\label{sec:conclusion}

We have proposed \name, a type-safe and coherent calculus with disjoint
intersection types, and support for nested composition/subtyping. \name
improves upon earlier work with a more
flexible notion of disjoint intersection types, which leads to
a clean and elegant formulation of the type system. Due to the added
flexibility we have had to employ a more powerful proof method based on logical
relations to rigorously prove coherence.
We also show how \name supports essential features of family
polymorphism, such as nested composition. We believe \name provides insights into family polymorphism, and
has potential for practical applications for extensible software designs.

A natural direction for future work is to enrich \name with parametric
polymorphism. There is abundant literature on logical relations for parametric
polymorphism~\cite{reynolds1983types} and we foresee no fundamental difficulties
in extending our proof method.\footnote{ Our prototype implementation already
  supports polymorphism, but we are still in the process of extending our Coq
  development with polymorphism.} The main challenge in the definition of the
logical relation is the clause for type variables with arbitrary types. Careful
measures are to be taken to avoid the potential circularity due to
impredicativity. With the combination of parametric polymorphism and nested
composition, an interesting application that we intend to investigate is native
support for a highly modular form of \textit{Object Algebras}~\cite{oliveira2012extensibility, xuan_traits} and \textsc{Visitor}s
(or the finally tagless approach~\cite{CARETTE_2009}).

Another direction for future work is to add mutable references, which would
touch two places in our metatheory: type safety and coherence. For type safety,
we expect that lessons learned from previous work on family polymorphism and
mutability on OO to apply to our work. For example, it is well-known that
subtyping in the presence of mutable state often needs restrictions. Given such
suitable restrictions we expect that type-safety in the presence of mutability
still works. For coherence, it would be a major technical challenge to adjust
our coherence proof and its Coq mechanisation. Logical relations that account
for mutable state (e.g., see Ahmed's thesis~\cite{ahmed2004semantics}) introduce significant complexity.





% For example, we can
% define the object algebra interfaces for the Expression Problem example in
% \cref{sec:overview} as follows:
% \lstinputlisting[linerange=77-78]{../../impl/examples/overview.sl}% APPLY:linerange=LANG_EXT_INTER
% By instantiating \lstinline{E} with \lstinline{IPrint}, i.e.,
% \lstinline{ExpAlg[IPrint]}, we get the interface of the \lstinline{Lang} family.
% In that sense, object algebra interfaces can be viewed as family interfaces.
% Moreover, combining algebras implementing \lstinline{ExpAlg[IPrint]} and
% \lstinline{ExpAlg[IEval]} to form \lstinline{ExpAlg[IPrint & IEval]} is trivial
% with nested composition. Polymorphism also improves code reuse across expressions in the
% base and extended languages. For example, the following creates two expressions,
% one in the base language, the other in the extended language:
% \lstinputlisting[linerange=83-84]{../../impl/examples/overview.sl}% APPLY:linerange=LANG_EXT
% Notice how we can  reuse \lstinline{e1} of the base language in the definition
% of \lstinline{e2}.



% \jeremy{creating expressions using base and extended expressions, and show more reuse}

% \jeremy{future work} \jeremy{mention in passing this rule is unsound with
%   effects, see ``Intersection types and computational effects''}

% Local Variables:
% mode: latex
% TeX-master: "../paper"
% org-ref-default-bibliography: ../paper.bib
% End:



%%% Local Variables:
%%% mode: latex
%%% TeX-master: "../Thesis"
%%% org-ref-default-bibliography: ../Thesis.bib
%%% End:

  
%%%%%%%%%%%%%%%%%%%%%%%%%%%%%%%%%%%%%%%%%%%%%%%%%%%%%%%%%%%%%%%%%%%%%%%%
\chapter{Extensible Designs}
%%%%%%%%%%%%%%%%%%%%%%%%%%%%%%%%%%%%%%%%%%%%%%%%%%%%%%%%%%%%%%%%%%%%%%%%

To further illustrate the applicability of \sedel, we present a case
study using Object Algebras~\cite{oliveira2012extensibility} and
Extensible \textsc{Visitor}s~\cite{oliveira09modular, togersen:2004}. Encodings
of extensible designs for Object Algebras and Extensible \textsc{Visitor}s have
been presented in mainstream languages~\cite{oliveira09modular, togersen:2004, oliveira2012extensibility, oliveira2013feature, rendel14attributes}.
However, prior approaches are not entirely satisfactory
due to the limitations in existing mainstream OO languages. In \cref{sec:ob}, we show how \sedel makes those designs significantly simpler and
convenient to use. In particular, \name's encoding of extensible visitors gives true ASTs and supports
conflict-free Object Algebra combinators, thanks to first-class traits and disjoint polymorphism.
Based on this technique, \cref{sec:case} gives a bird-view of several orthogonal features of
a small JavaScript-like language from a textbook on Programming
Languages~\cite{poplcook}, and illustrates how various features can
be modularly developed and composed to assemble a complete language with various
operations baked in. \Cref{sec:evaluate} compares our \name's implementation
with that of the textbook using Haskell in terms of lines of code.


\section{Object Algebras and Extensible Visitors in \sedel}
\label{sec:ob}

First we give a simple introduction to Object Algebras, a design pattern that
can solve the Expression Problem~\cite{wadler1998expression} (EP) in languages like
Java. The objective of EP is to \emph{modularly} extend a datatype in two
dimensions: by adding more cases to the datatype and by adding new operations
for the datatype.
Our starting point is the following code:
\lstinputlisting[linerange=4-9]{./examples/application.sl}% APPLY:linerange=ALGEBRA_DEF
\lstinline{ExpAlg[E]} is the generic interface of a simple arithmetic language
with two cases, \lstinline{lit} for literals and \lstinline{add} for addition.
\lstinline{ExpAlg[E]} is also called an Object Algebra interface. A concrete
Object Algebra will implement such an interface by instantiating \lstinline{E}
with a suitable type. Here we also define one operation \lstinline{IEval},
modelled by a single-field record type. A concrete Object Algebra that
implements the evaluation rules is given by a trait
\lstinline{evalAlg}.

\paragraph{First-Class Object Algebra Values.}
The actual AST of this simple arithmetic language is given as an internal
visitor~\cite{Oliveira_2008}:
\lstinputlisting[linerange=13-13]{./examples/application.sl}% APPLY:linerange=EXP_TYPE
Note that Object Algebras as implemented in languages like Java or Scala do not define the type
\lstinline{Exp} because this would make adding new variants very hard. Although extensible versions
of this visitor pattern do exist, they usually require complex types using advanced features of
generics~\cite{oliveira2012extensibility, togersen:2004}.
However, as we will see, this is not a problem in \sedel. We can build a value of \lstinline{Exp} as follows:
\lstinputlisting[linerange=17-17]{./examples/application.sl}% APPLY:linerange=VALUE_E1


\paragraph{Adding a New Operation.}
We add another operation \lstinline{IPrint} to the language:
\lstinputlisting[linerange=22-28]{./examples/application.sl}% APPLY:linerange=PRINT_DEF
This is done by giving another trait \lstinline{printAlg} that implements the
additional \lstinline{print} method.


\paragraph{Adding a New Case.}
A second dimension for extension is to add another case for negation:
\lstinputlisting[linerange=33-39]{./examples/application.sl}% APPLY:linerange=SUB_DEF
This is achieved by extending \lstinline{evalAlg} and \lstinline{printAlg}, implementing
missing operations for negation, respectively. We define the actual AST similarly:
\lstinputlisting[linerange=44-44]{./examples/application.sl}% APPLY:linerange=EXPEXT_TYPE
and build a value of \lstinline{-(2 + 3)} while reusing \lstinline{e1}:
\lstinputlisting[linerange=49-49]{./examples/application.sl}% APPLY:linerange=VALUE_E2

\paragraph{Relations between \lstinline{Exp} and \lstinline{ExpExt}}
At this stage, it is interesting to point out an interesting subtyping relation
between \lstinline{Exp} and \lstinline{ExtExp}: \lstinline{ExpExt}, though being an
\emph{extension} of \lstinline{Exp} is actually a \emph{supertype} of \lstinline{Exp}.
As Oliveira~\cite{oliveira09modular} observed, these relations are
important for legacy and performance reasons since it means that, a value of
type \lstinline{Exp} can be \emph{automatically} and \emph{safely}
coerced into a value of type \lstinline{ExpExt}, allowing some
interoperability between new functionality and legacy code.
However, to ensure type-soundness, Scala (or other common OO languages) forbids any kind of type-refinement on method
parameter types. The consequence of this is that in those languages, it is
impossible to express that \lstinline{ExtExp} is both an extension and a
supertype of \lstinline{Exp}.


% Encodings
% of extensible visitors in mainstream OO languages usually fail to
% correctly express these relations, or require sophisticated
% type system extensions~\cite{oliveira09modular}.


\section{Dynamic Object Algebra Composition Support}

When programming with Object Algebras, oftentimes it is necessary to pack
multiple operations in the same object. For example, in the simple language we
have been developing it can be useful to create an object that supports both
printing and evaluation. Oliveira and Cook~\cite{oliveira2012extensibility}
addressed this problem by proposing \emph{Object Algebra combinators} that
combine multiple algebras into one. However, as they noted, such combinators
written in Java are difficult to use in practice, and they require significant
amounts of boilerplate. Improved variants of Object Algebra combinators have
been encoded in Scala using intersection types and an encoding of the merge
construct~\cite{oliveira2013feature, rendel14attributes}. However, the
Scala encoding of the merge construct is quite complex as it relies on low-level
type-unsafe programming features such as dynamic proxies, reflection or other
meta-programming techniques. In \sedel, the combination of first-class
traits, dynamic inheritance and disjoint polymorphism allows type-safe, coherent
and boilerplate-free composition of Object Algebras.
\lstinputlisting[linerange=54-55]{./examples/application.sl}% APPLY:linerange=COMBINE
That is it. None of the boilerplate in other
approaches~\cite{oliveira2012extensibility}, or type-unsafe meta-programming
techniques of other approaches~\cite{oliveira2013feature,rendel14attributes} are
needed! Two points are worth noting: (1) \lstinline{combine} relies on
\emph{dynamic inheritance}. Notice how \lstinline{combine} inherits two traits
\lstinline{f} and \lstinline{g}, for which their implementations are unknown
statically; (2) the disjointness constraint (\lstinline{B * A}) is \emph{crucial} to
ensure two Object Algebras (\lstinline{f} and \lstinline{g}) are conflict-free
when being composed.

To conclude, let us see \lstinline{combine} in action. We combine \lstinline{negEvalAlg} and \lstinline{negPrintAlg}:
\lstinputlisting[linerange=59-59]{./examples/application.sl}% APPLY:linerange=NEW_ALG
The combined algebra \lstinline{combineAlg} is useful to avoid multiple interpretations
of the same AST when running multiple operations. For example, we can
create an object \lstinline{o} that supports both evaluation and printing in one go:
\lstinputlisting[linerange=72-73]{./examples/application.sl}% APPLY:linerange=USE




%%%%%%%%%%%%%%%%%%%%%%%%%%%%%%%%%%%%%%%%%%%%%%%%%%%%%%%%%%%%%%%%%%%%%%%%
\chapter{Case Study: Modularizing Language Components}
%%%%%%%%%%%%%%%%%%%%%%%%%%%%%%%%%%%%%%%%%%%%%%%%%%%%%%%%%%%%%%%%%%%%%%%%


\begin{figure}[t]
\centering
\begin{tabular}{lrclr}
  Types  & $\tau$ & ::= & $ \mathsf{int}  \mid  \mathsf{bool} $ & \\
  Expressions & $e$ & ::= & $ i  \mid \,  \ottnt{e_{{\mathrm{1}}}}  \ottsym{+}  \ottnt{e_{{\mathrm{2}}}} \mid \,  \ottnt{e_{{\mathrm{1}}}}  \ottsym{-}  \ottnt{e_{{\mathrm{2}}}} \mid \,  \ottnt{e_{{\mathrm{1}}}}  \times  \ottnt{e_{{\mathrm{2}}}} \mid \,  \ottnt{e_{{\mathrm{1}}}}  \div  \ottnt{e_{{\mathrm{2}}}} $ & $\mathit{natF}$ \\
              && $\mid$ & $ \mathbb{B}  \mid \ottkw{if} \, \ottnt{e_{{\mathrm{1}}}} \, \ottkw{then} \, \ottnt{e_{{\mathrm{2}}}} \, \ottkw{else} \, \ottnt{e_{{\mathrm{3}}}} $ & $\mathit{boolF}$\\
              && $\mid$ & $ \,  \ottnt{e_{{\mathrm{1}}}}  \ottsym{==}  \ottnt{e_{{\mathrm{2}}}} \mid \,  \ottnt{e_{{\mathrm{1}}}}  \ottsym{<}  \ottnt{e_{{\mathrm{2}}}} $ & $\mathit{compF}$ \\
              && $\mid$ & $ \,  \ottnt{e_{{\mathrm{1}}}}  \,\&\&\,  \ottnt{e_{{\mathrm{2}}}} \mid \,  \ottnt{e_{{\mathrm{1}}}}  \,||\,  \ottnt{e_{{\mathrm{2}}}} $ & $\mathit{logicF}$ \\
              && $\mid$ & $\ottmv{x} \mid \ottkw{var} \, \ottmv{x}  \ottsym{=}  \ottnt{e_{{\mathrm{1}}}}  \ottsym{;}  \ottnt{e_{{\mathrm{2}}}}$  &  $\mathit{varF}$ \\
              && $\mid$ & $\,  \ottnt{e_{{\mathrm{1}}}} \, \ottnt{e_{{\mathrm{2}}}}$ & $\mathit{funcF}$ \\
  Programs & $pgm$ & ::= & $decl_{{\mathrm{1}}} \dots decl_{\ottmv{n}} \, \ottnt{e}$ &  $\mathit{funcF}$ \\
  Functions & $decl$ & ::= & $\ottkw{function} \, \ottmv{f}  \ottsym{(}  \ottmv{x}  \ottsym{:}  \tau  \ottsym{)}  \ottsym{\{}  \ottnt{e}  \ottsym{\}}$ &  $\mathit{funcF}$ \\
  Values & $v$ & ::= & $ i  \mid  \mathbb{B} $ &
\end{tabular}

\caption{Mini-JS expressions, values, and types}
\label{fig:mini-js}
\end{figure}

\section{Case Study Overview}
\label{sec:case}

Now we are ready to see how the same technique scales to modularize different
language features. A \emph{feature} is an increment in program
functionality~\cite{zave1999faq,lopez2005evaluating}. \Cref{fig:mini-js}
presents the syntax of the expressions, values and types provided by the
features; each line is annotated with the corresponding feature sedel. Starting from a
simple arithmetic language, we gradually introduce new features and combine them
with some of the existing features to form various languages. Below we briefly
explain what constitutes each feature:
\begin{itemize}
\item $\mathit{natF}$ and $\mathit{boolF}$ contain, among others, literals, additions and conditional expressions.
\item $\mathit{compF}$ and $\mathit{logicF}$ introduce comparisons between numbers and logical connectives.
\item $\mathit{varF}$ introduces local variables and variable declarations.
\item $\mathit{funcF}$ introduces top-level functions and function calls.
\end{itemize}
Besides, each feature is packed with 3 operations: evaluator, pretty
printer and type checker.

Having the feature set, we can synthesize different languages by selecting one
or more operations, and one or more data variants, as shown in \cref{fig:langs}.
For example \lstinline{arith} is a simple language of arithmetic expressions,
assembled from $\mathit{natF}$, $\mathit{boolF}$ and $\mathit{compF}$. On top of
that, we also define an evaluator, a pretty printer and a type checker. Note
that for some languages (e.g., \lstinline{simplenat}), since they have only one
kind of value, we only define an evaluator and a pretty printer. We thus obtain
12 languages and 30 operations in total. The complete language
\lstinline{mini-JS} contains all the features and supports all the operations. % Besides, we also define
% a new algebra with the combined behavior of all the operations.
The reader can refer to our supplementary material for the source code of the case study.


\begin{table}[t]
  \centering
  \begin{small}
\begin{tabular}{|l||c|c|c||c|c|c|c|c|c|}
\hline
\multirow{2}{*}{Language} & \multicolumn{3}{c||}{Operations} & \multicolumn{6}{c|}{Data variants}           \\ \cline{2-10}
                      & eval     & print     & check    & $\mathit{natF}$ & $\mathit{boolF}$ & $\mathit{compF}$ & $\mathit{logicF}$ & $\mathit{varF}$ & $\mathit{funcF}$ \\ \hline \hline
\lstinline$simplenat$             &   \cmark       & \cmark          &          &  \cmark    &       &       &        &      &       \\ \hline
\lstinline$simplebool$          &  \cmark        &  \cmark         &          &      &  \cmark     &       &        &      &       \\ \hline
\lstinline$natbool$       &  \cmark        & \cmark          & \cmark         & \cmark     & \cmark      &       &        &      &       \\ \hline
\lstinline$varbool$       &  \cmark        &  \cmark         &          &      & \cmark      &       &        & \cmark     &       \\ \hline
\lstinline$varnat$      &   \cmark       &  \cmark         &   &  \cmark    &     &       &        & \cmark      &       \\ \hline
\lstinline$simplelogic$  &  \cmark        &  \cmark         &          &      &   \cmark    &       &    \cmark    &      &       \\ \hline
\lstinline$varlogic$   &    \cmark      &   \cmark        &          &      &  \cmark     &       &  \cmark  &  \cmark    &       \\ \hline
\lstinline$arith$     &  \cmark  &  \cmark &  \cmark &  \cmark    &  \cmark     &  \cmark     &        &      &       \\ \hline
\lstinline$arithlogic$ &  \cmark   &  \cmark &  \cmark  & \cmark     &  \cmark     & \cmark      & \cmark       &      &       \\ \hline
\lstinline$vararith$        &  \cmark   &  \cmark  &  \cmark  & \cmark     &  \cmark     &  \cmark     &        & \cmark     &       \\ \hline
\lstinline$vararithlogic$  &  \cmark &  \cmark  &  \cmark  & \cmark & \cmark & \cmark &  \cmark & \cmark &       \\ \hline
\lstinline$mini-JS$  &  \cmark &  \cmark  &  \cmark  & \cmark & \cmark & \cmark &  \cmark & \cmark & \cmark      \\ \hline
\end{tabular}

  \end{small}
\caption{Overview of the languages assembled}
\label{fig:langs}
\end{table}


\section{Evaluation}
\label{sec:evaluate}

To evaluate \name's implementation of the case study,
\Cref{fig:sloc} compares the number of source lines of code
(SLOC, lines of code without counting empty lines and comments) for
\name's \emph{modular} implementation with the vanilla
\emph{non-modular} AST-based implementations in Haskell. The Haskell
implementations are just straightforward AST interpreters, which duplicate code across the multiple language
components.

Since \sedel is a new language, we
had to write various code that is provided in Haskell by the standard library,
so they are not counted for fairness of comparison. In the left part, for each
feature, we count the lines of the algebra interface (number beside the feature
sedel), and the algebras for the operations. In the right part, for each
language, we count the lines of ASTs, and those to combine previously
defined operations. For example, here is the code that is needed to make the
\lstinline{arith} language.
\lstinputlisting[linerange=537-544]{./examples/case_study.sl}% APPLY:linerange=ARITH
We only need 8 lines in total: 2 lines for the AST, and 6 lines to combine the operations.

Therefore, the total SLOC of \name's implementation is the sum of all the
lines in the feature and language parts (237 SLOC of all features plus 94 SLOC
of ASTs and operations). Although \sedel is considerably more verbose than a
functional language like Haskell, \name's modular implementation for 12 languages and 30
operations in total reduces approximately 60\% in terms of SLOC. The reason is
that, the more frequently a feature is reused by other languages directly or
indirectly, the more reduction we see in the total SLOC. For example,
$\mathit{natF}$ is used across many languages. Even though \lstinline{simplenat}
itself \emph{alone} has more SLOC ($40 = 7+23+7+3$) than that of Haskell (which
has 33), we still get a huge gain when implementing other languages.

Finally, we acknowledge the limitation of our case study in that SLOC is just
one metric and we have not measured any other metrics. Nevertheless we believe
that the case study is already non-trivial in that we need to solve EP. Note
that Scala traits alone are not sufficient on their own to solve EP. While there
are solutions to EP in both Haskell and Scala, they
introduce significant complexity, as explained in \cref{sec:ob}.



\begin{table}[t]
  \centering
  \begin{small}
  \begin{tabular}{|r|ccc||l|ccc|}
    \hline
     Feature & \textbf{eval} & \textbf{print} & \textbf{check} & Lang sedel & \sedel & \textbf{Haskell} & \textbf{\% Reduced}  \\
    \hline
    $\mathit{natF}$(7) & 23 & 7 & 39 & \lstinline$simplenat$ & 3 & 33 & 91\%  \\
    $\mathit{boolF}$(4) & 9 & 4 & 17 & \lstinline$simplebool$ & 3 & 16 & 81\% \\
    $\mathit{compF}$(4) & 12 & 4 & 20 & \lstinline$natbool$ & 5 & 74 & 93\% \\
    $\mathit{logicF}$(4) & 12 & 4 & 20 & \lstinline$varbool$ & 4 & 24 & 83\% \\
    $\mathit{varF}$(4) & 7 & 4 & 7 & \lstinline$varnat$ & 4 & 41 & 90\% \\
    $\mathit{funcF}$(3) & 10 & 3 & 9 & \lstinline$simplelogic$ & 4 & 28 & 86\% \\
     & & & & \lstinline$varlogic$ & 6 & 36 & 83\% \\
     & & & & \lstinline$arith$ & 8 & 94 & 91\% \\
     & & & & \lstinline$arithlogic$ & 8 & 114 & 93\% \\
     & & & & \lstinline$vararith$ & 8 & 107 & 93\% \\
     & & & & \lstinline$vararithlogic$ & 8 & 127 & 94\% \\
     & & & & \lstinline$mini-JS$ & 33 & 149 & 78\% \\
    \hline
    \textbf{Total} & & & 237 & & 331 & 843 & 61\% \\
    \hline
  \end{tabular}
  \end{small}
  \caption{SLOC statistics: \sedel implementation vs. vanilla AST implementation}
  \label{fig:sloc}
\end{table}

  
%%%%%%%%%%%%%%%%%%%%%%%%%%%%%%%%%%%%%%%%%%%%%%%%%%%%%%%%%%%%%%%%%%%%%%%%
\chapter{Related Work}
%%%%%%%%%%%%%%%%%%%%%%%%%%%%%%%%%%%%%%%%%%%%%%%%%%%%%%%%%%%%%%%%%%%%%%%%



%%% Local Variables:
%%% mode: latex
%%% TeX-master: "../Thesis"
%%% End:

  %%%%%%%%%%%%%%%%%%%%%%%%%%%%%%%%%%%%%%%%%%%%%%%%%%%%%%%%%%%%%%%%%%%%%%%%
\chapter{Conclusion and Future Work}
\label{chap:conclusion}

%%%%%%%%%%%%%%%%%%%%%%%%%%%%%%%%%%%%%%%%%%%%%%%%%%%%%%%%%%%%%%%%%%%%%%%%


\section{Conclusion}




\section{Future Work}

In this section we discuss some areas where future research might extend and/or
complement the work described in this thesis.

\subsection{On Categorical Semantics}
\label{sec:category}

An interesting avenue for future work is to give a categorical semantics of
disjoint intersection types. The main reason for doing so is that, as
\citet{reynolds1988preliminary} nicely put it:
\begin{quote}
  ``by formulating succinct definitions in terms of a mathematical theory of
  great generality, we gain an assurance that our language will be uniform and
  general.''
\end{quote}
Using category theory as the basis for the type structure of a programming
language has a long history. \citet{lambek1985cartesian} discovered that
simply-typed lambda calculus can be interpreted in any Cartesian closed
category. \citet{Reynolds_1991} gives a category-theoretic presentation of a
lambda calculus extended to include records, fixed points and
intersection types, much similar to our \namee. Of particular interest to us is
his method for proving coherence. Let $[[D]]$ denote derivations of typing, then
the interpretation of a derivation $[[ D ; GG |- ee : A ]]$ is a morphism
$\bra{[[ D ; GG |- ee : A ]]} : \bra{[[GG]]} \rightarrow \bra{[[A]]} $ in a
suitable ``semantic'' category (i.e., being Cartesian closed and possessing
certain limits). Proving coherence in this presentation then amounts to
establishing the commutativity of all diagrams of the following
form\footnote{The proof actually needs a stronger inductive hypothesis.}:
\[
\begin{tikzcd}
\bra{[[  GG   ]]} \arrow[rrr, "\bra{[[ D1 ; GG |- ee : A  ]]}", bend left] \arrow[rrr, "\bra{[[ D2 ; GG |- ee : A  ]]} "', bend right] &  &  & \bra{[[ A ]]}
\end{tikzcd}
\]


\paragraph{Properties of Intersection Types.}

The key component of Reynolds' method is the interpretation of intersection
types. For the sake of precision in what follows, we pause to give some basic
properties of intersection types that are first proved by \citet{Reynolds_1991}.
First we give two definitions that are important for the discussion.

\begin{definition}[Type Equivalence]
  Two types $[[A]]$ and $[[B]]$ are equivalent, written $[[ A == B ]]$, when $[[ A <: B ]]$ and $[[B <: A]]$.
\end{definition}

\begin{definition}[Least Upper Bounds]
  A \textit{least upper bound} of $[[A]]$ and $[[B]]$, written $[[A =/ B]]$\footnote{Note
    that the meta-function $\sqcup$, unlike $\&$, is not a type constructor.},
  is a supertype of both $[[A]]$ and $[[B]]$ and a subtype of every common
  supertype of $[[A]]$ and $[[B]]$---i.e., a type $[[C]]$ such that:
  \begin{itemize}
  \item $[[A <: C]]$
  \item $[[B <: C]]$
  \item For any $[[C']]$, $[[A <: C']]$ and $[[B <: C']]$ implies $[[C <: C']]$
  \end{itemize}
\end{definition}

According to the subtyping rules in \cref{fig:subtype_decl}, we can derive the
following type equalities:

\begin{proposition} \label{prop:1}%
\begin{align*}
  [[A1 & (A2 & A3) ]]  &\approx  [[(A1 & A2) & A3]] \\
  [[ Top & A ]] &\approx [[A]] \\
  [[ A & Top ]] &\approx [[A]] \\
  [[A1 & A2 ]]  &\approx  [[ A2 & A1 ]] \\
  [[A & A ]]  &\approx  [[ A ]] \\
  [[ {l : A1 & A2}   ]] &\approx [[  {l : A1}  & {l : A2} ]] \\
  [[  A -> A1 & A2  ]] &\approx [[  (A -> A1) & (A -> A2)   ]] \\
  [[  {l : Top}    ]] &\approx [[  Top   ]] \\
  [[  A -> Top  ]] &\approx [[  Top   ]]
\end{align*}
\end{proposition}

It can be shown that every pair of types has a least upper bound (unique up to
$\approx$-equivalence). The following suffices to compute a least upper bound of
any types $[[A]]$ and $[[B]]$:

\begin{proposition} \label{prop:2}%
\begin{align*}
  [[  A =/ B   ]] &\approx [[B =/ A]] \\
  [[  A =/ Top   ]] &\approx [[Top]] \\
  [[  A1 =/ (A2 & A3)  ]] &\approx [[  (A1 =/ A2) & (A1 =/ A3)  ]] \\
  [[  pri =/ {l : A} ]] &\approx [[  Top  ]] \\
  [[  pri =/ (A1 -> A2) ]] &\approx [[  Top  ]] \\
  [[  {l : A} =/ (A1 -> A2) ]] &\approx [[  Top  ]] \\
  [[  {l : A1} =/ {l : A2} ]] &\approx [[  {l : A1 =/ A2}  ]] \\
  [[  {l1 : A1} =/ {l2 : A2} ]] &\approx [[  Top   ]] \quad \text{when} \ [[ l1 <> l2 ]] \\
  [[  (A1 -> A1') =/ (A2 -> A2') ]] &\approx [[  (A1 & A2) -> (A1' =/ A2')  ]]
\end{align*}
\end{proposition}


\paragraph{Connecting with Disjointness.}

With these properties stated, it turns out that our disjointness rules, as
given in \cref{fig:disjoint}, can be compactly formulated using $\approx$ and $\sqcup$:

\begin{theorem} \label{thm:disjoint_spec}
  $[[A ** B]]$ if and only if $[[   A =/ B == Top  ]]$.
\end{theorem}
\begin{proof}
  By induction on the derivation of disjointness. An interesting case is \rref{D-arr}
  \[
    \drule{D-arr}
  \]
  \begin{longtable}[l]{l|l}
    $[[A2 =/ B2 == Top]]$  & By i.h \\
    $[[  (A1 -> A2) =/ (B1 -> B2) ]] \approx [[(A1 & B1) -> (A2 =/ B2)]]$ & By \cref{prop:2} \\
    $[[  (A1 -> A2) =/ (B1 -> B2) ]] \approx [[(A1 & B1) -> Top]]$ & By above equality \\
    $[[(A1 & B1) -> Top == Top]]$  & By \cref{prop:1} \\
    $[[  (A1 -> A2) =/ (B1 -> B2) == Top]]$ & By above equality
  \end{longtable}
\end{proof}

\begin{remark}
  We can view \cref{thm:disjoint_spec} as a specification of disjointness.
\end{remark}


\paragraph{Interpretation of Intersection Types.}

Following Reynolds, a subtyping derivation is interpreted as a morphism $ \bra{[[ A <: B ]]} : \bra{[[A]]} \rightarrow \bra{[[B]]} $ with two requirements:
\begin{enumerate}
\item For all types $[[A]]$ the morphism from $ \bra{[[A]]}$ to $\bra{[[A]]}$ must be an identity arrow.
\item Whenever $[[A <: B]]$ and $[[ B <: C  ]]$, the composition of $\bra{[[ A <: B ]]}$ and $\bra{[[  B <: C   ]]}$ must be equal to $\bra{[[  A <: C  ]]}$, i.e., $ \bra{[[ A <: B ]]} ; \bra{[[  B <: C  ]]} = \bra{[[A <: C]]}$. (Here ``;'' denotes composition in diagrammatic order.)
\end{enumerate}
These requirements actually make $ \bra{\cdot} $ a functor from the
preordered set of types (viewed as a category) to the semantic category of
choice.

\begin{remark}
By definition, whenever $[[ A == B ]]$ we say $\bra{[[  A  ]]}$ is \textit{isomorphic} to $\bra{[[ B ]]}$, written $\bra{[[ A ]]} \cong \bra{[[B]]}$.
\end{remark}

Now we consider $\bra{[[ A1 & A2  ]]}$ in the following steps:
\begin{enumerate}
\item By \rref{S-andL,S-andR}, there must be two morphisms, $\bra{[[ pp1 ]]} : \bra{[[A1 & A2]]} \rightarrow \bra{[[A1]]}  $ and $\bra{[[pp2]]} : \bra{[[A1 & A2]]} \rightarrow \bra{[[A2]]}  $
  \[
\begin{tikzcd}
  \bra{[[A1]]} &  & \bra{[[A2]]} \\
  & \bra{[[A1 & A2]]} \arrow[lu, "\pi_1"'] \arrow[ru, "\pi_2"] &
\end{tikzcd}
  \]
\item For any types $[[A1]]$ and $[[A2]]$, there exists a least upper bound $[[
  A1 =/ A2 ]]$ (\cref{prop:2}), and two morphisms $\bra{[[A1 <: A1 =/ A2]]} : \bra{[[A1]]} \rightarrow \bra{[[A1 =/ A2]]}$
  and $\bra{[[A2 <: A1 =/ A2]]} : \bra{[[A2]]} \rightarrow \bra{[[A1 =/ A2]]}$, and the following diagram should commute:
  \[
\begin{tikzcd}
  & \bra{[[  A1 =/ A2 ]]} &  \\
  \bra{[[A1]]} \arrow[ru, "\bra{[[A1 <: A1 =/ A2]]}"] &  & \bra{[[A2]]} \arrow[lu, "\bra{[[A2 <: A1 =/ A2]]}"'] \\
  & \bra{[[A1 & A2]]} \arrow[lu, "\pi_1"'] \arrow[ru, "\pi_2"] &
\end{tikzcd}
  \]

\item For every type $[[A]]$ such that $[[A <: A1]]$ and $[[A <: A2]]$, \rref{S-and} implies that $[[A <: A1 & A2]]$, thus
  a morphism from $\bra{[[  A ]]}$ to $\bra{[[  A1 & A2  ]]}$. Call this $\mu_0$. The following diagram should commute:
  \[
\begin{tikzcd}
  & \bra{[[  A1 =/ A2 ]]} &  \\
  \bra{[[A1]]} \arrow[ru, "\bra{[[A1 <: A1 =/ A2]]}"] &  & \bra{[[A2]]} \arrow[lu, "\bra{[[A2 <: A1 =/ A2]]}"'] \\
  & \bra{[[A1 & A2]]} \arrow[lu, "\pi_1"'] \arrow[ru, "\pi_2"] & \\
  & \bra{[[A]]} \arrow[u, "\mu_0"] \arrow[luu, "\bra{[[A <: A1]]}"] \arrow[ruu, "\bra{[[A <: A2]]}"'] &
\end{tikzcd}
  \]
\item Furthermore, in the above diagram, we replace $\bra{[[A]]}$ by an
  arbitrary object $s$ and $\bra{[[A <: A1]]}$ and $\bra{[[A <: A1]]}$ by any
  morphisms $f_1$ and $f_2$ that make the outer diamond commutes, and we require
  the ``mediating morphism'' $\mu_0$ from $s$ to $\bra{[[A1 & A2]]}$ to be unique. Specifically,
  we define $\bra{[[A1 & A2]]}$ by requiring the following diagram must commute:
  \[
\begin{tikzcd}
  & \bra{[[  A1 =/ A2 ]]} &  \\
  \bra{[[A1]]} \arrow[ru, "\bra{[[A1 <: A1 =/ A2]]}"] &  & \bra{[[A2]]} \arrow[lu, "\bra{[[A2 <: A1 =/ A2]]}"'] \\
  & \bra{[[A1 & A2]]} \arrow[lu, "\pi_1"'] \arrow[ru, "\pi_2"] & \\
  & s \arrow[u, "\mu_0", dotted] \arrow[luu, "f_1"] \arrow[ruu, "f_2"'] &
\end{tikzcd}
  \]
\end{enumerate}
Thus we have defined $\bra{[[A1 & A2]]}$ to be the \textit{pullback} of
$\bra{[[A1]]}$, $\bra{[[A2]]}$ and $\bra{[[A1 =/ A2]]}$.

\paragraph{Interpretation of Disjoint Intersection Types.}

Given the interpretation of intersection types, it is fairly straightforward to
give the interpretation of disjoint intersection types. First recall that if
$[[A ** B]]$ then $[[ A =/ B == Top ]]$ (\cref{thm:disjoint_spec}). Also we have
$\bra{[[Top]]} = 1$---i.e., the terminal object. By specializing $\bra{[[A1 =/ A2]]}$ to be the terminal object
($\bra{[[A1 <: A1 =/ A2]]}$ and $\bra{[[A2 <: A1 =/ A2]]}$ are then uniquely
determined), then the pullback ``degrades'' to the \textit{product} of
$\bra{[[A1]]}$ and $\bra{[[A2]]}$. In other words, the interpretation of
disjoint intersection types is given by the following theorem:
\begin{theorem}
  If $[[A1 ** A2]]$ then $\bra{[[A1 & A2]]} \cong \bra{[[A1]]} \times \bra{[[A2]]} $.
\end{theorem}
\begin{remark}
It is reassuring to see that this theorem justifies our translation of
disjoint intersection types into product types, from the categorical
perspective.
\end{remark}



\paragraph{Coherence, from the categorical perspective?}

What we have developed so far is the (categorical) interpretation of disjoint intersection
types. We are still half way through the ultimate goal of (re-)establishing
coherence, now from the categorical perspective. The main difficulty is that we
do not know yet how to interpret bidirectional typing judgment---i.e., what are
$\bra{[[GG |- ee => A]]}$ and $\bra{[[GG |- ee <= A]]}$, and in particular the
interpretation of the merge operator. As we mentioned, bidirectional type checking
(besides disjointness) is essential to coherence. It would be exciting to see
some research along the lines of the above, so that we may have a solid
mathematical foundation for type systems with disjoint intersection types.

\subsection{On Implicit Polymorphism}
\label{sec:implicit}

Another interesting and practically useful extension is to study (predicative)
implicit polymorphism, in the spirit of Haskell. Our \fnamee calculus features
explicit polymorphism in the sense that we need to provide types during type
applications. A classic example of implicit polymorphism is the identity
function $[[\x . x]]$ of type $[[\X . X -> X]]$. When applied to $1$, for
example, the type variable $[[X]]$ will be implicitly instantiated to $[[nat]]$.
Moreover, we are interested in \textit{higher-rank polymorphism}, allowing
polymorphic quantifiers to appear anywhere in a type. There are several
approaches in the literature~\citep{odersky1996putting, dunfield2013complete,
  jones2007practical}. Since our declarative type system is already based on
bidirectional type-checking, the work by \citet{dunfield2013complete} is
particularly relevant for us. It turns out that coming up with a coherent
declarative system is already very challenging, especially the disjointness
relation. Below we sketch out some ideas for the initial design.

\paragraph{Declarative Subtyping.}

First we consider the subtyping rules. Obviously \rref{S-forall} needs to be modified.
We replace it with the following two rules:
\begin{mathpar}
  \drule{IS-allL}
  \drule{IS-allR}
\end{mathpar}
\Rref{IS-allL} says that a type $[[\X ** A1 . A2]]$ is a subtype of $[[B]]$ if
some instantiation $[[ [t / X] A2 ]]$ is a subtype of $[[B]]$. However, unlike
\citeauthor{dunfield2013complete}'s system, in our setting, not every monotype
$[[t]]$ works---those that do not respect the disjointness constraints should
not be considered, for the sake of coherence.
Otherwise, we would allow $[[ ((\x . x ,, 2) : \X ** nat . X -> X & nat) 1 ]]$ to type check,
which would be a disaster.
\Rref{IS-allR} says that $[[A]]$
is a subtype of $[[\X ** B1. B2]]$ if we can show that $[[A]]$ is a subtype of
$[[B2]]$ in a context extended with $[[X ** B1]]$. It is not immediately obvious
that these two rules subsume \rref{S-forall}, and in particular what happens to ``a universal quantifier is contravariant in its
disjointness constraints'', which is very important in the original subtyping.
It can be shown that they do subsume \rref{S-forall}, as is evident by the
following derivation:
\[
\inferrule*[right=\rref*{IS-allR}]{  \inferrule*[right=\rref*{IS-allL}]{ \inferrule*[right=\rref*{FD-tvarL}]{ [[  A2 <: A1  ]]    }{[[  X ** A2 |- X ** A1  ]]}  \\ [[  X ** A2 |- B1 <: B2  ]]   }{[[X ** A2 |- \X ** A1. B1 <: B2]]}    }{ [[  empty |- \X ** A1. B1 <: \X ** A2 . B2  ]] }
\]


\paragraph{Disjointness.}

The disjointness relation needs a major overhaul. For instance, subtyping allows
$[[ \X ** char . X -> X <: nat -> nat ]]$, and as such, $[[ \X ** char . X -> X]]$
is no longer disjoint with $[[nat -> nat]]$, whereas $[[ \X ** nat . X -> X]]$
is disjoint with $[[nat -> nat]]$. A seemingly intuitive rule is as follows:
\begin{mathpar}
  \inferrule*[lab=FD-implicit]{[[DD |- t1 ** A1]] \\  [[DD |- [t1 / X] A2 ** B2]]  }{  [[DD |- \ X ** A1 . A2 ** B2]]  }
\end{mathpar}
In the above rule, the monotype $[[t1]]$ is existentially quantified: it
suffices to exhibit a disjointness derivation of $[[  [t1 / X]  A2  ]]$ and $[[B2]]$
for one monotype in order to build a disjointness derivation of $[[\X ** A1 . A2]]$ and $[[B2]]$.
Unfortunately, this rule is incorrect as we could guess a
``wrong'' $[[t1]]$. Take $[[ \X ** char . X -> X ]]$ for example: one
instantiation is $[[bool -> bool]]$, which is disjoint with $[[nat -> nat]]$.
But as we saw, this does not mean $[[ \X ** char . X -> X ]]$ is disjoint with
$[[nat -> nat]]$. Instead we should require \textit{all possible}
instantiations are disjoint with $[[B2]]$:
\begin{mathpar}
  \inferrule*[lab=FD-implicit]{ \forall [[t1]] .\ [[DD |- t1 ** A1]] \Longrightarrow [[DD |- [t1 / X] A2 ** B2]]  }{  [[DD |- \ X ** A1 . A2 ** B2]]  }
\end{mathpar}
The universal rule is very convenient as an elimination form: if we have a
evidence of the disjointness between a polymorphic type and another type, we can
immediately obtain the knowledge that all suitable instantiations of the former
are disjoint with the latter. However, the universal rule is very hard to use as
an introduction rule: it requires us to inspect every possible instantiation;
it is getting even worse when we consider two polymorphic types. We do not
yet fully know all the consequences of this rule. Another idea is perhaps we
should focus on the opposite side---i.e., what constitutes a non-disjointness
relation. But this idea seems more radical.


\paragraph{Declarative Typing.}

Putting disjointness aside, now we consider the typing rules. Most of the rules
stay the same. We remove \rref{FT-tabs,FT-tapp}, since the syntax now does not
include type abstractions and type applications. We add one rule:
\[
  \drule{FT-gen}
\]
\Rref{FT-gen} says that $[[ee]]$ has type $[[\X ** A . B]]$ if $[[ee]]$ has type $[[B]]$ in a context extended with $[[ X ** A  ]]$.
Application becomes a little more complex:
\[
  \drule{FT-appI}
\]
The problem is that the inferred type $[[A]]$ for $[[ee1]]$ could be a
polymorphic quantifier.
We need to eliminate universals until we
reach an arrow type. To achieve this, we use a matching judgment $[[DD |- A tri A1 -> A2]]$,
which says that we can synthesize an arrow type $[[A1 -> A2]]$ from $[[A]]$.
Once we get an arrow type $[[A1 -> A2]]$, we use $[[A1]]$ to check against $[[ee2]]$.
The matching judgment~\citep{siek2015refined, xie2018consistent}, first used in gradual type systems, is inductively defined as follows:
\begin{mathpar}
  \drule{m-forall}
  \drule{m-arr}
\end{mathpar}
\Rref{M-forall}, as with \rref{IS-allL},
works by guessing instantiations of polymorphic quantifiers with the requirement
that the monotype $[[t]]$ must meet the disjointness constraints. \Rref{M-arr}
is trivial, returning $[[A1 -> A2]]$ as it is.



Of course the above is only a sketch; we have not studied the declarative system in full,
nor its metatheory. One potential problem is that now subtyping and
disjointness are mutually recursive (e.g., \rref{IS-allL} uses disjointness and
\rref{FD-tvarL} uses subtyping), which might pose difficulty in terms of
formalization. For coherence, we estimate that the proof method described in
this thesis should still work.



\paragraph{Algorithmic System.}

Having a declarative system is only a start. The major challenge is the
corresponding algorithmic system. It is known that full type inference is
undecidable for intersection types. Some restrictions are obviously in order,
leading to different points in the design space in terms of how much can be
inferred. We are interested to see some research into the algorithmic system.


\subsection{Disjoint Polymorphism vs. Row Polymorphism}

Row polymorphism, first proposed by \citet{wand1987complete}, was intended as a
mechanism to enable type inference for a simple object-oriented language based
on recursive records. These ideas were later adopted into type systems for
extensible records~\citep{Harper:1991:RCB:99583.99603, gaster1996polymorphic}.
As we have alluded to in \cref{sec:merge}, row polymorphism alone cannot express
the \lstinline{merge} function. It would be interesting to study the
relationship between disjoint polymorphism and row polymorphism, and in
particular, whether the former subsumes the latter. As noted by
\citet{alpuimdisjoint}, disjoint polymorphism can already encode polymorphic
extensible records. For the sake of comparison, we pick the record calculus
$\lambda^{\|}$ of \citet{Harper:1991:RCB:99583.99603}---an explicitly-typed,
second-order calculus that features single-field records and a symmetric merge
operator. In $\lambda^{\|}$, \textit{compatibility constraints} are used to
capture negative information about fields. For example, $r_1 \# r_2 $ denotes the
assertion that the record types $r_1$ and $r_2$ have disjoint sets of labels. To
illustrate polymorphic extensible records in $\lambda^{\|}$,
\citeauthor{Harper:1991:RCB:99583.99603} show a function that takes two
``disjoint'' records $x_1$ and $x_2$, where $x_1$ has at least a field $l_1$ of
type $[[nat]]$ and $x_2$ has at least a field $l_2$ of type $[[nat]]$, and
returns the result of merging $x_1$ and $x_2$ (altering their syntax slightly):
\begin{align*}
  \Lambda \alpha_1 \# (\{ l_1 : [[nat]] \}, \{ l_2 : [[nat]] \}) .\  \Lambda \alpha_2 \# (\alpha_1 , \{l_1 : [[nat]]\} , \{ l_2 : [[nat]]   \}) . \\
  \qquad \lambda x_1 : (\alpha_1 \| \{ l_1 : [[nat]] \}) .\  \lambda x_2 : (\alpha_2 \| \{ l_2 : [[nat]] \}) .\ x_1 \| x_2
\end{align*}
where $r_1 \| r_2$ is the record type obtained by merging $r_1$ and $r_2$, and
is only defined if $r_1 \# r_2$. The same operator is overloaded to merge two
records on the term level. Central to their system is the \textit{constrained
  quantification} $\forall \alpha \# R .\ t $, where each record type variable is
associated with a list of \textit{compatibility assumptions} $R$, whose elements
are record types (including record type variables). The \textit{constrained type abstraction} $\Lambda \alpha \# R.\ e$
is used to create values of constrained quantification.


In \fnamee, we can use disjoint quantification to express their constrained
qualification, intersection types to merge record types, and the merge operator
to merge records. The function mentioned above can be written in \fnamee as
follows:
\begin{align*}
  \Lambda (\alpha_1 * [[ {l1 : nat} & {l2 : nat} ]]) .\  \Lambda (\alpha_2 * [[  X1 & {l1 : nat} & {l2 : nat} ]]) . \\
  \qquad [[\x1 : X1 & {l1 : nat} . \x2 : X2 & {l2 : nat} . x1 ,, x2]]
\end{align*}
However, the merge operator in \fnamee is more general than its counterpart in
$\lambda^{\|}$---i.e., it works on any expressions, not just records. Another
important difference is that their compatibility judgment $r_1 \# r_2$
effectively implies that their records must have distinct fields, whereas
\fnamee accepts duplicate fields as long as their types are disjoint. On a
related note, $\lambda^{\|}$ is powerful enough to express a polymorphic,
conflict-free function that merges two records of statically-unknown fields:
\[
  \mathsf{mergeRcd} = \Lambda \alpha_1 \# \mathsf{Empty} .\ \Lambda \alpha_2 \# \alpha_1 .\ \lambda x_1 : \alpha_1 .\ \lambda x_2 : \alpha_2 .\ x_1 \| x_2
\]
where $\mathsf{Empty}$ is the empty record type. Compare it to our ``more expressive'' $\mathsf{mergeAny}$ function:
\[
  \mathsf{mergeAny} = [[\ X1 ** Top . \ X2 ** X1 . \x1 : X1 . \x2 : X2 . x1 ,, x2 ]]
\]
that merges \textit{any} two expressions of statically-unknown types.

We believe \fnamee completely subsumes $\lambda^{\|}$. To support this claim, we
need to show a bisimulation property between $\lambda^{\|}$ and (a subset of)
\fnamee:
\begin{inparaenum}[(1)]
\item the translation from $\lambda^{\|}$ to \fnamee is type-safe (i.e., it type check in \fnamee),
\item and the translation from a subset of \fnamee to $\lambda^{\|}$ is type-safe (i.e., it type check in $\lambda^{\|}$).
\end{inparaenum}


\subsection{Recursive Types}

One extension of particular importance for modeling object-oriented languages is
\textit{recursive types}. A great deal of lessons have been learned about
calculi with recursive types and subtyping (see \citet[chap.
20]{DBLP:books/daglib/0005958}). But previous work has been focused on type
systems with substantially simpler subtyping relations. For simplicity, we are
interested in adding \textit{iso-recursive types}, where a recursive type $[[mu XX . A]]$\footnote{We use $[[XX]]$ to distinguish from type variables introduced by universal quantifiers.} and its one-step unfolding are transformed back and forth by a pair
of functions $\mathsf{fold}$ and $\mathsf{unfold}$. The most common definition
of iso-recursive subtyping is the \textit{Amber rule}, popularized by
\citeauthor{DBLP:conf/litp/Cardelli85}'s Amber language~\citep{DBLP:conf/litp/Cardelli85}.
\begin{mathpar}
  \drule{RS-amber}
  \drule{RS-var}
\end{mathpar}
\Rref{RS-amber} says that to show $[[mu XX . A]] $ is a subtype of $ [[mu YY . B]]$
under some set of assumptions $[[SS]]$, it suffices to show $[[A <: B]]$ under
the additional assumption $[[XX]] <: [[YY]]$. \Rref{RS-var} allows us to
conclude $[[XX]] <: [[YY]]$ if the assumptions assume it.

While adding the above two rules to our subtyping relation in
\cref{fig:subtype_decl} (and extending the other rules so that they pass
$[[SS]]$ through from premises to conclusion) effectively yields a declarative
subtyping relation with recursive types, it is not entirely straightforward as
to how they affect disjointness, and in particular, under what conditions are
two recursive types disjoint. An initial attempt shows that the amber rule and
the disjointness rule for functions are in conflict.

\paragraph{The Problem.}

For the ease of discussion, we do not consider top types, polymorphic types, or
BCD subtyping; then a guiding principle of designing disjointness rules is the
simple disjointness specification (\cref{def:disjoint_spec}): two types are
disjoint if and only if they share no common supertypes. Now consider two
recursive types $[[ mu XX . XX -> nat ]]$ and $[[ mu YY . YY -> nat]]$. It is
not hard to see that they have no common supertypes (because of contravariance of function argument subtyping). According to
\cref{def:disjoint_spec}, they are disjoint. On the other hand, since the
disjointness relation is structural, we should inspect the disjointness relation
between $[[ XX -> nat ]]$ and $[[YY -> nat ]]$ under certain relation over
$[[XX]]$ and $[[YY]]$ we do not know yet. However, according to \rref{D-arr}, two functions are
disjoint if their range types are disjoint; thus $[[ XX -> nat ]]$ and $[[YY ->
nat ]]$ are not disjoint. So we have $[[ mu XX . XX -> nat]]$ and $[[ mu YY . YY -> nat]]$
are \textit{not} disjoint: a contradiction!


\paragraph{Positivity to the Rescue.}

It is not obvious how to change either \rref{RS-amber} or \rref{D-arr} without
disrupting the whole system. A possible solution is to restrict where type variables
can occur. Instead of
having a \textit{general} recursive type $[[mu XX . A]]$ where $[[XX]]$ may occur
anywhere in $[[A]]$, we require that $[[XX]]$ occurs \textit{positively} in
$[[A]]$. Specifically, $[[XX]]$ occurs positively in $[[A1 -> A2]]$, if
\begin{inparaenum}[(1)]
\item $[[XX]]$ \textit{does not occur} in $[[A1]]$,
\item and $[[XX]]$ occurs positively in $[[A2]]$.
\end{inparaenum}
In general, any occurrences of $[[XX]]$ within the domain of a function type are
\textit{negative occurrences}, whereas any occurrences of $[[XX]]$ within the
range of a function type are \textit{positive occurrences}. For example, the two
recursive types in the last paragraph are not positive. While positivity does
limit the expressiveness of types, most useful datatypes (e.g., natural numbers,
lists, streams) are positive. For us, the positivity restriction for recursive types does work with the
disjointness rule for function types.

With positive recursive types, here is the disjointness rule for recursive types:
\[
  \drule{D-rec}
\]
We also need a few more disjointness axioms:
\begin{mathpar}
  \drule{Dax-intRec}
  \drule{Dax-rcdRec}
  \drule{Dax-arrRec}
  \drule{Dax-intRVar}
  \drule{Dax-rcdRVar}
  \drule{Dax-arrRVar}
  \drule{Dax-recRVar}
\end{mathpar}
An important observation is that any two type variables are \textit{not}
disjoint. A few examples: $[[mu XX . nat -> XX]]$ and $[[mu YY . nat -> YY]]$
are not disjoint; $[[mu XX . nat -> XX]]$ and $[[mu YY . bool -> YY & nat]]$ are
not disjoint; $[[mu XX . nat -> XX]]$ and $[[mu YY . nat -> nat -> YY]]$ are
disjoint. Note that the above is only a sketch; it remains to see whether the disjointness
rules are equivalent to the specification.


Another gnarly issue is coherence. To model recursive types, we need to turn to
step-indexed logical relations~\citep{ahmed2006step}. We foresee it would be a
major technical challenge to adjust our coherence proof and its Coq
mechanization.

\subsection{Other Extensions}

There are several important extensions that should also be considered.


\paragraph{Union Types.}

Union types are dual to intersection types.



\paragraph{Nominal Typing.}

\paragraph{Mutable State.}






%%% Local Variables:
%%% mode: latex
%%% TeX-master: "../Thesis"
%%% org-ref-default-bibliography: ../Thesis.bib
%%% End:


% This ensures that the subsequent sections are being included as root
% items in the bookmark structure of your PDF reader.
\bookmarksetup{startatroot}
\backmatter

  \begingroup
    \let\clearpage\relax
    \glsaddall
    \printglossary[type=\acronymtype]
    \newpage
    \printglossary
  \endgroup

  \printindex
  \printbibliography

\end{document}
