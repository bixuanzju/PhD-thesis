
%%%%%%%%%%%%%%%%%%%%%%%%%%%%%%%%%%%%%%%%%%%%%%%%%%%%%%%%%%%%%%%%%%%%%%%%
\chapter{Coherence for \namee}
\label{chap:coherence:simple}
%%%%%%%%%%%%%%%%%%%%%%%%%%%%%%%%%%%%%%%%%%%%%%%%%%%%%%%%%%%%%%%%%%%%%%%%

This chapter constructs a logical relation to
establish coherence of \namee. Finding a
suitable definition of coherence for \namee is already challenging in its own
right. In what follows we reproduce the steps of finding a definition for coherence
that is both intuitive and applicable. Then we present the
construction of the logical (equivalence) relation tailored to this
definition, and the connection between logical equivalence and coherence.
\Cref{chap:coherence:poly} builds on the idea in this chapter to prove coherence for
\fnamee.

%-------------------------------------------------------------------------------
\section{Overview}

Coherence is easy to establish for \oname as its rigid rules mean that there is
at most one possible subtyping derivation between any two types.  As a
consequence there is only one possible elaboration and thus one
possible behavior for any program.

Two factors make establishing coherence for \namee much more difficult: the
relaxation of disjointness and the adoption of the more expressive subtyping
rules from the BCD system (for which \oname lacks). These two factors mean that subtyping proofs are no
longer unique and hence that there are multiple elaborations of the same source
program. For instance, $\inter{\mathsf{Int}}{\mathsf{Int}}$ is a subtype of $\mathsf{Int}$ in two
ways: by projection on either the first or second component.
Hence the fact that all elaborations yield the same result when evaluated has
become a much more subtle property that requires sophisticated reasoning.
For instance, in the example, we can see that coherence holds because  at
runtime any value of type $\inter{\mathsf{Int}}{\mathsf{Int}}$ has identical components, and
thus both projections yield the same result.

For \namee in general, we show coherence by capturing the non-ambiguity
invariant in a logical relation and showing that it is preserved by the
operational semantics. A complicating factor is that not one, but two languages
are involved: the source language \namee and the target language. In order to
deal with the complexity of the elaboration semantics of \namee, we employ
binary logical relations that are heterogeneous, parameterized by two types; the
fundamental property is also reformulated to account for bidirectional
type-checking. The logical relation does not hold for target programs and
program contexts in general, but only for those that are the image of a
corresponding source program or program context. Thus we must view everything
through the lens of elaboration.



\section{In Search of Coherence}

In \oname the definition of coherence is based on
$\alpha$-equivalence. More specifically, their coherence property states that
any two target terms that a source expression elaborates into must be exactly the same (up to
$\alpha$-equivalence). Unfortunately this syntactic notion of coherence is
very fragile with respect to extensions.
For example, it is not obvious how to retain this notion of coherence when adding more subtyping
rules such as those in \cref{fig:subtype_decl}.

If we permit ourselves to consider only the syntactic aspects of expressions,
then very few expressions can be considered equal. The syntactic view also conflicts
with the intuition that the significance of an expression lies in its
contribution to the \textit{outcome} of a computation~\citep{Harper_2016}.
Drawing inspiration from a wide range of literature on contextual
equivalence~\citep{morris1969lambda}, we want a context-based notion of
coherence. It is helpful to consider several examples before presenting the
formal definition of our new semantically-founded notion of coherence.

\begin{example} \label{eg:1}
The same \namee expression $3$ can be typed $[[nat]]$ in many ways: for instance, by \rref{T-lit}; by
\rref{T-sub,S-refl}; or by \rref{T-sub,S-trans,S-refl}, resulting in \tname
terms $3$, $\app{[[id]]}{3}$ and $\app{([[id o id]])}{3}$, respectively. It is apparent
that these three \tname terms are ``equal'' in the sense that all reduce to the
same numeral $3$.
\end{example}

\subsection{Expression Contexts and Contextual Equivalence.}

To formalize the intuition, we introduce the notion of \textit{expression
  contexts}. An expression context $[[cc]]$ is a term with a single hole
$[[__]]$ (possibly under some binders) in it. The syntax of \tname expression
contexts can be found in \cref{fig:contexts}. The only operation of expression
contexts is \textit{replacement}, which is the process of filling a hole in an
expression context $[[cc]]$ with an expression $[[e]]$, written $[[ cc{e} ]]$.
An important difference is that replacement \textit{is not} substitution, that
is, the free variables of $[[e]]$ that are exposed by $[[cc]]$ are captured by
replacement. The static semantics of \tname is extended to expression contexts
by defining the typing judgment
\[
  [[cc : (gg |- T) ~~> (gg' |- T')]]
\]
where $([[gg |- T]])$ indicates the type of the hole. This judgment is
inductively defined so that if $[[gg |- e : T]]$, then $[[gg' |- cc{e} : T']]$.
We define a \textit{complete program} to mean any closed term of type $[[nat]]$.
The following two definitions capture
the notion of \textit{contextual equivalence}.

\begin{figure}[t]
  \centering
\begin{tabular}{llll}\toprule
  \tname contexts & $[[cc]]$ & $\Coloneqq$ & $[[__]] \mid [[\x . cc]] \mid [[cc e]] \mid [[e cc]] \mid [[< cc, e >]] \mid [[< e, cc >]] \mid [[c cc]] $ \\
  \namee contexts & $[[CC]]$ & $\Coloneqq$ & $[[__]] \mid [[\x . CC]] \mid [[CC ee]] \mid [[ee CC]] \mid [[ee ,, CC]] \mid [[CC ,, ee]] \mid [[CC : A]] $ \\
  & & $\mid$ & $ [[ { l = CC } ]] \mid [[CC.l]]$ \\ \bottomrule
\end{tabular}
  \caption{Expression contexts of \tname and \namee}
  \label{fig:contexts}
\end{figure}

\begin{definition}[Kleene Equality]
  Two complete programs, $[[e]]$ and $[[e']]$, are Kleene equal, written
  $\kleq{[[e]]}{[[e']]}$, if there exists $i$ such that $[[e -->> ii]]$ and $[[e' -->> ii]]$.
\end{definition}

\begin{definition}[\tname Contextual Equivalence] \label{def:cxtx}
  \begin{align*}
    [[gg |- e1 ~= e2 : T]]  & \defeq [[gg |- e1 : T]] \land [[gg |- e2 : T]] \ \land \\
                                 & \qquad \forall [[cc]].\ [[cc : (gg |- T) ~~> (empty |- nat)]]  \Longrightarrow \kleq{[[cc{e1}]]}{[[cc{e2}]]}
  \end{align*}
\end{definition}

Regarding \cref{eg:1}, it seems adequate to say that $3$ and
$\app{[[id]]}{3}$ are contextually equivalent. Does this imply that coherence
can be based on \cref{def:cxtx}? Unfortunately it cannot, as
demonstrated by the following example.


\begin{example} \label{eg:2} It may be counter-intuitive that two \tname terms
  $[[\x . pp1 x]]$ and $[[\x . pp2 x]]$ should also be considered equal. To see
  why, first note that they are both the translations of the same \namee expression:
  $[[(\x . x) : nat & nat -> nat]]$. What can we do with this lambda
  abstraction? We can apply it to $1 : [[nat & nat]]$ for example, which leads to
  two translations $\app{[[(\x . pp1 x)]]}{\pair{1}{1}}$ and $\app{[[(\x . pp2 x)]]}{\pair{1}{1}}$.
  It is obvious that both reduce to the same numeral $1$. However, $[[\x . pp1 x]]$ and $[[\x . pp2 x]]$
  are definitely not equal according to \cref{def:cxtx}, as one can find a
  context $\app{[[__]]}{\pair{1}{2}}$ where the two terms reduce to two
  different numerals.
  The problem is that
  % not every well-typed \tname term
  % can be obtained from a well-typed \namee expression through the
  % elaboration semantics. For example,
  $\app{[[__]]}{\pair{1}{2}}$ should not be considered because the
  (non-disjoint) source expression $\mer{1}{2}$ is rejected by the type system
  of the source calculus \namee and thus never gets elaborated into $\pair{1}{2}$.
\end{example}




\subsection{\namee Contexts and Refined Contextual Equivalence.}

\cref{eg:2} hints at a shift from \tname contexts to \namee contexts $[[C]]$,
whose syntax is shown in \cref{fig:contexts}. Due to the bidirectional
nature of the type system, the typing judgment of $[[C]]$ features 4
different forms:
\begin{mathpar}
  [[CC : (GG => A) ~> (GG' => A') ~~> cc]] \and
  [[CC : (GG <= A) ~> (GG' => A') ~~> cc]] \and
  [[CC : (GG => A) ~> (GG' <= A') ~~> cc]] \and
  [[CC : (GG <= A) ~> (GG' <= A') ~~> cc]]
\end{mathpar}
We write $[[CC : (GG dir A) ~> (GG' dir' A') ~~> cc]]$ to abbreviate the above 4
different forms. Take $[[CC : (GG => A) ~> (GG' => A') ~~> cc]]$ for example
(whose typing rules are shown in \cref{fig:ctyp}), it reads that if
$[[GG |- ee => A]]$, then $[[GG' |- CC{ee} => A']]$. The judgment also generates
a \tname context $[[cc]]$ so that $[[cc : (|GG| |- |A|) ~~> (|GG'| |- |A'|)]]$
holds by construction. The full typing rules appear in the appendix. Now we are
ready to refine \cref{def:cxtx}'s contextual equivalence to take into
consideration both \namee and \tname contexts.


\begin{figure}
  \centering
\drules[CTyp]{$[[CC : ( GG => A ) ~> ( GG' => B ) ~~> cc]]$}{Context typing I}{emptyOne, appLOne, appROne, mergeLOne, mergeROne, rcdOne, projOne, annoOne}
\caption{\namee context typing (excerpt)}
\label{fig:ctyp}
\end{figure}



\begin{definition}[\namee Contextual Equivalence] \label{def:cxtx2}
  \begin{align*}
    [[GG |- ee1 ~= ee2 : A]]  & \defeq \forall [[e1]], [[e2]].\  [[GG |- ee1 => A ~~> e1]] \land [[GG |- ee2 => A ~~> e2]] \ \land \\
                                & \qquad \forall [[C]], [[cc]]. \ [[CC : (GG => A) ~> (empty => nat) ~~> cc]]   \\
                                & \qquad \Longrightarrow \kleq{[[cc{e1}]]}{[[cc{e2}]]}
  \end{align*}
\end{definition}


\begin{remark}
  That is, two source expressions are contextually equivalent if their
  translations are equivalent in any source context. For brevity we only
  consider expressions in the inference mode. Our Coq formalization is complete
  with two modes.
\end{remark}

Regarding \cref{eg:2}, a possible \namee context is
\[
[[ __ 1 : (empty => nat & nat -> nat) ~> (empty => nat) ~~> __ <1 , 1>]]
\]
We can verify that both $[[\x . pp1 x]]$ and $[[\x . pp2 x]]$ produce $1$ in the context $[[__ <1 , 1>]]$.
Of course we should consider all possible contexts to be certain that they are truly equal. From now on, we
use the symbol $\backsimeq_{ctx}$ to refer to contextual equivalence in
\cref{def:cxtx2}. With \cref{def:cxtx2} we can formally state that \namee is coherent
in the following sense:

\begin{restatable}[Coherence]{theorem}{coherence} \label{thm:coherence}
  We have that
  \begin{itemize}
  \item If $[[GG |- ee => A ]]$ then $[[GG |- ee ~= ee : A]]$.
  \item If $[[GG |- ee <= A ]]$ then $[[GG |- ee ~= ee : A]]$.
  \end{itemize}
\end{restatable}

That is, coherence is just a special case of
\cref{def:cxtx2} where we set $[[ee1]]$ and $[[ee2]]$ to be the same source
expression. % For the same reason as in \cref{def:cxtx2}, we only
% consider expressions in the inference mode.
The rest of the chapter is devoted to proving that \cref{thm:coherence}
holds.

\section{Logical Relation, Formally Defined}

Intuitive as \cref{def:cxtx2} may seem, it is generally very hard to prove
contextual equivalence directly, since it involves quantification over
\textit{all} possible contexts. Worse still, two kinds of contexts are involved
in \cref{thm:coherence}, which makes reasoning even more tedious. The key to
simplifying the reasoning is to exploit types using logical
relations~\citep{tait, statman1985logical, plotkin1973lambda}.


\paragraph{In Search of a Logical Relation.}

\begin{figure}
  \centering
  \begin{tabular}{lll}
  $[[(v1 , v2) in V ( nat ; nat ) ]]$  & $\defeq$ & $\exists [[i]].\, [[v1]] = [[v2]] = [[ii]]$ \\
  $[[(v1, v2) in V ( {l : A}  ; {l : B} ) ]]$ & $\defeq$ & $[[ (v1, v2) in V ( A ; B ) ]]$\\
  $[[(v1 , v2) in V ( A1 -> B1 ; A2 -> B2 ) ]]$  & $\defeq$ & $\forall [[(v2' , v1') in V ( A2 ; A1 ) ]].\, [[ (v1 v1' , v2 v2') in E ( B1 ; B2 ) ]]$ \\
  $[[( < v1 , v2 > , v3  )  in V ( A & B ;  C  ) ]]$  & $\defeq$ & $[[ (v1, v3)  in V (A ; C) ]] \land [[ (v2, v3)  in V (B ; C) ]]$  \\
  $[[( v3 , < v1 , v2 >  )  in V ( C; A & B  ) ]]$  & $\defeq$ & $[[ (v3, v1)  in V (C ; A) ]] \land [[ (v3, v2)  in V (C ; B) ]]$  \\
  $[[(v1 , v2) in V (A; B) ]]$  & $\defeq$ & $\mathsf{true} \quad \text{otherwise}$ \\ \\
    $[[(e1, e2) in E (A; B)]]$ & $\defeq$ & $\exists [[v1]], [[v2]].\, [[e1 -->> v1]] \land [[e2 -->> v2]] \ \land $ \\
                                       & & $[[(v1, v2) in V (A; B)]]$
  \end{tabular}
  \caption{Logical relation for \namee}
  \label{fig:logical}
\end{figure}



It is worth pausing to ponder what kind of relation we are looking for. % From
% \cref{eg:2}, it is clear that pairs have a special status in \tname. Indeed they
% ought to be, since pairs originate from merges or subtyping. Also disjointness
% on intersection types should correspond to some sort of constraints over pairs.
The high-level intuition behind the relation is to capture the
notion of ``coherent'' values. These values are unambiguous in every (source) context. A
moment of thought leads us to the following observations:

\begin{observation}[Disjoint values are unambiguous] \label{ob:1}

  The relation should relate values originating from disjoint intersection
  types. Those values are essentially translated from merges, and since
  \rref{T-merge} ensures disjointness, they are unambiguous. For example, pairs
  of values of the types $[[nat]]$ and $[[ { l : nat}]]$ are always unambiguous
  in any source context.
\end{observation}

\begin{observation}[Duplication is unambiguous] \label{ob:2}

  The relation should relate values originating from non-disjoint intersection
  types, only if the values are duplicates. This may sound baffling, since the
  whole point of disjointness is to rule out (ambiguous) expressions such as
  $\mer{1}{2}$. However, $\mer{1}{2}$ never gets elaborated, and the only values
  corresponding to $[[nat & nat]]$ are those pairs such as $\pair{1}{1}$,
  $\pair{2}{2}$, etc. Those values are essentially generated from \rref{T-sub}
  and are also unambiguous.
\end{observation}

\Cref{fig:logical} defines the logical relation intended to capture
``coherent'' values based on the above observations. The value relation
$\valR{[[A]]}{[[B]]}$ relates \textit{closed} values, i.e., well-typed values
with no free variables. Similarly, the expression relation $\eeR{[[A]]}{[[B]]}$
relates closed expressions. For succinctness, we write $\valRR{[[A]]}$ to mean
$\valR{[[A]]}{[[A]]}$, and $\eeRR{[[A]]}$ for $\eeR{[[A]]}{[[A]]}$.


\begin{remark}
  Our logical relation is a family of binary relations over \tname values.
  indexed by two \namee types. The definition seeks to combine equality checking
  from traditional (homogeneous) logical relations with disjointness checking.
\end{remark}

% \begin{remark}
%   The logical relations resemble those given by Biernacki and
%   Polesiuk~\citep{biernacki2015logical}, as both are heterogeneous. However, two
%   important differences are worth pointing out. Firstly, our value relation for
%   product types ($\valR{[[T1 * T2]]}{[[T3]]}$ and $\valR{[[T3]]}{[[T1 * T2]]}$)
%   is unusual. Secondly, their value relation disallows relating functions with
%   natural numbers, while ours does not. As we explain shortly, both points are
%   related to disjointness.
% \end{remark}


First let us consider the relation $\valR{[[A]]}{[[B]]}$, which specifies when
two closed values $[[v1]]$ and $[[v2]]$ are related at the types $[[A]]$ and
$[[B]]$. The definition for integers and records are straightforward. Two
integers are related if they are equal. Two values are related at two
record types of the same label if they are related at the two field types.

Functions $[[v1]]$ and $[[v2]]$ are related at the types $[[A1 -> B1]]$ and
$[[A2 -> B2]]$ if given two arguments $[[v1']]$ and $[[v2']]$ related at the
argument types $[[A1]]$ and $[[A2]]$, the functions applied to the arguments are
related expressions at the result types $[[B1]]$ and $[[B2]]$. Note that in
\tname, the values $[[v1]]$ and $[[v2]]$ may each be a lambda abstraction, or a
coercion application of a function type.

% Two functions are related if they map
% related arguments to related results.  These
% cases reflect \cref{ob:2}: values of the same type are duplicates.


The definition of $\valR{[[A]]}{[[B]]}$ is made more interesting when one of the
indexed types is an intersection type. In that case, the relation distributes
over the type constructor $[[&]]$. It is instructive to compare it with product
types. The traditional way of relating pairs is by relating their components
pairwise. That is, $[[<v1,v2>]]$ and $[[<v1', v2'>]]$ are related if (1)
$[[v1]]$ and $[[v1']]$ are related and (2) $[[v2]]$ and $[[v2']]$ are related.
According to our definition, we also require that (3) $[[v1]]$ and $[[v2']]$ are
related and (4) $[[v2]]$ and $[[v1']]$ are related. To see the reason, consider
whether $(\pair{1}{2}, \pair{1}{2}) \in \valRR{[[nat & nat]]}$. If we regard
$[[nat & nat]]$ as a normal product type, then these two pairs are related.
However, as we mentioned, $\pair{1}{2}$ should not be considered as the image of
some source expression at the type $[[nat & nat]]$, and our definition correctly
rejects it because $1$ is not equal to $2$, while accepting pairs such as
$\pair{1}{1}, \pair{2}{2}$, etc.

The acute reader may have noticed the structural similarity between the
definition and the disjointness rules for intersection types:
\begin{mathpar}
 \drule{D-andL} \and \drule{D-andR}
\end{mathpar}
This is not a coincidence -- we can show that disjointness and the value relation
are connected by the following lemma:

\begin{lemma}[Disjoint values are related] \label{lemma:disjoint}
  If $[[A ** B]]$, $[[  v1 : |A|  ]]$ and
  $[[  v2 : |B|  ]]$,
  then $[[   (v1, v2) in V ( A ; B  )    ]]$.
\end{lemma}
\begin{proof}
  By induction on the derivation of disjointness.
\end{proof}

Next we consider $\eeR{[[A]]}{[[B]]}$, which is standard. Informally it
expresses that two closed terms $[[e1]]$ and $[[e2]]$ are related if
they evaluate to two values $[[v1]]$ and $[[v2]]$ that are related.



\paragraph{Logical Equivalence.}

The logical relation can be lifted to open terms in the usual way. First we
give the semantic interpretation of typing contexts:

\begin{definition}[Interpretation of value contexts]
  \begin{mathpar}
    \ottaltinferrule{}{}{  }{ [[(emp, emp) in empty ]]  } \and
    \ottaltinferrule{}{}{ [[(g1, g2) in GG  ]] \\ [[(v1, v2) in V (A) ]] }{ [[(g1 [ x -> v1 ] , g2 [ x -> v2 ]  )  in GG , x : A  ]] }
  \end{mathpar}
\end{definition}

Two open terms are related if every pair of related closing substitutions
makes them related:
\begin{definition}[Logical equivalence]
  \begin{align*}
    [[GG |- e1 == e2 : A ; B]] & \defeq [[|GG| |- e1 : |A|]] \land [[|GG| |- e2 : | B | ]] \ \land \\
                                 & \qquad \forall [[g1]], [[g2]] .\, [[(g1, g2) in GG ]] \Longrightarrow [[(g1 (e1), g2 (e2))  in E (A ; B) ]]
  \end{align*}
\end{definition}
For succinctness, we write $[[GG |- e1 == e2 : A]]$ to mean $[[GG |- e1 == e2 : A ; A]]$.


\section{Establishing Coherence}

With all the machinery in place, we are now ready to prove \cref{thm:coherence}.
But we need several lemmas to set the stage.

Firstly we need the compatibility lemmas, which state that logical equivalence is
preserved by language constructs. Most of them are standard and are thus omitted.
We show only two compatibility lemmas that are specific to our logical relation:

\begin{lemma}[Coercion Compatibility]   \label{lemma:co-compa}
  Suppose that $[[A1 <: A2 ~~> c]]$,
  \begin{itemize}
  \item If $[[GG |- e1 == e2 : A1 ; A0]]$ then $[[GG |- c e1 == e2 : A2 ; A0]]$.
  \item If $[[GG |- e1 == e2 : A0 ; A1]]$ then $[[GG |- e1 == c e2 : A0 ; A2]]$.
  \end{itemize}
\end{lemma}
\begin{proof}
  By induction on the subtyping derivation.
\end{proof}

\begin{lemma}[Merge compatibility]
  If $[[   GG |- e1 == e1' : A ]]$, $[[  GG |- e2 == e2' : B ]]$ and $[[ A ** B ]]$,
  then $[[   GG |- < e1, e2 > == <e1', e2'> : A & B ]]$.
\end{lemma}
\begin{proof}
  By the definition of logical relation and \cref{lemma:disjoint}.
\end{proof}



The ``Fundamental Property'' states that any well-typed expression is related to
itself by the logical relation. In our elaboration setting, we rephrase it so
that any two \tname terms elaborated from the \textit{same} \namee expression are related
by the logical relation. To prove it, we require \cref{thm:uniq}.

\begin{theorem}[Inference Uniqueness] \label{thm:uniq}
  If $[[GG |- ee => A1]]$ and $[[GG |- ee => A2]]$, then $[[A1]] \equiv [[A2]]$.
\end{theorem}

\begin{theorem}[Fundamental Property]  \label{thm:co-log} We have that:
  \begin{itemize}
  \item If $[[GG |- ee => A ~~> e]]$ and $[[GG |- ee => A ~~> e']]$, then $[[GG |- e == e' : A ]]$.
  \item If $[[GG |- ee <= A ~~> e]]$ and $[[GG |- ee <= A ~~> e']]$, then $[[GG |- e == e' : A ]]$.
  \end{itemize}
\end{theorem}
\begin{proof}
  The proof follows by induction on the first derivation. The most interesting
  case is \rref{T-sub}
  \begin{mathpar}
    \drule{T-sub}
  \end{mathpar}
  where we need \cref{thm:uniq} to be able to apply the induction hypothesis.
  Then we apply \cref{lemma:co-compa} to say that the coercion generated
  preserves the relation between terms. For the other cases we use the
  appropriate compatibility lemmas.
\end{proof}


We show that logical equivalence is preserved by \namee contexts:

\begin{lemma}[Congruence] \label{lemma:cong}
 If $[[CC : (GG dir A) ~> (GG' dir' A') ~~> cc]]$, $[[GG |- ee1 dir A ~~> e1]]$, $[[GG |- ee2 dir A ~~> e2]]$
 and $[[GG |- e1 == e2 : A]]$, then $[[GG' |- cc{e1} == cc{e2} : A']]$.
\end{lemma}
\begin{proof}
  By induction on the typing derivation of the context $[[C]]$, and applying
  the compatibility lemmas where appropriate.
\end{proof}


\begin{lemma}[Adequacy] \label{lemma:ade}
  If $[[  empty |- e1 == e2 : nat ]]$ then $\kleq{[[e1]]}{[[e2]]}$.
\end{lemma}
\begin{proof}
  Adequacy follows easily from the definition of the logical relation.
\end{proof}


Next up is the proof that logical relation is sound with respect to contextual
equivalence -- that is, if two programs are logically related then they are
contextually equivalent -- which justifies the use of logical relation for
proving contextual equivalence of programs.

\begin{theorem}[Soundness w.r.t. Contextual Equivalence] \label{thm:log-sound}
  Given $[[GG |- e1 == e2 : A]]$, we have
  \begin{itemize}
  \item If $[[ GG |- ee1 => A ~~> e1]]$ and $[[ GG |- ee2 => A ~~> e2]]$ then
    $[[ GG |- ee1 ~= ee2 : A ]]$.
  \item If $[[ GG |- ee1 <= A ~~> e1]]$ and $[[ GG |- ee2 <= A ~~> e2]]$ then
    $[[ GG |- ee1 ~= ee2 : A ]]$.
  \end{itemize}
\end{theorem}
\begin{proof}
  From \cref{def:cxtx2}, we are given a context $[[  CC : (GG => A) ~> (empty => nat) ~~> cc ]]$. By \cref{lemma:cong}
  we have $[[  empty |- cc{e1} == cc{e2} : nat  ]]$, thus $  \kleq{[[ cc{e1} ]]}{ [[cc{e2} ]]}    $ by \cref{lemma:ade}.
\end{proof}


Armed with \cref{thm:co-log} and \cref{thm:log-sound}, coherence follows directly.
\coherence*
\begin{proof}
  Immediate from \cref{thm:co-log} and \cref{thm:log-sound}.
\end{proof}

\section{Some Interesting Corollaries}

To showcase the strength of the new proof method, we can derive some
interesting corollaries. For the most part, they are direct consequences of
logical equivalence which carry over to contextual equivalence.


\cref{lemma:neutral} says that merging an expression $[[ee1]]$ of some type with
an arbitrary expression $[[ee2]]$ does not affect the semantics of $[[ee1]]$ at
the same type. \cref{lemma:commu} and \cref{lemma:assoc} express that merges are
commutative and associative, respectively. \cref{lemma:coercion_same} states
that coercions from the same types are ``coherent'', i.e., they can be used
interchangeably.

\begin{corollary}[Neutrality] \label{lemma:neutral}
  If $[[GG |- ee1 => A ]]$ and $[[GG |- ee1 ,, ee2 => A ]]$, then
  $[[GG |- ee1 ~= ee1 ,, ee2 : A]]$
\end{corollary}

\begin{corollary}[Commutativity] \label{lemma:commu}
  If $[[GG |- ee1 ,, ee2 => A ]]$ and $[[GG |- ee2 ,, ee1 => A ]]$, then
  $[[GG |- ee1 ,, ee2 ~= ee2 ,, ee1 : A]]$.
\end{corollary}


\begin{corollary}[Associativity] \label{lemma:assoc}
  If $[[GG |- (ee1 ,, ee2) ,, ee3 => A  ]]$ and $[[GG |- ee1 ,, (ee2 ,, ee3) => A ]]$, then
  $[[GG |- (ee1 ,, ee2) ,, ee3 ~= ee1 ,, (ee2 ,, ee3) : A]]$.
\end{corollary}

\begin{corollary}[Coercions Preserve Semantics]
  \label{lemma:coercion_same}
  If $[[A <: B ~~> c1]]$ and $[[A <: B ~~> c2]]$, then $[[GG |- \x . c1 x == \x . c2 x :  A  ->  B ]]$.
\end{corollary}


% Local Variables:
% TeX-master: "../../Thesis"
% org-ref-default-bibliography: ../../Thesis.bib
% End:
