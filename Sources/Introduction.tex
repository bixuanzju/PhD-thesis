
%%%%%%%%%%%%%%%%%%%%%%%%%%%%%%%%%%%%%%%%%%%%%%%%%%%%%%%%%%%%%%%%%%%%%%%%
\chapter{Introduction}
%%%%%%%%%%%%%%%%%%%%%%%%%%%%%%%%%%%%%%%%%%%%%%%%%%%%%%%%%%%%%%%%%%%%%%%%


% \begin{center}
%   \begin{minipage}{0.5\textwidth}
%     \begin{small}
%       In which the reasons for creating this package are laid bare for the
%       whole world to see and we encounter some usage guidelines.
%     \end{small}
%   \end{minipage}
%   \vspace{0.5cm}
% \end{center}


This thesis investigates disjoint intersection types---a variant of intersection
types---focusing on its theoretical foundation and applications in the context
of object-oriented programming. The results are three new typed calculi, the
first two being core calculi and the last one a source calculus, combining the
power of parametric polymorphism, a rich subtyping relation with the
fine-grained expressiveness of disjoint intersection types. The key contribution
of the thesis is that it unifies concepts that are seemingly unrelated but
powerful on their own in object-oriented programming---dynamic inheritance,
multiple inheritance, family polymorphism, extensible design patterns---by a
single lightweight mechanism, thus providing new insights into software
modularity and extensibility.

\section{Motivation}

Programs are hard to write. It was so 50 years ago at the time of the so-called
\textit{software crisis}~\citep{Naur:1969:SER:1102020}; it still remains so
nowadays, as the software we use daily is getting more and more complex and hard
to maintain. Over the years, we have learned---the hard way---that software
should be constructed in a \textit{modular} way, i.e., as a network of smaller
and loosely connected modules. To help programmers write modular code,
researchers and software practitioners have developed new methodologies; new
programming paradigms; more expressive type systems; as well as better tooling
support. Still, this is not enough to cope with today's needs. We will mention
some limitations of current mainstream languages shortly. But before that, let
us identify the following well-established requirements for construction of
modular software:
\begin{enumerate}
\item \textbf{Extensibility in both dimensions:} Extensions may require new
  variants to the datatype and new operations on the datatype.
\item \textbf{Strong static type safety:} Extensions cannot cause run-time type errors.
\item \textbf{No modification or duplication:} Existing code must not be
  modified nor duplicated.
\item \textbf{Separate compilation and type-checking:} Safety checks or
  compilation steps must not be deferred until linking or at run time.
\item \textbf{Independent extensibility:} Independently developed extensions
  should be composable so that they can be used jointly.
\item \textbf{Scalability:} Extension should be scalable. The amount of code
  needed should be proportional to the functionality added.
\item \textbf{Non-destructive extension:} The base system should still be
  available for use within the extended system.
\end{enumerate}
The first four of these requirements correspond to
\citeauthor{wadler1998expression}'s expression
problem~\citep{wadler1998expression}. \citet{Zenger-Odersky2005} added the fifth
requirement. The last two requirements were proposed by \citet{Nystrom:2006}.
Scalability (6) is often but not necessarily satisfied by separate compilation;
it is important for extending large software. Non-destructive extension (7) is
an important requirement for legacy and performance reasons: it enables clients
of the extended system to reuse code and data of the base system, allowing some
interoperability between new functionality and legacy code. To address the
requirements, many solutions have been proposed over the years (for example, see
\citet{oliveira2012extensibility, wang2016expression, oliveira09modular,
  swierstra_2008, Zenger-Odersky2005}, to cite a few). They differ considerably
in the language context with varying degrees of extensibility they offer, as
well as the limitations they impose. Building on the prior solutions, this
thesis proposes a lightweight language design that addresses all of these
requirements.

Various programming language features support modular programming, with varying
degrees of limitations. Functional languages, notably ML and OCaml, use module
systems~\citep{MacQueen_1984} for flexible program construction. In particular,
``functor''---which are functions over modules---allows
one to develop and compile a module independently from the modules on which it
depends. One functor can then be instantiated with multiple different modules
during the execution of the program, enabling a powerful form of code reuse. One
prominent weakness of ML modules (at least in current module implementations) is
that they cannot be defined recursively, that is, mutually recursive functions
and datatypes must be written in the same module, even though they may belong
conceptually to different modules. Another limitation is that modules form a
separate, higher-order functional language on top of the core and therefore ML
is actually two languages in one. Moreover, module systems usually put more
emphasis on supporting data abstraction, which adds considerable complexity to
languages adopting module systems as the primary way to construct modular
programs.
% Datatype-genetic programming is another
% approach to modularity, where programs can be parameterized over
% datatypes. By abstracting from the differences in what would otherwise be
% separate but similar programs, one can write a single unified piece of program
% and instantiate it in various ways to retrieve specific programs. However, this
% approach usually requires quite advanced type system features, e.g., type classes, type
% families, or even dependent types!

Object-oriented languages, on the other hand, use classes and inheritance
mechanism to support code extensibility and reuse. Single inheritance found in
mainstream object-oriented languages (such as Java and C++) is perhaps the most
well-known and well-studied mechanism. However, programmers have long realized
that single inheritance is not flexible enough when it comes to structuring a
class hierarchy: it works for small and simple extensions, but does not work
well for larger software systems such as compilers and operating systems. There
has been great interest in the past several years in mechanisms for providing
greater extensibility in object-oriented languages. Of particular relevance to
the subject of this thesis is three powerful linguistic mechanisms to foster software
extensibility, providing increasing order of flexibility, as well as complexity:
first-class classes~\citep{DBLP:conf/oopsla/TakikawaSDTF12}, (first-class)
mixins/traits~\citep{bracha1990mixin, scharli2003traits}, and family
polymorphism~\citep{Ernst_2001}, as we will briefly discuss below.


\subsection{First-Class Classes}

Many dynamically-typed languages (including JavaScript, Ruby, Python or Racket)
support \emph{first-class classes}. In those languages classes are first-class
values and, like any other values, they can be passed as an argument, or
returned from a function. Furthermore, first-class classes support \emph{dynamic inheritance}:
i.e., they can inherit from other classes at \emph{run time},
enabling programmers to abstract over the inheritance hierarchy. % For example,
% mixins~\citep{bracha1990mixin} become programmer-defined constructs---a mixin is
% simply a function that takes a class as an argument and returns a subclass.
In contrast, most statically-typed languages do not have first-class classes and
dynamic inheritance. While all statically-typed object-oriented languages allow first-class
\emph{objects} (i.e., objects can be passed as arguments and returned as
results), the same is not true for classes. Classes in languages such as Scala,
Java or C++ are typically a second-class construct, and the inheritance
hierarchy is \emph{statically determined}.

Despite the popularity and expressive power of first-class classes in
dynamically-typed languages, there is surprisingly little work on typing of
first-class classes. First-class classes and dynamic inheritance pose well-known
difficulties in terms of typing. For example, in his thesis,
\citet{bracha1992programming} comments several times on the difficulties of
typing dynamic inheritance and first-class mixins, and proposes the restriction
to static inheritance that is also common in statically-typed languages. One of
the motivations in this thesis is to present a type discipline that can encode
first-class classes. Moreover, we push this one-step further: for the first
rime, this thesis shows how to encode \textit{first-class mixins/traits}. But
first thing first, let us briefly explain what are mixins and traits.

% Racket~\citep{DBLP:conf/aplas/FlattFF06} supports typed first-class classes, as
% well as a \emph{dynamically-typed model} of first-class traits. However, unlike
% Racket's first-class classes and mixins, there is no type system supporting the
% use of first-class traits. A key difficulty is \emph{statically} detecting
% conflicts, even in the presence of polymorphism. Type system limitations prevent
% most statically-typed languages from having typed first-class traits.

\subsection{(First-Class) Mixins and Traits}

As remarked earlier, single inheritance is inadequate and inflexible to write
large software. To overcome this limitation, multiple
inheritance~\citep{cardelli1984semantics} was proposed as a generalization of
single inheritance. However, multiple inheritance is renowned for being tricky
to get right, largely because of the possible ambiguity issues that arise when
conflicting features are inherited along different paths.
Mixins~\citep{bracha1990mixin} provide a simple mechanism for multiple
inheritance without the ambiguity issue. A mixin is a class declaration
parameterized over a superclass, able to extend a variety of parent classes with
the same set of features. Mixins are composed linearly, and that methods defined
in mixins appearing later override all the identically named methods of earlier
mixins. Because of the linear ordering of composition, a class may not be able
to access a member of a given super-mixin because the member is overridden by
another mixin.

Traits~\citep{scharli2003traits, Ducasse_2006} are an alternative to mixins, and
other models of multiple inheritance. The key difference between traits and
mixins lies on the treatment of conflicts when composing multiple traits/mixins.
Mixins adopt an \emph{implicit} resolution strategy for conflicts, where the
compiler automatically picks one implementation in case of conflicts. Traits, on
the other hand, employ an \emph{explicit} resolution strategy, where the
compositions with conflicts are rejected, and the conflicts are explicitly
resolved by programmers. \citet{scharli2003traits} make a good case for the
advantages of the trait model. In particular, traits avoid bugs that could arise
from accidental conflicts that were not noticed by programmers. With the mixin
model, such conflicts would be silently resolved, possibly resulting in
unexpected run-time behavior due to a wrong method implementation choice. From a
modularity point-of-view, the trait model also ensures that composition is
\emph{commutative}, thus the order of composition is irrelevant and does not
affect the semantics. \citet{bracha1992programming} claims that ``\emph{The only
  modular solution is to treat the name collisions as errors...}'',
strengthening the case for the use of a trait model of composition. Otherwise,
if the semantics is affected by the order of composition (like in the mixin model), global knowledge about
the full inheritance graph is required to determine which implementations are
chosen.

Mixins and traits as found in most statically-typed languages/calculi are
typically a second-class construct. Promoting mixins/traits to first-class
citizens adds considerable expressiveness and flexibility in terms of software
extensibility, as will be illustrated throughout this thesis. Only recently some
progress has been made in statically typing first-class classes and dynamic
inheritance~\citep{DBLP:conf/oopsla/TakikawaSDTF12,DBLP:conf/ecoop/LeeASP15}.
However, prior to this thesis, we did not know how to model \textit{typed
  first-class traits}. A key challenge, compared to models with first-class
classes or mixins, is how to detect conflicts at compile time even when \textit{not}
knowing all components being composed statically. This is important because in
the setting with dynamic inheritance and polymorphism, the possibility of
accidental conflicts caused by programmers is extremely high.



\subsection{Family Polymorphism}

The last mechanism---also the most powerful and complex one---is \emph{family
  polymorphism}. In family polymorphism~\citep{Ernst_2001}, inheritance is
extended to work on a \emph{whole family of classes}, rather than just a single
class. This enables high degrees of modularity and code reuse, enabling simple
solutions to hard programming language problems, like the expression
problem~\citep{wadler1998expression}. An essential feature of family
polymorphism is \emph{nested composition}~\citep{Corradi_2012,
  ErnstVirtual,Nystrom_2004}, which allows the automatic inheritance/composition
of nested (or inner) classes when the enclosing classes are composed.
\citet{Nystrom_2004} call this \emph{scalable extensibility}: ``the ability to
extend a body of code while writing new code proportional to the differences in
functionality''.

Not many mechanisms that support family polymorphism are available in existing
mainstream languages. The \textsc{Cake} pattern~\citep{odersky2005scalable,
  Zenger-Odersky2005} in Scala provides some form of family polymorphism. In
order to model this modest form of family polymorphism, this pattern uses
\textit{virtual types}, \textit{self types}, \textit{path-dependent types} and
\textit{static mixin composition}. Even with so many sophisticated features,
composition of families is still quite heavyweight and manual. The problem is
due to the lack of \textit{deep} mixin composition. Though solutions do
exist~\citep{oliveira2013feature}, they usually require low-level type-unsafe
programming features such as dynamic proxies, reflection or other
meta-programming techniques. It is known that designing a sound type system that
fully supports family polymorphism and nested composition is notoriously hard;
there are only a few, quite sophisticated, research languages that manage
this~\citep{ErnstVirtual, Nystrom_2004,
  pubsdoc:tribe-virtual-calculus,SAITO_2007}. But these mechanisms usually focus
on getting a relatively complex Java-like language with family polymorphism. One
of the motivations in this thesis is to come up with a minimal calculus that
supports nested composition.



\section{Our Proposed Solution}

This thesis sets out to explore an alternative object-oriented language design
that makes extending and composing existing code more easily and safely on the
language level. More specifically, we seek to rein in ideas that are seemingly
unrelated but powerful in object-oriented programming---dynamic inheritance,
first-class traits, family polymorphism---under a simple unifying mechanism:
they are but different manifestations of a single underlying type discipline:
\textit{disjoint intersection types}. Through many examples and rigorous
analysis in this thesis, we hope to convince readers that disjoint intersection
types is a feasible semantic tool to facilitate code reuse and modularity. In
particular, for family polymorphism, we show that the combination of the
\textit{merge operator} and a rich subtyping relation captures the essence of
nested composition; for traits, we show that with the merge operator and
disjoint intersection types, we are able to express \textit{typed first-class
  traits}. Combined with the power of parametric polymorphism, we can further
express a very dynamic form of mixin-style compositions, enabling writing highly
modular and reusable software components.

So what are disjoint intersecting types? Here only highlights are given---more
details are to be delivered in later chapters.

\subsection{Disjoint Intersection Types}

A central theme of this thesis is \textit{intersection types} (usually written
$\inter{A}{B}$). Intersection types~\citep{pottinger1980type, coppoInter} have a
long history in programming languages. They were originally introduced to
characterize exactly all strongly normalizing lambda terms. Since then, starting
with \citeauthor{reynolds1988preliminary}'s work on
Forsythe~\citep{reynolds1988preliminary}, they have also been employed to
express useful programming language constructs, such as key aspects of multiple
inheritance~\citep{compagnoni1996higher} in object-oriented programming. One
notable example is the Scala language~\citep{odersky2004overview} and its DOT
calculus~\citep{amin2012dependent}, which make fundamental use of intersection
types to express a class/trait that extends multiple other traits. Other modern
programming languages, such as TypeScript~\citep{typescript}, Flow~\citep{flow}
and Ceylon~\citep{ceylon}, also adopt some form of intersection types.

Intersection types come in different varieties in the literature. A far more
common form of intersection types are the so-called \emph{refinement
  types}~\citep{Freeman_1991, Davies_2000, dunfield2003type}. Refinement types
restrict the formation of intersection types so that the two types in an
intersection are refinements of the same simple (unrefined) type.
% For example,
% we can refine a type $\mathsf{Int}$ of integers with a subtype $\mathsf{Odd}$ of
% odd numbers, then an integer $1$ can be typed as follows:
% \[
%   1 : \inter{\mathsf{Int}}{\mathsf{Odd}}
% \]
% which satisfies the restriction: both $\mathsf{Int}$ and $\mathsf{Odd}$ refine a
% single simple type $\mathsf{Int}$.
Refinement intersection increases only the expressiveness of types (more precise
properties can be checked) and not of terms. For this reason,
\citet{dunfield2014elaborating} argues that refinement intersection is unsuited
for encoding various useful language features that require the \textit{merge operator}
(or an equivalent term-level operator).

Unrestrained intersection types with a merge operator as an \emph{explicit}
introduction form for intersections increase the expressiveness of the term
language. This operator was introduced by \citeauthor{reynolds1988preliminary}
in Forsythe~\citep{reynolds1988preliminary} and adopted by a few other
calculi~\citep{Castagna_1992, dunfield2014elaborating, oliveira2016disjoint,
  alpuimdisjoint}. Unfortunately, while the merge operator is powerful, it also
makes it hard to get a \emph{coherent} (or unambiguous) semantics.
%A semantics is said to be coherent if all valid programs have the
%same meaning.\tom{The previous sentence is easily misunderstood.}
Unrestricted uses of the merge operator can be ambiguous, leading to an incoherent semantics
where the same program can evaluate to different values.
%Perhaps because of this
%issue the merge operator has not been adopted by many language designs.
We shall come back to this form of intersection types in more details in
\cref{bg:sec:intersection}.

Recently, \citet{oliveira2016disjoint} proposed \oname: a calculus with a
variant of intersection types called \emph{disjoint intersection types}. Calculi
with disjoint intersection types also feature the merge operator, with restrictions
that all expressions in a merge operator must have disjoint types and all
well-formed intersections are also disjoint. A bidirectional type system and the
disjointness restrictions ensure that the semantics of the resulting calculi
remains coherent. As shown by \citet{alpuimdisjoint}, calculi with disjoint
intersection types are very expressive and can be used to statically type-check
JavaScript-style programs using mixins. Yet they retain both type safety and
coherence. While coherence may seem at first of mostly theoretical relevance, it
turns out to be very relevant for object-oriented programming. As remarked
earlier, a key issue for multiple inheritance is \emph{ambiguity} caused by the
same field/method names inherited from different parents. Disjoint intersection
types enforce that the types of parents are disjoint and thus that no conflicts
exist. Any violations are statically detected and can be manually resolved by
the programmer (for example by dropping one of the conflicting field/methods
from one of the parents). This is very similar to existing trait
models~\citep{scharli2003traits, Ducasse_2006}. Therefore in an object-oriented language
modeled on top of disjoint intersection types, coherence implies that no
ambiguity arises from multiple inheritance. This makes reasoning a lot simpler.

The main goal of this thesis is to significantly increase the expressiveness of
disjoint intersection types by extending the simple forms of multiple
inheritance/composition supported by prior work~\citep{alpuimdisjoint, oliveira2016disjoint}
into a more powerful form supporting nested composition.
On the pragmatic side, the outcome is a programming language with support for
first-class traits, dynamic inheritance and nested composition. On the
theoretical side, we put disjoint intersection types on a solid footing by
exploring their meta-theoretical properties using logical relations.


\section{Contributions}

In this thesis, we present three new typed calculi, starting from a simple
calculus with disjoint intersection types, then adding parametric polymorphism
and finally ending up with a relatively sophisticated object-oriented language
with support for first-class traits and nested composition.

\paragraph{The \namee Calculus.}

The first one, named \namee, is a simple calculus with records and disjoint
intersection types that supports \emph{nested composition}. The essential
novelty of \namee is adoption of BCD subtyping, which includes distributivity
rules between function/record types and intersection types. These rules are the
delta that enable extending the simple forms of multiple inheritance/composition
supported by prior work~\citep{oliveira2016disjoint} into a more powerful form
supporting nested composition. The incorporation of BCD subtyping is highly
challenging for two different reasons. The first difficulty is how to preserve
coherence. Although previous work on disjoint intersection types proposes a
solution to coherence, the solution imposes several ad-hoc restrictions to
guarantee the uniqueness of the elaboration and thus allow for a simple
syntactic proof of coherence. However such restrictions makes it hard or
impossible to adapt the proof to extensions of the calculus with distributivity
rules. To deal with coherence, we employ a more semantic proof method called the
\textit{canonicity} relation inspired by \emph{logical relations}~\citep{tait,
  plotkin1973lambda, statman1985logical}. The second difficulty is that BCD
subtyping is non-algorithmic. We adapted Pierce's decision procedure and propose
an equivalent algorithmic version.

\paragraph{The \fnamee Calculus.}

The second one, named \fnamee, is a polymorphic calculus with disjoint
intersection types. \fnamee is essentially \namee enriched with a variant of
parametric polymorphic called disjoint polymorphism~\citep{alpuimdisjoint}. The
addition of parametric polymorphic greatly increases the expressiveness power of
\namee: \fnamee is able to express \textit{deep} conflict-free mixin composition
in the presence of parametric polymorphism, which is extremely useful in the
encodings of extensible designs. The key contribution is the extension of BCD
subtyping with disjointness polymorphism. The extension is non-trivial in that
we need to carefully retain coherence. The technical difficulty is
\textit{well-foundedness}, stemming from the interaction between impredicativity
and disjointness. To address this, we extend the canonicity relation with the
restriction of predicativity.


\paragraph{Typed First-Class Traits.}

Lastly we present the design of \sedel: a polymorphic language with
\emph{first-class traits}, supporting \emph{parametric polymorphism}, \emph{dynamic inheritance} as well as
conventional object-oriented features such as \emph{dynamic dispatching} and \emph{abstract
  methods}. Traits pose additional challenges when compared to models with
first-class classes or mixins, because method conflicts should be detected
\emph{statically}, even in the presence of features such as dynamic inheritance and
parametric polymorphism. To address the challenges of
typing first-class traits and detecting conflicts statically, \sedel adopts the
well-established approach of elaborating high-level language constructs to a
low-level core calculus. The main contribution of \sedel is to show how to model
source language constructs for first-class traits and dynamic inheritance. The
work on \namee and \fnamee aimed at core record calculi, and omits important
features for practical object-oriented languages, including (dynamic) inheritance, dynamic
dispatching and abstract methods. Based on \citeauthor{cook1989denotational}'s
work on the denotational semantics for inheritance~\citep{cook1989denotational},
we show how to design a source language that is elaborated into \fnamee.
\sedel's elaboration into \fnamee is proved to be both type-safe and coherent.
Coherence ensures that the semantics of \sedel is unambiguous. In particular
this property is useful to ensure that programs using traits are free of
conflicts/ambiguities (even when the types of the object parts being composed
are not fully statically know). We illustrate the applicability of \sedel with
several example uses for first-class traits. Furthermore, we conduct a case study
that modularizes programming language interpreters using a highly modular form
of \visitor~\citep{oliveira09modular, togersen:2004}.

In summary the contributions of this thesis are:

\begin{itemize}

\item We present \namee, a calculus with disjoint intersection types that
  features both \emph{BCD-style subtyping} and \emph{the merge operator}. This
  calculus is both type-safe and coherent, and supports \emph{nested composition}.

\item We present \fnamee, a polymorphic calculus with disjoint intersection
  types. \fnamee is incorporated with a BCD-like subtyping relation extended
  with disjoint polymorphism. \fnamee is both type-safe and coherent and
  supports nested composition.

\item We present \sedel, a statically-typed language design that supports
  first-class traits, dynamic inheritance, as well as standard object-oriented
  features such as dynamic dispatching and abstract methods. We show how the
  semantics of \sedel can be defined by elaboration into \fnamee.

\item A more flexible notion of disjoint intersection types where only merges
  need to be checked for disjointness. This removes the need for enforcing
  disjointness for all well-formed types, making type systems with disjoint
  intersections more easily extensible.

\item The canonicity relation: a powerful proof method for establishing
  coherence of calculi with disjoint intersection types and BCD-like subtyping.

\item A comprehensive Coq mechanization of all metatheory, including type
  safety, coherence, algorithmic soundness and completeness, etc. This has
  notably revealed several missing lemmas and oversights in Pierce's manual
  proof of BCD's algorithmic subtyping~\citep{pierce1989decision}. As a
  by-product, we obtain the first mechanically verified BCD-style subtyping
  algorithm with coercions.

\item A full-blown implementation of \sedel; it runs and type-checks all the
  examples in this thesis.\footnote{The Coq formalization and implementation are
    available at \url{https://github.com/bixuanzju/phd-thesis-artifact}.} We
  also conduct a case study, which shows that support for composition of Object
  Algebras~\citep{oliveira2012extensibility} is greatly improved in \sedel.
  Using such improved design patterns we re-code the interpreters from an
  undergraduate textbook on programming languages~\citep{poplcook} in a modular
  way.

\end{itemize}


% The author also contributed to the following publications that do not directly
% relate to the topics of this thesis:
% \begin{itemize}
% \item Ningning Xie, Xuan Bi, Bruno C. d. S. Oliveira. 2018. ``Consistent Subtyping for All''.
%   In \textit{European Symposium on Programming (ESOP)}.
% \item Yanpeng Yang, Xuan Bi, Bruno C. d. S. Oliveira. 2016. ``Unified Syntax with
%   Iso-Types''. In \textit{Asian Symposium on Programming Languages and
%     Systems (APLAS)}.
% \item Tomas Tauber, Xuan Bi, Zhiyuan Shi, Weixin Zhang, Huang Li, Zhenrui Zhang,
%   Bruno C. d. S. Oliveira. 2015. ``Memory-efficient Tail Calls in the JVM with
%   Imperative Functional Objects''. In \textit{Asian Symposium on
%     Programming Languages and Systems (APLAS)}.
% \end{itemize}


\section{Structure of the Thesis}

We begin with some background in the main topics of this thesis in
\cref{chap:background} in order to keep this thesis as self-contained as
possible and also to put our methods and contributions into context. The
structure of the technical content in the thesis is divided into three parts:
\begin{description}
\item[Part I:] \Cref{chap:nested,chap:fi} formally define the type systems of
  \namee and \fnamee, respectively. We first give the syntax and semantics of
  the two calculi. The semantics is defined in two parts. The ``target''
  languages are two standard type systems (simply-typed lambda calculus and
  System F, respectively) that do not have intersection types, the merge
  operator or subtyping. The ``source'' languages, defined by translation into
  the target languages, contain intersection types, the merge operator and
  subtyping. We then prove some basic properties such as type safety of
  the elaboration, soundness and completeness of the algorithmic subtyping, etc.
\item[Part II:] \Cref{chap:coherence:simple,chap:coherence:poly} explore the
  issue of coherence. In \cref{chap:coherence:simple} we first propose a
  semantically-founded definition of coherence. We then use a proof method
  called the canonicity relation to establish coherence of \namee. In
  \cref{chap:coherence:poly} we follow the same technique in
  \cref{chap:coherence:simple} but encounter a severe issue of impredicativity.
  We impose a predicativity restriction and adapt the canonicity relation to
  establish coherence of \fnamee.
\item[Part III:] In \cref{chap:traits} we present the syntax and semantics of
  \sedel. In particular we show how to elaborate source-level constructs for
  first-class traits into expressions of \fnamee. In \cref{chap:case_study} we
  conduct a case study of modularizing programming language features using a
  highly modular form of \visitor.
\end{description}
\Cref{sec:related} reviews related work, \cref{chap:future} presents some future
work and finally \cref{chap:conclusion} concludes.

This thesis is largely based on two publications by the author, which are shown
in the list below with corresponding chapters indicated. The work on \fnamee is
based on an on-going draft by the author (\cref{chap:fi,chap:coherence:poly}).
In comparison to the original publications, this thesis contains a more in-depth
and consistent treatment of disjoint intersection types.

\begin{description}
\item[\cref{chap:nested,chap:coherence:simple}:] Xuan Bi, Bruno C. d. S.
  Oliveira, and Tom Schrijvers. 2018. ``The Essence of Nested Composition''. In
  \textit{European Conference on Object-Oriented Programming (ECOOP)}.
\item[\cref{chap:traits,chap:case_study}:] Xuan Bi and Bruno C. d. S. Oliveira.
  2018. ``Typed First-Class Traits''. In \textit{European Conference on Object-Oriented Programming (ECOOP)}.
\end{description}

\noindent\makebox[\linewidth]{\rule{0.7\textwidth}{0.4pt}}

\vspace{1.5\baselineskip}

This thesis assumes familiarity with basic knowledge of programming languages
theory and object-oriented programming. We recommend
\citeauthor{DBLP:books/daglib/0005958}'s excellent textbook on programming
languages~\citep{DBLP:books/daglib/0005958} for a general introduction.


%%% Local Variables:
%%% mode: latex
%%% TeX-master: "../Thesis"
%%% org-ref-default-bibliography: ../Thesis.bib
%%% End:
