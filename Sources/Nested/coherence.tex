
\section{Coherence}
\label{sec:cohe}

This section constructs logical relations to
establish the coherence of \name. Finding a
suitable definition of coherence for \name is already challenging in its own
right. In what follows we reproduce the steps of finding a definition for coherence
that is both intuitive and applicable. Then we present the
construction of logical (equivalence) relations tailored to this 
definition, and the connection between logical equivalence and coherence.



\subsection{In Search of Coherence}

In \oname the definition of coherence is based on
$\alpha$-equivalence. More specifically, their coherence property states that
any two target terms that a source expression elaborates into must be exactly the same (up to
$\alpha$-equivalence). Unfortunately this syntactic notion of coherence is
very fragile with respect to extensions.
For example, it is not obvious how to retain this notion of coherence when adding more subtyping
rules such as those in \cref{fig:subtype_decl}.

If we permit ourselves to consider only the syntactic aspects of expressions,
then very few expressions can be considered equal. The syntactic view also conflicts
with the intuition that the significance of an expression lies in its
contribution to the \textit{outcome} of a computation~\cite{Harper_2016}.
Drawing inspiration from a wide range of literature on contextual
equivalence~\cite{morris1969lambda}, we want a context-based notion of
coherence. It is helpful to consider several examples before presenting the
formal definition of our new semantically founded notion of coherence.

\begin{example} \label{eg:1}
The same \name expression $3$ can be typed $[[nat]]$ in many ways: for instance, by \rref{T-lit}; by
\rref{T-sub,S-refl}; or by \rref{T-sub,S-trans,S-refl}, resulting in \tname
terms $3$, $\app{[[id]]}{3}$ and $\app{([[id o id]])}{3}$, respectively. It is apparent
that these three \tname terms are ``equal'' in the sense that all reduce to the
same numeral $3$.
\end{example}

\paragraph{Expression Contexts and Contextual Equivalence.} To formalize the intuition,
we introduce the notion of \textit{expression contexts}. An expression context $[[cc]]$
is a term with a single hole $[[__]]$ (possibly under some binders) in it. The
syntax of \tname expression contexts can be found in \cref{fig:contexts}. The typing
judgement for expression contexts has the form $[[cc : (gg |- T) ~~> (gg' |- T')]]$ where $([[gg |- T]])$ indicates
the type of the hole. This judgement essentially says that plugging a well-typed
term $[[gg |- e : T]]$ into the context $[[cc]]$ gives another well-typed term
$[[gg' |- cc{e} : T']]$. We define a \textit{complete program} to mean any closed
term of type $[[nat]]$. The following two definitions capture the notion of
\textit{contextual equivalence}.

\begin{figure}[t]
  \centering
\begin{tabular}{llll}\toprule
  \tname contexts & $[[cc]]$ & $\Coloneqq$ & $[[__]] \mid [[\x . cc]] \mid [[cc e2]] \mid [[e1 cc]] \mid [[< cc, e2 >]] \mid [[< e1, cc >]] \mid [[c cc]] $ \\
  & & $\mid$ & $ [[ { l = cc }]] \mid [[ cc. l ]]$ \\
  \name contexts & $[[CC]]$ & $\Coloneqq$ & $[[__]] \mid [[\x . CC]] \mid [[CC ee2]] \mid [[ee1 CC]] \mid [[ee1 ,, CC]] \mid [[CC ,, ee2]] \mid [[CC : A]] $ \\
  & & $\mid$ & $ [[ { l = CC } ]] \mid [[CC.l]]$ \\ \bottomrule
\end{tabular}
  \caption{Expression contexts of \name and \tname}
  \label{fig:contexts}
\end{figure}

\begin{definition}[Kleene Equality]
  Two complete programs, $[[e]]$ and $[[e']]$, are Kleene equal, written
  $\kleq{[[e]]}{[[e']]}$, iff there exists $i$ such that $[[e -->> i]]$ and $[[e' -->> i]]$.
\end{definition}

\begin{definition}[\tname Contextual Equivalence] \label{def:cxtx} \leavevmode
  \begin{center}
  \begin{tabular}{lll}
    $[[gg |- e1 ~= e2 : T]] $ & $\defeq $ & $[[gg |- e1 : T]] \land [[gg |- e2 : T]] \ \land $ \\
                                 & & $\forall [[cc]].\ [[cc : (gg |- T) ~~> (empty |- nat)]]  \Longrightarrow \kleq{[[cc{e1}]]}{[[cc{e2}]]}  $
  \end{tabular}
  \end{center}
\end{definition}

\noindent Regarding Example~\ref{eg:1}, it seems adequate to say that $3$ and $\app{[[id]]}{3}$
are contextually equivalent. Does this imply that coherence
can be based on Definition~\ref{def:cxtx}? Unfortunately it cannot, as demonstrated by the
following example.


\begin{example} \label{eg:2} It may be counter-intuitive that two \tname terms
  $[[\x . pp1 x]]$ and $[[\x . pp2 x]]$ should also be considered equal. To see
  why, first note that they are both translations of the same \name expression:
  $[[(\x . x) : nat & nat -> nat]]$. What can we do with this lambda
  abstraction? We can apply it to $1 : [[nat & nat]]$ for example. In that case,
  we get two translations $\app{[[(\x . pp1 x)]]}{\pair{1}{1}}$ and $\app{[[(\x . pp2 x)]]}{\pair{1}{1}}$,
  which both reduce to the same numeral $1$. However, $[[\x . pp1 x]]$ and $[[\x . pp2 x]]$
  are definitely not equal according to Definition~\ref{def:cxtx}, as one can find a
  context $\app{[[__]]}{\pair{1}{2}}$ where the two terms reduce to two
  different numerals. The problem is that not every well-typed \tname term
  can be obtained from a well-typed \name expression through the
  elaboration semantics. For
  example, $\app{[[__]]}{\pair{1}{2}}$ should not be considered because the
  (non-disjoint) source expression $\mer{1}{2}$ is rejected by the type system
  of the source calculus \name and thus never gets elaborated into $\pair{1}{2}$.
\end{example}

\paragraph{\name Contexts and Refined Contextual Equivalence.}
Example~\ref{eg:2} hints at a shift from \tname contexts to \name contexts $[[C]]$,
whose syntax is shown in \cref{fig:contexts}. Due to the bidirectional
nature of the type system, the typing judgement of $[[C]]$ features 4
different forms:
\begin{mathpar}
  [[CC : (GG => A) ~> (GG' => A') ~~> cc]] \and
  [[CC : (GG <= A) ~> (GG' => A') ~~> cc]] \and
  [[CC : (GG => A) ~> (GG' <= A') ~~> cc]] \and
  [[CC : (GG <= A) ~> (GG' <= A') ~~> cc]]
\end{mathpar}
We write $[[CC : (GG dir A) ~> (GG' dir' A') ~~> cc]]$ to abbreviate the above 4 different forms.
Take $[[CC : (GG => A) ~> (GG' => A') ~~> cc]]$ for example, it reads given a
well-typed \name expression $[[GG |- ee => A]]$, we have $[[GG' |- CC{ee} => A']]$. The judgement also generates a \tname context $[[cc]]$ such that $[[cc : (|GG| |- |A|) ~~> (|GG'| |- |A'|)]]$ holds by
construction. The typing rules appear in the appendix. Now we are ready to
refine Definition~\ref{def:cxtx}'s contextual equivalence to take into
consideration both \name and \tname contexts.


\begin{definition}[\name Contextual Equivalence] \label{def:cxtx2} \leavevmode
  \begin{center}
  \begin{tabular}{lll}
    $[[GG |- ee1 ~= ee2 : A]] $ & $\defeq $ & $\forall [[e1]], [[e2]], [[C]], [[cc]].\  [[GG |- ee1 => A ~~> e1]] \land [[GG |- ee2 => A ~~> e2]] \ \land $ \\
                                 & & $[[CC : (GG => A) ~> (empty => nat) ~~> cc]]  \Longrightarrow \kleq{[[cc{e1}]]}{[[cc{e2}]]}  $
  \end{tabular}
  \end{center}
\end{definition}
\begin{remark}
  For brevity we only consider expressions in the inference mode. Our Coq formalization is complete with two modes.
\end{remark}
\noindent Regarding Example~\ref{eg:2}, a possible \name context is $[[ __ 1 : (empty => nat & nat -> nat) ~> (empty => nat) ~~> __ <1 , 1>]]$.
We can verify that both $[[\x . pp1 x]]$ and $[[\x . pp2 x]]$ produce $1$ in the context $[[__ <1 , 1>]]$.
Of course we should consider all possible contexts to be certain they are truly equal. From now on, we
use the symbol $\backsimeq_{ctx}$ to refer to contextual equivalence in
Definition~\ref{def:cxtx2}. With Definition~\ref{def:cxtx2} we can formally state that \name is coherent
in the following theorem:

\begin{restatable}[Coherence]{theorem}{coherence} \label{thm:coherence}
  If $[[GG |- ee => A ]]$ then $[[GG |- ee ~= ee : A]]$.
\end{restatable}
\noindent For the same reason as in Definition~\ref{def:cxtx2}, we only consider
expressions in the inference mode. The rest of the section is devoted to proving
that Theorem~\ref{thm:coherence} holds.

\subsection{Logical Relations}

Intuitive as Definition~\ref{def:cxtx2} may seem, it is generally very hard to prove
contextual equivalence directly, since it involves quantification over
\textit{all} possible contexts. Worse still, two kinds of contexts are involved
in Theorem~\ref{thm:coherence}, which makes reasoning even more tedious. The key to
simplifying the reasoning is to exploit types by using logical
relations~\cite{tait, statman1985logical, plotkin1973lambda}.


\paragraph{In Search of a Logical Relation.}
It is worth pausing to ponder what kind of relation we are looking for. % From
% Example~\ref{eg:2}, it is clear that pairs have a special status in \tname. Indeed they
% ought to be, since pairs originate from merges or subtyping. Also disjointness
% on intersection types should correspond to some sort of constraints over pairs.
The high-level intuition behind the relation is to capture the
notion of ``coherent'' values. These values are unambiguous in every context. A
moment of thought leads us to the following important observations:

\begin{observation}[Disjoint values are unambiguous] \label{ob:1} The relation should relate values originating
  from disjoint intersection types. Those values are essentially translated from
  merges, and since \rref{T-merge} ensures disjointness, they are unambiguous.
  For example, $\pair{1}{\recordCon{l}{1}}$ corresponds to the type $[[nat & {l : nat}]]$.
  It is always clear which one to choose ($1$ or $\recordCon{l}{1}$) no matter how this pair is used in certain contexts.
\end{observation}

\begin{observation}[Duplication is unambiguous] \label{ob:2} The relation should
  relate values originating from non-disjoint intersection types, only if the values are duplicates.
  This may sound baffling since the whole point of disjointness is to rule out
  (ambiguous) expressions such as $\mer{1}{2}$. However, $\mer{1}{2}$ never gets
  elaborated, and the only values corresponding to $[[nat & nat]]$ are those
  pairs such as $\pair{1}{1}$, $\pair{2}{2}$, etc. Those values are essentially
  generated from \rref{T-sub} and are also unambiguous.
\end{observation}

To formalize values being ``coherent'' based on the above observations,
\Cref{fig:logical} defines two (binary) logical relations for \tname, one for
values ($\valR{[[T1]]}{[[T2]]}$) and one for terms ($\eeR{[[T1]]}{[[T2]]}$). We
require that any two values $([[v1]], [[v2]]) \in \valR{[[T1]]}{[[T2]]}$ are
closed and well-typed. For succinctness, we write $\valRR{\tau}$ to mean
$\valR{\tau}{\tau}$, and similarly for $\eeRR{\tau}$.

\begin{figure}[t]
  \centering
\begin{align*}
  [[(v1 , v2) in V ( nat ; nat )]] &\defeq \exists i, v_1 = v_2 = i \\
  [[(v1, v2) in V (T1 -> T2 ; T1' -> T2')]] &\defeq \forall [[(v, v') in V (T1 ; T1')]], [[(v1 v, v2 v') in E (T2 ; T2')]] \\
  [[( {l = v1} , {l = v2}) in V ( {l : T1} ; { l : T2 } )]] &\defeq [[(v1, v2) in V (T1; T2)]] \\
  [[(<v1, v2> , v3) in V (T1 * T2 ; T3)]] &\defeq [[(v1, v3) in V (T1; T3)]] \land [[(v2, v3) in V (T2; T3)]] \\
  [[(v3, <v1, v2> ) in V (T3; T1 * T2)]] &\defeq [[(v3, v1) in V (T3; T1)]] \land [[(v3, v2) in V (T3; T2)]] \\
  [[(v1 , v2) in V (T1; T2)]]  &\defeq \mathsf{true} \quad \text{otherwise}
\end{align*}
\[
  [[(e1, e2) in E (T1; T2)]] \defeq \exists [[v1]] [[v2]], [[e1 -->> v1]] \land [[e2 -->> v2]] \land [[(v1, v2) in V (T1; T2)]]
\]
  \caption{Logical relations for \tname}
  \label{fig:logical}
\end{figure}


\begin{remark}
  The logical relations are heterogeneous, parameterized by two types, one for
  each argument. This is intended to relate values of different types.
\end{remark}


\begin{remark}
  The logical relations resemble those given by Biernacki and Polesiuk~\cite{biernacki2015logical}, as
  both are heterogeneous. However, two important differences are worth pointing
  out. Firstly, our value relation for product types ($\valR{[[T1 * T2]]}{[[T3]]}$ and $\valR{[[T3]]}{[[T1 * T2]]}$) is unusual.
  Secondly, their value relation disallows relating functions with natural numbers, while
  ours does not. As we explain shortly, both points are related to disjointness.
\end{remark}


First let us consider $\valR{[[T1]]}{[[T2]]}$. The first three cases are
standard: Two natural numbers are related iff they are the same numeral. Two
functions are related iff they map related arguments to related results. Two
singleton records are related iff they have the same label and their fields are
related. These cases reflect Observation~\ref{ob:2}: the same type
corresponds to the same value.

In the next two cases one of the parameterized types is a product type. In those
cases, the relation distributes over the product constructor $[[*]]$. This may
look strange at first, since the traditional way of relating pairs is by
relating their components pairwise. That is, $[[<v1,v2>]]$ and $[[<v1', v2'>]]$
are related iff (1) $[[v1]]$ and $[[v1']]$ are related and (2) $[[v2]]$ and
$[[v2']]$ are related. According to our definition, we also require that (3)
$[[v1]]$ and $[[v2']]$ are related and (4) $[[v2]]$ and $[[v1']]$ are related.
The design of these two cases is influenced by the disjointness of intersection
types. Below are two rules dealing with intersection types:
\begin{mathpar}
 \drule{D-andL} \and \drule{D-andR}
\end{mathpar}
Notice the structural similarity between these two rules and the two cases. Now
it is clear that the cases for products manifests disjointness of intersection
types, reflecting Observation~\ref{ob:1}. Together with the last case, we can
show that disjointness and the value relation are connected by the following
lemma:

\begin{lemma}[Disjoint values are in a value relation]\label{lemma:disjoint}
  If $[[A1 ** A2]]$ and $v_1 : |A_1|$ and $v_2 : |A_2|$, then $[[(v1, v2) in V (|A1| ; |A2|)]]$.
\end{lemma}
\begin{proof}
  By induction on the derivation of disjointness.
\end{proof}

Next we consider $\eeR{[[T1]]}{[[T2]]}$, which is standard. Informally it expresses
that two closed terms $[[e1]]$ and $[[e2]]$ are related iff
they evaluate to two values $[[v1]]$ and $[[v2]]$ that are related.



\paragraph{Logical Equivalence.}
The logical relations can be lifted to open terms in the usual way. First we give the
semantic interpretation of typing contexts:
\begin{definition}[Interpretation of Typing Contexts]
  $(\gamma_1, \gamma_2) \in \ggR{[[gg1]]}{[[gg2]]}$ is defined as follows:
  \begin{mathpar}
    \ottaltinferrule{}{}{ }{(\emptyset, \emptyset) \in  \ggR{[[empty]]}{[[empty]]}} \and
    \ottaltinferrule{}{}{ [[(v1, v2) in V (T1; T2)]]  \\ (\gamma_1,\gamma_2) \in \ggR{[[gg1]]}{[[gg2]]} \\ \mathsf{fresh}\, x }{(\gamma_1[x \mapsto v_1], \gamma_2[x \mapsto v_2]) \in \ggR{[[gg1, x : T1]]}{[[gg2 , x : T2]]}}
  \end{mathpar}
\end{definition}
Two open terms are related if every pair of related closing substitutions
makes them related:
\begin{definition}[Logical equivalence]
  Let $[[gg1 |- e1 : T1]]$ and $[[gg2 |- e2 : T2]]$.
  \[
    [[gg1 ; gg2 |- e1 == e2 : T1 ; T2]] \defeq \forall \gamma_1, \gamma_2 . \  (\gamma_1, \gamma_2) \in \ggR{[[gg1]]}{[[gg2]]} \Longrightarrow (\app{\gamma_1}{[[e1]]}, \app{\gamma_2}{[[e2]]}) \in \eeR{[[T1]]}{[[T2]]}
  \]
\end{definition}
For succinctness, we write $[[gg |- e1 == e2 : T]]$ to mean $[[gg ; gg |- e1 == e2 : T ; T]]$.


\subsection{Establishing Coherence}

With all the machinery in place, we are now ready to prove Theorem~\ref{thm:coherence}. But we need
several lemmas to set the stage.

First we show compatibility lemmas, which state that logical equivalence
is preserved by every language construct. Most are standard and thus are
omitted. We show only one compatibility lemma that is specific to our relations:

\begin{lemma}[Coercion Compatibility]   \label{lemma:co-compa}
  Suppose that $[[c |- T1 tri T2]]$,
  % The logical relation is preserved by coercion application.
  \begin{itemize}
  \item If $[[gg1 ; gg2 |- e1 == e2 : T1 ; T0]]$ then $[[gg1 ; gg2 |- c e1 == e2 : T2 ; T0]]$.
  \item If $[[gg1 ; gg2 |- e1 == e2 : T0 ; T1]]$ then $[[gg1 ; gg2 |- e1 == c e2 : T0 ; T2]]$.
  \end{itemize}
\end{lemma}
\begin{proof}
  By induction on the typing derivation of the coercion $[[c]]$.
\end{proof}


The ``Fundamental Property'' states that any well-typed expression is related to
itself by the logical relation. In our elaboration setting, we rephrase it so
that any two \tname terms elaborated from the \textit{same} \name expression are related
by the logical relation. To prove it, we require Theorem~\ref{thm:uniq}.

\begin{theorem}[Inference Uniqueness] \label{thm:uniq}
  If $[[GG |- ee => A1]]$ and $[[GG |- ee => A2]]$, then $[[A1]] \equiv [[A2]]$.
\end{theorem}

\begin{theorem}[Fundamental Property]  \label{thm:co-log} We have that:
  \begin{itemize}
  \item If $[[GG |- ee => A ~~> e]]$ and $[[GG |- ee => A ~~> e']]$, then $[[|GG| |- e == e' : |A| ]]$.
  \item If $[[GG |- ee <= A ~~> e]]$ and $[[GG |- ee <= A ~~> e']]$, then $[[|GG| |- e == e' : |A| ]]$.
  \end{itemize}
\end{theorem}
\begin{proof}
  The proof follows by induction on the first derivation. The most interesting case is \rref{T-sub}
  % \begin{mathpar}
  %   \drule{T-sub}
  % \end{mathpar}
  where we need Theorem~\ref{thm:uniq} to be able to
  apply the induction hypothesis. Then we apply Lemma~\ref{lemma:co-compa} to say
  that the coercion generated preserves the relation between terms. For the other cases
  we use the appropriate compatibility lemmas.
\end{proof}
\begin{remark}
  It is interesting to ask whether the Fundamental Property holds in the target language. That is,
  if $[[gg |- e : T]]$ then $[[gg |- e == e : T]]$. Clearly this does not
  hold for every well-typed \tname term. However, as we have emphasized, we do
  not need to consider every \tname term. Our logical relation is carefully formulated
  so that the Fundamental Property holds in the source language.
\end{remark}


We show that logical equivalence is preserved by \name contexts:

\begin{lemma}[Congruence] \label{lemma:cong}
 If $[[CC : (GG dir A) ~> (GG' dir' A') ~~> cc]]$, $[[GG |- ee1 dir A ~~> e1]]$, $[[GG |- ee2 dir A ~~> e2]]$
 and $[[|GG| |- e1 == e2 : |A|]]$, then $[[|GG'| |- cc{e1} == cc{e2} : |A'|]]$.
\end{lemma}
\begin{proof}
  By induction on the typing derivation of the context $[[C]]$, and applying
  the compatibility lemmas where appropriate.
\end{proof}


\begin{lemma}[Adequacy] \label{lemma:ade}
  If $[[  empty |- e1 == e2 : nat ]]$ then $\kleq{[[e1]]}{[[e2]]}$.
\end{lemma}
\begin{proof}
  Adequacy follows easily from the definition of the logical relation.
\end{proof}


Next up is the proof that logical relation is sound with respect to contextual equivalence:
\begin{theorem}[Soundness w.r.t. Contextual Equivalence] \label{thm:log-sound}
  If $[[ GG |- ee1 => A ~~> e1]]$ and $[[ GG |- ee2 => A ~~> e2]]$ and $  [[|GG| |- e1 == e2 : |A|]]  $ then
  $[[   GG |- ee1 ~= ee2 : A ]]$.
\end{theorem}
\begin{proof}
  From Definition~\ref{def:cxtx2}, we are given a context $[[  CC : (GG => A) ~> (empty => nat) ~~> cc ]]$. By Lemma~\ref{lemma:cong}
  we have $[[  empty |- cc{e1} == cc{e2} : nat  ]]$, thus $  \kleq{[[ cc{e1} ]]}{ [[cc{e2} ]]}    $ by Lemma~\ref{lemma:ade}.
\end{proof}


Armed with Theorem~\ref{thm:co-log} and Theorem~\ref{thm:log-sound}, coherence follows directly.
\coherence*
\begin{proof}
  Immediate from Theorem~\ref{thm:co-log} and Theorem~\ref{thm:log-sound}.
\end{proof}

\subsection{Some Interesting Corollaries}

To showcase the strength of the new proof method, we can derive some
interesting corollaries. For the most part, they are direct consequences of
logical equivalence which carry over to contextual equivalence.


Corollary~\ref{lemma:neutral} says that merging a term of some type with something else
does not affect its semantics. Corollary~\ref{lemma:commu} and Corollary~\ref{lemma:assoc} express that
merges are commutative and associative, respectively. Corollary~\ref{lemma:coercion_same}
states that coercions from the same types are ``coherent''.

\begin{corollary}[Merge is Neutral] \label{lemma:neutral}
  If $[[GG |- ee1 => A ]]$ and $[[GG |- ee1 ,, ee2 => A ]]$, then
  $[[GG |- ee1 ~= ee1 ,, ee2 : A]]$
\end{corollary}

\begin{corollary}[Merge is Commutative] \label{lemma:commu}
  If $[[GG |- ee1 ,, ee2 => A ]]$ and $[[GG |- ee2 ,, ee1 => A ]]$, then
  $[[GG |- ee1 ,, ee2 ~= ee2 ,, ee1 : A]]$.
\end{corollary}


\begin{corollary}[Merge is Associative] \label{lemma:assoc}
  If $[[GG |- (ee1 ,, ee2) ,, ee3 => A  ]]$ and $[[GG |- ee1 ,, (ee2 ,, ee3) => A ]]$, then
  $[[GG |- (ee1 ,, ee2) ,, ee3 ~= ee1 ,, (ee2 ,, ee3) : A]]$.
\end{corollary}



\begin{corollary}[Coercions Preserve Semantics]
  \label{lemma:coercion_same}
  If $[[A <: B ~~> c1]]$ and $[[A <: B ~~> c2]]$, then $[[gg |- \x . c1 x == \x . c2 x : | A | -> | B |]]$.
\end{corollary}


% Local Variables:
% org-ref-default-bibliography: ../paper.bib
% End:
