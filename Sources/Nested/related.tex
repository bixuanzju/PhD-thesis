
\section{Related Work}
\label{sec:related}

\paragraph{Coherence.}
In calculi that feature coercive subtyping, a semantics that interprets the
subtyping judgement by introducing explicit coercions is typically defined on
typing derivations rather than on typing judgements. A natural
question that arises
for such systems is whether the semantics is \textit{coherent}, i.e.,
distinct typing derivations of the same typing judgement possess the same
meaning. Since Reynolds~\citep{Reynolds_1991} proved the coherence of a calculus with
intersection types, based on the denotational semantics for intersection types,
many researchers have studied the problem of coherence in a variety of typed
calculi. Below we summarize two commonly-found approaches in the literature.

%\paragraph{Normalization-based Approach}
% The first approach is based on normalization.
Breazu-Tannen et al.~\citep{Breazu_Tannen_1991} proved
the coherence of a coercion translation from Fun~\citep{cardelli1985understanding}
extended with recursive types to System F by showing that any two
typing derivations of the same judgement are normalizable to a unique
normal derivation. % where the correctness of the normalization steps is justified
% by an equational theory in the target calculus, i.e., System F.
Ghelli~\citep{Curien_1992} presented a translation of System F$_\leq$ into a calculus
with explicit coercions and showed that any derivations of the same judgement are
translated to terms that are normalizable to a unique normal form. Following the
same approach, Schwinghammer~\citep{SCHWINGHAMMER_2008} proved the coherence of coercion
translation from Moggi's computational lambda calculus~\citep{Moggi_1991} with subtyping.


%\paragraph{Logical-relations-based Approach}

Central to the first approach is to find a normal form for a representation of
the derivation and show that normal forms are unique for a given typing
judgement. However, this approach cannot be directly applied to Curry-style
calculi, i.e, where the lambda abstractions are not type annotated. Also this
line of reasoning cannot be used when the calculus has general recursion.
Biernacki and Polesiuk~\citep{biernacki2015logical} considered the coherence
problem of coercion semantics. Their criterion for coherence of the translation is
\textit{contextual equivalence} in the target calculus. They presented a construction of
logical relations for establishing so constructed coherence for coercion semantics,
applicable in a variety of calculi, including delimited continuations and
control-effect subtyping.

As far as we know, our work is the first to use logical relations
to show the coherence for intersection types and the merge operator. The BCD
subtyping in our setting poses a non-trivial complication over Biernacki and
Polesiuk's simple structural subtyping. Indeed, because any two coercions
between given types are behaviorally equivalent in the target language, their
coherence reasoning can all take place in the target language. This is not true
in our setting, where coercions can be distinguished by arbitrary target
programs, but not those that are elaborations of source programs. Hence, we have to
restrict our reasoning to the latter class, which is reflected in a more
complicated notion of contextual equivalence and our logical relation's
non-trivial treatment of pairs.

\paragraph{Intersection Types and the Merge Operator.}
Forsythe~\citep{reynolds1988preliminary} has intersection types and a merge-like operator. However to ensure coherence, various
restrictions were added to limit the use of merges. For example, in Forsythe
merges cannot contain more than one function. Castagna et al.~\citep{Castagna_1992} proposed a
coherent calculus with a special merge operator that works on functions only.
More recently, Dunfield~\citep{dunfield2014elaborating} shows significant expressiveness
of type systems with intersection types and a merge operator. However his
calculus lacks coherence. The limitation was addressed by
Oliveira et al.~\citep{oliveira2016disjoint}, who introduced disjointness to ensure
coherence. The combination of intersection types, a merge operator and
parametric polymorphism, while achieving coherence was first studied in the
\fname calculus~\citep{alpuimdisjoint}. Compared to prior work, \name simplifies type systems
with disjoint intersection types by removing several restrictions. 
Furthermore, \name adopts a more powerful subtyping relation based 
on BCD subtyping, which in turn requires the use of a more powerful
logical relations based method for proving coherence.


% \bruno{This is an example of a typical mistake when writting related
%   work: you describe existing work, but do not say how it
%   relates/differs from the work presented in the paper.}


\paragraph{BCD Type System and Decidability.}
The BCD type system was first introduced by Barendregt et al.~\citep{Barendregt_1983}. It is
derived from a filter lambda model in order to characterize exactly the strongly
normalizing terms. The BCD type system features a powerful subtyping relation,
which serves as a base for our subtyping relation. Bessai et al.~\citep{DBLP:journals/corr/BessaiDDCd15}
showed how to type classes and mixins in a BCD-style record calculus with Bracha-Cook's merge operator~\citep{bracha1990mixin}.
Their merge can only operate on records, and they only study a type assignment system.
The decidability of BCD subtyping has been shown in several
works~\citep{pierce1989decision, Kurata_1995, Rehof_2011, Statman_2015}.
Laurent~\citep{laurent2012intersection} has formalized the relation in Coq in
order to eliminate transitivity cuts from it, but his formalization does not
deliver an algorithm. Based on Statman's work~\citep{Statman_2015}, Bessai et
al.~\citep{bessaiextracting} show a formally verified subtyping algorithm in Coq.
Our Coq formalization follows a different idea based on Pierce's decision
procedure~\citep{pierce1989decision}, which is shown to be easily extensible to
coercions and records. In the course of our mechanization we identified several
mistakes in Pierce's proofs, as well as some important missing lemmas.

\paragraph{Family Polymorphism.}
There has been much work on family polymorphism since Ernst's
original proposal~\citep{Ernst_2001}. Family polymorphism provides an elegant
solution to the Expression Problem. Although a simple Scala solution does exist without
requiring family polymorphism (e.g., see Wang and Oliveira~\citep{wang2016expression}), Scala
does not support nested composition: programmers need to manually compose
all the classes from multiple extensions. Generally speaking, systems that support
family polymorphism usually require quite sophisticated mechanisms such as
dependent types.

% Broadly speaking, systems that support
% family polymorphism can be divided into two categories: those that support
% \textit{object families} and those that support \textit{class families}. In
% object families, classes are nested in objects, whereas in class families,
% classes are nested in other classes. The former choice is considered more
% expressive~\citep{ErnstVirtual}, but requires a complex type system usually
% involving dependent types.

%\paragraph{Object Families}
There are two approaches to family polymorphism: the original \textit{object family}
approach of Beta (e.g., virtual classes~\citep{Madsen_1989}) treats nested classes
as attributes of objects of the family classes. Path-dependent types are used to
ensure type safety for virtual types and virtual classes in the calculus
\textit{vc}~\citep{ErnstVirtual}. As for conflicts, \textit{vc} follows the
mixin-style by allowing the rightmost class to take precedence. This is in
contrast to \name where conflicts are detected statically and resolved
explicitly. In the \textit{class family} approach of Concord~\citep{jolly2004simple},
Jx and J\&~\citep{Nystrom_2004,Nystrom:2006:JNI:1167473.1167476},
nested classes and types are attributes of the family classes directly.
% Unlike virtual classes, subclass and subtype relationships are preserved by inheritance: the
% overriding class is also a subtype of the class it overrides.
%Nested inheritance
%does not support generic types. As we discussed in \cref{sec:diss}, \name can be
%easily extended to incorporate parametric polymorphism.
Jx supports \textit{nested inheritance}, a class family mechanism that allows
nesting of arbitrary depth. J\& is a language that supports \textit{nested
  intersection}, building on top of Jx. Similar to \name, intersection types
play an important role in J\&, which are used to compose packages/classes.
Unlike \name, J\& does not have a merge-like operator. When conflicts arise,
prefix types can be exploited to resolve the ambiguity.
J\&$_s$~\citep{Qi:2009:SCF:1542476.1542508} is an extension of the Java language
that adds class sharing to J\&. Saito et al.~\citep{SAITO_2007} identified a
minimal, lightweight set of language features to enable family polymorphism,
% though inheritance relations between nested classes are not preserved by
% extension.
Corradi et al.~\citep{Corradi_2012} present a language design that
integrates modular composition and nesting of Java-like classes. It features a
set of composition operators that allow to manipulate nested classes at any
depth level. More recently, a Java-like language called
Familia~\citep{Zhang_2017} were proposed to combine subtyping polymorphism,
parametric polymorphism and family polymorphism. The object and class family approaches have even been
combined by the work on Tribe~\citep{pubsdoc:tribe-virtual-calculus}.

% This % flexibility allows a further bound class to be a subtype of the class it
% overrides.


% Interestingly, conflicts are distinguished in J\&. If a name was introduced in a
% common ancestor of the intersected namespaces, it is not considered a conflict.
% This is somewhat similar to \name where a value such as $1 :
% \inter{\mathsf{Nat}}{\mathsf{Nat}}$ is allowed. Otherwise, the name is assumed
% to refer to distinct members that coincidentally have the same name, then
% ambiguity occurs, and resolution is required by exploiting prefix type notation.





% The original design of virtual
% classes in BETA is not statically safe.
% To ensure type safety, virtual classes were formalized using in the calculus \textit{vc}. Due to the introduction
% of dependent types, their system is rather complex. Subtyping in \textit{vc} is
% more restrictive, compared to \name, in that there is no subtyping relationship
% between classes in the base family and classes in the derived family, nor is there between
% the base family and the derived family.






Compared with those systems, which usually focus on getting
a relatively complex Java-like language with family polymorphism,
\name focuses on a minimal calculus that supports
nested composition. \name shows that a calculus with the merge
operator and a variant of BCD captures the
essence of nested composition. Moreover \name enables new insights 
on the subtyping relations of families.
 \name's goal is not to support full family
polymorphism which, besides nested composition, also requires 
dealing with other features such as
self types~\citep{bruce95thistype,saito09matching} and mutable state. Supporting these features is not the
focus of this paper, but we expect to investigate those features in the future.


% \bruno{Probably comparison, in this last paragraph, can be improved a
%   bit. Also perhaps the discussion does not need to be so detailed. }


%%% Local Variables:
%%% org-ref-default-bibliography: ../paper.bib
%%% End:
