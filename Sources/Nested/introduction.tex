
\section{Introduction}
\label{sec:intro}

Intersection types~\cite{pottinger1980type,coppo1981functional} have a long
history in programming languages. They were originally introduced to
characterize exactly all strongly normalizing lambda terms. Since then,
starting with Reynolds's work on
Forsythe~\cite{reynolds1988preliminary}, they have also been employed
to express useful programming language constructs, such as key
aspects of \emph{multiple inheritance}~\cite{compagnoni1996higher} in
Object-Oriented Programming (OOP). One notable
example is the Scala
language~\cite{odersky2004overview} and its DOT
calculus~\cite{amin2012dependent}, which make fundamental use of intersection
types to express a class/trait that extends multiple other traits. Other
modern languages, such as TypeScript~\cite{typescript}, Flow~\cite{flow} and
Ceylon~\cite{ceylon}, also adopt some form of intersection types.

Intersection types come in different varieties in the literature. Some calculi
provide an \emph{explicit} introduction form for intersections, called the
\emph{merge operator}. This operator was introduced by Reynolds in Forsythe~\cite{reynolds1988preliminary} and
adopted by a few other calculi~\cite{Castagna_1992,
  dunfield2014elaborating, oliveira2016disjoint, alpuimdisjoint}. Unfortunately,
while the merge operator is powerful, it also makes it hard to get a \emph{coherent}
(or unambiguous) semantics. 
%A semantics is said to be coherent if all valid programs have the
%same meaning.\tom{The previous sentence is easily misunderstood.} 
Unrestricted uses of the merge operator can be ambiguous, leading to an incoherent semantics
where the same program can evaluate to different values. 
%Perhaps because of this
%issue the merge operator has not been adopted by many language designs. 
A far more common form of intersection types are the so-called \emph{refinement
  types}~\cite{Freeman_1991, Davies_2000, dunfield2003type}. Refinement types
restrict the formation of intersection types so that the two types in an
intersection are refinements of the same simple (unrefined) type. Refinement
intersection increases only the expressiveness of types and not of terms. 
For this reason, \citet{dunfield2014elaborating} argues that refinement
intersection is unsuited for encoding various useful language features
that require the merge operator (or an equivalent term-level operator).

\paragraph{Disjoint Intersection Types.}
\oname is a recent calculus with a variant of intersection types 
called \emph{disjoint intersection types}~\cite{oliveira2016disjoint}.
Calculi with disjoint intersection types feature the merge 
operator, with restrictions that all expressions in a merge 
operator must have disjoint types and all well-formed intersections 
are also disjoint. A bidirectional type system and the disjointness restrictions 
ensure that the semantics of the resulting calculi remains 
coherent. 

\begin{comment}
The merge operator was introduced by Reynolds 
and Forsythe and adopted by a few other calculi as well~\cite{}.
Unfortunately, while the merge operator is powerful, it makes 
it hard to get a \emph{coherent} semantics. \bruno{what is coherence}
Perhaps because 
of this issue the merge operator has not been adopted by 
many language designs. Disjoint intersection types provide 
a remedy for the coherence problem, by imposing restrictions 
on the uses of merges and on the formation of intersection types. 
\bruno{merge operator ==> models inheritance; intersection types ==> 
model subtyping}

In essence disjoint intersection types retain most of the 
expressive power of the merge operator.
For example, they can 
be used to model powerful forms of extensible records~\cite{}.
\end{comment}

Disjoint intersection types have great potential to serve as a foundation for
powerful, flexible and yet type-safe OO languages that are easy to reason
about. As shown by \citet{alpuimdisjoint}, calculi with disjoint intersection
types are very
expressive and can be used to statically type-check JavaScript-style programs
using mixins. Yet they retain both type safety and coherence. While
coherence may seem at first of mostly theoretical relevance, it
turns out to be very relevant for OOP. Multiple
inheritance is renowned for being tricky to get right, largely because of the
possible \emph{ambiguity} issues caused by the same field/method names
inherited from different parents~\cite{bracha1990mixin, scharli2003traits}. Disjoint intersection types
enforce that the types of parents are disjoint and thus that no conflicts exist.
Any violations are statically detected and can be manually resolved by the
programmer.
%(for example by dropping one
%of the conflicting field/methods from one of the parents). 
This is very similar
to existing trait models~\cite{fisher2004typed,
  ducasse2006traits}. Therefore in an OO language
modelled on top of disjoint intersection types, coherence implies
that no ambiguity arises from multiple inheritance. This makes
reasoning a lot simpler.

\paragraph{Family Polymorphism.}
One powerful and long-standing idea in OOP is \emph{family
  polymorphism}~\cite{Ernst_2001}. In family polymorphism inheritance is
extended to work on a \emph{whole family of classes}, rather than just a single
class. This enables high degrees of modularity and
reuse, including simple solutions to hard programming language problems, like
the Expression Problem~\cite{wadler1998expression}. An essential feature of
family polymorphism is \emph{nested composition}~\cite{Corradi_2012,
  ErnstVirtual, Nystrom_2004}, which allows the automatic
inheritance/composition of nested (or inner) classes when the top-level classes
containing them are composed. Designing a sound type system that fully supports family
polymorphism and nested composition is notoriously hard; there are only
a few, quite sophisticated, languages that manage this~\cite{ErnstVirtual, Nystrom_2004, pubsdoc:tribe-virtual-calculus, SAITO_2007}.
%To make matters worse, combining
%multiple inheritance with family polymorphism requires dealing with the various
%issues of both ideas.

\paragraph{\name.}
This paper presents an improved variant of \oname called
\name\footnote{\name stands for \textbf{Ne}sted \textbf{Co}mposition
  calcu\textbf{lus}.} (or \namee): a simple calculus with records and
disjoint intersection types that supports \emph{nested composition}. Nested composition enables
encoding simple forms of family polymorphism. More complex forms of
family polymorphism, involving binary methods~\cite{bruce1995binary} and mutable state are
not yet supported, but are interesting avenues for future work.
Nevertheless, in \name, it is
possible, for example, to encode Ernst's elegant family-polymorphism solution~\cite{Ernst_2001} to
the Expression Problem. 
Compared to \oname the essential novelty of \name are
distributivity rules between function/record types and intersection
types. These rules are the delta that enable extending the simple
forms of multiple inheritance/composition supported by \oname into a
more powerful form supporting nested composition. The distributivity
rule between function types and intersections is
common in calculi with intersection types aimed at capturing the set
of all strongly normalizable terms, and was first proposed by
Barendregt et al.~\cite{Barendregt_1983} (BCD). However the
distributivity rule is not common in calculi or languages with intersection types
aimed at programming. For example the rules employed in languages that
support intersection types (such as Scala, TypeScript, Flow or Ceylon)
lack distributivity rules. Moreover distributivity is also missing from
several calculi with a merge operator. This includes all calculi with
disjoint intersection types and Dunfield's work on elaborating
intersection types, which was the original foundation for \oname.
A possible reason for this omission in the past is that distributivity adds
substantial complexity (both algorithmically and metatheoretically), without having any obvious practical
applications. 
This paper shows how to deal with the complications of BCD subtyping, while
identifying a major reason to include it in a programming
language: BCD enables nested composition and subtyping, which is of
significant practical interest.

%The distributivity rules for records are
%new. Moreover, as far as we know, no previous work
%establishes the relation between BCD-style subtyping and nested composition.

\name differs significantly from previous BCD-based calculi in that it has to
deal with the possibility of incoherence, introduced by the merge operator. Incoherence
is a non-issue in the previous BCD-based calculi because they do not feature
this merge operator or any other source of incoherence.
Although previous work on disjoint intersection types
proposes a solution to coherence, the solution imposes several ad-hoc restrictions
to guarantee the uniqueness of the elaboration and thus allow for a simple
syntactic proof of coherence. Most
importantly, it makes it hard or impossible to adapt the proof to extensions of
the calculus, such as the new subtyping rules required by the BCD system.

In this work we remove the brittleness of the previous syntactic method to prove
coherence, by employing a more semantic proof method based on \emph{logical
  relations}~\cite{tait, plotkin1973lambda, statman1985logical}. This new proof method has several
advantages. Firstly, with the new proof method, several restrictions that were
enforced by \oname to enable the syntactic proof method are removed. For example
the work on \oname has to carefully distinguish between so-called \emph{top-like
  types} and other types. 
%This is necessary because top-like types can be
%non-disjoint (unlike other types), and yet they need to be allowed in a calculus
%with top types. 
In \name this distinction is not necessary; top-like types are handled like all
other types. Secondly, the method based on logical relations is more powerful
because it is based on semantic rather than syntactic equality. Finally, the
removal of the ad-hoc side-conditions makes adding new extensions, such as
support for BCD-style subtyping, easier. In order to deal with the complexity of
the elaboration semantics of \name, we employ binary logical relations that are
heterogeneous, parameterized by two types; the fundamental property is also
reformulated to account for bidirectional type-checking.

\begin{comment}
\tom{We need to be careful about how to formulate the above. The proof
     method itself is not novel. What is novel is that we use it for
     this purpose.}
\jeremy{adding to tom's comment, the novelty lines in the
     design of the logical relations that are tightly related to the disjointness
     judgment, esp the relation of products. Also, compared with traditional logical
     relations, there are two differences: 1) traditional logical
     relations are usually indexed by one type, ours are heterogeneous,
     parameterized by two types; 2) fundamental property doesn't hold in our
     target language  }
\end{comment}

In summary the contributions of this paper are:
\begin{itemize}

\item \name: a calculus with (disjoint) intersection types
that features both \emph{BCD-style subtyping} and \emph{the merge operator}.
This calculus is both type-safe and coherent, and supports \emph{nested
composition}. 

\item A more flexible notion of disjoint intersection types where 
only merges need to be checked for disjointness. This removes the need 
for enforcing disjointness for all well-formed types, making the
calculus more easily extensible.

\item An extension of BCD subtyping with both records and elaboration into coercions, 
  and
  algorithmic subtyping rules with coercions, inspired by
  Pierce's decision
  procedure~\cite{pierce1989decision}.

\item A more powerful proof strategy for coherence of disjoint
  intersection types based on logical
  relations. 

\item Illustrations of how the calculus can encode essential features 
of \emph{family polymorphism} through nested composition.

\item A comprehensive Coq mechanization of all meta-theory. This has notably revealed
  several missing lemmas and oversights
  in Pierce's manual proof~\cite{pierce1989decision} of BCD's algorithmic
  subtyping. % As a by-product, we obtain the first mechanically verified
  % BCD-style subtyping algorithm with coercions.
  We also have an implementation of a language built on top of \name; it runs and type-checks all examples shown in
  the paper.\footnote{The Coq formalization and implementation are available at \url{https://goo.gl/R5hUAp}.}


\end{itemize}

% Local Variables:
% TeX-master: "../paper"
% org-ref-default-bibliography: ../paper.bib
% End:
