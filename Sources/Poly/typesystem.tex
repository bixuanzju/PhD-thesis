
\section{Syntax and Semantics}

\begin{figure}[t]
  \centering
\begin{tabular}{llll} \toprule
  Types & $[[A]], [[B]], [[C]]$ & $\Coloneqq$ & $[[pri]] \mid [[Top]] \mid [[A -> B]]  \mid [[A & B]] \mid [[{l : A}]] \mid [[X]] \mid [[\ X ** A . B]] $\\
  Monotypes & $[[t]]$ & $\Coloneqq$ & $[[pri]] \mid [[Top]] \mid [[t1 -> t2]]  \mid [[t1 & t2]] \mid [[X]] \mid [[{l : t}]]$\\
  Expressions & $[[ee]]$ & $\Coloneqq$ & $[[x]] \mid [[i]] \mid [[Top]] \mid [[\x . ee]] \mid [[ee1 ee2]] \mid [[ ee1 ,, ee2 ]]   \mid [[ ee : A ]] $ \\
        & & $\mid$ & $ [[{l = ee}]] \mid [[ ee.l  ]] \mid [[\X ** A . ee]] \mid [[ ee A ]]  $ \\
  Value Context & $[[GG]]$ & $\Coloneqq$ &  $[[empty]] \mid [[GG , x : A]] $ \\
  Type Context & $[[DD]]$ & $\Coloneqq$ &  $[[empty]] \mid [[DD , X ** A]] $  \\ \bottomrule
  % Expression Contexts & $[[CC]]$ & $\Coloneqq$ &  $[[__]] \mid [[\ x . CC]] \mid [[\ X ** A. CC]] \mid [[ CC A  ]] \mid [[CC ee]] \mid [[ee CC]] \mid [[ CC ,, ee  ]]  $ \\
  % & & $\mid$ & $[[ ee ,, CC  ]] \mid  [[ { l = CC}  ]]  \mid [[ CC . l]] $
\end{tabular}
  \caption{Syntax of \fnamee}
  \label{fig:syntax:fi}
\end{figure}

\Cref{fig:syntax:fi} shows the syntax of \fnamee. Metavariables $[[A]], [[B]],
[[C]]$ range over types. Apart from \namee types, \fnamee also includes type
variables $[[X]]$ and disjoint quantification $[[ \X ** A . B ]]$. Monotypes
$[[t]]$ are the same, minus the universal quantification. Metavariable $[[ee]]$
ranges over expressions. We extend \namee expressions with two standard
constructs in System F: type abstractions $[[ \X ** A . ee ]]$ and type
applications $[[ee A]]$. The former also includes an extra disjointness
constraint $[[A]]$ associated with the type variable $[[X]]$.

\paragraph{Contexts.}

In the traditional formulation of System F, there is a single context that is
used to keep track of both type variables and term variables. Here we use
another style of presentation~\citep[chap. 16]{Harper_2016} where contexts are
split into \textit{value contexts} $[[GG]]$ and \textit{type contexts} $[[DD]]$.
The former track bound term variables $[[x]]$ with their types $[[A]]$; and the
latter track bound type variables $[[X]]$ with their disjointness constraints
$[[A]]$. This formulation is also convenient for the presentation of logical
relations in \cref{chap:coherence:poly}.

\begin{figure}[t]
  \centering
  \drules[swft]{$[[DD |- A]]$}{Well-formedness of types}{top, int, var, arrow, all, and, rcd}
  \drules[swfe]{$[[DD ||- GG]]$}{Well-formedness of value context}{empty, var}
  \drules[swfte]{$[[||- DD]]$}{Well-formedness of type context}{empty, var}
  \caption{Well-formedness of types and contexts}
  \label{fig:well-formedness:fi}
\end{figure}

\paragraph{Well-formedness of types and contexts.}

Our well-formedness judgments for types and contexts in
\cref{fig:well-formedness:fi}, though quite standard in System F, is the key
difference from \fname. The \fname calculus follows \oname where they add a
disjointness condition $[[ A ** B ]]$ in \rref{swft-and}. This is crucial for
them to ensure coherence due to their syntactic proof method, it also
complicates the metatheory a lot with extra efforts required to show that all
types in \fname is well-formed. In particular, they only have a weaker version
of the substitution lemma, shown in \cref{lemma:weaker_lemma}: the general
substitution lemma as shown in \cref{lemma:general_lemma} does not hold in
\fname, but holds in \fnamee.

\begin{lemma}[Types are stable under substitution in \fname] \label{lemma:weaker_lemma}
  If $[[DD |- A]]$ and $[[DD |- B]]$ and $[[ X ** C in DD  ]]$ and $\hlmath{[[ DD |- B ** C ]]}$
  and well-formed context $[[ [ B / X ] DD    ]]$, then $[[  [ B / X ] DD |- [B / X] A      ]]$.
\end{lemma}

\begin{lemma}[Types are stable under substitution in \fnamee] \label{lemma:general_lemma}
  If $[[DD |- A]]$ and $[[DD |- B]]$ and $[[ X ** C in DD  ]]$
  and well-formed context $[[ [ B / X ] DD    ]]$, then $[[  [ B / X ] DD |- [B / X] A      ]]$.
\end{lemma}



\begin{figure}[t]
  \centering
  \begin{drulepar}[S]{$ [[A <: B ~~> c]]  $}{Declarative subtyping}{}
    \drule{refl}
    \drule{trans}
    \drule{top}
    \drule{rcd}
    \drule{arr}
    \drule{andl}
    \drule{andr}
    \drule{and}
    \drule{distArr}
    \drule{topArr}
    \drule{distRcd}
    \drule{topRcd} \and
    \hlmath{\drule{forall}}
  \end{drulepar}
  \caption{Declarative subtyping of \fnamee}
  \label{fig:subtype_decl:fi}
\end{figure}

\paragraph{Declarative Subtyping.}


% \Cref{fig:subtyping:fi} presents the subtyping relation of \fnamee. For now, we
% ignore the coercion parts ($[[~~>]] [[c]]$) and explain them in
% \cref{sec:elaboration:fi}.
We naturally extend the subtyping rules of \namee with only one rule
\rref*{S-forall}, highlighted in \cref{fig:subtype_decl:fi}, which specifies the
subtyping relation between two universal quantifiers. In \rref{S-forall}, a
universal quantifier is covariant in its body, and contravariant in its
disjointness constraint. We also need a new coercion $[[ \ c ]]$ to expression the
conversion between two polymorphic types.
A minor comment is that since
\fnamee features explicit polymorphism, type variables are neutral to subtyping,
i.e., $[[X <: X]]$, which is already contained in \rref{S-refl}. As with \namee
subtyping, the subtyping relation of \fnamee is trivially \textit{reflexive} and
\textit{transitive}.

\begin{remark}
  In our Coq formalization, we require that the two types $[[A]]$ and $[[B]]$ are
  well-formed relative to some type context, resulting in the subtyping
  judgment $[[DD |- A <: B]]$. But this is not very important
  for the purpose of presentation, thus we omit contexts.
\end{remark}


\paragraph{Typing.}

\begin{figure}
  \centering
  \drules[FT]{$[[DD; GG |- ee => A ~~> e]]$}{Inference}{top, int, var, app, merge, anno, tabs, tapp, rcd, proj}
  \drules[FT]{$[[DD ; GG |- ee <= A ~~> e]]$}{Checking}{abs, sub}
  \caption{Bidirectional type system of \fnamee}
  \label{fig:typing:fi}
\end{figure}


The bidirectional type system of \fnamee follows that of \namee, as shown
in \cref{fig:typing:fi}. Again we ignore the translation parts ($[[~~>]] [[e]]$) and explain them in
\cref{sec:elaboration:fi}. The inference judgment $[[ DD; GG |- ee => A  ]]$
says that we can synthesize the type $[[A]]$ in the contexts $[[DD]]$ and $[[GG]]$. The checking judgment
$[[ DD ; GG |- ee <= A  ]]$ asserts that $[[ee]]$ checks against the type $[[A]]$
in the contexts $[[DD]]$ and $[[GG]]$. The rules directly ported from \namee are inferring rules \rref*{FT-top} for top values,
\rref*{FT-int} for integers, \rref*{FT-var} for variables, \rref*{FT-app} for applications, \rref*{FT-merge} for merges,
\rref*{FT-anno} for annotated terms, \rref*{FT-rcd,FT-proj} for records; checking rules \rref*{FT-abs} for term abstractions, and
the subsumption rule \rref*{FT-sub}. Note that in \rref{FT-merge}, the disjointness judgment has an extra type context, which will be
explained in \cref{sec:disjoint:fi}.

\paragraph{Disjoint quantification.}

The new rules are the inferring rules for type abstractions \rref*{FT-tabs} and
type applications \rref*{FT-tapp}. In \rref{FT-tabs}, the disjointness
constraint is added to the type context. During a type application in
\rref{FT-tapp}, the type system checks that the type argument agrees with the
disjointness constraint; this is the rule where we diverge from System F and
\fname, and allow only monotype instantiations ($[[t]]$ rather than $[[A]]$).  We shall return to this point in
\cref{chap:coherence:poly}. \Rref{FT-merge,FT-tapp} are the
only two rules that use the disjointness checking.



% Local Variables:
% TeX-master: "../../Thesis"
% org-ref-default-bibliography: ../../Thesis.bib
% End:
