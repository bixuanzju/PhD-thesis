
\section{Motivation}

Parametric polymorphism~\citep{reynolds1983types} is a well-beloved (and
well-studied) programming feature. It enables a single piece of code to be
reused on data of different types. So it is quite natural and theoretically
interesting to study combining parametric polymorphism with disjoint
intersection types, especially how it affects disjointness and coherence. On a more
pragmatic note, the combination of parametric polymorphism and disjoint intersection types also reveals new
insights into practical applications. Dynamic languages (such as JavaScript)
usually embrace quite flexible programming patterns, e.g., mixin composition
where objects can be composed at run time, and their types are not necessarily
statically known. The use of intersection types in TypeScript is inspired by
such flexible programming patterns. For example, an important use of
intersection types in TypeScript is the following function for mixin
composition:
\begin{lstlisting}[language=JavaScript]
function extend<T, U>(first: T, second : U) : T & U {...}
\end{lstlisting}
which is analogous to our merge operator in that it takes two objects and
produces an object with the intersection of the types of the argument objects.
However, the types of the two objects are not known, i.e., they are generic. An
important point is that, while it is possible to define such function in
TypeScript (albeit using some low-level (and type-unsafe) features of
JavaScript), it can also cause, as pointed out by \citet{alpuimdisjoint},
run-time type errors! Clearly a well-defined meaning for intersection types with
type variables is needed.


\paragraph{Disjoint Polymorphism.}

Motivated by the above two points, \citet{alpuimdisjoint} proposed disjoint
polymorphism, a variant of parametric polymorphism. The
main novelty is \textit{disjoint (universal) quantification} of the form $[[ \ X ** A . B ]]$.
Inspired by bounded quantification~\citep{cardelli1994extension}
where a type variable is constrained by a type bound, disjoint quantification
allows type variables to be associated with \textit{disjointness constraints}.
Correspondingly, a term construct of the form $[[ \ X ** A. ee ]]$ is used to
create values of disjoint quantification.
To understand the purpose of disjointness constraints, consider the following program (adapted from \citet{alpuimdisjoint}):
\begin{lstlisting}
mergeBad X (x : X) : X & Int = x ,, 2;
\end{lstlisting}
\lstinline{mergeBad} takes an argument \lstinline{x} of type \lstinline{X} (which is itself a type variable), and merges it with \lstinline{2}.
However, if we were to allow such definition, we could easily create an example where incoherence occurs again:
\begin{lstlisting}
(mergeBad Int 1) : Int -- 1 or 2
\end{lstlisting}
This is essentially the same problem of allowing $\mer{1}{2}$, which as we
discussed in \cref{bg:sec:intersection} will cause ambiguity. For \namee, we
know the concrete type for each variable and thus disjointness checking can help
avoid this problematic expression. However, with parametric polymorphism, a
variable could have any types,
including those that are already in the intersection. So a question to ask is to
decide under what conditions a type variable is disjoint with, say,
\lstinline{Int}. This is where disjointness constraints come into stage. The key
idea is that since we do not know \textit{a priori} what is the type with which
a type variable can be instantiated, we can restrict the set of types it can be
instantiated to. Let us rewrite the above program as follows:
\begin{lstlisting}
mergeGood [X * Int] (x : X) : X & Int = x ,, 2;
\end{lstlisting}
The only change is the notation \lstinline{[X * Int]}, where the left-side of
\lstinline{*} denotes the type variable being declared, and the right-side
denotes the disjointness constraint(s). Here the disjointness constraint
(\lstinline{Int}) effectively states that the type variable \lstinline{X} can be
instantiated to any types disjoint with \lstinline{Int}. For instance, the expression \lstinline{mergeGood Bool True}
type checks but the expression \lstinline{mergeGood Int 1}
is rejected because \lstinline{Int} (the type argument) is not disjoint with
\lstinline{Int} (the disjointness constraint). What is more, we can express multiple
constraints using intersection types, for example,
\begin{lstlisting}
mergeThree [X * Int & Bool] (x : X) : X & Int & Bool = x ,, 2 ,, True;
\end{lstlisting}
Here the type variable \lstinline{X} can only be instantiated to types disjoint with both
\lstinline{Int} and \lstinline{Bool}.


With disjointness constraints and a built-in merge operator, a type-safe and
conflict-free \lstinline{extend} function can be naturally defined as follows:
\begin{lstlisting}
extend T [U * T] (first : T) (second : U) : T & U = first ,, second;
\end{lstlisting}
The disjointness constraint on the type variable \lstinline{U} ensures that no
conflicts can occur when composing two objects, which is quite similar to
trait-based approach~\citep{scharli2003traits} in object-orientated programming.
We shall devote a whole chapter (\cref{chap:traits}) to further this point.


\paragraph{Adding BCD Subtyping.}

While \citet{alpuimdisjoint} studied the combination of disjoint intersection
types and parametric polymorphism, they follow the then-standard approach
of \citet{oliveira2016disjoint} to ensure coherence, thus excluding the
possibility of adding BCD subtyping. The combination of BCD subtyping and
disjoint polymorphism opens doors for more expressiveness. For example, we can
define the following function
\begin{lstlisting}
combine A [B * A] (f : {foo : Int -> A})
                  (g : {foo : Int -> B}) : {foo : Int -> A & B} = f ,, g;
\end{lstlisting}
which ``combines'' two singleton records with parts of types unknown and returns
another singleton record containing an intersection type. A variant of this
function plays a fundamental role in defining Object Algebra combinators (cf.
\cref{chap:case_study}).




In what follows, we first present the syntax and semantics (subtyping and
typing) of \fnamee. We then discuss the disjointness judgment in detail, in
particular, the disjointness relation between type variables and arbitrary
types. Finally we talk about the elaboration semantics of \fnamee and its target
calculus \tnamee, a variant of System F with explicit coercions.



% Local Variables:
% TeX-master: "../../Thesis"
% org-ref-default-bibliography: ../../Thesis.bib
% End:
