%%%%%%%%%%%%%%%%%%%%%%%%%%%%%%%%%%%%%%%%%%%%%%%%%%%%%%%%%%%%%%%%%%%%%%%%
\chapter{First-Class Traits}
\label{chap:traits}
%%%%%%%%%%%%%%%%%%%%%%%%%%%%%%%%%%%%%%%%%%%%%%%%%%%%%%%%%%%%%%%%%%%%%%%%


In this chapter and \cref{chap:case_study}, we present two applications of
\fnamee. This chapter is primarily concerned with building a source-level
language called \sedel that features \emph{typed first-class traits},
\emph{dynamic inheritance} and \emph{nested composition} among others. We show
how to model source-level constructs for first-class traits and dynamic
inheritance, supporting standard object-oriented features such as dynamic
dispatching and abstract methods. It is remarkable that all of these can be
explained by plain \fnamee expressions, showing its expressive power. In
\cref{chap:case_study} we conduct a case study that modularizes programming
language features by the means of first-class traits.



\section{Motivation: First-Class Classes and Dynamic Inheritance}

Many dynamically typed languages (including JavaScript, Ruby, Python
or Racket) support \emph{first-class classes}~\citep{DBLP:conf/aplas/FlattFF06}, or related concepts
such as first-class mixins and/or traits. In those languages classes
are first-class values and, like any other values, they can be
passed as an argument, or returned from a function. Furthermore,
first-class classes support \emph{dynamic inheritance}: i.e., they
can inherit from other classes at \emph{run time}, enabling
programmers to abstract over the inheritance hierarchy.
Those features make first-class classes very powerful and expressive,
and enable highly modular and reusable pieces of code, such as:
\begin{lstlisting}[language=JavaScript]
const mixin = Base => {
  return class extends Base { ... }
};
\end{lstlisting}
In this piece of JavaScript code, \lstinline{mixin} is
parameterized by a class \lstinline{Base}. Note that the concrete
implementation of \lstinline{Base} can be
even dynamically determined at run time, for example
after reading a configuration file to decide which
class to use as the base class.  When applied to an argument,
\lstinline{mixin} will create a new class on-the-fly and return that
as a result. Later that class can be instantiated and used to create
new objects, as any other classes.

In contrast, most statically typed
languages do not have first-class classes and dynamic
inheritance. While all statically typed object-oriented languages allow first-class
\emph{objects} (i.e., objects can be passed as arguments and returned
as results), the same is not true for classes. Classes in languages such as
Scala, Java or C++ are typically a second-class construct, and the
inheritance hierarchy is \emph{statically determined}. The closest thing
to first-class classes in
languages like Java or Scala are classes such as
\lstinline[language=java]{java.lang.Class} that enable representing classes and
interfaces as part of their reflective framework. \lstinline[language=java]{java.lang.Class} can be used to
mimic some of the uses of first-class classes, but in an essentially
dynamically typed way. Furthermore, simulating first-class classes
using such mechanisms is highly cumbersome because classes need to be
manipulated programmatically. For example instantiating a new class
cannot be done using the standard \lstinline{new} construct, but
rather requires going through API methods of
\lstinline[language=java]{java.lang.Class}, such as \lstinline{newInstance}, for
creating a new instance of a class.

Despite the popularity and expressive power of first-class classes in dynamically typed
languages, there is surprisingly little work on typing of first-class
classes (or related concepts such as first-class mixins or traits).
First-class classes and dynamic inheritance pose well-known
difficulties in terms of typing. For example, in his thesis,
\citet{bracha1992programming} comments several times on the difficulties of typing
dynamic inheritance and first-class mixins, and proposes the
restriction to static inheritance that is also common in
statically typed languages. He also observes that such restriction
poses severe limitations in terms of expressiveness, but that appeared
(at the time)
to be a necessary compromise when typing was also desired.
Only recently some progress has been made in statically typing
first-class classes and dynamic inheritance. In particular there are
two works in this area: Racket's gradually
typed first-class classes~\citep{DBLP:conf/oopsla/TakikawaSDTF12}; and \citeauthor{DBLP:conf/ecoop/LeeASP15}'s model of
typed first-class classes~\citep{DBLP:conf/ecoop/LeeASP15}. Both works provide typed models of
first-class classes, and they enable encodings of mixins~\citep{bracha1990mixin}
similar to those employed in dynamically typed languages.

However, as far as we known no previous work supports statically typed
\emph{first-class traits}. Traits~\citep{scharli2003traits, Ducasse_2006} are an
alternative to mixins, and other models of (multiple) inheritance. The key
difference between traits and mixins lies on the treatment of conflicts when
composing multiple traits/mixins. Mixins adopt an \emph{implicit} resolution
strategy for conflicts, where the compiler automatically picks one
implementation in case of conflicts. For example, Scala uses the order of mixin
composition to determine which implementation to pick in case of conflicts.
Traits, on the other hand, employ an \emph{explicit} resolution strategy, where
the compositions with conflicts are rejected, and the conflicts are explicitly
resolved by programmers. This gives programmers fine-grained control, when
conflicts arise, of selecting desired features from different components. Thus
we believe traits are a better model for multiple inheritance in
statically typed object-oriented languages.
% In what follows, we present \sedel: the first design of typed first-class traits.




\section{Overview}
\label{sec:trait:overview}

This section aims at introducing first-class classes and traits, their possible
uses and applications, as well as the typing challenges that arise
from their use.
We start by describing a hypothetical JavaScript library for text editing
widgets, inspired and adapted from Racket's GUI
toolkit~\citep{DBLP:conf/oopsla/TakikawaSDTF12}. The example is illustrative of
typical uses of dynamic inheritance/composition, and also the typing challenges
in the presence of first-class classes/traits. Without diving into
technical details, we then give the corresponding typed version in
\sedel, and informally presents its salient features.

\subsection{First-Class Classes in JavaScript}

A class construct was officially added to JavaScript in the ECMAScript
2015 Language Specification~\citep{EcmaScript:15}. One purpose of
adding classes to JavaScript was to support a construct that is more
familiar to programmers who come from mainstream class-based languages,
such as Java or C++. However classes in JavaScript are
\emph{first-class} and support functionality not easily mimicked in
statically-typed class-based languages.

\paragraph{Conventional Classes.}

Before diving into the more advanced features of JavaScript classes, we first
review the more conventional class declarations supported in JavaScript as well
as many other languages. Even for conventional classes there are some
interesting points to note about JavaScript that will be important when we move
into a typed setting. An example of a JavaScript class declaration is:
\begin{lstlisting}[language=JavaScript]
class Editor {
  onKey(key) {
    return "Pressing " + key;
  }
  doCut() {
    return this.onKey("C-x") + " for cutting text";
  }
  showHelp() {
    return "Version: " + this.version() + " Basic usage...";
  }
};
\end{lstlisting}
This form of class definition is standard and very similar to declarations in
class-based languages (for example Java). The \lstinline{Editor} class
defines three methods: \lstinline{onKey} for handling key events,
\lstinline{doCut} for cutting text and \lstinline{showHelp} for displaying help
message. For the purpose of demonstration, we elide the actual implementation,
and replace it with plain messages.

We wish to bring the readers' attention to two points in the above class.
Firstly, note that the \lstinline{doCut} method is defined in terms of the
\lstinline{onKey} method via the keyword
\lstinline[language=JavaScript]{this}. In other words the call to
\lstinline{onKey} is enabled by the \emph{self} reference and is
\emph{dynamically dispatched} (i.e., the particular implementation of
\lstinline{onKey} will only be determined when the class or subclass
is instantiated). % Typically an
% OO programmer seeing this definition would expect the \lstinline{doCut} method
% to call the \lstinline{onKey} method of a subclass of \lstinline{Editor}, even though
% the subclass does not exist when the superclass \lstinline{Editor} is being
% defined.
Secondly, notice that there is no definition of
the \lstinline{version} method in the class body, but such method is used in the body of the
\lstinline{showHelp} method. In an untyped language, such as JavaScript, using
undefined methods is error prone---accidentally instantiating \lstinline{Editor}
and then calling \lstinline{showHelp} will cause a run-time error!
Statically-typed languages usually provide some means to protect us from this
situation. For example, in Java, we would need an \textit{abstract} \lstinline{version}
method, which effectively makes \lstinline{Editor} an abstract class and
prevents it from being instantiated. As we will see, \sedel's treatment of
abstract methods is quite different from mainstream languages. In fact, \sedel
has a unified (typing) mechanism for dealing with both dynamic dispatch and abstract
methods. We will describe \sedel's mechanism for dealing with both features and
justify our design in \cref{sec:traits}.

% A couple of things worth pointing out in the above code snippet: (1) the class
% \lstinline{Editor} has no definition of the method
% \lstinline{version}, but such method
% is used in the body of the method \lstinline{showHelp}. In a strongly-typed OO
% language, such as Java, we would need to define an abstract method for
% \lstinline{version}. (2) The \lstinline{Editor} class requires
% \emph{dynamic dispatching}.
%  In the body of the method \lstinline{doCut} we invoke
% the method \lstinline{onKey} defined in the same class through the keyword
% \lstinline[language=JavaScript]{this}. This has the implication that when a
% subclass of \lstinline{Editor} overrides the method \lstinline{onKey}, a call to
% \lstinline{doCut} should invoke \lstinline{onKey} defined in the subclass
% instead of the original one.\bruno{punchline?}
%As we will see later, the type system of \sedel correctly handles it.

\paragraph{First-Class Classes and Class Expressions.}

Another way to define a class in JavaScript is via a \emph{class expression}. This is where the class
model in JavaScript is very different from the traditional class model found in
many mainstream OO languages, such as Java, where classes are second-class
(static) entities. JavaScript embraces a dynamic class model that treats classes
as \emph{first-class} expressions: a function can take classes as arguments,
or return them as a result. First-class classes enable programmers to
abstract over patterns in the class hierarchy and to experiment with new forms of OOP
such as mixins and traits. In particular, mixins become programmer-defined
constructs. We illustrate this by presenting a simple mixin that adds
spell checking to an editor:
\begin{lstlisting}[language=JavaScript]
const spellMixin = Base => {
  return class extends Base {
    check() {
      return super.onKey("C-c") + " for spell checking";
    }
    onKey(key) {
      return "Process " + key + " on spell editor";
    }
  }
};
\end{lstlisting}

\paragraph{Dynamic Inheritance.}

In JavaScript, a mixin is simply a function with a superclass as input and a
subclass extending that superclass as an output. Concretely, \lstinline{spellMixin}
adds a method \lstinline{check} for spell checking. It also provides
a method \lstinline{onKey}.
The function \lstinline{spellMixin} shows the typical use of what we call \emph{dynamic inheritance}.
Note that \lstinline{Base}, which is supposed to be a superclass being inherited, is \emph{parameterized}.
Therefore \lstinline{spellMixin} can be applied to any base class at
\emph{run time}. This is impossible to do, in a type-safe way, in
conventional statically-typed class-based languages like Java or
C++.\footnote{With C++ templates, it is possible to
  implement a so-called mixin pattern~\citep{DBLP:conf/gcse/SmaragdakisB00}, which enables extending
a parameterized class. However C++ templates defer type-checking until
instantiation, and such pattern still does not allow selection of the
base class at run time (only at up to class instantiation time).}

It is noteworthy that not all applications of \lstinline{spellMixin} to base
classes are successful. Notice the use of the \lstinline{super} keyword in the
\lstinline{check} method. If the base class does not implement the
\lstinline{onKey} method, then mixin application fails with a run-time error. In
a typed setting, a type system must express this requirement (i.e., the presence of
the \lstinline{onKey} method) on the (statically unknown) base class being inherited.


% The class expression inside the function body has no
% definition of the method \lstinline{version}, but which is used in the body of
% the method \lstinline{showHelp}. In a statically-typed OO language, such as Java,
% we would need an \emph{abstract method} for
% \lstinline{version}.


We invite the readers to pause for a while and think about what the type of
\lstinline{spellMixin} would look like. Clearly our type system should be
flexible enough to express this kind of dynamic pattern of composition in order
to accommodate mixins (or traits), but also not too lenient to allow any
composition.


\paragraph{Mixin Composition and Conflicts.}
The powerful part of mixins is that \lstinline{spellMixin}'s functionality is not
tied to a particular class hierarchy and is composable with other features. For
example, we can define another mixin that adds simple modal editing---as in Vim---to an arbitrary editor:
\begin{lstlisting}[language=JavaScript]
const modalMixin = Base => {
  return class extends Base {
    constructor() {
      super();
      this.mode = "command";
    }
    toggleMode() {
      return "toggle succeeded";
    }
    onKey(key) {
      return "Process " + key + " on modal editor";
    }
  };
};
\end{lstlisting}
\lstinline{modalMixin} adds a \lstinline{mode} field that controls which
keybindings are active, initially set to the command mode, and a method
\lstinline{toggleMode} that is used to switch between modes. It also provides a method \lstinline{onKey}.

Now we can compose \lstinline{spellMixin} with \lstinline{modalMixin} to produce
a combination of functionality, mimicking some form of multiple inheritance:
\begin{lstlisting}[language=JavaScript]
class IDEEditor extends modalMixin(spellMixin(Editor)) {
  version() {
    return 0.2;
  }
}
\end{lstlisting}
The class \lstinline{IDEEditor} extends the base class \lstinline{Editor} with
modal editing and spell checking capabilities. It also defines the missing
\lstinline{version} method.

At first glance, \lstinline{IDEEditor} looks quite fine, but it has a subtle
issue. Recall that two mixins \lstinline{modalMixin} and \lstinline{spellMixin}
both provide a method \lstinline{onKey}, and the \lstinline{Editor} class also
defines an \lstinline{onKey} method of its own. Now we have a name clash. A
question arises as to which one gets picked inside the \lstinline{IDEEditor}
class. A typical mixin model resolves this issue by looking at the order of mixin applications. Mixins appearing later in the order
overrides \emph{all} the identically named methods of earlier mixins. So in our
case, \lstinline{onKey} in \lstinline{modalMixin} gets picked. If we
change the order of application to \lstinline{spellMixin(modalMixin(Editor))},
then \lstinline{onKey} in \lstinline{spellMixin} is inherited.

\paragraph{The Problem of Mixin Composition.}
From the above discussion, we can see that mixin are composed linearly: all the
mixins used by a class must be applied one at a time. However, when we wish to
resolve conflicts by selecting features from different mixins, we may not be
able to find a suitable order. For example, when we compose the two mixins to
make the class \lstinline{IDEEditor}, we can choose which of them comes first,
but in either order, \lstinline{IDEEditor} cannot access to the \lstinline{onKey}
method from the \lstinline{Editor} class.

\paragraph{Trait Model.}
Because of the total ordering and the limited means for resolving conflicts imposed by the mixin model,
researchers have proposed a simple compositional model called
traits~\citep{scharli2003traits, Ducasse_2006}. Traits are lightweight entities and serve as
the primitive units of code reuse. Among others, the key difference from
mixins is that the order of trait composition is irrelevant, and conflicting
methods must be resolved \emph{explicitly}. This gives programmers
fine-grained control, when conflicts arise, of selecting desired features from
different components. Thus we believe traits are a better model for multiple
inheritance in statically-typed object-oriented languages, and in \sedel we realize this
vision by giving traits a first-class status in the language,
achieving more expressive power compared with traditional (second-class) traits.


\paragraph{Summary of Typing Challenges.}
From our previous discussion, we can identify the following typing challenges
for a type system to accommodate the programming patterns (first-class classes/mixins)
we have just seen in a typed setting:
\begin{itemize}
\item How to account for, in a typed way, abstract methods and dynamic dispatch.
\item What are the types of first-class classes or mixins.
\item How to type dynamic inheritance.
\item How to express constraints on method presence and absence (the use of
  \lstinline{super} clearly demands that).
% \item How to ensure that composition of mixins is going to be valid, i.e., how
%   to reflect linearity in a type system.
\item In the presence of first-class traits, how to detect conflicts statically,
  even when the traits involved are not statically known.
\end{itemize}
\sedel elegantly solves the above challenges in a unified way, as
we will see next.


% From a pragmatic point of view, this implicit conflict resolution
% sometimes give programmers more surprises than convenience. What if the compiler can alarm us when a
% potential conflict may occur. Because of the dynamic nature of JavaScript, we
% would not know before actually running the code that there is a conflict. We
% miss the guarantee that a static type system can provide: such conflict can be
% detected at compile-time.

% Given the flexibility of first-class classes in dynamically-typed languages, we
% -- being advocates of statically-typed languages -- were wondering how to
% incorporate this same expressive power into statically-typed
% languages. As it
% turns out, designing a sound type system that fully supports first-class classes
% is notoriously hard; there are only a few, quite sophisticated, languages that
% manage this~\citep{DBLP:conf/oopsla/TakikawaSDTF12, DBLP:conf/ecoop/LeeASP15}. We
% pushed it further: \sedel has support for typed first-class
% traits.\bruno{Better to say there's no work on typed first-class
%   traits, and little work on first-class classes/mixins, despite
%  many dynamic languages prominently supporting such features.}

\subsection{A Glance at Typed First-Class Traits in \sedel}

We now rewrite the above code in \sedel, but this time with types. The resulting code has the same functionality as the dynamic version, but it is
statically typed. All code snippets in this and later sections are runnable in
our prototype. Before proceeding, we ask the reader to bear in mind that in this section we are not using traits
in the most canonical way, i.e., we use traits as if they are classes (but with
built-in conflict detection). This is because we are trying to stay as close as possible
to the structure of the JavaScript version for ease of comparison. In
\cref{sec:traits} we will remedy this to make better use of traits.

\paragraph{Simple Traits.}
Below is a simple trait \lstinline{editor}, which corresponds to the JavaScript
class \lstinline{Editor}. The \lstinline{editor} trait defines the same set of
methods: \lstinline{on_key}, \lstinline{do_cut} and \lstinline{show_help}:
\lstinputlisting[linerange=23-27]{./examples/overview2.sl}% APPLY:linerange=OVERVIEW_EDITOR
The first thing to notice is that \sedel uses a syntax (similar to Scala's
self type annotations~\citep{odersky2004overview}) where we can give a type annotation to the
\lstinline{self} reference. In the type of \lstinline{self} we use
\lstinline{&} construct to create intersection types. \lstinline{Editor} and \lstinline{Version} are two record types:
\lstinputlisting[linerange=10-17]{./examples/overview2.sl}% APPLY:linerange=OVERVIEW_EDITOR_TYPES
For the sake of conciseness, \sedel uses \lstinline{type} aliases to abbreviate types.

\paragraph{The type of \lstinline{self} Encodes Abstract Methods.}
Recall that in the JavaScript class \lstinline{Editor}, the \lstinline{version}
method is undefined, but is used inside \lstinline{showHelp}. How can we express
this in the typed setting, if not with an abstract method? In \sedel, the type of \lstinline{self}
plays the role of trait requirements. As the first approximation, we
can justify the invocation of \lstinline{version} on \lstinline{self} by noticing that (part of) the
type of \lstinline{self} (i.e., \lstinline{Version}) contains the declaration of
\lstinline{version}. An interesting aspect of \sedel's trait model is that there
is no need for abstract methods. Instead, abstract methods can be simulated as
requirements of a trait. Later, when the trait is composed with other
traits, \emph{all} requirements on the type of \lstinline{self} must be
satisfied and one of the traits in the composition must provide an
implementation of the method \lstinline{version}.
%to this point in \cref{sec:traits}.

As with the JavaScript version, the \lstinline{on_key} method is invoked on
\lstinline{self} in the body of \lstinline{do_cut}. This is allowed as (part of)
the type of \lstinline{self} (i.e., \lstinline{Editor}) contains the signature
of \lstinline{on_key}. Compared to the JavaScript class
\lstinline{Editor}, almost everything stays the same, except that we now have
a typed version. As a side note, since \sedel is currently a pure functional OO
language, there is no difference between fields and methods, so we can omit
empty arguments and parameter parentheses.

\paragraph{First-Class Traits and Trait Expressions.}

\sedel treats traits as first-class expressions, putting them in the same
syntactic category as objects, functions, and other primitive forms. To
illustrate this, we give the \sedel version of \lstinline{spellMixin}:
\lstinputlisting[linerange=31-43]{./examples/overview2.sl}% APPLY:linerange=OVERVIEW_HELP
This looks daunting at first, but \lstinline{spell_mixin} has almost the same structure as
its JavaScript cousin \lstinline{spellMixin}, albeit with
some type annotations. In \sedel, we use capital letters (\lstinline{A}, \lstinline{B}, $\dots$) to denote type variables, and trait
expressions \lstinline$trait [self : ...] inherits ... => {...}$ to create
first-class traits. Trait expressions have trait
types of the form \lstinline{Trait[T1, T2]} where \lstinline{T1} and \lstinline{T2} denote trait requirements and functionality respectively.
We will explain trait types in \cref{sec:traits}. Despite the structural similarities, there are several significant
features that are unique to \sedel (e.g., the disjointness operator \lstinline{*}).
We discuss these in the following.



\paragraph{Disjoint Polymorphism and Conflict Detection.}

\sedel uses a type system based on \emph{disjoint intersection types} (cf. \cref{chap:nested}) and
\emph{disjoint polymorphism} (cf. \cref{chap:fi}). Disjoint intersections
empower \sedel to detect conflicts statically when trying to compose two
traits with identically named features. For example, composing two traits
\lstinline{a} and \lstinline{b} that both provide \lstinline{foo} gives a
type error (the overloaded \lstinline{&} operator denotes trait composition):
\begin{lstlisting}
trait a => { foo = 1 };
trait b => { foo = 2 };
trait c inherits a & b => {}; -- type error!
\end{lstlisting}
Disjoint polymorphism, as a more advanced mechanism, allows detecting conflicts
even in the presence of polymorphism---for example when a trait is parameterized and its
full set of methods is not statically known. As can be seen,
\lstinline{spell_mixin} is actually a polymorphic function. Unlike ordinary
parametric polymorphism, in \sedel, a type variable can also have a disjointness
constraint. For instance, \lstinline{A * Spelling & OnKey}
means that \lstinline{A} can be instantiated to any type as long as it \emph{does not}
contain \lstinline{check} and \lstinline{on_key}. Note that these are the minimal constraints of \lstinline{A}, as
\begin{inparaenum}[(1)]
  \item \lstinline{A} cannot contain the \lstinline{on_key} method because otherwise it will conflict with \lstinline{Editor};
  \item \lstinline{A} cannot contain the \lstinline{check} method because otherwise it will conflict with that in the trait body.
\end{inparaenum}
To mimic mixins, the
argument \lstinline{base}, which is supposed to be some trait, serves as the
``base'' trait being inherited. Notice that the type variable
\lstinline{A} appears in the type of \lstinline{base}, which essentially states
that \lstinline{base} is a trait that contains at least those methods specified
by \lstinline{Editor}, and possibly more (which we do not know statically).
% In summary, \lstinline{Trait[Editor & Version, Editor & A]} (the assigned type
% of \lstinline{base}) specifies that both method \emph{presence} and \emph{absence}.
Also note that leaving out the \lstinline{override} keyword will result in a
type error. The type system is forcing us to be very specific as to what is the
intention of the \lstinline{on_key} method because it sees the same method is
also declared in \lstinline{base}, and blindly inheriting \lstinline{base}
will definitely cause a method conflict. As a final note, the use of \lstinline{super}
inside \lstinline{check} is allowed because the ``super'' trait \lstinline{base}
implements \lstinline{on_key}, as can be seen from its type.


\paragraph{Dynamic Inheritance.}

Disjoint polymorphism enables us to correctly type dynamic inheritance:
\lstinline{spell_mixin} is able to take any trait that conforms with its
assigned type, equips it with the \lstinline{check} method and overrides its
old \lstinline{on_key} method. As a side note, the use of disjoint polymorphism
is essential to correctly model the mixin semantics. From the type we know
\lstinline{base} has some features specified by \lstinline{Editor}, plus
something more denoted by \lstinline{A}. By inheriting \lstinline{base}, we are
guaranteed that the resulting trait will have everything that is already contained
in \lstinline{base}, plus more features. This is in some sense similar to row
polymorphism~\citep{wand1994type} in that the result trait is prohibited from
forgetting methods from the argument trait.
% As we will discuss in \cref{sec:related}, disjoint polymorphism is more expressive than row polymorphism.


\paragraph{Typing Mixin Composition.}
Next we give the typed version of \lstinline{modalMixin} as follows:
\lstinputlisting[linerange=48-56]{./examples/overview2.sl}% APPLY:linerange=OVERVIEW_MODAL
Now the definition of \lstinline{modal_mixin} should be self-explanatory.
Finally we can apply both ``mixins'' to \lstinline{editor} one at a time to create
an IDE editor:
\lstinputlisting[linerange=61-66]{./examples/overview2.sl}% APPLY:linerange=OVERVIEW_LINE
As with the JavaScript class \lstinline{IDEEditor}, we need to fill in the missing
\lstinline{version} method. It is easy to verify that the \lstinline{on_key} method
in \lstinline{modal_mixin} is inherited. Compared with the untyped version,
here this behaviour is reasonable because we specifically tag each
\lstinline{on_key} method to be an overriding method. Let us take a close look
at the mixin applications. Since \sedel is currently explicitly typed, we need to
provide concrete types when applying \lstinline{modal_mixin} and \lstinline{spell_mixin}.
In the inner application (\lstinline{spell_mixin Top editor}), we use the top
type \lstinline{Top} to instantiate \lstinline{A} because the \lstinline{editor} trait
provides exactly those method specified by \lstinline{Editor} and nothing more
(hence \lstinline{Top}). In the outer application, we use \lstinline{Spelling}
to instantiate \lstinline{A} because the resulting trait of the inner application
contains the \lstinline{check} method.
In summary, mixin applications are simply normal function applications,
and conflict resolution code is implicitly embedded via the keyword \lstinline{override}
and the order of mixin applications.
Unsurprisingly, changing the mixin application order to
\begin{lstlisting}
  inherits spell_mixin ModalEdit (modal_mixin Top editor)
\end{lstlisting}
gives the same (expected) behavior.


Admittedly the typed version is unnecessarily complicated as we were
mimicking mixins by functions over traits. The final trait
\lstinline{ide_editor} suffers from the same problem as the class
\lstinline{IDEEditor}, since there is no obvious way to access the
\lstinline{on_key} method in the \lstinline{editor} trait.\footnote{In fact, as
  we will see in \cref{sec:traits}, we can still access \lstinline{on_key} in
  \lstinline{editor} by the forwarding operator.} \cref{sec:traits}
makes better use of traits to simplify the editor code.



% Note that the use of \lstinline{override} is valid because the type system knows the inherit clause contains \lstinline{on_key}.
% As a bonus, since \sedel guarantees that there are no potential conflicts in a program,
% we can reason that the version number in \lstinline{modal_editor} is
% \lstinline{0.1}.

%%% Local Variables:
%%% mode: latex
%%% TeX-master: "../paper"
%%% org-ref-default-bibliography: ../paper.bib
%%% End:

\section{Typed First-Class Traits}
\label{sec:traits}

In \cref{sec:trait:overview} we have seen some examples of first-class traits at work
in \sedel. In this section we give a detailed account of \sedel's support for
typed first-class traits, to complement what has been presented so far. In doing so,
we simplify the examples in \cref{sec:trait:overview} to make better use of traits.
\Cref{sec:trait:typesystem} presents the formal type system of first-class traits.

\subsection{Traits in \sedel}

\sedel supports a simple, yet expressive form of traits~\citep{scharli2003traits}.
Traits provide a simple mechanism for fine-grained code reuse, which
can be regarded as a disciplined form of multiple inheritance. A trait is
similar to a mixin in that it encapsulates a collection of related methods to be
added to a class. The
practical difference between traits and mixins is the way conflicting features
that typically arise in multiple inheritance are dealt with. Instead of
automatically resolved by scoping rules, conflicts are, in \sedel,
detected by the type system, and explicitly resolved by the programmer. Compared
with traditional trait models, there are three interesting points about
\sedel's traits:
\begin{inparaenum}[(1)]
\item they are \emph{statically typed};
\item they are \emph{first-class} values;
\item they support \emph{dynamic inheritance}.
\end{inparaenum}
The support for such combination of features is one of the key novelties of \sedel.
Another minor difference from traditional traits (e.g., in Scala) is that,
due to the use of structural types, a trait name is \emph{not} a type.

\subsection{Two Roles of Traits in \sedel}

\paragraph{Traits as templates for creating objects.}

An obvious difference between traits in \sedel and many other models of
traits~\citep{scharli2003traits,fisher2004typed,odersky2005scalable} is that they
directly serve as templates for objects. In many other trait models, traits are
complemented by classes, which take the responsibility for object creation. In
particular, most models of traits do not allow constructors for traits. However,
a trait in \sedel has a single constructor of the same name. Take our last trait
\lstinline{ide_editor} in \cref{sec:trait:overview} for example:
\lstinputlisting[linerange=81-81]{./examples/overview2.sl}% APPLY:linerange=EDITOR_INST
As with conventional object-oriented languages, the keyword \lstinline{new} is used to create
an object. A difference to other object-oriented languages is that the keyword
\lstinline{new} also specifies the intended type of the object. We instantiate
the \lstinline{ide_editor} trait and create an object \lstinline{a_editor1} of
type \lstinline{IDEEditor}. As we will see in \cref{subsec:cons}, constructors
with parameters can also be expressed.

It is tempting to instantiate the \lstinline{editor} trait such as
\lstinline{new[Editor] editor}. However this would result in a type error, because, as
we discussed, \lstinline{editor} has no definition of \lstinline{version}, and
blindly instantiating it would cause run-time errors. This behavior is on a par
with Java's abstract classes---i.e., traits with undefined methods cannot be instantiated on their own.

\paragraph{Traits as units of code reuse.}

The traditional role of traits is to serve as units of code reuse. \sedel's traits
can have this role as well.
Our \lstinline{spell_mixin} function in \cref{sec:trait:overview} is more complicated than it should be.
This is because we were mimicking classes as traits, and
mixins as functions over traits. Instead, traits already provide a mechanism of
code reuse. To illustrate this, we simplify \lstinline{spell_mixin} as follows:
\lstinputlisting[linerange=111-114]{./examples/overview2.sl}% APPLY:linerange=HELP
This is much cleaner. The trait \lstinline{spell} adds a method
\lstinline{check}. It also defines a method \lstinline{on_key}.
A key difference from \lstinline{spell_mixin} is that \lstinline{on_key} is invoked on the \lstinline{self}
parameter instead of \lstinline{super}. Note that this does not necessarily mean \lstinline{check} will call \lstinline{on_key}
defined in the same trait. As we will see, the actual behavior entirely depends on how we compose \lstinline{spell}
with other traits. One minor difference is that we do not need to tag \lstinline{on_key}
with the \lstinline{override} keyword, because \lstinline{spell} stands as a standalone entity.
Another interesting point is that the type of \lstinline{self} (i.e., \lstinline{OnKey})
is not the same as that of the trait body, which also contains the \lstinline{check} method.
In \sedel's traits, the type of \lstinline{self} serves as trait \emph{requirements}.


\paragraph{Classes and/or traits.}

In the literature on traits~\citep{Ducasse_2006, scharli2003traits}, the
aforementioned two roles are considered as competing. One reason of the two
roles conflicting in class-based languages is because a class must adopt a fixed
position in the class hierarchy and therefore it can be difficult to reuse and
resolve conflicts, whereas in \sedel, a trait is a standalone entity and is not
tied to any particular hierarchy. Therefore we can view our traits either as templates for creating objects,
or as units of code reuse. Another important reason why our
model can do just with traits is because we have a pure language. Mutable state
can often only appear in classes in imperative models of traits, which is a good
reason for having both classes and traits.



% \lstinline{version} method is not defined. Like mixins, \lstinline{help} can be
% combined with other traits to produce several combinations of functionality. For
% instance, we create another editor that inherits two traits \lstinline{editor}
% and \lstinline{help}.
% \lstinputlisting[linerange=92-95]{./examples/overview2.sl}% APPLY:linerange=HELP2
% Due to the lack of multiple inheritance in JavaScript, we were forced to use
% mixins. In \sedel, this can be easily achieved because traits support multiple
% inheritance. In general, \lstinline{inherits ...} can take one or more trait
% expressions (delimited by \lstinline{&}).


\subsection{Trait Types and Trait Requirements}

\paragraph{Object types and trait types.}

\sedel adopts a relatively standard foundational model of object-oriented
constructs~\citep{DBLP:conf/ecoop/LeeASP15} where objects are encoded as records
with a structural type. This is why the object \lstinline{a_editor1}
has the record type \lstinline{IDEEditor}. In \sedel, an object type is
different from a trait type. A trait type is specified via \lstinline{Trait[T1, T2]}.
% For example, the type of the \lstinline{spell} trait is \lstinline{Trait[OnKey, OnKey & Spelling]}.

\paragraph{Trait requirements and functionality.}

In general, a trait type
\lstinline{Trait[T1, T2]} specifies both the \emph{requirements} \lstinline{T1}
and the \emph{functionality} \lstinline{T2} of a trait. The requirements of a trait denote the types/methods that the
trait needs to support for defining the functionality it provides. % Both are
% reflected in the trait type.
For example, \lstinline{spell} has type
\lstinline{Trait[OnKey, OnKey & Spelling]}, meaning that \lstinline{spell}
requires some implementation of the \lstinline{on_key} method, and it provides
implementations for the \lstinline{on_key} and \lstinline{check} methods.
When a trait
has no requirements, the absence of a requirement is denoted by using
the top type \lstinline{Top}. A simplified sugar \lstinline{Trait[T]} is
used to denote a trait without requirements, but providing functionality \lstinline{T}.




\paragraph{Trait requirements as abstract methods.}

Let us go back to our very first trait \lstinline{editor} in
\cref{sec:trait:overview}. Note how in \lstinline{editor} the type of the
\lstinline{self} parameter is \lstinline{Editor & Version}, where
\lstinline{Version} contains a declaration of the \lstinline{version} method
that is needed for the definition of \lstinline{show_help}. Note also that the
trait itself does not actually contain a \lstinline{version} definition. In many
other object-oriented models a similar program could be achieved by having an \emph{abstract}
definition of \lstinline{version}. In \sedel there are no abstract definitions
(methods or fields), but a similar result can be achieved via trait
requirements. Requirements of a trait are met at the object creation point. For
example, as we mentioned before, the \lstinline{editor} trait alone cannot be
instantiated since it lacks \lstinline{version}. However, when it is composed
with a trait that provides \lstinline{version}, the composition can be
instantiated, as shown below:
\lstinputlisting[linerange=92-95]{./examples/overview2.sl}% APPLY:linerange=HELP2
\sedel uses a syntax where the self parameter can be explicitly named (not
necessarily named \lstinline{self}) with a type annotation. When the self
parameter is omitted (for example in the \lstinline{foo} trait above), its type
defaults to \lstinline{Top}. This is different from most object-oriented languages, where
the default type of the self parameter is the same as the class being defined.
This also makes trait requirements ``pay as you go'' in the sense that if the
self parameter is not used in the body, then there is no requirements on the
trait. Otherwise, suppose the type of the self parameter in the trait
\lstinline{foo} implicitly defaults to \lstinline{Version}:
\lstinputlisting[linerange=104-106]{./examples/overview2.sl}% APPLY:linerange=HELP3
then \lstinline{Version} will pollute the type of the self parameter of any trait that
uses \lstinline{foo}, cascading down the inheritance hierarchy, even though \lstinline{self}
is not used in the body of \lstinline{foo}.



\paragraph{Intersection types model subtyping.}

The type \lstinline{IDEEditor} is defined as an intersection type.
Intersection types~\citep{coppo1981functional,pottinger1980type} have been woven
into many modern languages these days. A notable example is Scala, which makes
fundamental use of intersection types to express a class/trait that extends
multiple other traits. An intersection type such as \lstinline{T1 & T2} contains
exactly those values which can be used as values of type \lstinline{T1} and of
type \lstinline{T2}, and as such, \lstinline{T1 & T2} immediately introduces a
subtyping relation between itself and its two constituent types \lstinline{T1}
and \lstinline{T2}. Unsurprisingly, \lstinline{IDEEditor} is a subtype of
\lstinline{Editor}.


% \paragraph{Composition of traits}


% The definition of the object \lstinline{my_editor2} also shows the second way to
% introduce inheritance, namely by \emph{composition} of traits. Composition of
% traits is denoted by the operator \lstinline{&}. Thus \sedel offers two options
% when it comes to inheritance: we can either compose beforehand when declaring
% traits (using \lstinline{inherits}), or compose at the object creation point
% (using \lstinline{new} and the \lstinline{&} operator).

%Under the hood, inheritance is accomplished by using the \emph{merge operator}
%(denoted by \lstinline{,,}). The merge operator~\citep{dunfield2014elaborating}
%allows two arbitrary values to be merged, with the resulting type being an
%intersection type.
%For example the type of \lstinline{2 ,, true} is
%\lstinline{Int & Bool}.


\subsection{Traits with Parameters and First-Class Traits}

\label{subsec:cons}

So far our uses of traits involve no parameters. Instead of inventing another trait
syntax with parameters, a trait with parameters is just a function that produces
a trait expression, since functions already have parameters of their own. This
is one benefit of having first-class traits in terms of language economy. To
illustrate, let us simplify \lstinline{modal_mixin} in a similar way as in \lstinline{spell_mixin}:
\lstinputlisting[linerange=118-122]{./examples/overview2.sl}% APPLY:linerange=MODAL2
The first thing to notice is that \lstinline{modal} is a function with one
argument, and returns a trait expression, which essentially makes
\lstinline{modal} a trait with one parameter.
Now it is easy to see that a trait declaration
\lstinline$trait name [self : ...] => {...}$ is just syntactic sugar for
function definition \lstinline$name = trait [self : ...] => {...}$. The body of
the \lstinline{modal} trait is straightforward. We initialize the
\lstinline{mode} field to \lstinline{init_mode}.
The \lstinline{modal} trait also comes with a constructor with one parameter---e.g., we can create an object via \lstinline{new[ModalEdit] (modal "insert")}.

\subsection{Detecting and Resolving Conflicts in Trait Composition}
\label{sec:trait:forward}

A common problem in multiple inheritance is how to detect and/or resolve conflicts. For example, when
inheriting from two traits that have the same field, then it is unclear which
implementation to choose. There are various approaches to dealing with
conflicts. The trait-based approach requires conflicts to be resolved at the
level of the composition, otherwise the program is rejected by
the type system. \sedel provides a means to help resolve conflicts.

We start by assembling all the traits defined in this section
to create the final editor with the same functionality as
\lstinline{ide_editor} in \cref{sec:trait:overview}. Our first try is as follows:
\lstinputlisting[linerange=132-136]{./examples/overview2.sl}% APPLY:linerange=MODAL_CONFLICT
Unfortunately the above trait gets rejected by \sedel because
\lstinline{editor}, \lstinline{spell} and \lstinline{modal} all define an \lstinline{on_key} method.
Recall that in \cref{sec:trait:overview}, when we use a mixin-style composition,
the conflict resolution code has been hardwired in the definition.
However, in a trait-style composition, this is not the case: conflicts must be resolved \emph{explicitly}.
The
above definition is ill-typed precisely because there is a conflicting
method \lstinline{on_key}, thus violating the disjointness conditions
imposed by disjoint intersection types.

\paragraph{Resolving conflicts.}

To resolve the conflict, we need to explicitly state which implementation of the method
\lstinline{on_key} gets to stay. \sedel provides such a means---the \emph{exclusion} operator (denoted by \lstinline$\$)---which allows one to
exclude a field/method from a given trait. The following matches the behavior
in \cref{sec:trait:overview} where \lstinline{on_key} from the \lstinline{modal} trait
is selected:
\lstinputlisting[linerange=143-149]{./examples/overview2.sl}% APPLY:linerange=MODAL_OK
Now the above code type checks. We can also select \lstinline{on_key} from the \lstinline{spell} trait as easily:
\lstinputlisting[linerange=154-160]{./examples/overview2.sl}% APPLY:linerange=MODAL_OK2
In \cref{sec:trait:overview} we mentioned that in the mixin style, it is impossible
to select \lstinline{on_key} from the \lstinline{editor} trait, but this is not a problem now:
\lstinputlisting[linerange=164-170]{./examples/overview2.sl}% APPLY:linerange=MODAL_OK3
Using the exclusion operator, we can drop \lstinline{on_key} in \lstinline{spell} and \lstinline{modal} while
keeping it in \lstinline{editor}.


\paragraph{The forwarding operator.}

Another operator that \sedel provides is the \emph{forwarding} operator, which can be useful when we want to access some method that has been
explicitly excluded in the \lstinline{inherits} clause. This is a common scenario in
diamond inheritance, where \lstinline{super} is not enough. Below we show a
variant of \lstinline{ide_editor}:
\lstinputlisting[linerange=175-183]{./examples/overview2.sl}% APPLY:linerange=MODAL_WIRE
Notice that \lstinline{on_key} in \lstinline{spell} has been
excluded. However, we can
still access it by using the forwarding operator as in \lstinline{spell ^ self},
which gives full access to all the methods in \lstinline{spell}. Also note that
using \lstinline{super} only gives us access to \lstinline{on_key} in the
\lstinline{modal} trait. To see \lstinline{ide_editor4} in action, we create a
small test:
\lstinputlisting[linerange=187-190]{./examples/overview2.sl}% APPLY:linerange=MODAL_USE
% \jeremy{Compare this to trait alias operator?}
% \jeremy{would it be better to use forwarding operator in mix-style \lstinline{ide_editor} in
%   \cref{sec:trait:overview}, where we show how to access \lstinline{on_key} from editor, which is impossible in JavaScript?}

% Since the result editor trait has such an exciting feature, we increment the version number to \lstinline{0.2}!



\subsection{Disjoint Polymorphism and Dynamic Composition}

\sedel supports disjoint polymorphism. The combination of disjoint
polymorphism and first-class traits enables the highly modular code
where traits with \emph{statically unknown} types can be instantiated
and composed in a type-safe way! The following is illustrative of this:
%However, this is not a problem in \sedel. Thanks to disjoint polymorphism and
%disjoint intersection types, we can define the same \lstinline{merge} that is able to take two
%traits (where the full set of the members may not be known statically), combine and instantiate them.
\lstinputlisting[linerange=6-6]{./examples/overview2.sl}% APPLY:linerange=MERGE
The \lstinline{mergeTraits} function takes two traits \lstinline{x} and \lstinline{y} of
some arbitrary types \lstinline{Trait[A]} and \lstinline{Trait[B]}, composes them,
and creates an object from the resulting composed trait. Clearly
such composition cannot always work if \lstinline{A} and
\lstinline{B} can have conflicts. However, \lstinline{mergeTraits} has a
constraint \lstinline{B * A} that ensures that whatever types are used
to instantiate \lstinline{A} and \lstinline{B} they must be disjoint.
Thus, under the assumption that \lstinline{A} and \lstinline{B} are
disjoint the code type-checks.


%%% Local Variables:
%%% mode: latex
%%% TeX-master: "../paper"
%%% org-ref-default-bibliography: ../paper.bib
%%% End:

\renewcommand{\rulehl}[2][gray!40]{%
  \colorbox{#1}{$\displaystyle#2$}}

\section{Formalizing Typed First-Class Traits}
\label{sec:typesystem}

This section presents the syntax and semantics of \sedel. In particular,
we show how to elaborate high-level source language constructs (self-references, abstract methods, first-class traits, dynamic inheritance, etc)
in \sedel to \fname, a pure record calculus with disjoint
polymorphism. The treatment of the self-reference and dynamic dispatching is
inspired by Cook and Palsberg's work on the denotational semantics for
inheritance~\cite{cook1989denotational}. We then prove the elaboration is type
safe, i.e., well-typed \sedel expressions are translated to well-typed \fname
terms. Finally we show that \sedel is coherent. Full proofs can be found in the appendix.

\subsection{Syntax}

\begin{figure}[t]
\centering
\begin{tabular}{lrcl}
  Types  & $[[AA]], [[BB]], [[CT]]$ & $\Coloneqq$ & $[[Top]] \mid [[nat]] \mid [[AA -> BB]] \mid [[AA & BB]] \mid  [[{ l : AA }]] \mid [[X]] \mid [[\ X ** AA  . BB]] \mid \hlmath{[[ Trait[AA,BB] ]]}$ \\
  Expressions & $[[E]]$ & $\Coloneqq$ & $[[Top]] \mid [[ii]] \mid [[x]] \mid [[\ x . E]] \mid [[E1 E2]] \mid [[\ X ** AA  . E]] \mid [[E AA]] \mid [[E1 ,, E2]] \mid [[E : AA]] $ \\
         & & $\mid$ & $[[{ l = E }]] \mid [[E . l]] \mid [[letrec x : AA = E1 in E2]] \mid \hlmath{[[new [ AA ] (</ Ei // i />) ]]} \mid \hlmath{[[E1 ^ E2]]} $ \\
  Value contexts & $[[SG]]$ & $\Coloneqq$ & $[[empty]] \mid [[SG , x : AA]] $ \\
  Type contexts & $[[SD]]$ & $\Coloneqq$ & $[[empty]] \mid [[SD , X ** AA]]$ \\ \\
\end{tabular}
\begin{tabular}{llll}
  Record types & $[[ { l1 : AA1 , ... , ln : AAn } ]] $ & := & $[[ { l1 : AA1} & ... & { ln : AAn } ]]$ \\
  Records &  $[[ { l1 = E1 , ... , ln = En } ]] $ & := & $ [[ { l1 = E1 } ,, ... ,, { ln = En } ]]$
\end{tabular}
\caption{\sedel core syntax and syntactic abbreviations}
\label{fig:sedel_syntax}
\end{figure}

The core syntax of \sedel is shown in \cref{fig:sedel_syntax}, with trait related
constructs \hll{highlighted}. For brevity of the meta-theoretic study, we do not
consider definitions, which can be added in standard ways.
%We omit mutable fields and other practical
%constructs in order to focus on the basic mechanisms of traits. The omitted
%constructs can be added in standard ways~\cite{DBLP:books/daglib/0005958}.

\paragraph{Types.}
Metavariables $[[AA]]$, $[[BB]]$, $[[CT]]$ range over types. Types include a top
type $[[Top]]$, type of integers $[[nat]]$, function types $[[AA -> BB]]$, intersection types $[[AA & BB]]$,
singleton record types $[[{l : AA}]]$,  type variables $[[X]]$ and disjoint
(universal) quantification $[[\ X ** A . B]]$. The main
novelty is the type of first-class traits $[[ Trait[AA, BB] ]]$, which expresses
the requirement $[[AA]]$ and the functionality $[[BB]]$. We will use $[[ [ AA / X ] BB ]]$
to denote capture-avoiding substitution of $[[AA]]$ for $[[X]]$ inside $[[BB]]$.


\paragraph{Expressions.}
Metavariable $[[E]]$ ranges over expressions. We start with constructs required
to encode objects based on records: term variables $[[x]]$, lambda abstractions $[[\x. E]]$, function
applications $[[E1 E2]]$, singleton records $[[{l = E}]]$, record projections
$[[E.l]]$, recursive let bindings $[[letrec x : AA = E1 in E2]]$, disjoint type
abstraction $[[\ X ** AA . E]]$ and type application $[[E AA]]$.
The calculus also supports a merge construct $[[E1 ,, E2]]$ for creating values of intersection
types and annotated expressions $[[E : AA]]$. We also include a canonical top
value $[[Top]]$ and integer literals $[[ii]]$.

\paragraph{First-class traits and trait expressions.}
The central construct of \sedel is the trait
expression%\footnote{The abstract syntax of trait expressions is slightly different from the concrete syntax.}
$[[ trait [ self : BB ] inherits </ Ei // i /> { </ lj = Ej' // j /> } : AA]]$,
which specifies a (possibly empty) list
of trait expressions $\overline{[[Ei]]}$ in the \lstinline{inherits} clause, an explicit
$[[self]]$ reference (with type annotation $[[B]]$), and a set of
methods $\{ \overline{l_j = E'_j} \}$. Intuitively this trait expression has
type $[[ Trait[BB, AA] ]]$. Unlike the conventional trait model, a trait
expression denotes a first-class value: it may occur anywhere where an
expression is expected. Trait instantiation expressions $[[new [ AA ] (</ Ei // i />) ]]$
instantiate a composition of trait expressions $\overline{[[Ei]]}$ to create an
object of type $[[AA]]$. Finally $[[E1 ^ E2]]$ is the forwarding expression,
where $[[E1]]$ should be some trait.

\paragraph{Abbreviations.}
For ease of programming, multiple-field record types are merely syntactic sugar
for intersections of single-field record types. Similarly, multi-field record
expressions are syntactic sugar for merges of single-field records.

\subsection{Semantics}

\begin{figure}[t]
  \centering
  \drules[TS]{$[[ AA <: BB ]]$}{Subtyping}{arr, trait}
  \drules[WF]{$[[ SD |- AA ]]$}{Well formedness}{and, trait}
  \caption{Subtyping and well-formedness of \sedel (excerpt)}
  \label{fig:typesystem}
\end{figure}


\paragraph{Subtyping and Well-formedness.}
\Cref{fig:typesystem} shows the most relevant subtyping and well-formedness
rules for \sedel. Omitted rules are standard and can be found in previous
work~\cite{alpuimdisjoint}. The
subtyping rule for trait types (\rref{TS-trait}) resembles the one for function
types (\rref{TS-arr}) in that it is contravariant on the first type $[[AA]]$
and covariant on the second type $[[BB]]$. The well-formedness rule for trait
types is straightforward.

\begin{figure}[t]
  \centering
\begin{small}
  \drules[SD]{$[[ SD |- AA ** BB ]]$}{Disjointness}{top, topSym, var, varSym, forall, rec, recn, arrow, andL, andR, trait, traitArrOne, traitArrTwo, ax}
  \drules[Dax]{$[[ AA **a BB ]]$}{Disjointness axiom}{intTrait, traitForall, traitRec}
\end{small}
\caption{Disjointness rules of \sedel (excerpt)}
  \label{fig:disjoint}
\end{figure}


\paragraph{Disjointness.}
\Cref{fig:disjoint} shows the disjointness judgment $[[SD |- AA ** BB]]$, which is
used for example in \rref{WF-and}. The disjointness checking is the underlying
mechanism of conflict detection. We naturally extend the disjointness rules in
\fname to cover trait types. We refer to
their paper~\cite{alpuimdisjoint} for further explanation. Here we discuss
the rules
related with traits. \Rref{SD-trait} says that as long as the functionalities
that two traits provide are disjoint, the two trait types are disjoint.
\Rref{SD-traitArr1,SD-traitArr2} deal with situations where one of the two types
is a function type. At first glance, these two look strange because a trait type is
\textit{different} from a function type, and they ought to be disjoint as an axiom. The reason
is that \sedel has an elaboration semantics, and as we will see, trait types are translated to function
types. In order to ensure the elaboration is type-safe, we have to have special treatment for trait
and function types. In principle, if \sedel has its own semantics, then trait types are always disjoint
to function types. The axiom rules of the form $[[ AA **a BB ]]$ take care of two types with different language constructs.
% These rules capture the notion that any two types are disjoint unless one of
% them is an intersection types, a type variable or $[[top]]$.

\begin{figure}[t]
  \centering
  \begin{small}
  \drules[ST]{$[[ SD ; SG  |- E => AA ~~> ee]]$}{Infer}{trait,traitSuper,forward,new}
  \end{small}
  \caption{Typing of \sedel (excerpt)}
  \label{fig:type}
\end{figure}

\paragraph{Typing Traits.}
The typing rules of trait related constructs are shown in \cref{fig:type}. The full set of rules can be found in the appendix.
The reader is advised to ignore the \hll{highlighted} parts for now.
% As with conventional bidirectional type systems,
\sedel employs two modes: the inference mode
($[[=>]]$) and the checking mode ($[[<=]]$). The inference judgment $[[ SD ; SG |- E => AA]]$
says that we can synthesize a type $[[AA]]$ for expression $[[E]]$.
The checking judgment $[[SD; SG |- E <= AA]]$ checks $[[E]]$ against $[[AA]]$. One representative of inference rules is
% \begin{mathpar}
%  \drule{st-merge}
% \end{mathpar}
which says that a merge of two expressions is valid only if their types are disjoint. This is the underlying
mechanism for conflict detection. One representative of checking rules is
% \begin{mathpar}
%  \drule{ST-sub}
% \end{mathpar}
% typically known as the subsumption rule,
where subtyping is used to coerce expressions of one type to another.


To type-check a trait (\rref{ST-trait}) we first type-check if its inherited traits $\overline{[[Ei]]}$ are valid
traits. Note that each trait $[[Ei]]$ can possibly refer to $[[self]]$. Methods
must all be well-typed in the usual sense. Apart from these, we have several
side-conditions to make sure traits are well-behaved. The well-formedness
judgment $[[SD |- CT1 & .. & CTn & CT]]$ ensures that we do not have conflicting
methods (in inherited traits and the body). The subtyping judgments $\overline{[[BB <: BBi]]}$ ensure that the
$[[self]]$ parameter satisfies the requirements imposed by each
inherited trait. Finally the subtyping judgment $[[CT1 & .. & CTn & CT <: AA]]$
sanity-checks that the assigned type $[[AA]]$ is compatible.

Trait instantiation (\rref{ST-new}) requires that each instantiated trait is valid.
There are also several side-conditions, which serve the same
purposes as in \rref{ST-trait}.
\Rref{ST-forward} says that the first operand $[[E1]]$ of the forwarding operator must be a trait. Moreover, the type of the second operand
$[[E2]]$ must satisfy the requirement of $[[E1]]$.



\paragraph{Treatments of Exclusion, Super and Override.}
One may have noticed that in \cref{fig:sedel_syntax} we did not include the
exclusion operator in the core \sedel syntax, neither do \lstinline{super} and
\lstinline{override} appear. The reason is that in principle all
uses of the exclusion operator can be replaced by type annotations. For example
to exclude a \lstinline{bar} field from \lstinline${foo = a, bar = b, baz = c}$,
all we need is to annotate the record with type \lstinline${foo : A, baz : C}$
(suppose \lstinline{a} has type \lstinline{A}, etc). By \rref{Chk-sub}, the resulting
record is guaranteed to contain no \lstinline{bar} field. In the same vein,
the use of \lstinline{override} can be explained using the exclusion operator.
The \lstinline{super} keyword is internally a variable pointing to the \lstinline{inherits} clause
(its typing rule is similar to \rref{inf-trait} and can be found in the appendix).
We omit all of these features in the meta-theoretic study in order to focus our attention on
the essence of first-class traits.
However in practice, this is rather inconvenient as we need to write down
all types we wish to retain rather than the one to exclude. So in our
implementation we offer all of them.

\paragraph{Elaboration.}

The operational semantics of \sedel is given by means of a type-directed
translation into \fname extended with (lazy) recursive let bindings.
This extension is standard and type-safe. Let us go back to
\cref{fig:type}, now focusing on the \hll{highlighted} parts, which
denote the elaborated \fname terms. Most of them
are straightforward translations and are thus omitted. We explain the most
involved rules regarding traits. In \rref{Inf-trait}, a trait is translated into
a lambda abstraction with $[[self]]$ as the formal parameter.
In essence a trait corresponds to what Cook and Palsberg~\cite{cook1989denotational} call a \emph{generator}.
 The translations
of the inherited traits (i.e., $\overline{[[eei]]}$) are each applied to
$[[self]]$ and then merged with the translation of the trait body $[[ee]]$. Now
it is clear why we require $[[BB]]$ (the type of $[[self]]$) to be a subtype of each
$[[BBi]]$ (the requirement of each inherited trait). Note that we abuse the bar
notation here with the intention that $[[</ (eei self) // i IN 1..n />]]$ means
$[[ee1 self ,, .. ,, een self]]$.
Here is an example of translating the \lstinline{ide_editor} trait from \cref{sec:overview} into
plain \fname terms equipped with definitions (suppose \lstinline{modal_mixin} and \lstinline{spell_mixin}
have been translated accordingly):
\lstinputlisting[linerange=62-63]{./examples/overview2.sl}% APPLY:linerange=TRANS

The gray parts in \rref{ST-new} show the translation of trait instantiation.
First we apply every translation (i.e., $[[eei]]$) of the instantiated traits to the $[[self]]$ parameter,
and then merge the applications together. The bar notation is
interpreted similarly to the translation in \rref{ST-trait}. Finally we compute the \emph{lazy}
fixed-point of the resulting merge term, i.e., self-reference must be updated to refer to
the whole composition. Taking the fixed-point of the
traits/generators again follows the denotational inheritance model by
Cook and Palsberg.
 This is the key to the correct implementation of dynamic
 dispatching. Finally,
\rref{ST-forward} translates forwarding expressions to function
applications. We show the translation of the
\lstinline{a_editor1} object in \cref{sec:traits} to illustrate the
translation of instantiation:
\lstinputlisting[linerange=71-71]{./examples/overview2.sl}% APPLY:linerange=NEW

One remarkable point is that, while Cook and Palsberg work is done in
an untyped setting, here we apply their ideas in a setting with
disjoint intersection types and disjoint polymorphism. Our work shows that
disjoint intersection types blend in quite nicely with Cook and
Palsberg's denotational model of inheritance.

\paragraph{Flattening Property.}

In the literature of traits~\cite{Ducasse_2006, scharli2003traits, JOT:issue_2006_05/article4},
a distinguished feature of traits is the
so-called \textit{flattening property}. This property says that a (non-overridden) method in a
trait has the same semantics as if it were implemented directly in the class
that uses the trait. It would be interesting to see if our trait model has this
property. One problem in formulating such a property is that flattening is a
property that talks about the equivalence between a flattened class (i.e., a
class where all trait methods have been inlined) and a class that reuses code
from traits. Since \sedel does not have classes, we cannot state exactly the same
property. However, we believe that one way to talk about a similar property for \sedel is to have something
along the lines of the following example:
\begin{example}[Flattening]
  Suppose we have \lstinline$m$ well-typed (i.e, conflict-free) traits \lstinline$trait t1 {l11 = E11, ..}, ..., trait tm {lm1 = Em1, ..}$,
  each with some number of methods, then
  \begin{center}
   \lstinline|new (trait inherits t1 & ... & tm {})|  $=$  \lstinline|new (trait {l11 = E11,..,lm1 = Em1,..})|
  \end{center}
\end{example}
If we elaborate these two expressions, the property boils down to whether two merge terms
$[[(ee1 ,, ee2) ,, ee2]]$ and $[[ee1 ,, (ee2 ,, ee3)]]$
have the same semantics. As is shown by Bi et al.~\cite{xuan_nested}, merges are
associative and commutative, so it is not hard to see that the above two expressions
are semantically equivalent. We leave it as future work to formally state and prove flattening.


% no gray anymore after this point
\renewcommand{\rulehl}[1]{#1}

\subsection{Type Soundness and Coherence}

Since the semantics of \sedel is defined by elaboration into \fname it
is easy to show that key properties of \fname are also guaranteed by \sedel.
In particular, we show that the type-directed elaboration is
type-safe in the sense that well-typed \sedel expressions are elaborated into
well-typed \fname terms. We also show that the source language is
coherent and each valid source program has a unique (unambiguous)
elaboration.

We need a meta-function $| \cdot |$ that translates \sedel types to \fname types, whose definition is
straightforward. Only the translation of trait types deserves attention:
\begin{mathpar}
  | [[Trait[AA,BB] ]] | = [[|AA| -> |BB|]]
\end{mathpar}
That is, trait types are translated to
function types. $| \cdot |$ extends naturally to typing contexts.
Now we show several lemmas that are useful in the type-safety proof.

\begin{lemma}
  If $[[SD |- AA]]$ then $[[|SG| |- |AA|]]$.
\end{lemma}
\begin{proof}
  By structural induction on the well-formedness judgment.
\end{proof}

\begin{lemma}
  If $[[AA <: BB]]$ then $[[|AA| <: |BB|]]$.
\end{lemma}
\begin{proof}
  By structural induction on the subtyping judgment.
\end{proof}

\begin{lemma}
  If $[[SD |- AA ** BB]]$ then $[[ |SD| |- |AA| ** |BB| ]]$.
\end{lemma}
\begin{proof}
  By structural induction on the disjointness judgment.
\end{proof}
% \begin{remark}
%  Due to the elaboration semantics, \rref{D-traitArr1,D-traitArr2} are needed to make this lemma hold.
% \end{remark}


Finally we are in a position to establish the type safety property:
\begin{theorem}[Type-safe translation]
  We have that:
  \begin{itemize}
  \item If $[[SD ; SG  |- E => AA ~~> ee]]$ then $ [[ |SD| ;  |SG|  |- ee => |AA| ]] $.
  \item If $[[SD ; SG  |- E <= AA ~~> ee]]$ then $ [[ |SD| ;  |SG|  |- ee <= |AA| ]] $.
  \end{itemize}
\end{theorem}
\begin{proof}
    By structural induction on the typing judgment.
\end{proof}

\begin{theorem}[Coherence] Each well-typed \sedel expression has a unique elaboration.
\end{theorem}
\begin{proof}
  By examining every elaboration rule, it is easy to see that the elaborated
  \fname term in the conclusion is uniquely determined by the elaborated \fname
  terms in the premises. Then by the coherence property of \fname, we conclude
  that each well-typed \sedel expression has a unique unambiguous elaboration,
  thus \sedel is coherent.
\end{proof}


%%% Local Variables:
%%% mode: latex
%%% TeX-master: "../Thesis"
%%% org-ref-default-bibliography: ../Thesis.bib
%%% End:
