%%%%%%%%%%%%%%%%%%%%%%%%%%%%%%%%%%%%%%%%%%%%%%%%%%%%%%%%%%%%%%%%%%%%%%%%
\chapter{Conclusion and Future Work}
\label{chap:conclusion}

%%%%%%%%%%%%%%%%%%%%%%%%%%%%%%%%%%%%%%%%%%%%%%%%%%%%%%%%%%%%%%%%%%%%%%%%


\section{Conclusion}

In this thesis we have argued that the combination of disjoint intersection
types, a powerful subtyping relation and parametric polymorphism greatly improve
the state-of-art technique for modularity and code reuse. In the course of our
investigation, we have gradually introduced three new typed calculi, each
building on top of the previous one:
\begin{itemize}
\item The \namee calculus is the basic calculus with disjoint intersection types
  and a powerful subtyping relation. We have shown that it captures the essence
  of nested composition, enabling a simple solution to the expression problem.
  In order to prove coherence, we have introduced a powerful proof method based
  on logical relations.
\item The \fnamee calculus, building on \namee, supports parametric
  polymorphism. We have shown that it can express a very dynamic (and
  conflict-free) form of composition. We have also adapted the proof method to
  establish its coherence property.
\item \sedel, building on \fnamee, supports, among others, typed first traits.
  We have shown its usefulness in building highly reusable software components
  using a improved form of Object Algebras.
\end{itemize}

We hope that the concepts and the methods described in this thesis may serve as
a helpful guide to researchers and programmers alike in their attempts to
understand and build better software. In a near future, we are more than
thrilled to see some of the ideas in this thesis can be woven into existing languages
of today.


\section{Future Work}

In this section we discuss some areas where future research might extend and/or
complement the work described in this thesis.

\subsection{On Categorical Semantics}
\label{sec:category}

An interesting avenue for future work is to give a categorical semantics of
disjoint intersection types. The main reason for doing so is that, as
\citet{reynolds1988preliminary} nicely put it:
\begin{quote}
  ``by formulating succinct definitions in terms of a mathematical theory of
  great generality, we gain an assurance that our language will be uniform and
  general.''
\end{quote}
Using category theory as the basis for the type structure of a programming
language has a long history. \citet{lambek1985cartesian} discovered that
simply-typed lambda calculus can be interpreted in any Cartesian closed
category. \citet{Reynolds_1991} gives a category-theoretic presentation of a
lambda calculus extended to include records, fixed points and
intersection types, much similar to our \namee. Of particular interest to us is
his method for proving coherence. Let $[[D]]$ denote derivations of typing, then
the interpretation of a derivation $[[ D ; GG |- ee : A ]]$ is a morphism
$\bra{[[ D ; GG |- ee : A ]]} : \bra{[[GG]]} \rightarrow \bra{[[A]]} $ in a
suitable ``semantic'' category (i.e., being Cartesian closed and possessing
certain limits). Proving coherence in this presentation then amounts to
establishing the commutativity of all diagrams of the following
form\footnote{The proof actually needs a stronger inductive hypothesis.}:
\[
\begin{tikzcd}
\bra{[[  GG   ]]} \arrow[rrr, "\bra{[[ D1 ; GG |- ee : A  ]]}", bend left] \arrow[rrr, "\bra{[[ D2 ; GG |- ee : A  ]]} "', bend right] &  &  & \bra{[[ A ]]}
\end{tikzcd}
\]


\paragraph{Properties of Intersection Types.}

The key component of Reynolds' method is the interpretation of intersection
types. For the sake of precision in what follows, we pause to give some basic
properties of intersection types that are first proved by \citet{Reynolds_1991}.
First we give two definitions that are important for the discussion.

\begin{definition}[Type Equivalence]
  Two types $[[A]]$ and $[[B]]$ are equivalent, written $[[ A == B ]]$, when $[[ A <: B ]]$ and $[[B <: A]]$.
\end{definition}

\begin{definition}[Least Upper Bounds]
  A \textit{least upper bound} of $[[A]]$ and $[[B]]$, written $[[A =/ B]]$\footnote{Note
    that the meta-function $\sqcup$, unlike $\&$, is not a type constructor.},
  is a supertype of both $[[A]]$ and $[[B]]$ and a subtype of every common
  supertype of $[[A]]$ and $[[B]]$---i.e., a type $[[C]]$ such that:
  \begin{itemize}
  \item $[[A <: C]]$
  \item $[[B <: C]]$
  \item For any $[[C']]$, $[[A <: C']]$ and $[[B <: C']]$ implies $[[C <: C']]$
  \end{itemize}
\end{definition}

According to the subtyping rules in \cref{fig:subtype_decl}, we can derive the
following type equalities:

\begin{proposition} \label{prop:1}%
\begin{align*}
  [[A1 & (A2 & A3) ]]  &\approx  [[(A1 & A2) & A3]] \\
  [[ Top & A ]] &\approx [[A]] \\
  [[ A & Top ]] &\approx [[A]] \\
  [[A1 & A2 ]]  &\approx  [[ A2 & A1 ]] \\
  [[A & A ]]  &\approx  [[ A ]] \\
  [[ {l : A1 & A2}   ]] &\approx [[  {l : A1}  & {l : A2} ]] \\
  [[  A -> A1 & A2  ]] &\approx [[  (A -> A1) & (A -> A2)   ]] \\
  [[  {l : Top}    ]] &\approx [[  Top   ]] \\
  [[  A -> Top  ]] &\approx [[  Top   ]]
\end{align*}
\end{proposition}

Furthermore, it can be shown that every pair of \namee types has a least upper bound (unique up to
$\approx$-equivalence). The following suffices to compute a least upper bound of
any types $[[A]]$ and $[[B]]$:

\begin{proposition} \label{prop:2}%
\begin{align*}
  [[  A =/ B   ]] &\approx [[B =/ A]] \\
  [[  A =/ Top   ]] &\approx [[Top]] \\
  [[  A1 =/ (A2 & A3)  ]] &\approx [[  (A1 =/ A2) & (A1 =/ A3)  ]] \\
  [[  pri =/ {l : A} ]] &\approx [[  Top  ]] \\
  [[  pri =/ (A1 -> A2) ]] &\approx [[  Top  ]] \\
  [[  {l : A} =/ (A1 -> A2) ]] &\approx [[  Top  ]] \\
  [[  {l : A1} =/ {l : A2} ]] &\approx [[  {l : A1 =/ A2}  ]] \\
  [[  {l1 : A1} =/ {l2 : A2} ]] &\approx [[  Top   ]] \quad \text{when} \ [[ l1 <> l2 ]] \\
  [[  (A1 -> A1') =/ (A2 -> A2') ]] &\approx [[  (A1 & A2) -> (A1' =/ A2')  ]]
\end{align*}
\end{proposition}


\paragraph{Connecting with Disjointness.}

With the above propositions stated, it turns out that our disjointness rules, as
given in \cref{fig:disjoint}, can be compactly formulated using $\approx$ and $\sqcup$:

\begin{theorem} \label{thm:disjoint_spec}
  $[[A ** B]]$ if and only if $[[   A =/ B == Top  ]]$.
\end{theorem}
\begin{proof}
  By induction on the derivation of disjointness. An interesting case is \rref{D-arr}
  \[
    \drule{D-arr}
  \]
  \begin{longtable}[l]{l|l}
    $[[A2 =/ B2 == Top]]$  & By i.h \\
    $[[  (A1 -> A2) =/ (B1 -> B2) ]] \approx [[(A1 & B1) -> (A2 =/ B2)]]$ & By \cref{prop:2} \\
    $[[  (A1 -> A2) =/ (B1 -> B2) ]] \approx [[(A1 & B1) -> Top]]$ & By above equality \\
    $[[(A1 & B1) -> Top == Top]]$  & By \cref{prop:1} \\
    $[[  (A1 -> A2) =/ (B1 -> B2) == Top]]$ & By above equality
  \end{longtable}
\end{proof}

\begin{remark}
  We can view \cref{thm:disjoint_spec} as a specification of disjointness.
\end{remark}


\paragraph{Interpretation of Intersection Types.}

Following Reynolds, a subtyping derivation is interpreted as a morphism $ \bra{[[ A <: B ]]} : \bra{[[A]]} \rightarrow \bra{[[B]]} $ with two requirements:
\begin{enumerate}
\item For all types $[[A]]$ the morphism from $ \bra{[[A]]}$ to $\bra{[[A]]}$ must be an identity arrow.
\item Whenever $[[A <: B]]$ and $[[ B <: C  ]]$, the composition of $\bra{[[ A <: B ]]}$ and $\bra{[[  B <: C   ]]}$ must be equal to $\bra{[[  A <: C  ]]}$, i.e., $ \bra{[[ A <: B ]]} ; \bra{[[  B <: C  ]]} = \bra{[[A <: C]]}$. (Here ``;'' denotes composition in diagrammatic order.)
\end{enumerate}
These requirements actually make $ \bra{\cdot} $ a functor from the
preordered set of types (viewed as a category) to the semantic category of
choice.

\begin{remark}
By definition, whenever $[[ A == B ]]$ we say $\bra{[[  A  ]]}$ is \textit{isomorphic} to $\bra{[[ B ]]}$, written $\bra{[[ A ]]} \cong \bra{[[B]]}$.
\end{remark}

Now we consider $\bra{[[ A1 & A2  ]]}$ in the following steps:
\begin{enumerate}
\item By \rref{S-andL,S-andR}, there must be two morphisms, $[[ pp1 ]] : \bra{[[A1 & A2]]} \rightarrow \bra{[[A1]]}  $ and $[[pp2]] : \bra{[[A1 & A2]]} \rightarrow \bra{[[A2]]}  $
  \[
\begin{tikzcd}
  \bra{[[A1]]} &  & \bra{[[A2]]} \\
  & \bra{[[A1 & A2]]} \arrow[lu, "\pi_1"'] \arrow[ru, "\pi_2"] &
\end{tikzcd}
  \]
\item For any types $[[A1]]$ and $[[A2]]$, there exists a least upper bound
  $[[A1 =/ A2 ]]$ (\cref{prop:2}), and two morphisms $\bra{[[A1 <: A1 =/ A2]]} :
  \bra{[[A1]]} \rightarrow \bra{[[A1 =/ A2]]}$ and $\bra{[[A2 <: A1 =/ A2]]} :
  \bra{[[A2]]} \rightarrow \bra{[[A1 =/ A2]]}$, and the following diagram should
  commute:
  \[
\begin{tikzcd}
  & \bra{[[  A1 =/ A2 ]]} &  \\
  \bra{[[A1]]} \arrow[ru, "\bra{[[A1 <: A1 =/ A2]]}"] &  & \bra{[[A2]]} \arrow[lu, "\bra{[[A2 <: A1 =/ A2]]}"'] \\
  & \bra{[[A1 & A2]]} \arrow[lu, "\pi_1"'] \arrow[ru, "\pi_2"] &
\end{tikzcd}
  \]

\item For every type $[[A]]$ such that $[[A <: A1]]$ and $[[A <: A2]]$, \rref{S-and} implies that $[[A <: A1 & A2]]$, thus
  a morphism from $\bra{[[  A ]]}$ to $\bra{[[  A1 & A2  ]]}$. Call this $\mu_0$. The following diagram should commute:
  \[
\begin{tikzcd}
  & \bra{[[  A1 =/ A2 ]]} &  \\
  \bra{[[A1]]} \arrow[ru, "\bra{[[A1 <: A1 =/ A2]]}"] &  & \bra{[[A2]]} \arrow[lu, "\bra{[[A2 <: A1 =/ A2]]}"'] \\
  & \bra{[[A1 & A2]]} \arrow[lu, "\pi_1"'] \arrow[ru, "\pi_2"] & \\
  & \bra{[[A]]} \arrow[u, "\mu_0"] \arrow[luu, "\bra{[[A <: A1]]}"] \arrow[ruu, "\bra{[[A <: A2]]}"'] &
\end{tikzcd}
  \]
\item Furthermore, in the above diagram, we replace $\bra{[[A]]}$ by an
  arbitrary object $s$ and $\bra{[[A <: A1]]}$ and $\bra{[[A <: A1]]}$ by any
  morphisms $f_1$ and $f_2$ that make the outer diamond commutes, and we require
  the ``mediating morphism'' $\mu_0$ from $s$ to $\bra{[[A1 & A2]]}$ to be unique. Specifically,
  we define $\bra{[[A1 & A2]]}$ by requiring the following diagram must commute:
  \[
\begin{tikzcd}
  & \bra{[[  A1 =/ A2 ]]} &  \\
  \bra{[[A1]]} \arrow[ru, "\bra{[[A1 <: A1 =/ A2]]}"] &  & \bra{[[A2]]} \arrow[lu, "\bra{[[A2 <: A1 =/ A2]]}"'] \\
  & \bra{[[A1 & A2]]} \arrow[lu, "\pi_1"'] \arrow[ru, "\pi_2"] & \\
  & s \arrow[u, "\mu_0", dotted] \arrow[luu, "f_1"] \arrow[ruu, "f_2"'] &
\end{tikzcd}
  \]
\end{enumerate}
Thus we have defined $\bra{[[A1 & A2]]}$ to be the \textit{pullback} of
$\bra{[[A1]]}$, $\bra{[[A2]]}$ and $\bra{[[A1 =/ A2]]}$.

\paragraph{Interpretation of Disjoint Intersection Types.}

Given the interpretation of intersection types, it is fairly straightforward to
give the interpretation of disjoint intersection types. First recall that if
$[[A ** B]]$ then $[[ A =/ B == Top ]]$ (\cref{thm:disjoint_spec}). Also we have
$\bra{[[Top]]} = 1$---i.e., the terminal object. By specializing $\bra{[[A1 =/ A2]]}$ to be the terminal object
($\bra{[[A1 <: A1 =/ A2]]}$ and $\bra{[[A2 <: A1 =/ A2]]}$ are then uniquely
determined), then the pullback ``degrades'' to the \textit{product} of
$\bra{[[A1]]}$ and $\bra{[[A2]]}$. In other words, the interpretation of
disjoint intersection types is given by the following theorem:
\begin{theorem}
  If $[[A1 ** A2]]$ then $\bra{[[A1 & A2]]} \cong \bra{[[A1]]} \times \bra{[[A2]]} $.
\end{theorem}
\begin{remark}
It is reassuring to see that this theorem justifies our translation of
disjoint intersection types into product types, from the categorical
perspective.
\end{remark}



\paragraph{Coherence, from the categorical perspective?}

What we have developed so far is the (categorical) interpretation of disjoint intersection
types. We are still half way through the ultimate goal of (re-)establishing
coherence, now from the categorical perspective. The main difficulty is that we
do not know yet how to interpret bidirectional typing judgment---i.e., what are
$\bra{[[GG |- ee => A]]}$ and $\bra{[[GG |- ee <= A]]}$, and in particular the
interpretation of the merge operator. As we mentioned, bidirectional type checking
(besides disjointness) is essential to coherence. It would be exciting to see
some research along the lines of the above, so that we may have a solid
mathematical foundation for type systems with disjoint intersection types.

\subsection{On Implicit Polymorphism}
\label{sec:implicit}

Another interesting and practically useful extension is to study (predicative)
implicit polymorphism, in the spirit of Haskell. Our \fnamee calculus features
explicit polymorphism in the sense that we need to provide types during type
applications. A classic example of implicit polymorphism is the identity
function $[[\x . x]]$ of type $[[\X . X -> X]]$. When applied to $1$, for
example, the type variable $[[X]]$ will be implicitly instantiated to $[[nat]]$.
Moreover, we are interested in \textit{higher-rank polymorphism}, allowing
polymorphic quantifiers to appear anywhere in a type. There are several
approaches in the literature~\citep{odersky1996putting, dunfield2013complete,
  jones2007practical}. Since our declarative type system is already based on
bidirectional type-checking, the work by \citet{dunfield2013complete} is
particularly relevant for us. It turns out that coming up with a coherent
declarative system is already very challenging, especially the disjointness
relation. Below we sketch out some ideas for the initial design.

\paragraph{Declarative Subtyping.}

First we consider the subtyping rules. Obviously \rref{S-forall} needs to be modified.
We replace it with the following two rules:
\begin{mathpar}
  \drule{IS-allL}
  \drule{IS-allR}
\end{mathpar}
\Rref{IS-allL} says that a type $[[\X ** A1 . A2]]$ is a subtype of $[[B]]$ if
some instantiation $[[ [t / X] A2 ]]$ is a subtype of $[[B]]$. However, unlike
\citeauthor{dunfield2013complete}'s system, in our setting, not all monotypes
$[[t]]$ that make the subtyping go through are equally fine---those that do not respect the disjointness constraints should
not be considered, for the sake of coherence.
Otherwise, we would allow $[[ ((\x . x ,, 2) : \X ** nat . X -> X & nat) 1 ]]$ to type check,
which would be a disaster.
\Rref{IS-allR} says that $[[A]]$
is a subtype of $[[\X ** B1. B2]]$ if we can show that $[[A]]$ is a subtype of
$[[B2]]$ in a context extended with $[[X ** B1]]$. It is not immediately obvious
that these two rules subsume \rref{S-forall}, and in particular what happens to ``a universal quantifier is contravariant in its
disjointness constraints'', which is very important in the original subtyping.
It can be shown that they do subsume \rref{S-forall}, as exhibited by the
following derivation:
\[
\inferrule*[right=\rref*{IS-allR}]{  \inferrule*[right=\rref*{IS-allL}]{ \inferrule*[right=\rref*{FD-tvarL}]{ [[  A2 <: A1  ]]    }{[[  X ** A2 |- X ** A1  ]]}  \\ [[  X ** A2 |- B1 <: B2  ]]   }{[[X ** A2 |- \X ** A1. B1 <: B2]]}    }{ [[  empty |- \X ** A1. B1 <: \X ** A2 . B2  ]] }
\]


\paragraph{Disjointness.}

The disjointness relation needs a major overhaul. For instance, subtyping allows
$[[ \X ** char . X -> X <: nat -> nat ]]$, and as such, $[[ \X ** char . X -> X]]$
is no longer disjoint with $[[nat -> nat]]$, whereas $[[ \X ** nat . X -> X]]$
is disjoint with $[[nat -> nat]]$. A seemingly intuitive rule is as follows:
\begin{mathpar}
  \inferrule*[lab=FD-implicit]{[[DD |- t1 ** A1]] \\  [[DD |- [t1 / X] A2 ** B2]]  }{  [[DD |- \ X ** A1 . A2 ** B2]]  }
\end{mathpar}
In the above rule, the monotype $[[t1]]$ is existentially quantified: it
suffices to exhibit a disjointness derivation of $[[  [t1 / X]  A2  ]]$ and $[[B2]]$
for one monotype in order to build a disjointness derivation of $[[\X ** A1 . A2]]$ and $[[B2]]$.
Unfortunately, this rule is incorrect as we could guess a
``wrong'' $[[t1]]$. Take $[[ \X ** char . X -> X ]]$ for example: one
instantiation is $[[bool -> bool]]$, which is disjoint with $[[nat -> nat]]$.
But as we saw, this does not mean $[[ \X ** char . X -> X ]]$ is disjoint with
$[[nat -> nat]]$. Instead we should require \textit{all possible}
instantiations are disjoint with $[[B2]]$:
\begin{mathpar}
  \inferrule*[lab=FD-implicit]{ \forall [[t1]] .\ [[DD |- t1 ** A1]] \Longrightarrow [[DD |- [t1 / X] A2 ** B2]]  }{  [[DD |- \ X ** A1 . A2 ** B2]]  }
\end{mathpar}
The universal rule is very convenient as an elimination form: if we have a
evidence of the disjointness between a polymorphic type and another type, we can
immediately obtain the knowledge that all suitable instantiations of the former
are disjoint with the latter. However, the universal rule is very hard to use as
an introduction rule: it requires us to inspect every possible instantiation;
it is getting even worse when we consider two polymorphic types. We do not
yet fully understand all the consequences of this rule. Another idea is perhaps we
should focus on the opposite side---i.e., what constitutes a non-disjointness
relation. But this idea seems more radical.


\paragraph{Declarative Typing.}

Putting disjointness aside, now we consider the typing rules. Most of the rules
stay the same. We remove \rref{FT-tabs,FT-tapp}, since the syntax now does not
include type abstractions and type applications. We add one rule:
\[
  \drule{FT-gen}
\]
\Rref{FT-gen} says that $[[ee]]$ has type $[[\X ** A . B]]$ if $[[ee]]$ has type $[[B]]$ in a context extended with $[[ X ** A  ]]$.
Application becomes a little more complex:
\[
  \drule{FT-appI}
\]
The problem is that the inferred type $[[A]]$ for $[[ee1]]$ could be a
polymorphic quantifier.
We need to eliminate universals until we
reach an arrow type. To achieve this, we use a matching judgment $[[DD |- A tri A1 -> A2]]$,
which says that we can synthesize an arrow type $[[A1 -> A2]]$ from $[[A]]$.
Once we get an arrow type $[[A1 -> A2]]$, we use $[[A1]]$ to check against $[[ee2]]$.
The matching judgment~\citep{siek2015refined, xie2018consistent}, first used in gradual type systems, is inductively defined as follows:
\begin{mathpar}
  \drule{m-forall}
  \drule{m-arr}
\end{mathpar}
\Rref{M-forall}, as with \rref{IS-allL},
works by guessing instantiations of polymorphic quantifiers with the requirement
that the monotype $[[t]]$ must meet the disjointness constraints. \Rref{M-arr}
is trivial, returning $[[A1 -> A2]]$ as it is.



Of course the above is only a sketch; we have not studied the declarative system in full,
nor its metatheory. One potential problem is that now subtyping and
disjointness are mutually recursive (e.g., \rref{IS-allL} uses disjointness and
\rref{FD-tvarL} uses subtyping), which might pose difficulty in terms of
formalization. For coherence, we estimate that the proof method described in
this thesis should still work.



\paragraph{Algorithmic System.}

Having a declarative system is only a start. The major challenge is the
corresponding algorithmic system. It is known that full type inference is
undecidable for intersection types. Some restrictions are obviously in order,
leading to different points in the design space with varying degrees of
expressiveness and technical difficulties. We are interested to see some
research into the algorithmic system.


\subsection{Disjoint Polymorphism vs. Row Polymorphism}

Row polymorphism, first proposed by \citet{wand1987complete}, was intended as a
mechanism to enable type inference for a simple object-oriented language based
on recursive records. These ideas were later adopted into type systems for
extensible records~\citep{Harper:1991:RCB:99583.99603, gaster1996polymorphic}.
As we have alluded to in \cref{sec:merge}, row polymorphism alone cannot express
the \lstinline{merge} function. It would be interesting to study the
relationship between disjoint polymorphism and row polymorphism, and in
particular, whether the former subsumes the latter. As noted by
\citet{alpuimdisjoint}, disjoint polymorphism can already encode polymorphic
extensible records. For the sake of comparison, we pick the record calculus
$\lambda^{\|}$ of \citet{Harper:1991:RCB:99583.99603}---an explicitly-typed,
second-order calculus that features single-field records and a symmetric merge
operator. In $\lambda^{\|}$, \textit{compatibility constraints} are used to
capture negative information about fields. For example, $r_1 \# r_2 $ denotes the
assertion that the record types $r_1$ and $r_2$ have disjoint sets of labels. To
illustrate polymorphic extensible records in $\lambda^{\|}$,
\citeauthor{Harper:1991:RCB:99583.99603} show a function that takes two
``disjoint'' records $x_1$ and $x_2$, where $x_1$ has at least a field $l_1$ of
type $[[nat]]$ and $x_2$ has at least a field $l_2$ of type $[[nat]]$, and
returns the result of merging $x_1$ and $x_2$ (altering their syntax slightly):
\begin{align*}
  \Lambda \alpha_1 \# (\{ l_1 : [[nat]] \}, \{ l_2 : [[nat]] \}) .\  \Lambda \alpha_2 \# (\alpha_1 , \{l_1 : [[nat]]\} , \{ l_2 : [[nat]]   \}) . \\
  \qquad \lambda x_1 : (\alpha_1 \| \{ l_1 : [[nat]] \}) .\  \lambda x_2 : (\alpha_2 \| \{ l_2 : [[nat]] \}) .\ x_1 \| x_2
\end{align*}
where $r_1 \| r_2$ is the record type obtained by merging $r_1$ and $r_2$, and
is only defined if $r_1 \# r_2$. The same operator is overloaded to merge two
records on the term level. Central to their system is the \textit{constrained
  quantification} $\forall \alpha \# R .\ t $, where each record type variable is
associated with a list of \textit{compatibility assumptions} $R$, whose elements
are record types (including record type variables). The \textit{constrained type abstraction} $\Lambda \alpha \# R.\ e$
is used to create values of constrained quantification.


In \fnamee, we can use disjoint quantification to express their constrained
qualification, intersection types to merge record types, and the merge operator
to merge records. The function mentioned above can be written in \fnamee as
follows:
\begin{align*}
  \Lambda (\alpha_1 * [[ {l1 : nat} & {l2 : nat} ]]) .\  \Lambda (\alpha_2 * [[  X1 & {l1 : nat} & {l2 : nat} ]]) . \\
  \qquad [[\x1 : X1 & {l1 : nat} . \x2 : X2 & {l2 : nat} . x1 ,, x2]]
\end{align*}
However, the merge operator in \fnamee is more general than its counterpart in
$\lambda^{\|}$---i.e., it works on any expressions, not just records. Another
important difference is that their compatibility judgment $r_1 \# r_2$
effectively implies that their records must have distinct fields, whereas
\fnamee accepts duplicate fields as long as their types are disjoint. On a
related note, $\lambda^{\|}$ is powerful enough to express a polymorphic,
conflict-free function that merges two records of statically-unknown fields:
\[
  \mathsf{mergeRcd} = \Lambda \alpha_1 \# \mathsf{Empty} .\ \Lambda \alpha_2 \# \alpha_1 .\ \lambda x_1 : \alpha_1 .\ \lambda x_2 : \alpha_2 .\ x_1 \| x_2
\]
where $\mathsf{Empty}$ is the empty record type. Compare it to our ``more expressive'' $\mathsf{mergeAny}$ function:
\[
  \mathsf{mergeAny} = [[\ X1 ** Top . \ X2 ** X1 . \x1 : X1 . \x2 : X2 . x1 ,, x2 ]]
\]
that merges \textit{any} two expressions of statically-unknown types.

We believe \fnamee completely subsumes $\lambda^{\|}$. To support this claim, we
need to show a bisimulation property between $\lambda^{\|}$ and (a subset of)
\fnamee:
\begin{inparaenum}[(1)]
\item the translation from $\lambda^{\|}$ to \fnamee is type-safe (i.e., it type check in \fnamee),
\item and the translation from a subset of \fnamee to $\lambda^{\|}$ is type-safe (i.e., it type check in $\lambda^{\|}$).
\end{inparaenum}


\subsection{Recursive Types}

One extension of particular importance for modeling object-oriented languages is
\textit{recursive types}. A great deal of lessons have been learned about
calculi with recursive types and subtyping (see \citet[chap.
20]{DBLP:books/daglib/0005958}). But previous work has been focused on type
systems with substantially simpler subtyping relations. For simplicity, we are
interested in adding \textit{iso-recursive types}, where a recursive type $[[mu XX . A]]$\footnote{We use $[[XX]]$ to distinguish from type variables introduced by universal quantifiers.} and its one-step unfolding are transformed back and forth by a pair
of functions $\mathsf{fold}$ and $\mathsf{unfold}$. The most common definition
of iso-recursive subtyping is the \textit{Amber rule}, popularized by
\citeauthor{DBLP:conf/litp/Cardelli85}'s Amber language~\citep{DBLP:conf/litp/Cardelli85}.
\begin{mathpar}
  \drule{RS-amber}
  \drule{RS-var}
\end{mathpar}
\Rref{RS-amber} says that to show $[[mu XX . A]] $ is a subtype of $ [[mu YY . B]]$
under some set of assumptions $[[SS]]$, it suffices to show $[[A <: B]]$ under
the additional assumption $[[XX]] <: [[YY]]$. \Rref{RS-var} allows us to
conclude $[[XX]] <: [[YY]]$ if the assumptions assume it.

While adding the above two rules to our subtyping relation in
\cref{fig:subtype_decl} (and extending the other rules so that they pass
$[[SS]]$ through from premises to conclusion) effectively yields a declarative
subtyping relation with recursive types, it is not entirely straightforward as
to how they affect disjointness, and in particular, under what conditions are
two recursive types disjoint. An initial attempt shows that the amber rule and
the disjointness rule for functions are in conflict.

\paragraph{The Problem.}

For the ease of discussion, we do not consider top types, polymorphic types, or
BCD subtyping; then a guiding principle of designing disjointness rules is the
simple disjointness specification (\cref{def:disjoint_spec}): two types are
disjoint if and only if they share no common supertypes. Now consider two
recursive types $[[ mu XX . XX -> nat ]]$ and $[[ mu YY . YY -> nat]]$. It is
not hard to see that they have no common supertypes (because of contravariance of function argument subtyping). According to
\cref{def:disjoint_spec}, they are disjoint. On the other hand, since the
disjointness relation is structural, we should inspect the disjointness relation
between $[[ XX -> nat ]]$ and $[[YY -> nat ]]$ under certain relation over
$[[XX]]$ and $[[YY]]$ we do not know yet. However, according to \rref{D-arr}, two functions are
disjoint if their range types are disjoint; thus $[[ XX -> nat ]]$ and $[[YY ->
nat ]]$ are not disjoint. So we have $[[ mu XX . XX -> nat]]$ and $[[ mu YY . YY -> nat]]$
are \textit{not} disjoint: a contradiction!


\paragraph{Positivity to the Rescue.}

It is not obvious how to change either \rref{RS-amber} or \rref{D-arr} without
disrupting the whole system. A possible solution is to restrict where type variables
can occur. Instead of
having a \textit{general} recursive type $[[mu XX . A]]$ where $[[XX]]$ may occur
anywhere in $[[A]]$, we require that $[[XX]]$ occurs \textit{positively} in
$[[A]]$. Specifically, $[[XX]]$ occurs positively in $[[A1 -> A2]]$, if
\begin{inparaenum}[(1)]
\item $[[XX]]$ \textit{does not occur} in $[[A1]]$,
\item and $[[XX]]$ occurs positively in $[[A2]]$.
\end{inparaenum}
In general, any occurrences of $[[XX]]$ within the domain of a function type are
\textit{negative occurrences}, whereas any occurrences of $[[XX]]$ within the
range of a function type are \textit{positive occurrences}. For example, the two
recursive types in the last paragraph are not positive. While positivity does
limit the expressiveness of types, most useful datatypes (e.g., natural numbers,
lists, streams) are positive. For us, the positivity restriction for recursive types does work with the
disjointness rule for function types.

With positive recursive types, here is the disjointness rule for recursive types:
\[
  \drule{D-rec}
\]
We also need a few more disjointness axioms:
\begin{mathpar}
  \drule{Dax-intRec}
  \drule{Dax-rcdRec}
  \drule{Dax-arrRec}
  \drule{Dax-intRVar}
  \drule{Dax-rcdRVar}
  \drule{Dax-arrRVar}
  \drule{Dax-recRVar}
\end{mathpar}
An important observation is that any two type variables are \textit{not}
disjoint. A few examples: $[[mu XX . nat -> XX]]$ and $[[mu YY . nat -> YY]]$
are not disjoint; $[[mu XX . nat -> XX]]$ and $[[mu YY . bool -> YY & nat]]$ are
not disjoint; $[[mu XX . nat -> XX]]$ and $[[mu YY . nat -> nat -> YY]]$ are
disjoint. Note that the above is only a sketch; it remains to see whether the disjointness
rules are equivalent to the specification.


Another gnarly issue is coherence. To model recursive types, we need to turn to
step-indexed logical relations~\citep{ahmed2006step}. We foresee it would be a
major technical challenge to adjust our coherence proof and its Coq
mechanization.

\subsection{Other Extensions}

There are several important extensions that should also be considered.


\paragraph{Union Types.}

Union types---as intersections' dual---are also widely used in languages such as
Ceylon and Flow. Union types introduce an interesting subtyping relation: a
union type $[[A | B]] $ is a common supertype of $[[A]]$ and $[[B]]$;
or more precisely, it is a \textit{least upper bound} of $[[A]]$ and $[[B]]$, as
exhibited by the following subtyping rules.
\begin{mathpar}
  \inferrule*[lab=union]{  [[A <: C]] \\ [[B <: C]]   }{[[ A | B <: C]]} \and
  \inferrule*[lab=unionL]{   }{[[A <: A | B]]} \and
  \inferrule*[lab=unionR]{   }{[[B <: A | B]]}
\end{mathpar}
\citet{dunfield2014elaborating} has shown that unions can be elaborated into
sums, and the merge operator also supports union elimination with two
computationally distinct branches. We think that adapting his approach to our
setting should not be difficult. One immediate issue is disjointness of union
types; obviously \cref{thm:disjoint_spec} cannot serve as a specification any more.
More thoughts are needed to come up with a coherent system with union types.

\paragraph{Nominal Typing.}

Many widely-used OO languages feature nominal type systems where type names play
a crucial role. In previous chapters, we often define short names for long or
complex compound types to improve readability, e.g., in
\cref{sec:trait:overview}, we have seen:
\lstinputlisting[linerange=10-17]{./examples/overview2.sl}% APPLY:linerange=OVERVIEW_EDITOR_TYPES
Such definitions are purely cosmetic: the name \lstinline{Editor} is just an
abbreviation for the record on the right-hand side, and the two are
interchangeable in every context. By contract, in OO languages such as Java,
every compound type (class declaration or interface definition) has a name, and
subtyping is explicitly declared between type \textit{names}.

To blend in with our powerful structural subtyping relation, we need to clearly
decide which types are based on nominal subtyping, which are based on structural
subtyping and how they interact. A rough idea, following the \textsf{Moby} type
system~\citep{fisher2000extending}, is to separate \textit{class types} from
\textit{object types}, as we did for trait types. Subtyping on class types is
nominal, while objects are compared structurally. This is just a high-level
intuition; of course there are other details (e.g., disjointness) that need to
be worked out. A pleasant property of nominal systems, and also related to our
extension of recursive types, is that they offer a natural account of recursive
types: if we look at the amber rule \rref*{RS-amber}, an explicit subtyping
relation $[[XX <: YY]]$ is added to the context when two recursive types are
compared.



\paragraph{Mutable State.}

Another direction for future work is to add mutable state, which would affect
two aspects of our metatheory: type safety and coherence. For type safety, we
expect that lessons learned from previous work on family polymorphism and
mutability on OO languages apply to our work. For example, it is well-known that
subtyping in the presence of mutable state often needs restrictions. Given such
suitable restrictions we expect that type-safety in the presence of mutability
still holds. For coherence, it would be a major technical challenge to adjust
our coherence proof and its Coq mechanization: logical relations that account
for mutable state introduce significant complexity (e.g., see
\citet{ahmed2004semantics}).





%%% Local Variables:
%%% mode: latex
%%% TeX-master: "../Thesis"
%%% org-ref-default-bibliography: ../Thesis.bib
%%% End:
