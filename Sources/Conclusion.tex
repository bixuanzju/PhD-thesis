%%%%%%%%%%%%%%%%%%%%%%%%%%%%%%%%%%%%%%%%%%%%%%%%%%%%%%%%%%%%%%%%%%%%%%%%
\chapter{Conclusion}
\label{chap:conclusion}

%%%%%%%%%%%%%%%%%%%%%%%%%%%%%%%%%%%%%%%%%%%%%%%%%%%%%%%%%%%%%%%%%%%%%%%%

In this thesis we have argued that the combination of disjoint intersection
types, a powerful subtyping relation and parametric polymorphism greatly improve
the state-of-art technique for modularity and code reuse. In the course of our
investigation, we have gradually introduced three new typed calculi with
increasing expressiveness:
\begin{itemize}
\item The \namee calculus is the basic calculus with disjoint intersection types
  and a powerful subtyping relation. We have shown that it captures the essence
  of nested composition, enables a simple solution to the expression problem. In
  order to prove coherence, we have proposed the canonicity relation based on
  logical relations.
\item The \fnamee calculus, building on \namee, supports parametric
  polymorphism. We have shown that it improves upon the finally
  tagless~\citep{CARETTE_2009} and object
  algebra~\citep{oliveira2012extensibility} approaches and support advanced
  compositional designs, and enables the development of highly modular and
  reusable programs. We have also extended the canonicity relation to establish
  coherence property of \fnamee.
\item \sedel---an object-oriented language design---building on \fnamee,
  supports, among others, typed first traits. We have illustrated the
  applicability of \sedel with several example uses for first-class traits.
  Furthermore, we have conducted a case study that modularizes programming
  language interpreters. The case study demonstrates that the state-of-art
  encodings of extensible designs are greatly improved by \sedel.
\end{itemize}


Of course there are several noteworthy limitations in our proposed calculi. (1)
Lack of mutable state, which is a desirable feature in modern programming
languages, and also very important in order for mainstream languages to adopt
some of the proposed language mechanisms in this thesis. (2) Lack of recursive
types, which is important for modeling object-oriented language. (3) Poor
runtime performance due to redundant coercion applications (e.g., applying multiple
$\mathsf{id}$s). As we remarked earlier, our generated coercions are not
efficient in terms of space. There is existing work on space-efficient
coercions~\citep{Siek_2015, herman2010space}, which we expect should be
applicable to the work in this thesis.


Hopefully we have convinced the reader that disjoint intersection types have
great potential to serve as a foundation for powerful, flexible and yet
type-safe and easy to reason object-oriented languages. We hope that the
concepts and the methods described in this thesis may serve as a helpful guide
to researchers and programmers alike in their attempts to understand and build
better software. Thus this thesis serves as a stepping stone for further
investigation of disjoint intersection types in conjunction with other type
disciplines. A great number of open questions, new research directions lie ahead
(cf. \cref{chap:future}).



%%% Local Variables:
%%% mode: latex
%%% TeX-master: "../Thesis"
%%% org-ref-default-bibliography: ../Thesis.bib
%%% End:
