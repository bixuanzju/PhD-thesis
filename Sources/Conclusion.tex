%%%%%%%%%%%%%%%%%%%%%%%%%%%%%%%%%%%%%%%%%%%%%%%%%%%%%%%%%%%%%%%%%%%%%%%%
\chapter{Conclusion}
\label{chap:conclusion}

%%%%%%%%%%%%%%%%%%%%%%%%%%%%%%%%%%%%%%%%%%%%%%%%%%%%%%%%%%%%%%%%%%%%%%%%

In this thesis we have argued that the combination of disjoint intersection
types, a powerful subtyping relation and parametric polymorphism greatly improve
the state-of-art technique for modularity and code reuse. In the course of our
investigation, we have gradually introduced three new typed calculi with
increasing expressiveness:
\begin{itemize}
\item The \namee calculus is the basic calculus with disjoint intersection types
  and a powerful subtyping relation. We have shown that it captures the essence
  of nested composition, enabling a simple solution to the expression problem.
  In order to prove coherence, we have introduced a powerful proof method based
  on logical relations.
\item The \fnamee calculus, building on \namee, supports parametric
  polymorphism. We have shown that it can express a very dynamic (and
  conflict-free) form of composition, which could serve as a foundation for more
  sophisticated compositional models. We have also adapted the proof method to
  establish its coherence property.
\item \sedel---an object-oriented language design---building on \fnamee,
  supports, among others, typed first traits. Through a case study, we have
  shown the usefulness of \fnamee in building highly reusable software
  components using a improved form of Object Algebras. The case study
  demonstrates that the state-of-art encodings of extensible designs are greatly
  improved by \fnamee.
\end{itemize}

We hope that the concepts and the methods described in this thesis may serve as
a helpful guide to researchers and programmers alike in their attempts to
understand and build better software. Thus this thesis serves as a stepping
stone for further investigation of disjoint intersection types in conjuncture
with other type disciplines. A great number of open questions, new research
directions lie ahead!




%%% Local Variables:
%%% mode: latex
%%% TeX-master: "../Thesis"
%%% org-ref-default-bibliography: ../Thesis.bib
%%% End:
