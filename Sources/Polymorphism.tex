
%%%%%%%%%%%%%%%%%%%%%%%%%%%%%%%%%%%%%%%%%%%%%%%%%%%%%%%%%%%%%%%%%%%%%%%%
\chapter{Semantics of the \fnamee Calculus}
\label{chap:fi}
%%%%%%%%%%%%%%%%%%%%%%%%%%%%%%%%%%%%%%%%%%%%%%%%%%%%%%%%%%%%%%%%%%%%%%%%

In this chapter, we are going to enrich \namee with parametric polymorphism. As
we will see in later chapters, the combination is very expressive, able to express sophisticated
concepts such as mixins/traits and a highly modular form of \visitor~\citep{oliveira09modular, togersen:2004}. The
combination is also highly challenging in that coherence becomes even harder to
prove than in the simply typed setting. We present \fnamee, the first typed
calculus combining disjoint polymorphism with BCD subtyping. \fnamee is a
variant of \citeauthor{leivant1991finitely}'s predicative System
F~\citep{leivant1991finitely}. The choice of predicativity is due to the
simplicity of the coherence proof. We will further discuss this in
\cref{chap:coherence:poly}. \fnamee serves as the theoretical foundation of
typed first-class traits, which will be introduced in \cref{chap:traits}.




\section{Motivation}

Parametric polymorphism~\citep{reynolds1983types} is a well-beloved (and
well-studied) programming feature. It enables a single piece of code to be
reused on data of different types. So it is quite natural and theoretically
interesting to study combining parametric polymorphism with disjoint
intersection types, especially how it affects disjointness and coherence. On a more
pragmatic note, the combination of parametric polymorphism also reveals new
insights into practical applications. Dynamic languages (such as JavaScript)
usually embrace quite flexible programming patterns, e.g., mixin composition
where objects can be composed at run time, and their types are not necessarily
statically known. The use of intersection types in TypeScript is inspired by
such flexible programming patterns. For example, an important use of
intersection types in TypeScript is the following function for mixin
composition:
\begin{lstlisting}[language=JavaScript]
function extend<T, U>(first: T, second : U) : T & U {...}
\end{lstlisting}
which is analogous to our merge operator in that it takes two objects and
produces an object with the intersection of the types of the argument objects.
However, the types of the two objects are not known, i.e., they are generic. An
important point is that, while it is possible to define such function in
TypeScript (albeit using some low-level (and type-unsafe) features of
JavaScript), it can also cause, as pointed out by \citet{alpuimdisjoint},
run-time type error! Clearly a well-defined meaning for intersection types with
type variables is needed.


\paragraph{Disjoint Polymorphism.}

Motivated by the above two points, \citet{alpuimdisjoint} proposed disjoint
polymorphism, a variant of parametric polymorphism. The
main novelty is \textit{disjoint (universal) quantification} of the form $[[ \ X ** A . B ]]$.
Inspired by bounded quantification~\citep{cardelli1994extension}
where a type variable is constrained by a type bound, disjoint quantification
allows type variables to be associated with \textit{disjointness constraints}.
Correspondingly, a term construct of the form $[[ \ X ** A. ee ]]$ is used to
create values of disjoint quantification.
To understand the purpose of disjointness constraints, consider the following program (adapted from \citet{alpuimdisjoint}):
\begin{lstlisting}
mergeBad X (x : X) : X & Int = x ,, 2;
\end{lstlisting}
\lstinline{mergeBad} takes an argument \lstinline{x} of type \lstinline{X} (which is itself a type variable), and merges it with \lstinline{2}.
However, if we were to allow such definition, we could easily create an example where incoherence occurs again:
\begin{lstlisting}
(mergeBad Int 1) : Int -- 1 or 2
\end{lstlisting}
This is essentially the same problem of allowing $\mer{1}{2}$, which as we
discussed in \cref{bg:sec:intersection} will cause ambiguity. For \namee, we
know the concrete type for each variable and thus disjointness checking can help
avoid this problematic expression. However, with parametric polymorphism, a
variable could have any types,
including those that are already in the intersection. So a question to ask is to
decide under what conditions a type variable is disjoint with, say,
\lstinline{Int}. This is where disjointness constraints come into stage. The key
idea is that since we do not know \textit{a priori} what is the type with which
a type variable can be instantiated, we can restrict the set of types it can be
instantiated to. Let us rewrite the above program as follows:
\begin{lstlisting}
mergeGood [X * Int] (x : X) : X & Int = x ,, 2;
\end{lstlisting}
The only change is the notation \lstinline{[X * Int]}, where the left-side of
\lstinline{*} denotes the type variable being declared, and the right-side
denotes the disjointness constraint(s). Here the disjointness constraint
(\lstinline{Int}) effectively states that the type variable \lstinline{X} can be
instantiated to any types disjoint with \lstinline{Int}. For instance, the expression \lstinline{mergeGood Bool True}
type checks but the expression \lstinline{mergeGood Int 1}
is rejected because \lstinline{Int} (the type argument) is not disjoint with
\lstinline{Int} (the disjointness constraint). What is more, we can express multiple
constraints using intersection types, for example,
\begin{lstlisting}
mergeThree [X * Int & Bool] (x : X) : X & Int & Bool = x ,, 2 ,, True;
\end{lstlisting}
Here the type variable \lstinline{X} can only be instantiated to types disjoint with both
\lstinline{Int} and \lstinline{Bool}.


With disjointness constraints and a built-in merge operator, a type-safe and
conflict-free \lstinline{extend} function can be naturally defined as follows:
\begin{lstlisting}
extend T [U * T] (first : T) (second : U) : T & U = first ,, second;
\end{lstlisting}
The disjointness constraint on the type variable \lstinline{U} ensures that no
conflicts can occur when composing two objects, which is quite similar to
trait-based approach~\citep{scharli2003traits} in object-orientated programming.
We shall devote a whole chapter (\cref{chap:traits}) to further this point.


\paragraph{Adding BCD Subtyping.}

While \citet{alpuimdisjoint} studied the combination of disjoint intersection
types and parametric polymorphism, they follow the then-standard approach
of \citet{oliveira2016disjoint} to ensure coherence, thus excluding the
possibility of adding BCD subtyping. The combination of BCD subtyping and
disjointness polymorphism opens room for more expressiveness. For example, we can
define the following function
\begin{lstlisting}
combine A [B * A] (f : {foo : Int -> A})
                  (g : {foo : Int -> B}) : {foo : Int -> A & B} = f ,, g;
\end{lstlisting}
which ``combines'' two singleton records with parts of types unknown and returns
another singleton record containing an intersection type. A variant of this
function plays a fundamental role in defining Object Algebra combinators (cf.
\cref{chap:case_study}).




In what follows, we first present the syntax and semantics (subtyping and
typing) of \fnamee. We then discuss the disjointness judgment in detail, in
particular, the disjointness relation between type variables and arbitrary
types. Finally we talk about the elaboration semantics of \fnamee and its target
calculus \tnamee, a variant of System F with explicit coercions.



% Local Variables:
% TeX-master: "../../Thesis"
% org-ref-default-bibliography: ../../Thesis.bib
% End:



\section{Syntax and Semantics}

\begin{figure}[t]
  \centering
\begin{tabular}{llll} \toprule
  Types & $[[A]], [[B]], [[C]]$ & $\Coloneqq$ & $[[nat]] \mid [[Top]] \mid [[A -> B]]  \mid [[A & B]] \mid [[{l : A}]] \mid [[X]] \mid [[\ X ** A . B]] $\\
  Monotypes & $[[t]]$ & $\Coloneqq$ & $[[nat]] \mid [[Top]] \mid [[t1 -> t2]]  \mid [[t1 & t2]] \mid [[X]] \mid [[{l : t}]]$\\
  Expressions & $[[ee]]$ & $\Coloneqq$ & $[[x]] \mid [[i]] \mid [[Top]] \mid [[\x . ee]] \mid [[ee1 ee2]] \mid [[ ee1 ,, ee2 ]]   \mid [[ ee : A ]] $ \\
        & & $\mid$ & $ [[{l = ee}]] \mid [[ ee.l  ]] \mid [[\X ** A . ee]] \mid [[ ee A ]]  $ \\
  Value Contexts & $[[GG]]$ & $\Coloneqq$ &  $[[empty]] \mid [[GG , x : A]] $ \\
  Type Contexts & $[[DD]]$ & $\Coloneqq$ &  $[[empty]] \mid [[DD , X ** A]] $  \\ \bottomrule
  % Expression Contexts & $[[CC]]$ & $\Coloneqq$ &  $[[__]] \mid [[\ x . CC]] \mid [[\ X ** A. CC]] \mid [[ CC A  ]] \mid [[CC ee]] \mid [[ee CC]] \mid [[ CC ,, ee  ]]  $ \\
  % & & $\mid$ & $[[ ee ,, CC  ]] \mid  [[ { l = CC}  ]]  \mid [[ CC . l]] $
\end{tabular}
  \caption{Syntax of \fnamee}
  \label{fig:syntax:fi}
\end{figure}

\Cref{fig:syntax:fi} shows the syntax of \fnamee. Metavariables $[[A]], [[B]],
[[C]]$ range over types. Apart from \namee types, \fnamee also includes type
variables $[[X]]$ and disjoint quantification $[[ \X ** A . B ]]$. Monotypes
$[[t]]$ are the same, less the universal quantification. Metavariable $[[ee]]$
ranges over expressions. We extend \namee expressions with two standard
constructs in System F: type abstractions $[[ \X ** A . ee ]]$ and type
applications $[[ee A]]$. The former also includes an extra disjointness
constraint $[[A]]$ associated with the type variable $[[X]]$.

\paragraph{Contexts.}

In the traditional formulation of System F, there is a single context that is
used to keep track of both type variables and term variables. Here we use
another style of presentation~\citep[chap. 16]{Harper_2016} where contexts are
split into \textit{value contexts} $[[GG]]$ and \textit{type contexts} $[[DD]]$.
The former track bound term variables $[[x]]$ with their types $[[A]]$; and the
latter track bound type variables $[[X]]$ with their disjointness constraints
$[[A]]$. This formulation is also convenient for the presentation of logical
relations in \cref{chap:coherence:poly}.

\begin{figure}
  \centering
  \drules[swft]{$[[DD |- A]]$}{Well-formedness of types}{top, int, var, arrow, all, and, rcd}
  \drules[swfe]{$[[DD ||- GG]]$}{Well-formedness of value contexts}{empty, var}
  \drules[swfte]{$[[||- DD]]$}{Well-formedness of type contexts}{empty, var}
  \caption{Well-formedness of contexts and types}
  \label{fig:well-formedness:fi}
\end{figure}

\paragraph{Well-formedness of contexts.}

The well-formedness judgments for contexts and types, as shown in
\cref{fig:well-formedness:fi} are quite standard. They together ensure that each
type appearing in the contexts is well-formed in the sense that there are no
unbound free variables.


\paragraph{Declarative Subtyping.}


\begin{figure}[h]
  \centering
  \drules[FS]{$[[ A <|: B ~~> c]]$}{Declarative subtyping}{refl,trans,top,rcd, arr,andr,andl,and,distArr,topArr,distRcd,topRcd,forall}
  \caption{Declarative subtyping of \fnamee}
  \label{fig:subtyping:fi}
\end{figure}

\Cref{fig:subtyping:fi} presents the subtyping relation of \fnamee. For now, we
ignore the coercion parts ($[[~~>]] [[c]]$) and explain them in
\cref{sec:elaboration:fi}. We naturally extend the subtyping rules of \namee
with only one rule \rref*{FS-forall}, which specifies the subtyping relation
between two universal quantifiers. In \rref{FS-forall}, a universal quantifier
is covariant in its body, and contravariant in its disjointness constraint. A
minor comment is that since \fnamee features explicit polymorphism, type
variables are neutral to subtyping, i.e., $[[X <: X]]$, which is contained in
\rref{FS-refl}. As with \namee subtyping, the subtyping relation of \fnamee is
trivially \textit{reflexive} and \textit{transitive}.

\begin{remark}
  In our Coq formalization, we require that the two types $[[A]]$ and $[[B]]$ are
  well-formed with respect to some type context, resulting in the subtyping
  judgment $[[DD |- A <: B]]$. But this is not very important
  for the purpose of presentation, thus we omit contexts.
\end{remark}


\paragraph{Typing.}

\begin{figure}
  \centering
  \drules[FT]{$[[DD; GG |- ee => A ~~> e]]$}{Inference}{top, int, var, app, merge, anno, tabs, tapp, rcd, proj}
  \drules[FT]{$[[DD ; GG |- ee <= A ~~> e]]$}{Checking}{abs, sub}
  \caption{Bidirectional type system of \fnamee}
  \label{fig:typing:fi}
\end{figure}


The bidirectional type system of \fnamee follows that of \namee, as shown
in \cref{fig:typing:fi}. Again we ignore the translation parts ($[[~~>]] [[e]]$) and explain them in
\cref{sec:elaboration:fi}. The inference judgment $[[ DD; GG |- ee => A  ]]$
says that we can synthesize the type $[[A]]$ in the contexts $[[DD]]$ and $[[GG]]$. The checking judgment
$[[ DD ; GG |- ee <= A  ]]$ asserts that $[[ee]]$ checks against the type $[[A]]$
in the contexts $[[DD]]$ and $[[GG]]$. The rules directly ported from \namee are inferring rules \rref*{FT-top} for top values,
\rref*{FT-int} for integers, \rref*{FT-var} for variables, \rref*{FT-app} for applications, \rref*{FT-merge} for merges,
\rref*{FT-anno} for annotated terms, \rref*{FT-rcd,FT-proj} for records; checking rules \rref*{FT-abs} for term abstractions, and
the subsumption rule \rref*{FT-sub}. Note that in \rref{FT-merge}, the disjointness judgment has an extra type context, which will be
explained in \cref{sec:disjoint:fi}.

\paragraph{Disjoint quantification.}

The new rules are the inferring rules for type abstractions \rref*{FT-tabs} and
type applications \rref*{FT-tapp}. In \rref{FT-tabs}, the disjointness
constraint is added to the type context. During a type application in
\rref{FT-tapp}, the type system checks that the type argument agrees with the
disjointness constraint. This, together with \rref{FT-merge} are the only two
rules that use the disjointness checking. Moreover, since \fnamee is
predicative, we require that the type being instantiated is a monotype.



% Local Variables:
% TeX-master: "../../Thesis"
% org-ref-default-bibliography: ../../Thesis.bib
% End:



\section{Disjointness}
\label{sec:disjoint:fi}


\begin{figure}[t]
  \centering
  \drules[FD]{$[[DD |- A ** B]]$}{Disjointness}{topL, topR, arr, andL, andR, rcdEq, rcdNeq, tvarL, tvarR, forall,ax}
  \drules[Dax]{$[[A **a B]]$}{Disjointness axioms}{sym, intArr, intRcd,intAll,arrAll,arrRcd,allRcd}
  \caption{Disjointness of \fnamee}
  \label{fig:disjoint:fi}
\end{figure}


In this section we present the formal rules of disjointness, as show in
\cref{fig:disjoint:fi}. The disjointness rules of \fnamee are directly inherited
from \fname~\citep{alpuimdisjoint}, which consists of two judgments.


\paragraph{Main judgment.}

The main judgment $[[DD |- A ** B]]$ says that the two types $[[A]]$ and $[[B]]$
are disjoint in the context $[[DD]]$. As a precondition, $[[A]]$
and $[[B]]$ are required to be both well-formed in the context $[[DD]]$.
Most of the rules are similar to those of
\namee. The major additions are the two rules \rref*{FD-tvarL,FD-tvarR} for
type variables, and \rref{FD-forall} for disjoint quantification.
\Rref{FD-tvarL} and the symmetric one \rref*{FD-tvarR} state that a type
variable $[[X]]$ is disjoint with some type $[[B]]$ if its
disjointness constraint (i.e., $[[A]]$) in the context $[[DD]]$ is a subtype of
$[[B]]$. These two rules are a specialization of a more general lemma, which
says that disjointness is covariant with respect to subtyping. In a more precise
sense, we have the following:

\begin{lemma}[Covariance of disjointness] \label{lemma:covariance:disjoint}
  If $[[DD |- A ** B]]$ and $[[B <: C]]$, then $[[DD |- A ** C]]$.
\end{lemma}
\begin{proof}
  By double induction, first on the subtyping derivation, and then on the
  type $[[A]]$. In the case for \rref{FS-forall}, we need \cref{lemma:narrow:disjoint}.
\end{proof}

\begin{lemma}[Narrowing of disjointness] \label{lemma:narrow:disjoint}
  If $[[DD, X ** C1 |- A ** B]]$ and $[[C2 <: C1]]$, then $[[DD, X ** C2 |- A ** B]]$.
\end{lemma}
\begin{proof}
  We need to slightly generalize the lemma in the sense that the type variable is inserted
  in the middle, then by induction on the disjointness derivation.
\end{proof}

An intuition of the following may help better understanding
\cref{lemma:covariance:disjoint}. As we will see in \cref{sec:category}, another
way to interpret two types being disjoint is that their least upper bound is
(isomorphic to) $[[Top]]$. Following this interpretation, it is obvious that if
the least upper bound of two given types is already $[[Top]]$, a supertype of
one of them will not change this fact.

We now turn to \rref{FD-forall}. To illustrate this rule, consider the following two types:
\begin{mathpar}
  [[ \X ** nat . X & nat ]] \and  [[ \X ** char . X & char ]]
\end{mathpar}
Under what conditions are the two types disjoint? In the first type, $[[X]]$
cannot be instantiated to $[[nat]]$ (among others) and in the second type
$[[X]]$ cannot be instantiated to $[[char]]$. Therefore for both bodies to be disjoint,
$[[X]]$ can only be instantiated to types that are disjoint with both $[[nat]]$
and $[[char]]$. More formally, in \rref{FD-forall}, we add to the context a new
constraint $[[A1 & A2]]$ by intersecting the two constraints $[[A1]]$ and $[[A2]]$, and check for disjointness in the bodies
under the extended context.

\paragraph{Disjointness axioms.}

Disjointness axioms $[[ A **a B ]]$  take care of two types with different type constructs,
except for when one of them is $[[Top]]$, an intersection type or a type
variable, which are all dealt with by the main judgment.

To conclude this section, we show that disjointness is symmetric:

\begin{lemma}[Symmetry of disjointness]
  If $[[ DD |- A ** B  ]]$, then $[[  DD |- B ** A   ]]$.
\end{lemma}
\begin{proof}
  By induction on the disjointness derivation. In the case for \rref{FD-forall},
  apply \cref{lemma:narrow:disjoint}.
\end{proof}

% Local Variables:
% TeX-master: "../../Thesis"
% org-ref-default-bibliography: ../../Thesis.bib
% End:



\section{Elaboration and Type Safety}
\label{sec:elaboration:fi}



\begin{figure}
  \centering
\begin{tabular}{llll} \toprule
  Types & $[[T]]$ & $\Coloneqq$ & $[[nat]] \mid [[Unit]] \mid [[T1 -> T2]]  \mid [[T1 * T2]] \mid \hlmath{[[X]] } \mid \hlmath{[[\ X . T]]}$\\
  Expressions & $[[e]]$ & $\Coloneqq$ & $[[x]] \mid [[i]] \mid [[unit]] \mid [[\x . e]] \mid [[e1 e2]] \mid [[< e1 , e2>]]  \mid [[c e]] \mid \hlmath{[[\X . e]]} \mid \hlmath{[[ e T ]]}$ \\
  Coercions & $[[c]]$ & $\Coloneqq$ & $[[id]] \mid [[c1 o c2]] \mid [[top]] \mid [[c1 -> c2]] \mid [[< c1 , c2 >]] \mid [[pp1]] \mid [[pp2]] $ \\
  & & $\mid$ & $ [[distArr]] \mid [[topArr]] \mid \hlmath{[[\ c]]} $ \\
  Values & $[[v]]$ & $\Coloneqq$ & $[[i]] \mid [[unit]] \mid [[\x . e]] \mid [[< v1 , v2>]] \mid [[ (c1 -> c2) v ]] \mid [[distArr v]] \mid [[topArr v]] $ \\
  & & $\mid$ & $ \hlmath{[[\X . e]]} \mid \hlmath{[[\c v]]}  $ \\
  Value Context & $[[gg]]$ & $\Coloneqq$ &  $[[empty]] \mid [[gg , x : T]] $ \\
  Type Context & $[[dd]]$ & $\Coloneqq$ &  $[[empty]] \mid [[dd , X ]] $ \\
  Evaluation Context & $[[EE]]$ & $\Coloneqq$ &  $  [[__]] \mid [[EE e]] \mid [[v EE]] \mid [[ < EE , e >  ]] \mid [[ < v , EE > ]] \mid [[ c EE  ]] \mid \hlmath{[[ EE T  ]]}  $ \\ \bottomrule
\end{tabular}
\caption{Syntax of \tnamee}
\label{fig:syntax:fco}
\end{figure}


Like \namee, the dynamic semantics of \fnamee is given by elaboration into
a target calculus. The target calculus \tnamee is the standard call-by-value
System F extended with products and coercions. The syntax of \tnamee is shown in
\cref{fig:syntax:fco}, with the differences from \tname \hll{highlighted}. We naturally
extend the type translation function $| \cdot |$ to cover type variables and
disjoint quantification as shown in \cref{def:type:translate:fi}. For disjoint
quantification, we simply erase the disjointness constraints and translate the body.

\begin{definition}[Type translation from \fnamee to \tnamee] \label{def:type:translate:fi}
  \begin{align*}
    | [[nat]] | &= [[nat]] \\
    | [[Top]] | &= \langle \rangle \\
    | [[A -> B]]  | &= [[ | A | -> | B |  ]] \\
    | [[ A & B  ]] | &= [[ | A | * | B |  ]] \\
    | [[ X  ]] | &= [[ X ]] \\
    | [[ \X ** A . B ]] | &= [[ \ X . | B | ]]
  \end{align*}
\end{definition}


\paragraph{Coercions and Coercive Subtyping.}

As shown in \cref{fig:syntax:fco}, we extend the coercions of \tname with a new
coercion form $[[ \ c ]]$, which expresses the transformation between two
universal quantifiers. Now we go back to the coercion part in \rref{S-forall}.
Since the disjointness constraint is erased during elaboration, it does not contribute to the
overall coercion; we only need the coercion generated by the subtyping of the
bodies $[[B1]]$ and $[[B2]]$. As a cognitive aid, it is instructive to mentally
``desugar'' the coercion $[[\ c]]$ to the regular term $[[ \f . \ X . c (f X)]]$, then
the expression $ [[\c v]] $ is ``equal'' to $[[  \X . c (v X) ]]$. This is why we treat
$[[ \c v ]]$ as a value.


\paragraph{\tnamee Static Semantics.}

\begin{figure}
  \centering
  \drules[wfe]{$[[ dd |- gg   ]]$}{Well-formedness of value context}{empty, var}
  \drules[wft]{$[[ dd |- T   ]]$}{Well-formedness of types}{int, var, arrow,prod, all}
  \drules[Ft]{$[[ dd ; gg |- e : T ]]$}{Static semantics}{unit, int, var, abs, app, tabs, tapp, pair, capp}
  \caption{Typing rules of \tnamee}
  \label{fig:typing:fco}
\end{figure}

\Cref{fig:typing:fco} presents the typing rules of \tnamee. Most of the rules
are quite standard. \Rref{Ft-capp} uses the coercion typing judgment $[[ c |- T1 tri T2 ]]$.
We extend the coercion typing of \tname in \cref{fig:co} with one new rule
\rref*{ct-forall} as shown below:
\[
  \drule{ct-forall}
\]


\paragraph{\tnamee Dynamic Semantics.}


\begin{figure}[t]
  \centering
  \begin{drulepar}[r]{$[[e --> e']]$}{Single-step reduction}{}
    \drule{id}
    \drule{trans}
    \drule{top}
    \drule{topArr}
    \drule{pair}
    \drule{arr}
    \drule{distArr}
    \drule{projl}
    \drule{projr} \and
    \hlmath{\drule{forall}} \and
    \hlmath{\drule{tapp}} \and
    \drule{app}
    \drule{ctxt}
  \end{drulepar}
  \caption{Dynamic semantics of \tnamee}
  \label{fig:red:fi}
\end{figure}


We extend the evaluation context with one new form $[[EE T]]$ for type
applications, as shown in \cref{fig:syntax:fco}. The set of reduction rules for \tnamee in \cref{fig:red:fi}
is a straightforward extension of \tname. We
have a new reduction rule \rref*{r-forall} for the new coercion. This rule might look
strange at first. To explain, let us use our old trick of treating the coercion
$[[\ c]]$ as the term $[[ \f . \ X . c (f X) ]]$, then the application
$[[(\f . \ X . c (f X)) v T ]]$ reduces to $[[ c (v T) ]]$. Also we add the
reduction rule \rref*{r-tapp} for type applications. Now we can show that
\tnamee is type-safe in the usual sense:

\begin{theorem}[Preservation of \tnamee]
  If $[[empty; empty |- e : T]]$ and $[[e --> e']]$, then $[[empty; empty |- e' : T]]$.
\end{theorem}

\begin{theorem}[Progress of \tnamee]
  If $[[empty; empty |- e : T]]$, then either $[[e]]$ is a value, or there exists $[[e']]$ such
  that $[[e --> e']]$.
\end{theorem}


\paragraph{Elaboration.}

We go back to the translation parts in \cref{fig:typing:fi}. The key idea of the
translation remains the same: we translate merges to pairs. For disjoint
quantification and disjoint type applications (\rref{FT-tabs,FT-tapp}), we
translate them to regular universal quantification and type applications,
respectively. For \rref{FT-rcd,FT-proj} we simply erase
the labels and translate the corresponding underlying term. All the remaining
rules are ported from \namee. To conclude, we show an example translation:
\begin{align*}
  & [[ (\X ** nat . (\x . x) : X -> X)  : \ X ** nat . X & nat -> X ]] \\
  \rightsquigarrow & \\
  & [[\ (pp1 -> id)  (\ X . \x . x)]]
\end{align*}

As with \namee, we show two lemmas that relate \fnamee to \tnamee.

\begin{lemma}[Coercions preserve types]
  If $[[A <: B ~~> c]]$, then $[[c |-  |A| tri |B|]]$.
  \label{lemma:sub-correct:fi}
\end{lemma}
\begin{proof}
  By structural induction on the derivation of subtyping.
\end{proof}


\begin{lemma}[Elaboration soundness] We have that:
  \begin{itemize}
  \item If $[[DD ; GG |- ee => A ~~> e]]$, then $[[ |DD| ; |GG| |- e : |A | ]]$.
  \item If $[[DD ; GG |- ee <= A ~~> e]]$, then $[[ |DD| ; |GG| |- e : |A | ]]$.
  \end{itemize}
\end{lemma}
\begin{proof}
  By structural induction on the derivation of typing.
\end{proof}


\paragraph{Algorithmic subtyping.}



\begin{figure}[t]
  \centering
  \begin{drulepar}[A]{$[[fs |- A <: B ~~> c]]$}{Algorithmic subtyping}{}
    \drule{prim}
    \drule{and}
    \drule{arr}
    \drule{rcd}
    \drule{top} \and
    \hlmath{\drule{forall}} \and
    \hlmath{\drule{var}} \and
    \drule{arrR}
    \drule{rcdR}
    \drule{andROne}
    \drule{andRTwo}
  \end{drulepar}
  \caption{Algorithmic subtyping of \fnamee}
  \label{fig:algo:sub:fi}
\end{figure}

We extend the algorithmic subtyping for \namee with two rules
\rref*{A-forall,A-var}, as shown in \cref{fig:algo:sub:fi}. They are simple
adaptions from their declarative counterparts. We also need to extend the
definition of rigid types to include type variables and disjoint quantification,
shown in \cref{def:rigid:extended}.

\begin{definition}[Rigid types, extended] \label{def:rigid:extended}
  \begin{mathpar}
    [[  pri rigid  ]] \and
    [[ X  rigid ]] \and
    [[ \X ** A . B rigid ]]
  \end{mathpar}
\end{definition}

Finally we show the correctness of the algorithmic subtyping:

\begin{theorem}[Soundness]
  If $[[ fs |- A <: B ~~> c]]$ then $ [[   A <: fs -> B ~~> c  ]]   $.
\end{theorem}

\begin{theorem}[Completeness] \label{thm:complete}
  If $[[A <: B ~~> c]]$ then there exists $[[c']]$ such that $[[ [] |- A <: B ~~> c']]$.
\end{theorem}



% \begin{remark}
%   As already can be seen, one drawback of our algorithmic subtyping is that the
%   size of rules grows exponentially as more constructs are added.
% \end{remark}



%%% Local Variables:
%%% mode: latex
%%% TeX-master: "../Thesis"
%%% org-ref-default-bibliography: ../Thesis.bib
%%% End:
