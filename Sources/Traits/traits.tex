\section{Typed First-Class Traits}
\label{sec:traits}

In \cref{sec:trait:overview} we have seen some examples of first-class traits at work
in \sedel. In this section we give a detailed account of \sedel's support for
typed first-class traits, to complement what has been presented so far. In doing so,
we simplify the examples in \cref{sec:trait:overview} to make better use of traits.
\Cref{sec:trait:typesystem} presents the formal type system of first-class traits.

\subsection{Traits in \sedel}

\sedel supports a simple, yet expressive form of traits~\citep{scharli2003traits}.
Traits provide a simple mechanism for find-grained code reuse, which
can be regarded as a disciplined form of multiple inheritance. A trait is
similar to a mixin in that it encapsulates a collection of related methods to be
added to a class. The
practical difference between traits and mixins is the way conflicting features
that typically arise in multiple inheritance are dealt with. Instead of
automatically resolved by scoping rules, conflicts are, in \sedel,
detected by the type system, and explicitly resolved by the programmer. Compared
with traditional trait models, there are three interesting points about
\sedel's traits:
\begin{inparaenum}[(1)]
\item they are \emph{statically typed};
\item they are \emph{first-class} values;
\item they support \emph{dynamic inheritance}.
\end{inparaenum}
The support for such combination of features is one of the key novelties of \sedel.
Another minor difference from traditional traits (e.g., in Scala) is that,
due to the use of structural types, a trait name is \emph{not} a type.

\subsection{Two Roles of Traits in \sedel}

\paragraph{Traits as templates for creating objects.}

An obvious difference between traits in \sedel and many other models of
traits~\citep{scharli2003traits,fisher2004typed,odersky2005scalable} is that they
directly serve as templates for objects. In many other trait models, traits are
complemented by classes, which take the responsibility for object creation. In
particular, most models of traits do not allow constructors for traits. However,
a trait in \sedel has a single constructor of the same name. Take our last trait
\lstinline{ide_editor} in \cref{sec:trait:overview} for example:
\lstinputlisting[linerange=81-81]{./examples/overview2.sl}% APPLY:linerange=EDITOR_INST
As with conventional object-oriented languages, the keyword \lstinline{new} is used to create
an object. A difference to other object-oriented languages is that the keyword
\lstinline{new} also specifies the intended type of the object. We instantiate
the \lstinline{ide_editor} trait and create an object \lstinline{a_editor1} of
type \lstinline{IDEEditor}. As we will see in \cref{subsec:cons}, constructors
with parameters can also be expressed.

It is tempting to instantiate the \lstinline{editor} trait such as
\lstinline{new[Editor] editor}. However this would result in a type error, because, as
we discussed, \lstinline{editor} has no definition of \lstinline{version}, and
blindly instantiating it would cause run-time errors. This behavior is on a par
with Java's abstract classes---i.e., traits with undefined methods cannot be instantiated on their own.

\paragraph{Traits as units of code reuse.}

The traditional role of traits is to serve as units of code reuse. \sedel's traits
can have this role as well.
Our \lstinline{spell_mixin} function in \cref{sec:trait:overview} is more complicated than it should be.
This is because we were mimicking classes as traits, and
mixins as functions over traits. Instead, traits already provide a mechanism of
code reuse. To illustrate this, we simplify \lstinline{spell_mixin} as follows:
\lstinputlisting[linerange=111-114]{./examples/overview2.sl}% APPLY:linerange=HELP
This is much cleaner. The trait \lstinline{spell} adds a method
\lstinline{check}. It also defines a method \lstinline{on_key}.
A key difference from \lstinline{spell_mixin} is that \lstinline{on_key} is invoked on the \lstinline{self}
parameter instead of \lstinline{super}. Note that this does not necessarily mean \lstinline{check} will call \lstinline{on_key}
defined in the same trait. As we will see, the actual behavior entirely depends on how we compose \lstinline{spell}
with other traits. One minor difference is that we do not need to tag \lstinline{on_key}
with the \lstinline{override} keyword, because \lstinline{spell} stands as a standalone entity.
Another interesting point is that the type of \lstinline{self} (i.e., \lstinline{OnKey})
is not the same as that of the trait body, which also contains the \lstinline{check} method.
In \sedel's traits, the type of \lstinline{self} serves as trait \emph{requirements}.


\paragraph{Classes and/or traits.}

In the literature on traits~\citep{Ducasse_2006, scharli2003traits}, the
aforementioned two roles are considered as competing. One reason of the two
roles conflicting in class-based languages is because a class must adopt a fixed
position in the class hierarchy and therefore it can be difficult to reuse and
resolve conflicts, whereas in \sedel, a trait is a standalone entity and is not
tied to any particular hierarchy. Therefore we can view our traits either as templates for creating objects,
or as units of code reuse. Another important reason why our
model can do just with traits is because we have a pure language. Mutable state
can often only appear in classes in imperative models of traits, which is a good
reason for having both classes and traits.



% \lstinline{version} method is not defined. Like mixins, \lstinline{help} can be
% combined with other traits to produce several combinations of functionality. For
% instance, we create another editor that inherits two traits \lstinline{editor}
% and \lstinline{help}.
% \lstinputlisting[linerange=92-95]{./examples/overview2.sl}% APPLY:linerange=HELP2
% Due to the lack of multiple inheritance in JavaScript, we were forced to use
% mixins. In \sedel, this can be easily achieved because traits support multiple
% inheritance. In general, \lstinline{inherits ...} can take one or more trait
% expressions (delimited by \lstinline{&}).


\subsection{Trait Types and Trait Requirements}

\paragraph{Object types and trait types.}

\sedel adopts a relatively standard foundational model of object-oriented
constructs~\citep{DBLP:conf/ecoop/LeeASP15} where objects are encoded as records
with a structural type. This is why the object \lstinline{a_editor1}
has the record type \lstinline{IDEEditor}. In \sedel, an object type is
different from a trait type. A trait type is specified via \lstinline{Trait[T1, T2]}.
% For example, the type of the \lstinline{spell} trait is \lstinline{Trait[OnKey, OnKey & Spelling]}.

\paragraph{Trait requirements and functionality.}

In general, a trait type
\lstinline{Trait[T1, T2]} specifies both the \emph{requirements} \lstinline{T1}
and the \emph{functionality} \lstinline{T2} of a trait. The requirements of a trait denote the types/methods that the
trait needs to support for defining the functionality it provides. % Both are
% reflected in the trait type.
For example, \lstinline{spell} has type
\lstinline{Trait[OnKey, OnKey & Spelling]}, meaning that \lstinline{spell}
requires some implementation of the \lstinline{on_key} method, and it provides
implementations for the \lstinline{on_key} and \lstinline{check} methods.
When a trait
has no requirements, the absence of a requirement is denoted by using
the top type \lstinline{Top}. A simplified sugar \lstinline{Trait[T]} is
used to denote a trait without requirements, but providing functionality \lstinline{T}.




\paragraph{Trait requirements as abstract methods.}

Let us go back to our very first trait \lstinline{editor} in
\cref{sec:trait:overview}. Note how in \lstinline{editor} the type of the
\lstinline{self} parameter is \lstinline{Editor & Version}, where
\lstinline{Version} contains a declaration of the \lstinline{version} method
that is needed for the definition of \lstinline{show_help}. Note also that the
trait itself does not actually contain a \lstinline{version} definition. In many
other object-oriented models a similar program could be achieved by having an \emph{abstract}
definition of \lstinline{version}. In \sedel there are no abstract definitions
(methods or fields), but a similar result can be achieved via trait
requirements. Requirements of a trait are met at the object creation point. For
example, as we mentioned before, the \lstinline{editor} trait alone cannot be
instantiated since it lacks \lstinline{version}. However, when it is composed
with a trait that provides \lstinline{version}, the composition can be
instantiated, as shown below:
\lstinputlisting[linerange=92-95]{./examples/overview2.sl}% APPLY:linerange=HELP2
\sedel uses a syntax where the self parameter can be explicitly named (not
necessarily named \lstinline{self}) with a type annotation. When the self
parameter is omitted (for example in the \lstinline{foo} trait above), its type
defaults to \lstinline{Top}. This is different from most object-oriented languages, where
the default type of the self parameter is the same as the class being defined.
This also makes trait requirements ``pay as you go'' in the sense that if the
self parameter is not used in the body, then there is no requirements on the
trait. Otherwise, suppose the type of the self parameter in the trait
\lstinline{foo} implicitly defaults to \lstinline{Version}:
\lstinputlisting[linerange=104-106]{./examples/overview2.sl}% APPLY:linerange=HELP3
then \lstinline{Version} will pollute the type of the self parameter of any trait that
uses \lstinline{foo}, cascading down the inheritance hierarchy, even though \lstinline{self}
is not used in the body of \lstinline{foo}.



\paragraph{Intersection types model subtyping.}

The type \lstinline{IDEEditor} is defined as an intersection type.
Intersection types~\citep{coppo1981functional,pottinger1980type} have been woven
into many modern languages these days. A notable example is Scala, which makes
fundamental use of intersection types to express a class/trait that extends
multiple other traits. An intersection type such as \lstinline{T1 & T2} contains
exactly those values which can be used as values of type \lstinline{T1} and of
type \lstinline{T2}, and as such, \lstinline{T1 & T2} immediately introduces a
subtyping relation between itself and its two constituent types \lstinline{T1}
and \lstinline{T2}. Unsurprisingly, \lstinline{IDEEditor} is a subtype of
\lstinline{Editor}.


% \paragraph{Composition of traits}


% The definition of the object \lstinline{my_editor2} also shows the second way to
% introduce inheritance, namely by \emph{composition} of traits. Composition of
% traits is denoted by the operator \lstinline{&}. Thus \sedel offers two options
% when it comes to inheritance: we can either compose beforehand when declaring
% traits (using \lstinline{inherits}), or compose at the object creation point
% (using \lstinline{new} and the \lstinline{&} operator).

%Under the hood, inheritance is accomplished by using the \emph{merge operator}
%(denoted by \lstinline{,,}). The merge operator~\citep{dunfield2014elaborating}
%allows two arbitrary values to be merged, with the resulting type being an
%intersection type.
%For example the type of \lstinline{2 ,, true} is
%\lstinline{Int & Bool}.


\subsection{Traits with Parameters and First-Class Traits}

\label{subsec:cons}

So far our uses of traits involve no parameters. Instead of inventing another trait
syntax with parameters, a trait with parameters is just a function that produces
a trait expression, since functions already have parameters of their own. This
is one benefit of having first-class traits in terms of language economy. To
illustrate, let us simplify \lstinline{modal_mixin} in a similar way as in \lstinline{spell_mixin}:
\lstinputlisting[linerange=118-122]{./examples/overview2.sl}% APPLY:linerange=MODAL2
The first thing to notice is that \lstinline{modal} is a function with one
argument, and returns a trait expression, which essentially makes
\lstinline{modal} a trait with one parameter.
Now it is easy to see that a trait declaration
\lstinline$trait name [self : ...] => {...}$ is just syntactic sugar for
function definition \lstinline$name = trait [self : ...] => {...}$. The body of
the \lstinline{modal} trait is straightforward. We initialize the
\lstinline{mode} field to \lstinline{init_mode}.
The \lstinline{modal} trait also comes with a constructor with one parameter---e.g., we can create an object via \lstinline{new[ModalEdit] (modal "insert")}.

\subsection{Detecting and Resolving Conflicts in Trait Composition}
\label{sec:trait:forward}

A common problem in multiple inheritance is how to detect and/or resolve conflicts. For example, when
inheriting from two traits that have the same field, then it is unclear which
implementation to choose. There are various approaches to dealing with
conflicts. The trait-based approach requires conflicts to be resolved at the
level of the composition, otherwise the program is rejected by
the type system. \sedel provides a means to help resolve conflicts.

We start by assembling all the traits defined in this section
to create the final editor with the same functionality as
\lstinline{ide_editor} in \cref{sec:trait:overview}. Our first try is as follows:
\lstinputlisting[linerange=132-136]{./examples/overview2.sl}% APPLY:linerange=MODAL_CONFLICT
Unfortunately the above trait gets rejected by \sedel because
\lstinline{editor}, \lstinline{spell} and \lstinline{modal} all define an \lstinline{on_key} method.
Recall that in \cref{sec:trait:overview}, when we use a mixin-style composition,
the conflict resolution code has been hardwired in the definition.
However, in a trait-style composition, this is not the case: conflicts must be resolved \emph{explicitly}.
The
above definition is ill-typed precisely because there is a conflicting
method \lstinline{on_key}, thus violating the disjointness conditions
imposed by disjoint intersection types.

\paragraph{Resolving conflicts.}

To resolve the conflict, we need to explicitly state which implementation of the method
\lstinline{on_key} gets to stay. \sedel provides such a means---the \emph{exclusion} operator (denoted by \lstinline$\$)---which allows one to
exclude a field/method from a given trait. The following matches the behavior
in \cref{sec:trait:overview} where \lstinline{on_key} from the \lstinline{modal} trait
is selected:
\lstinputlisting[linerange=143-149]{./examples/overview2.sl}% APPLY:linerange=MODAL_OK
Now the above code type checks. We can also select \lstinline{on_key} from the \lstinline{spell} trait as easily:
\lstinputlisting[linerange=154-159]{./examples/overview2.sl}% APPLY:linerange=MODAL_OK2
In \cref{sec:trait:overview} we mentioned that in the mixin style, it is impossible
to select \lstinline{on_key} from the \lstinline{editor} trait, but this is not a problem now:
\lstinputlisting[linerange=163-168]{./examples/overview2.sl}% APPLY:linerange=MODAL_OK3
Using the exclusion operator, we can drop \lstinline{on_key} in \lstinline{spell} and \lstinline{modal} while
keeping it in \lstinline{editor}.


\paragraph{The forwarding operator.}

Another operator that \sedel provides is the \emph{forwarding} operator, which can be useful when we want to access some method that has been
explicitly excluded in the \lstinline{inherits} clause. This is a common scenario in
diamond inheritance, where \lstinline{super} is not enough. Below we show a
variant of \lstinline{ide_editor}:
\lstinputlisting[linerange=173-181]{./examples/overview2.sl}% APPLY:linerange=MODAL_WIRE
Notice that \lstinline{on_key} in \lstinline{spell} has been
excluded. However, we can
still access it by using the forwarding operator as in \lstinline{spell ^ self},
which gives full access to all the methods in \lstinline{spell}. Also note that
using \lstinline{super} only gives us access to \lstinline{on_key} in the
\lstinline{modal} trait. To see \lstinline{ide_editor4} in action, we create a
small test:
\lstinputlisting[linerange=185-188]{./examples/overview2.sl}% APPLY:linerange=MODAL_USE
% \jeremy{Compare this to trait alias operator?}
% \jeremy{would it be better to use forwarding operator in mix-style \lstinline{ide_editor} in
%   \cref{sec:trait:overview}, where we show how to access \lstinline{on_key} from editor, which is impossible in JavaScript?}

% Since the result editor trait has such an exciting feature, we increment the version number to \lstinline{0.2}!



\subsection{Disjoint Polymorphism and Dynamic Composition}

\sedel supports disjoint polymorphism. The combination of disjoint
polymorphism and first-class traits enables the highly modular code
where traits with \emph{statically unknown} types can be instantiated
and composed in a type-safe way! The following is illustrative of this:
%However, this is not a problem in \sedel. Thanks to disjoint polymorphism and
%disjoint intersection types, we can define the same \lstinline{merge} that is able to take two
%traits (where the full set of the members may not be known statically), combine and instantiate them.
\lstinputlisting[linerange=6-6]{./examples/overview2.sl}% APPLY:linerange=MERGE
The \lstinline{mergeTraits} function takes two traits \lstinline{x} and \lstinline{y} of
some arbitrary types \lstinline{Trait[A]} and \lstinline{Trait[B]}, composes them,
and creates an object from the resulting composed trait. Clearly
such composition cannot always work if \lstinline{A} and
\lstinline{B} can have conflicts. However, \lstinline{mergeTraits} has a
constraint \lstinline{B * A} that ensures that whatever types are used
to instantiate \lstinline{A} and \lstinline{B} they must be disjoint.
Thus, under the assumption that \lstinline{A} and \lstinline{B} are
disjoint the code type-checks.


%%% Local Variables:
%%% mode: latex
%%% TeX-master: "../paper"
%%% org-ref-default-bibliography: ../paper.bib
%%% End:
