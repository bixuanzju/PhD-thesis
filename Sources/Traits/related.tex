
\paragraph{Typed First-Class Classes/Mixins/Traits.}
First-class~\citep{DBLP:conf/aplas/FlattFF06} classes have been used in
Racket, along with mixin support, and
have shown great practical value. For example, DrRacket
IDE~\citep{DBLP:journals/jfp/FindlerCFFKSF02} makes extensive use of layered
combinations of mixins to implement text editing features. The topic of
first-class classes with static typing has been explored by Takikawa et
al.~\citep{DBLP:conf/oopsla/TakikawaSDTF12} in Typed Racket. They designed a
gradual type system that supports first-class classes. Of particular interest is
their use of row polymorphism~\citep{wand1994type} to type mixins.
% For example,
% \lstinline{modal_mixin} from \cref{sec:overview} implemented in Typed Racket has
% type:
% \begin{lstlisting}
% (All (r / on-key toggle-mode)
%      (Class ([on-key : (String -> Void)] | r)) ->
%      (Class ([toggle-mode : (-> Void)] [on-key : (String -> Void)] | r)))
% \end{lstlisting}
As with our use of disjoint polymorphism, row polymorphism can express
constraints on the presence or absence of members. Unlike disjoint polymorphism,
row polymorphism prohibits forgetting class members.
% While this is reasonable in
% the setting of mixins, in some cases, a function taking one class as an argument
% can return another class that has fewer methods.
For example, in \name we can write:
\lstinputlisting[linerange=100-100]{./examples/overview2.sl}% APPLY:linerange=FOO
where \lstinline{foo} drops \lstinline{bar} from its argument trait \lstinline{t},
which is impossible to express in Typed Racket. Also as we pointed out in \cref{sec:merge}, row polymorphism
alone cannot express the \lstinline{merge} function that is able to compose objects of statically unknown types.
In this sense, we argue disjoint polymorphism is more powerful than row polymorphism in terms of expressivity.
It would be interesting to investigate the relationship between disjoint polymorphism and row polymorphism. We leave it
as future work.


% As a consequence, Typed Racket ends up with two subtyping mechanisms:
% one for first-class classes (via row polymorphism) and the other for objects (via normal width subtyping). In
% contrast, \name uses only one mechanism -- disjoint polymorphism -- to deal with
% both.

More recently, Lee et al.~\citep{DBLP:conf/ecoop/LeeASP15} proposed a model for typed first-class
classes based on tagged objects. Like our development, the semantics
of their source language is defined by a translation into a target language.
One notable difference to \name is that they require the use of a variable
rather than an expression in the \lstinline{extends} clause, whereas we do not
have this restriction. In their source language, subclasses define subtypes,
which limits its applicability to extensible designs. Also their target calculus is significantly more complex
than ours due to the use of dependent function types and dependent sum types. As
they admitted, they omit inheritance in their formalization.

Racket also supports a \emph{dynamically-typed model} of first-class
traits~\citep{DBLP:conf/aplas/FlattFF06}. However, unlike Racket's first-class classes and
mixins, there's no type system supporting the use of first-class
traits. A key difficulty is \emph{statically} detecting conflicts.
% In the mixin model this is not a problem because conflicts are implicitly
% resolved using the order of composition.
As far as we know, \name is the first design for typed first-class traits.


% There are multiple flavours of inheritance. To avoid confusion, since the same
% terminology is often used in the literature to mean different things, we use the
% following 3 terms when comparing related work with ours.

% \begin{itemize}
% \item{{\bf Static inheritance:}} Static inheritance refers to what the typical
%   model of inheritance in class-based languages. The inheritance model is said
%   to be static because when using class extension, the extended classes are
%   statically known at compile-time.
% \item{{\bf Mutable Inheritance:}} Prototype-based languages allow another model
%   of inheritance, which we call \emph{mutable inheritance}. In this inheritance
%   model, self-references are mutable and changeable at any point.
% \item{{\bf Dynamic Inheritance:}} Dynamic inheritance is a less well-known model
%   which stands in between static and mutable inheritance. Unlike the static
%   inheritance model, with dynamic inheritance objects can inherit from other
%   objects which are not statically known. However, unlike mutable inheritance,
%   the self-reference is not mutable and cannot be arbitrarily changed at
%   run-time.
% \end{itemize}

% \Cref{fig:comparision} shows the comparison between \name and various
% similar languages that follow \cite{cook1989inheritance}'s ``Inheritance is not
% Subtyping'' (i.e. the flexible model), as we will explain below.

% \begin{figure}[t]
%   \centering
%   \begin{tabular}{|l||c|c|c|c|}
%     \hline
%     & \bf{Statically typed} & \bf{Polymorphism} & \bf{Meta-theory} & \bf{Inheritance}  \\
%     \hline
%     \name & \cmark & \cmark & \cmark & Dynamic \\
%     \hline
%     \textsc{Self} & \xmark & \xmark & \xmark & Mutable \\
%     \hline
%     Cecil & \cmark & \cmark & \xmark & Static \\
%     \hline
%     Cook's Modula-3 & \cmark & \xmark & \xmark & Static \\
%     \hline
%     IFJ & \cmark & \xmark & \cmark & Dynamic \\
%     \hline
%     \textsc{Darwin} & \cmark & \xmark & \xmark & Dynamic \\
%     \hline
%   \end{tabular}
%   \caption{Comparison between \name and various similar languages that
%   adopt the \emph{flexible model}.}
%   \label{fig:comparision}
% \end{figure}



% \paragraph{Dynamically-typed Languages with Delegation Mechanism}

% \begin{itemize}
% \item Clojure Protocols
%   % http://www.ibm.com/developerworks/library/j-clojure-protocols/
% \item Ruby mixin
% \item JS mixin
% \end{itemize}

% They are all dynamically typed.


% \paragraph{Delegation-based languages}

% \cite{lieberman1986using} is the first to promote the use of prototypes and
% delegation as the mechanism to code sharing between objects. Since then many
% researchers have studied the mechanisms of
% delegation~\citep{wegner1987dimensions,malenfant1995semantic,goldberg1989smalltalk}.

% There is not much work on statically-typed, delegation-based languages.
% \cite{kniesel1999type} provides a good overview of problems when combining
% delegation with a static type discipline. Cecil~\citep{chambers1992object,
%   chambers1993cecil} is a prototype-based language, where delegation is the
% mechanism for method call and code reuse. Cecil supports a polymorphic static
% type system, although no meta-theory of any kind is given. Its type system is
% able to detect statically when a message might be ambiguously defined as a
% result of multiple inheritance or multiple dispatching. However, one major
% omission of Cecil, which is also one of the interesting features of \name, is
% dynamic inheritance. There are other
% works~\citep{fisher1995delegation,anderson2003can} on delegation in a
% statically-typed setting, but none of them provide means (such as the merge
% construct, disjointness constraints, etc.) that are needed for extensible
% designs.

% \cite{cook1989inheritance} were the first to propose a typed model of
% inheritance where subtyping and inheritance are two separate concepts. In
% particular, they introduce the notion of \emph{type inheritance} and show that
% inherited objects have inherited types, not subtypes. An interesting aspect of
% their calculus is the \textbf{with} construct, used to join two records. This is
% somewhat similar to our merge construct. However two major differences are worth
% pointing out: 1) the \textbf{with} construct operates only on records; and 2) it
% is a biased operator, favoring values from its right argument. This biased
% operator is good for modelling mixins, but not traits. The
% \textbf{with} construct seems to be unable to merge two arbitrary (and possible
% polymorphic) values, since this seems to require something like
% \emph{row polymorphism}~\citep{wand1987complete,wand1989type}, which is not available in their language.
% The \emph{onion} construct in the Big Bang
% language~\citep{palmer2015building,menon2012big} has a similar bias problem -- it is a
% left-associative operator which gives rightmost precedence to one
% implementation when conflicts exist.

\paragraph{Mixin-Based Inheritance.}

% Mixins have become very popular in many OO languages~\citep{flatt1998classes,bono1999core, ancona2003jam}.
Bracha and Cook's seminal paper~\citep{bracha1990mixin} extends Modula-3 with
mixins. Since then, many mixin-based models have been proposed~\citep{flatt1998classes,bono1999core, ancona2003jam}.
Mixin-based inheritance requires that mixins are composed linearly, and
as such, conflicts are resolved implicitly. In comparison, the trait
model in \name requires conflicts to be resolved explicitly. We want to emphasize
that conflict detection is essential in expressing composition operators
for Object Algebras, without running into ambiguities. Bracha's
Jigsaw~\citep{bracha1992programming} formalized mixin composition, along with a rich
trait algebra including merge, restrict, select, project, overriding and rename operators.
Lagorio et al.~\citep{LAGORIO201286} proposed \textsc{FJig} that reformulates Jigsaw constructs
in a Java-like setting.
Allen et al.~\citep{DBLP:conf/oopsla/AllenBC03} described how to add first-class generic
types -- including mixins -- to OO languages with nominal typing.
As such, classes and mixins, though they enjoy static typing, are still second-class
constructs, and thus their system cannot express dynamic inheritance. Bessai et
al.~\citep{DBLP:journals/corr/BessaiDDCd15} showed how to type classes and mixins
with intersection types and Bracha-Cook's merge operator~\citep{bracha1990mixin}.
% In contrast to our merge operator, their merge operator can only work for
% records, thus limiting its expressive power.


\paragraph{Trait-Based Inheritance.}


Traits were proposed by Sch{\"a}rli et al.~\citep{scharli2003traits, Ducasse_2006} as a mechanism
for fine-grained code reuse to overcome many limitations of class-based inheritance. The original proposal
of traits were implemented in the dynamically-typed class-based language \textsc{Squeak/Smalltalk}.
Since then various formalizations of traits in a Java-like (statically-typed) setting
have been proposed~\citep{fisher2004typed,scharli2003traitsformal,chai_trait, JOT:issue_2006_05/article4}.
% Explicit conflict resolution is the hallmark of trait-based inheritance, compared with
% mixin-based inheritance.
In most of the above proposals, trait composition and class-based inheritance live together.
\name, in the spirit of \textit{pure trait-based programming languages}~\citep{BETTINI2013521, BETTINI2017419}, embraces
traits as the sole mechanism for code reuse.
% differs from prior models of (second-class)
% traits in that they support \emph{classes} in addition to traits, and consider
% the interaction between them, whereas in \name the mechanism for code reuse is
% purely traits.
The deviation from traditional class-based inheritance is not only
because of its simplicity, but also because we need a very \emph{dynamic} form
of inheritance.

% Compared to the traditional trait mode, traits in \name have the following
% differences: 1) traditional traits cannot be instantiated but only composed with
% a class, whereas traits in \name can be instantiated directly; 2) traditional
% traits cannot take constructor parameters whereas ours can; 3) the trait system
% in \name lacks a proper notation of inheritance relationship. For example in the
% traditional trait model, if the same method (i.e., from the same trait) is
% obtained more than once via different paths, there is no conflict. This is not
% the case in \name; and 4) traits in \name support dynamic
% inheritance.
%In the
%traditional trait model, when it comes to inheritance, the traits being
%inherited must be statically known.




% \cite{flatt1998classes} proposed MIXEDJAVA, an extension to a subset of
% sequential Java called CLASSICJAVA with mixins. In their model, mixins
% completely subsume the role of classes (classes are mixins that do not inherit
% any services). One interesting aspect in their system is that two identically
% named methods are allowed to coexist, and are resolved at run-time with run-time
% context information provided by the current \emph{view} of an object. In
% comparison, conflicts in \name are detected statically, and resolved by the
% programmers. Like \name, their model also enforces the distinction between
% implementation inheritance and subtyping.

% \cite{bono1999core} develop an imperative class-based calculus that provides a
% formal model for both single and mixin inheritance. Objects are represented by
% records and produced by instantiating classes. In their calculus, the class
% construct is extensible but not subtypable, while objects are subtypable but not
% extensible. Like \name, their system has a clean separation between subtyping
% and inheritance. Also, their type system does not have polymorphism.

% \cite{ancona2003jam} extends the Java language to support mixins, called Jam.
% Since Jam is an upward-compatible extension of Java 1.0, it is inheritantly a
% covariant mode. Unlike MIXEDJAVA, mixins can be only instantiated on classes,
% and there is no notion of mixin composition.


% Incomplete Featherweight Java (IFJ), proposed by \cite{bettini2008type}, is a
% conservative extension of Featherweight Java with incomplete objects. Besides
% standard classes, programmers can also define incomplete classes, whose
% instances are incomplete objects. Incomplete objects can be composed (by object
% composition) with complete objects, yielding new complete objects at run-time,
% while ensuring statically that the composition is type-safe. Incomplete objects
% are quite flexible, and support dynamic inheritance. However, object composition
% in IFJ is quite restrictive, compared to \name, in that it can only compose an
% incomplete object with a complete object. In that regard, and also because IFJ's
% type system is not polymorphic, IFJ is unable to encode composition operators of
% Object Algebras. \cite{kniesel1999type} showed that type-safe integration of
% delegation with subtyping into a class-based model is possible, resulting in the
% \textsc{Darwin} model. In \textsc{Darwin}, the type of the parent object must be
% a declared class and this limits the flexibility of dynamic composition.
% Ostermann's delegation layers~\citep{ostermann2002dynamically} use delegation for
% doing dynamic composition in a system with virtual classes. This is in contrast
% with most other approaches that use class-based composition, but closer to the
% dynamic composition that we use in \name.

\paragraph{Languages with More Advanced Forms of Inheritance.}
\textsc{Self}~\citep{ungar1988self} is a dynamically-typed, prototype-based
language with a simple and uniform object model. \textsc{Self}'s inheritance
model is typical of what we call \textit{mutable inheritance}, because an object's parent
slot may be assigned new values at run time. Mutable inheritance is rather
unstructured, and oftentimes access to any clashing methods will generate a
``messageAmbiguous'' error at run time. Although \name's dynamic inheritance is
not as powerful as mutable inheritance, its static type system can guarantee
that no such errors occur at run time. Eiffel~\citep{meyer1987eiffel} supports a
sophisticated class-based multiple inheritance with deep renaming, exclusion
and repeated inheritance. Of particular interest is that in Eiffel, name
collisions are considered programming errors, and ambiguities must be resolved
explicitly by the programmer (by means of renaming). In this regard, \name is
quite like Eiffel. However, the type system in \name is more lenient in that two
identically named methods with different signatures can coexist. Grace~\citep{DBLP:journals/jot/NobleBBHJ17, DBLP:conf/ecoop/0002HNB16}
is an object-based language designed for education, where objects are created by
\textit{object constructors}.
% In that regard, Grace is similar to \name in that
% both are not class-based.
Since Grace has mutable fields, it has to consider
many concerns when it comes to inheritance, resulting in a rather complex
inheritance mechanism with various restrictions.
% For example, Grace imposes the
% constraint that the object being inherited must be \emph{fresh}. The effect of
% the freshness constraint is that the expression in the inherit clause must
% generate a new object. As a consequence, the expression after
% \lstinline{inherit} could not be a variable, which seems to preclude
% dynamic inheritance.
Since \name is pure, a relatively simple
% delegation-based
encoding of traits with late binding of \lstinline{self}
suffices for our applications. Grace's support for multiple inheritance is
based on so-called \emph{instantiable traits}.
% Still the freshness constraint applies.
We believe that there is plenty to be learned from
Grace's design of traits if we want to extend our trait model with
features such as mutable state. \textsc{MetaFJig}~\citep{SERVETTO2014219} (an extension of \textsc{FJig})
supports \textit{dynamic trait replacement}~\citep{chai_trait, BETTINI2013907, Ducasse_2006},
a feature for changing the behavior of an object at run time by replacing one trait for another.




% There are many other class-based OO languages that are equipped with more
% advanced forms of
% inheritance~\citep{meyer1987eiffel,buchi2000generic,ostermann2001object}. Most of
% them are heavyweight and are specific to classes. \name is object-centered, more
% lightweight, and is dedicated to express extensible designs in a simpler way.



% \cite{kniesel1999type} is the first to show that type-safe integration of
% delegation with subtyping into a class-based model is possible, resulting in the
% DARWIN model. In the DARWIN model, the type of the parent object must be a
% declared class and this limits the flexibility of dynamic composition, whereas
% in \name, the merge operator can merge/compose any objects. Another difference
% with \name lies in the conflict resolution, where DARWIN relies on method
% overriding with the assumption that the author of the overriding method is aware
% of the effect.

% Generic wrappers~\citep{buchi2000generic} supports aggregating objects at
% run-time. In their model, once a ``wrappee'' is assigned to a ``wrapper'', the
% wrappee is fixed. GBETA~\citep{ernst2000gbeta} has some dynamic features that are
% related to delegation. Like Generic wrappers, parents in GBETA are fixed at
% run-time.

% \cite{ostermann2001object} proposed compound references (CR) as a abstraction
% for object references, which provides explicit linguistic support for combining
% different composition properties on-demand. The model is statically typed, and
% decouples subtype declaration from implementation reuse.


% \cite{ostermann2002dynamically} proposed delegation layers as an approach to
% decompose a collaboration into layers and compose these layers dynamically at
% run-time. This combines and generalizes delegation and virtual classes concepts.

% \cite{ostermann2008nominal} compared the nominal and structural subtyping
% mechanisms. They argue nominal subtyping gives more safety guarantee, whereas
% structural subtyping is more flexible from a component-based perspective. The
% type system of \name chooses structural subtyping.

\paragraph{Module Systems.}

In parallel to OOP, the ML module system originally proposed by
MacQueen~\citep{MacQueen_1984} also offers powerful support for flexible program
construction, data abstraction and code reuse. Mixin modules in the
Jigsaw framework~\citep{Bracha92modularitymeets} provides a suite of operators
for adapting and combining modules. The MixML~\citep{Rossberg_2013} module
system incorporates mixin module composition, while retaining the full
expressive powerful of ML modules. Module systems usually put more emphasis on
supporting type abstraction. Support for type abstraction
adds considerable complexity, which is not needed in \name. \name is focused on
OOP and supports, among others, method overriding, self references and dynamic
dispatching, which (generally speaking) are all missing features in module systems.
% One similarity between MixML and SEDEL is that
% both use an elaboration approach into a System F-like calculus.



\paragraph{Intersection Types, Polymorphism and Merge Construct.} There
is a large body of work on intersection types. Here we only talk about work that
has direct influences on ours. Dunfield~\citep{dunfield2014elaborating} shows
significant expressiveness of type systems with intersection types and a merge
construct. However his calculus lacks coherence. The limitation was addressed by
Oliveira el at.~\citep{oliveira2016disjoint}, where they introduced the notion of
disjointness to ensure coherence. The combination of intersection types, a merge
construct and parametric polymorphism, while achieving coherence was first
studied in the \fname calculus~\citep{alpuimdisjoint}.
% where they proposed the notion of disjoint polymorphism.
\fname serves as the target language
of \name. Dynamic inheritance, self-references and abstract methods
are all missing from \fname but, as shown in this paper, they can be
encoded using an elaboration that employs ideas from Cook and
Palsberg's denotational model of inheritance~\citep{cook1989denotational}.


%%% Local Variables:
%%% mode: latex
%%% TeX-master: "../paper"
%%% org-ref-default-bibliography: ../paper.bib
%%% End:
