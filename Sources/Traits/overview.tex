
\section{Overview}
\label{sec:trait:overview}

This section aims at introducing first-class classes and traits, their possible
uses and applications, as well as the typing challenges that arise
from their use.
We start by describing a hypothetical JavaScript library for text editing
widgets, inspired and adapted from Racket's GUI
toolkit~\citep{DBLP:conf/oopsla/TakikawaSDTF12}. The example is illustrative of
typical uses of dynamic inheritance/composition, and also the typing challenges
in the presence of first-class classes/traits. Without diving into
technical details, we then give the corresponding typed version in
\sedel, and informally presents its salient features.

\subsection{First-Class Classes in JavaScript}

A class construct was officially added to JavaScript in the ECMAScript
2015 Language Specification~\citep{EcmaScript:15}. One purpose of
adding classes to JavaScript was to support a construct that is more
familiar to programmers who come from mainstream class-based languages,
such as Java or C++. However classes in JavaScript are
\emph{first-class} and support functionality not easily mimicked in
statically typed class-based languages.

\paragraph{Conventional classes.}

Before diving into the more advanced features of JavaScript classes, we first
review the more conventional class declarations supported in JavaScript as well
as many other languages. Even for conventional classes there are some
interesting points to note about JavaScript that will be important when we move
into a typed setting. An example of a JavaScript class declaration is:
\begin{lstlisting}[language=JavaScript]
class Editor {
  onKey(key) {
    return "Pressing " + key;
  }
  doCut() {
    return this.onKey("C-x") + " for cutting text";
  }
  showHelp() {
    return "Version: " + this.version() + " Basic usage...";
  }
};
\end{lstlisting}
This form of class definition is standard and very similar to declarations in
class-based languages (for example Java). The \lstinline{Editor} class
defines three methods: \lstinline{onKey} for handling key events,
\lstinline{doCut} for cutting text and \lstinline{showHelp} for displaying help
message. For the purpose of demonstration, we elide the actual implementation,
and replace it with plain messages.

We wish to bring the reader's attention to two points in the above class.
Firstly, note that the \lstinline{doCut} method is defined in terms of the
\lstinline{onKey} method via the keyword
\lstinline[language=JavaScript]{this}. In other words the call to
\lstinline{onKey} is enabled by the \emph{self} reference and is
\emph{dynamically dispatched} (i.e., the particular implementation of
\lstinline{onKey} will only be determined when the class or subclass
is instantiated). % Typically an
% OO programmer seeing this definition would expect the \lstinline{doCut} method
% to call the \lstinline{onKey} method of a subclass of \lstinline{Editor}, even though
% the subclass does not exist when the superclass \lstinline{Editor} is being
% defined.
Secondly, notice that there is no definition of
the \lstinline{version} method in the class body, but such method is used in the body of the
\lstinline{showHelp} method. In a dynamically typed language, such as JavaScript, using
undefined methods is error prone---accidentally instantiating \lstinline{Editor}
and then calling \lstinline{showHelp} will cause a run-time error!
Statically-typed languages usually provide some means to protect us from this
situation. For example, in Java, we would need an \emph{abstract} \lstinline{version}
method, which effectively makes \lstinline{Editor} an abstract class and
prevents it from being instantiated. As we will see, \sedel's treatment of
abstract methods is quite different from mainstream languages. In fact, \sedel
has a unified (typing) mechanism for dealing with both dynamic dispatch and abstract
methods. We will describe \sedel's mechanism for dealing with both features and
justify our design in \cref{sec:traits}.

% A couple of things worth pointing out in the above code snippet: (1) the class
% \lstinline{Editor} has no definition of the method
% \lstinline{version}, but such method
% is used in the body of the method \lstinline{showHelp}. In a strongly-typed OO
% language, such as Java, we would need to define an abstract method for
% \lstinline{version}. (2) The \lstinline{Editor} class requires
% \emph{dynamic dispatching}.
%  In the body of the method \lstinline{doCut} we invoke
% the method \lstinline{onKey} defined in the same class through the keyword
% \lstinline[language=JavaScript]{this}. This has the implication that when a
% subclass of \lstinline{Editor} overrides the method \lstinline{onKey}, a call to
% \lstinline{doCut} should invoke \lstinline{onKey} defined in the subclass
% instead of the original one.\bruno{punchline?}
%As we will see later, the type system of \sedel correctly handles it.

\paragraph{First-class classes and class expressions.}

Another way to define a class in JavaScript is via a \emph{class expression}. This is where the class
model in JavaScript is very different from the traditional class model found in
many mainstream object-oriented languages, such as Java, where classes are second-class
(static) entities. JavaScript embraces a dynamic class model that treats classes
as \emph{first-class} expressions: a function can take classes as arguments,
or return them as a result. First-class classes enable programmers to
abstract over patterns in the class hierarchy and to experiment with new forms of OOP
such as mixins and traits. In particular, mixins become programmer-defined
constructs. We illustrate this by presenting a simple mixin that adds
spell checking to an editor:
\begin{lstlisting}[language=JavaScript]
const spellMixin = Base => {
  return class extends Base {
    check() {
      return super.onKey("C-c") + " for spell checking";
    }
    onKey(key) {
      return "Process " + key + " on spell editor";
    }
  }
};
\end{lstlisting}

\paragraph{Dynamic inheritance.}

In JavaScript, a mixin is simply a function with a superclass as input and a
subclass extending that superclass as an output. Concretely, \lstinline{spellMixin}
adds a method \lstinline{check} for spell checking. It also provides
a method \lstinline{onKey}.
The function \lstinline{spellMixin} shows the typical use of what we call \emph{dynamic inheritance}.
Note that \lstinline{Base}, which is supposed to be a superclass being inherited, is \emph{parameterized}.
Therefore \lstinline{spellMixin} can be applied to any base class at
\emph{run time}. This is impossible to do, in a type-safe way, in
conventional statically typed class-based languages like Java or
C++.\footnote{With C++ templates, it is possible to
  implement a so-called mixin pattern~\citep{DBLP:conf/gcse/SmaragdakisB00}, which enables extending
a parameterized class. However C++ templates defer type-checking until
instantiation, and such pattern still does not allow selection of the
base class at run time (only at up to class instantiation time).}

It is noteworthy that not all applications of \lstinline{spellMixin} to base
classes are successful. Notice the use of the \lstinline{super} keyword in the
\lstinline{check} method. If the base class does not implement the
\lstinline{onKey} method, then mixin application will fail with a run-time error. In
a typed setting, a type system must express this requirement (i.e., the presence of
the \lstinline{onKey} method) on the (statically unknown) base class being inherited.


% The class expression inside the function body has no
% definition of the method \lstinline{version}, but which is used in the body of
% the method \lstinline{showHelp}. In a statically typed OO language, such as Java,
% we would need an \emph{abstract method} for
% \lstinline{version}.


We invite the reader to pause for a while and think about what the type of
\lstinline{spellMixin} would look like. Clearly our type system should be
flexible enough to express this kind of dynamic pattern of composition in order
to accommodate mixins (or traits), but also not too lenient to allow any
composition.


\paragraph{Mixin composition and conflicts.}
The powerful part of mixins is that \lstinline{spellMixin}'s functionality is not
tied to a particular class hierarchy and is composable with other features. For
example, we can define another mixin that adds simple modal editing---as in Vim---to an arbitrary editor:
\begin{lstlisting}[language=JavaScript]
const modalMixin = Base => {
  return class extends Base {
    constructor() {
      super();
      this.mode = "command";
    }
    toggleMode() {
      return "toggle succeeded";
    }
    onKey(key) {
      return "Process " + key + " on modal editor";
    }
  };
};
\end{lstlisting}
\lstinline{modalMixin} adds a \lstinline{mode} field that controls which
keybindings are active, initially set to the command mode, and a method
\lstinline{toggleMode} that is used to switch between modes. It also provides a method \lstinline{onKey}.

Now we can compose \lstinline{spellMixin} with \lstinline{modalMixin} to produce
a combination of functionality, mimicking some form of multiple inheritance:
\begin{lstlisting}[language=JavaScript]
class IDEEditor extends modalMixin(spellMixin(Editor)) {
  version() {
    return 0.2;
  }
}
\end{lstlisting}
The class \lstinline{IDEEditor} extends the base class \lstinline{Editor} with
modal editing and spell checking capabilities. It also defines the missing
\lstinline{version} method.

At first glance, \lstinline{IDEEditor} looks quite fine, but it has a subtle
issue. Recall that two mixins \lstinline{modalMixin} and \lstinline{spellMixin}
both provide a method \lstinline{onKey}, and the \lstinline{Editor} class also
defines an \lstinline{onKey} method of its own. Now we have a name clash. A
question arises as to which one gets picked inside the \lstinline{IDEEditor}
class. A typical mixin model resolves this issue by looking at the order of mixin applications. Mixins appearing later in the order
overrides \emph{all} the identically named methods of earlier mixins. So in our
case, \lstinline{onKey} in \lstinline{modalMixin} gets picked. If we
change the order of application to \lstinline{spellMixin(modalMixin(Editor))},
then \lstinline{onKey} in \lstinline{spellMixin} is inherited.

\paragraph{The problem of mixin composition.}
From the above discussion, we can see that mixin are composed linearly: all the
mixins used by a class must be applied one at a time. However, when we wish to
resolve conflicts by selecting features from different mixins, we may not be
able to find a suitable order. For example, when we compose the two mixins to
make the class \lstinline{IDEEditor}, we can choose which of them comes first,
but in either order, \lstinline{IDEEditor} cannot access to the \lstinline{onKey}
method from the \lstinline{Editor} class.

\paragraph{Trait model.}
Because of the total ordering and the limited means for resolving conflicts imposed by the mixin model,
researchers have proposed a simple compositional model called
traits~\citep{scharli2003traits, Ducasse_2006}. Traits are lightweight entities and serve as
the primitive units of code reuse. Among others, the key difference from
mixins is that the order of trait composition is irrelevant, and conflicting
methods must be resolved \emph{explicitly}. This gives programmers
fine-grained control, when conflicts arise, of selecting desired features from
different components. Thus we believe traits are a better model for multiple
inheritance in statically typed object-oriented languages, and in \sedel we realize this
vision by giving traits a first-class status in the language,
achieving more expressive power compared with traditional (second-class) traits.


\paragraph{Summary of typing challenges.}
From our previous discussion, we can identify the following typing challenges
for a type system to accommodate the programming patterns (first-class classes/mixins)
we have just seen:
\begin{itemize}
\item How to account for, in a typed way, abstract methods and dynamic dispatch.
\item What are the types of first-class classes or mixins.
\item How to type dynamic inheritance.
\item How to express constraints on method presence and absence (the use of
  \lstinline{super} clearly demands that).
% \item How to ensure that composition of mixins is going to be valid, i.e., how
%   to reflect linearity in a type system.
\item In the presence of first-class traits, how to detect conflicts statically,
  even when the traits involved are not statically known.
\end{itemize}
\sedel elegantly solves the above challenges in a unified way, as
we will see next.


% From a pragmatic point of view, this implicit conflict resolution
% sometimes give programmers more surprises than convenience. What if the compiler can alarm us when a
% potential conflict may occur. Because of the dynamic nature of JavaScript, we
% would not know before actually running the code that there is a conflict. We
% miss the guarantee that a static type system can provide: such conflict can be
% detected at compile-time.

% Given the flexibility of first-class classes in dynamically typed languages, we
% -- being advocates of statically typed languages -- were wondering how to
% incorporate this same expressive power into statically typed
% languages. As it
% turns out, designing a sound type system that fully supports first-class classes
% is notoriously hard; there are only a few, quite sophisticated, languages that
% manage this~\citep{DBLP:conf/oopsla/TakikawaSDTF12, DBLP:conf/ecoop/LeeASP15}. We
% pushed it further: \sedel has support for typed first-class
% traits.\bruno{Better to say there's no work on typed first-class
%   traits, and little work on first-class classes/mixins, despite
%  many dynamic languages prominently supporting such features.}

\subsection{A Glance at Typed First-Class Traits in \sedel}

We now rewrite the above code in \sedel, but this time with types. The resulting code has the same functionality as the dynamic version, but it is
statically typed. All code snippets in this and later sections are runnable in
our prototype. Before proceeding, we ask the reader to bear in mind that in this section we are not using traits
in the most canonical way, i.e., we use traits as if they are classes (but with
built-in conflict detection). This is because we are trying to stay as close as possible
to the structure of the JavaScript code for ease of comparison. In
\cref{sec:traits} we will remedy this to make better use of traits.

\paragraph{Simple traits.}
Below is a simple trait \lstinline{editor}, which corresponds to the JavaScript
class \lstinline{Editor}. The \lstinline{editor} trait defines the same set of
methods: \lstinline{on_key}, \lstinline{do_cut} and \lstinline{show_help}:
\lstinputlisting[linerange=23-27]{./examples/overview2.sl}% APPLY:linerange=OVERVIEW_EDITOR
The first thing to notice is that \sedel uses a syntax (similar to Scala's
self type annotations~\citep{odersky2004overview}) where we can give a type annotation to the
\lstinline{self} reference. In the type of \lstinline{self} we use
\lstinline{&} construct to create intersection types. \lstinline{Editor} and \lstinline{Version} are two record types:
\lstinputlisting[linerange=10-17]{./examples/overview2.sl}% APPLY:linerange=OVERVIEW_EDITOR_TYPES
For the sake of conciseness, \sedel uses \lstinline{type} aliases to abbreviate types.

\paragraph{The type of \lstinline{self} encodes abstract methods.}
Recall that in the JavaScript class \lstinline{Editor}, the \lstinline{version}
method is undefined, but is used inside \lstinline{showHelp}. How can we express
this in the typed setting, if not with an abstract method? In \sedel, the type of \lstinline{self}
plays the role of trait requirements. As a first approximation,
the invocation of \lstinline{version} on \lstinline{self} is justified by noticing that (part of) the
type of \lstinline{self} (i.e., \lstinline{Version}) contains the declaration of
\lstinline{version}. An interesting aspect of \sedel's trait model is that there
is no need for abstract methods. Instead, abstract methods can be simulated as
requirements of a trait. Later, when the trait is composed with other
traits, \emph{all} requirements on the type of \lstinline{self} must be
satisfied and one of the traits in the composition must provide an
implementation of the method \lstinline{version}.
%to this point in \cref{sec:traits}.

As with the JavaScript code, the \lstinline{on_key} method is invoked on
\lstinline{self} in the body of \lstinline{do_cut}. This is allowed as (part of)
the type of \lstinline{self} (i.e., \lstinline{Editor}) contains the signature
of \lstinline{on_key}. Compared to the JavaScript class
\lstinline{Editor}, almost everything stays the same, except that we now have
a typed version. As a side note, since \sedel is currently a pure ``functional'' object-oriented
language, there is no difference between fields and methods, so we can omit
empty arguments and parameter parentheses.

\paragraph{First-class traits and trait expressions.}

\sedel treats traits as first-class expressions, putting them in the same
syntactic category as objects, functions, and other primitive forms. To
illustrate this, we give the \sedel version of \lstinline{spellMixin}:
\lstinputlisting[linerange=31-43]{./examples/overview2.sl}% APPLY:linerange=OVERVIEW_HELP
This looks daunting at first, but \lstinline{spell_mixin} has almost the same structure as
its JavaScript counterpart \lstinline{spellMixin}, albeit with
some type annotations. In \sedel, we use capital letters (\lstinline{A}, \lstinline{B}, $\dots$) to denote type variables, and trait
expressions \lstinline$trait [self : ...] inherits ... => {...}$ to create
first-class traits. Trait expressions have trait
types of the form \lstinline{Trait[T1, T2]} where \lstinline{T1} and \lstinline{T2} denote trait requirements and functionality respectively.
We will explain trait types in \cref{sec:traits}. Despite the structural similarities, there are several significant
features that are unique to \sedel (e.g., the disjointness operator \lstinline{*}).
We discuss these in the following.



\paragraph{Disjoint polymorphism and conflict detection.}

\sedel uses a type system based on \emph{disjoint intersection types} (cf. \cref{chap:nested}) and
\emph{disjoint polymorphism} (cf. \cref{chap:fi}). Disjoint intersections
empower \sedel to detect conflicts statically when trying to compose two
traits with identically named features. For example, composing two traits
\lstinline{a} and \lstinline{b} that both provide \lstinline{foo} gives a
type error (the overloaded \lstinline{&} operator denotes trait composition):
\begin{lstlisting}
trait a => { foo = 1 };
trait b => { foo = 2 };
trait c inherits a & b => {}; -- type error!
\end{lstlisting}
Disjoint polymorphism, as a more advanced mechanism, allows detecting conflicts
even in the presence of polymorphism---for example when a trait is parameterized and its
full set of methods is not statically known. As can be seen,
\lstinline{spell_mixin} is actually a polymorphic function. Unlike ordinary
parametric polymorphism, in \sedel, a type variable can also have a disjointness
constraint. For instance, \lstinline{A * Spelling & OnKey}
means that \lstinline{A} can be instantiated to any type as long as it \emph{does not}
contain \lstinline{check} and \lstinline{on_key}. Note that these are the minimal constraints of \lstinline{A}, as
\begin{inparaenum}[(1)]
  \item \lstinline{A} cannot contain the \lstinline{on_key} method because otherwise it will conflict with \lstinline{Editor};
  \item \lstinline{A} cannot contain the \lstinline{check} method because otherwise it will conflict with that in the trait body.
\end{inparaenum}
To mimic mixins, the
argument \lstinline{base}, which is supposed to be some trait, serves as the
``base'' trait being inherited. Notice that the type variable
\lstinline{A} appears in the type of \lstinline{base}, which essentially states
that \lstinline{base} is a trait that contains at least those methods specified
by \lstinline{Editor}, and possibly more (which we do not know statically).
% In summary, \lstinline{Trait[Editor & Version, Editor & A]} (the assigned type
% of \lstinline{base}) specifies that both method \emph{presence} and \emph{absence}.
Also note that leaving out the \lstinline{override} keyword will result in a
type error. The type system is forcing us to be very specific as to what is the
intention of the \lstinline{on_key} method because it sees the same method is
also declared in \lstinline{base}, and blindly inheriting \lstinline{base}
will definitely cause a method conflict. As a final note, the use of \lstinline{super}
inside \lstinline{check} is allowed because the ``super-trait'' \lstinline{base}
implements \lstinline{on_key}, as can be seen from its type.


\paragraph{Typing dynamic inheritance.}

Disjoint polymorphism enables us to correctly type dynamic inheritance:
\lstinline{spell_mixin} is able to take any trait that conforms with its
assigned type, equips it with the \lstinline{check} method and overrides its
old \lstinline{on_key} method. As a side note, the use of disjoint polymorphism
is essential to correctly model the mixin semantics. From the type we know
\lstinline{base} has some features specified by \lstinline{Editor}, plus
something more denoted by \lstinline{A}. By inheriting \lstinline{base}, it is
guaranteed that the resulting trait will have everything that is already contained
in \lstinline{base}, plus more features. This is in some sense similar to row
polymorphism~\citep{wand1994type} in that the result trait is prohibited from
forgetting methods from the argument trait. (\Cref{sec:future:row} has further discussion about the relationship
between row polymorphism and disjoint polymorphism.)
% As we will discuss in \cref{sec:related}, disjoint polymorphism is more expressive than row polymorphism.


\paragraph{Typing mixin composition.}

Next we give the typed version of \lstinline{modalMixin} as follows:
\lstinputlisting[linerange=48-56]{./examples/overview2.sl}% APPLY:linerange=OVERVIEW_MODAL
Now the definition of \lstinline{modal_mixin} should be self-explanatory.
Finally we can apply both ``mixins'' to \lstinline{editor} one at a time to create
an IDE editor:
\lstinputlisting[linerange=61-66]{./examples/overview2.sl}% APPLY:linerange=OVERVIEW_LINE
As with the JavaScript class \lstinline{IDEEditor}, we need to fill in the missing
\lstinline{version} method. It is easy to verify that the \lstinline{on_key} method
in \lstinline{modal_mixin} is inherited. Compared with the untyped version,
here this behavior is reasonable because we specifically tag each
\lstinline{on_key} method to be an overriding method. Let us take a close look
at the mixin applications. Since \sedel is currently explicitly typed, we need to
provide concrete types when applying \lstinline{modal_mixin} and \lstinline{spell_mixin}.
In the inner application (\lstinline{spell_mixin Top editor}), we use the top
type \lstinline{Top} to instantiate \lstinline{A} because the \lstinline{editor} trait
provides exactly those method specified by \lstinline{Editor} and nothing more
(hence \lstinline{Top}). In the outer application, we use \lstinline{Spelling}
to instantiate \lstinline{A} because the resulting trait of the inner application
contains the \lstinline{check} method.
In summary, mixin applications are simply normal function applications,
and conflict resolution code is implicitly embedded via the keyword \lstinline{override}
and the order of mixin applications.
Unsurprisingly, changing the mixin application order to
\begin{lstlisting}
  inherits spell_mixin ModalEdit (modal_mixin Top editor)
\end{lstlisting}
gives the same difference as in the JavaScript version (\lstinline{on_key} from \lstinline{spell_mixin} is inherited).


Admittedly the typed version is unnecessarily complicated as we were
mimicking mixins by functions over traits. The final trait
\lstinline{ide_editor} suffers from the same problem as the class
\lstinline{IDEEditor}, since there is no obvious way to access the
\lstinline{on_key} method in the \lstinline{editor} trait.\footnote{In fact, as
  we will see in \cref{sec:trait:forward}, we can still access it by the forwarding operator.} \cref{sec:traits}
makes better use of traits to simplify the editor code.



% Note that the use of \lstinline{override} is valid because the type system knows the inherit clause contains \lstinline{on_key}.
% As a bonus, since \sedel guarantees that there are no potential conflicts in a program,
% we can reason that the version number in \lstinline{modal_editor} is
% \lstinline{0.1}.

%%% Local Variables:
%%% mode: latex
%%% TeX-master: "../paper"
%%% org-ref-default-bibliography: ../paper.bib
%%% End:
